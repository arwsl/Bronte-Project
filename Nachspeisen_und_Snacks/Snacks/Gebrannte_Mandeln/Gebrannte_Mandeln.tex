\begin{recipeDP}
    [
        preparationtime = {\SI{20}{\minute}},
        % bakingtime = {\SI{ZEIT}{\minute} bis \SI{ZEIT}{\minute}},
        % bakingtemperature = {\protect\bakingtemperature{fanoven=\SI{TEMPERATUR}{\celsius}}},
        % portion = {PORTIONEN/MENGE},
        source = {oetker.de}
    ]
    {Gebrannte Mandeln\index{Mandeln!gebrannte}}

    \graph
        {
            big=Nachspeisen_und_Snacks/Snacks/Gebrannte_Mandeln/big.jpg,
            % small=TEIL/KAPITEL/REZEPT/small.jpg
        }

    % \introduction{
    %     EINLEITUNG
    % }

    \ingredients{
        \SI[]{100}{\g} & Zucker \\
        \SI[]{4}{\EL} & Wasser \\
        \SI[]{200}{\g} & ganze, ungeschälte Mandeln \\
        \SI[]{2}{\pck} & Vanillezucker \index{Zucker!Vanille-}
    }

    \preparation{
        \step Zucker und Wasser in einer beschichteten Pfanne oder in einer Edelstahlpfanne mischen und unter Rühren aufkochen. Dann die Mandeln hinzugeben und alles gut verrühren. 
        \step Bei mittlerer Hitze so lange köcheln, bis das Wasser nach und nach verdampft ist. Das dauert etwa 5 Min. Der Zucker legt sich als weiße krümelige Schicht um die Mandeln. Die Mandeln nun weiter unter Rühren in der Pfanne karamellisieren lassen. Es bilden sich Karamellfäden und die Mandeln bräunen. Gut aufpassen, dass sie nicht zu dunkel werden!
        \step Nun den Vanille-Zucker unter Rühren einrieseln lassen. Die gebrannten Mandeln noch einmal gut durchmischen und auf einen Bogen Backpapier geben. Die Mandeln voneinander trennen und erkalten lassen. 
    }

    \suggestion[Gebrannte Haselnüsse]{
		Statt Mandeln kann man auch super gebrannte Haselnüsse zubereiten.
    }

    \hint{
        Die gebrannten Mandeln kann man etwa 2 Wochen aufbewahren.
    }

\end{recipeDP}