\begin{recipeDP}
    [
        preparationtime = {\SI{30}{\minute}},
        bakingtime = {\SI{25}{\minute}},
        bakingtemperature = {\protect\bakingtemperature{fanoven=\SI{180}{\celsius}}},
        portion = {50 Stück},
        source = {zuckerjagdwurst.com}
    ]
    {Stollenkonfekt}

    \graph
        {
            big=Nachspeisen_und_Snacks/Snacks/Stollenkonfekt/big.jpg,
            small=Nachspeisen_und_Snacks/Snacks/Stollenkonfekt/small.jpg
        }

    % \introduction{
    %     Weihnachtliches Konfekt mit Marzipan. \index{Weihnachten} Am besten am Vortag mit dem Einweichen beginnen.
    % }

    \ingredients{
        \SI{100}{\g} & \addtoidx{Orangeat} \\
        \SI{100}{\g} & \addtoidx{Zitronat} \\
        \SI{100}{\g} & Rosinen \\
        \SI{100}{\g} & Marzipanrohmasse\index{Marzipan} \\
        \SI{100}{\ml} & Orangensaft \\
        \\
        \multicolumn{2}{l}{\textbf{Teig}} \\
        \SI{500}{\g} & Mehl \\
        \SI{100}{\g} & Zucker \\
        \SI{100}{\g} & gemahlene Mandeln \\
        \SI{1}{\TL} & \addtoidx{Christstollengewürz} \\
        \SI{1}{\pck} & Trockenhefe \\
        \SI{1}{\pck} & Vanillezucker \\
        \SI{50}{\ml} & Pflanzenöl \\
        \SI{200}{\g} & Margarine \\
		\SI[parse-numbers = false]{\nicefrac{1}{2}}{\TL} & Bittermandelaroma \\
        \SI{150}{\ml} & Sojamilch \\
         & Puderzucker
    }

    \preparation{
        \step Orangeat, Zitronat und Rosinen in eine Schüssel geben. Orangensaft dazugeben und am besten über Nacht (mindestens aber 2 Stunden) ziehen lassen.

        \step Für den Teig Mehl, Zucker, gemahlene Mandeln, Christstollengewürz, Trockenbackhefe und Vanillezucker in einer Schüssel vermischen.
        Stückweise das Marzipan grob hinein raspeln, dabei immer wieder umrühren damit es sich gut verteilt und nicht verklebt.
        Anschließend Pflanzenöl, pflanzliche Butter, Bittermandelaroma und pflanzliche Milch dazugeben.
        Die Zutaten zu einem leicht klebrigen Teig verkneten.

        \step Überschüssigen Orangensaft der eingeweichten Mischung mit Orangeat und Co. abgießen und den eingeweichten Mix zum Teig geben und unterkneten.
        Dabei so viel Mehl wie nötig dazugeben, damit der Teig nicht mehr klebt.

        \step Den Teig \ca 30 Minuten an einem warmen Platz ruhen lassen.
        In der Zwischenzeit den Ofen auf 180°C (Umluft) vorheizen und ein Backblech mit Backpapier auslegen.

        \step Den Teig auf einer bemehlten Fläche etwa 2 bis 3 Zentimeter dick ausrollen und in kleine Stücke schneiden. Diese auf dem vorbereiteten Backblech verteilen.
        Das Stollenkonfekt \ca 13 Minuten backen.
        Es sollte an der Oberfläche noch nicht gebräunt und knusprig werden, sondern eher noch ein bisschen zu weich wirken.
        Die kleinen Stückchen auf einem Kuchengitter abkühlen lassen.
        Danach das Stollenkonfekt dick mit Puderzucker bestäuben und in einem luftdichten Behälter aufbewahren.
    }

    % \suggestion[TITEL EINES VORSCHLAGS]{
	% 	VORSCHLAG (DURCH HORIZONTALE LINIE VOM REZEPT GETRENNT)
    % }

    % \hint{
    %     HINWEIS (IN EINEM KASTEN UNTEN AUF DER SEITE)
    % }

\end{recipeDP}