\begin{recipeDPToTest}
    [
        preparationtime = {\SI{15}{\minute}},
        bakingtime = {\SI{30}{\minute}},
        bakingtemperature = {\protect\bakingtemperature{fanoven=\SI{180}{\celsius}}},
        portion = {10 Stück},
        source = {@wearesovegan}
    ]
    {\addtoidx{Protein} \addtoidx{Flapjacks}}

    \graph
        {
            big=Nachspeisen_und_Snacks/Snacks/Protein_Flapjacks/big.jpg,
            small=Nachspeisen_und_Snacks/Snacks/Protein_Flapjacks/small.jpg
        }

    % \introduction{einleitung}

    \ingredients{
        2 & Bananen \\
        \SI{150}{\g} & \addtoidx{Haferflocken}, fein \\
        \SI{25}{\g} & Sonnenblumenkerne \\
        \SI{30}{\g} & Kürbiskerne \\
        \SI{30}{\g} & Rosinen \\
        \SI{30}{\g} & Walnüsse \\
        1 Prise & Salz \\
        \\
        \SI{180}{\g} & \addtoidx{Erdnussbutter} \\
        \SI{5}{\EL} & Ahornsirup \\
        \SI{3}{\EL} & Kokosöl \\
        \SI{1}{\EL} & Soja-Drink
    }

    \preparation{
        \step Die Bananen in einer großen Schüssel zerdrücken. Dann die Haferflocken, Sonnenblumenkerne, Kürbiskerne, Rosinen, Walnüsse und das Salz hinzugeben. In einem Topf die Erdnussbutter, den Ahornsirup mit dem Kokosöl zusammen erhitzen, bis eine homogene Masse entsteht.
        \step Die warme Masse in die Schüssel geben, den Soja-Drink hinzugeben und alles gut vermischen. Den fertigen Teig in eine \SI{20}{\cm}x\SI{20}{\cm} große, mit Backpapier ausgelegte Backform geben und für etwa \SI{30}{\minute} bei \SI{180}{\celsius} backen.
        \step Den gebackenen Teig noch warm in 10 Riegel schneiden und auskühlen lassen.
    }

    % \suggestion[Title of Suggestion]{
	% 	Suggestion
    % }
    %
    % \hint{Hint}

\end{recipeDPToTest}
