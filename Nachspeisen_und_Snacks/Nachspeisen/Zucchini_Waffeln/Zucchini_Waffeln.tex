\begin{recipeDP}
    [
        preparationtime = {\SI{15}{\minute}},
        % bakingtime = {\SI{ZEIT}{\minute} bis \SI{ZEIT}{\minute}},
        % bakingtemperature = {\protect\bakingtemperature{fanoven=\SI{TEMPERATUR}{\celsius}}},
        portion = {8 Waffeln},
        source = {chefkoch.de}
    ]
    {Zucchini Waffeln}

    \graph
        {
            big=Nachspeisen_und_Snacks/Nachspeisen/Zucchini_Waffeln/big.jpg,
            % small=TEIL/KAPITEL/REZEPT/small.jpg
        }

    % \introduction{
    %     EINLEITUNG
    % }

    \ingredients{
        \SI{200}{\g} & Mehl \\
        \SI{7}{\g} & Backpulver \\
        \SI{20}{\g} & Tomatenmark \\
        \SI[parse-numbers = false]{\nicefrac{1}{2}}{} & Zwiebel \\
        2 & \addtoidx{Zucchini}, mittelgroß \\
        \SI{100}{\g} & Reibekäse \\
        \SI{1}{\TL} & Salz \\
        \SI[parse-numbers = false]{1\nicefrac{1}{2}}{\TL} & Paprikapulver \\
        \SI[parse-numbers = false]{1\nicefrac{1}{2}}{\TL} & Oregano \\
        \SI[parse-numbers = false]{1\nicefrac{1}{2}}{\TL} & Basilikum \\
        \SI[parse-numbers = false]{1\nicefrac{1}{2}}{\TL} & Thymian \\
        \SI[parse-numbers = false]{\nicefrac{1}{2}}{\TL} & Pfeffer \\
        \SI[parse-numbers = false]{\nicefrac{1}{2}}{\TL} & Knoblauchpulver \\
        \SI{150}{\ml} & Mineralwasser
    }

    \preparation{
        \step Die Zucchini waschen und grob raspeln.
        Die Zwiebel würfeln.
        \step Das Mehl mit dem Backpulver und den Gewürzen in der Schüssel mischen.
        Dann das Gemüse und das Tomatenmark dazugeben.
        Unter Rühren das Mineralwasser hinzufügen, bis ein homogener, dickflüssiger Teig entsteht.
        Ggf. mehr oder weniger Mineralwasser verwenden.
        \step Das Waffeleisen aufheizen, einfetten und etwa 1,5 Esslöffel Teig in die Mitte des Waffeleisens geben.
        Die Waffeln nach Wunsch kross durchbacken.
    }

    % \suggestion[TITEL EINES VORSCHLAGS]{
	% 	VORSCHLAG (DURCH HORIZONTALE LINIE VOM REZEPT GETRENNT)
    % }

    \hint{
        Nach Belieben können auch z. B. Paprikawürfel mit hinein gegeben werden.
    }

\end{recipeDP}