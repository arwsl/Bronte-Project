\begin{recipeDP}
    [
        preparationtime = {\SI{45}{\minute}},
        % bakingtime = {\SI{ZEIT}{\minute} bis \SI{ZEIT}{\minute}},
        % bakingtemperature = {\protect\bakingtemperature{fanoven=\SI{TEMPERATUR}{\celsius}}},
        portion = {4-5 Pancakes},
        source = {zuckerjagdwurst.com}
    ]
    {\addtoidx{Pancakes}}

    \graph
        {
            big=Nachspeisen_und_Snacks/Nachspeisen/Pancakes/big.jpg,
            small=Nachspeisen_und_Snacks/Nachspeisen/Pancakes/small.jpg
        }

    % \introduction{
    %     EINLEITUNG
    % }

    \ingredients{
        \SI{125}{\g} & Weizenmehl (Type 405) \\
        \SI{1}{\EL} & Zucker \\
        1 Prise & Salz \\
        \SI{4}{\g} & \addtoidx{Backpulver} \\
        \SI{150}{\ml} & Sprudelwasser
    }

    \preparation{
        \step Das Mehl in eine große Schüssel hinein sieben.
        Zucker, Salz und Backpulver hinein geben und alles vermischen.
        \step Das Wasser dazu geben und (unbedingt) per Hand mit einem Schneebesen kurz und vorsichtig verrühren.
        Dabei dürfen auch noch kleine Klumpen übrig bleiben, bloß nicht zu lange oder zu starkm schlagen.
        \step Den Teig für 15 Minuten ruhen lassen.
        Dabei zieht auch in die kleinen Klümpchen noch Feuchtigkeit ein.
        \step Eine beschichtete Pfanne auf mittlere Stufe vorheizen und dann dann kleine Pancakes darin ausbraten.
        Wenn oben der Teig nicht mehr glasig ist, wenden.
        Von beiden Seiten goldbraun braten.
    }

    % \suggestion[TITEL EINES VORSCHLAGS]{
	% 	VORSCHLAG (DURCH HORIZONTALE LINIE VOM REZEPT GETRENNT)
    % }

    % \hint{
    %     HINWEIS (IN EINEM KASTEN UNTEN AUF DER SEITE)
    % }

\end{recipeDP}