\documentclass[
a4paper,
11pt,
oneside,
]{scrartcl}

\usepackage[T1]{fontenc}
\usepackage[utf8]{inputenc}
\usepackage{lmodern}
\usepackage[ngerman]{babel}

\usepackage{textcomp}
\usepackage{gensymb}

\usepackage{paracol}
\usepackage{supertabular}
\usepackage{lipsum}

\usepackage{amssymb}                                % mathematische Symbole importieren
\usepackage{amsmath}                                % Mathematikumgebungen (align) importieren
\usepackage{nicefrac}
\usepackage{siunitx}

\newcommand{\ca}{\acs{ca}\,}
\newcommand{\pck}{\acs{Pck}\,}
\newcommand{\msp}{\acs{Msp}\,}
\newcommand{\geh}{\acs{geh}\,}
\newcommand{\Ta}{\acs{Ta}\,}
\newcommand{\TL}{\acs{TL}}
\newcommand{\EL}{\acs{EL}}
\newcommand{\dmesser}{\(\varnothing\)}

\usepackage{xcolor}

%% Hyperlinks setzen, PDF bearbeiten %%%%%%%%%%%%%%%%%%%%%%%%%%%%%%%%%%
\usepackage[
		bookmarks=true,
		% bookmarksopen=true,							% Kapitelmarken in Acrobat anzeigen
		% bookmarksopenlevel=1,						% nu oberste Ebene der Kapitel anzeigen
		pagebackref=true,							% Links: aus Literatur in Text
		% pdfpagelayout=TwoColumnRight,				% Anzeige: scollen und zweiseitig
		breaklinks, 								% Allows link text to break across lines; This makes links on multiple lines into different PDF links to the same target
		hidelinks,									% Farbe und Rahmen der Links entfernen
		linktoc=all,									% Text, Seitenzahlen in TOC, LOF, LOT
		% Fit,										% Seite an Fenster anpassen
	]{hyperref}
\hypersetup{										% Einstellungen zum pdf
	%pdftoolbar=false,							% Acrobat toolbar aus
	% pdfmenubar=false,							% Acrobat menu aus
	pdftitle={Vegane Gerichte und Getränke},
	pdfsubject={Rezeptsammlung},
	pdfauthor={arwsl},
}

\usepackage{todonotes}

\usepackage[printonlyused]{acronym}					% Abkürzungsverzeichnis
\makeatletter										% \acs-Befehl neu definieren: \mbox entfernt, sodass Umbrüche möglich sind
\renewcommand*\AC@acs[1]{%
    \expandafter\AC@get\csname fn@#1\endcsname\@firstoftwo{#1}}
\makeatother


\usepackage[ngerman]{minitoc} % kleine TOC einfügen
\mtcsetfeature{secttoc}{after}{\cleardoublepage}
% \mtcsetfeature{minitoc}{pagestyle}{\thispagestyle{empty}}
% \mtcselectlanguage{german}


%%% Farbendefinitionen                                                %%
\definecolor{ostfaliaBlau}{HTML}{003A79}
\definecolor{PTBBlau}{HTML}{009CCF}
\definecolor{PTBRot}{HTML}{CF3300}

\definecolor{petrol}{HTML}{11697A}
\definecolor{sundown}{HTML}{DB6400}
\definecolor{sand}{HTML}{FFA62B}
\definecolor{ivory}{HTML}{F8F1F1}
