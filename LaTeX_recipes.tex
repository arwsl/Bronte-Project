%%%%%%%%%%%%%%%%%%%%%%%%%%%%%%%%%%%%%%%%%%%%%%%%%%%%%%%%%%%%%%%%%%%%%%%
%% Diese tex-Datei kompilieren, Vorgaben können so übernommen werden %%
%% Diese tex-Datei anpassen:                                         %%
%% 1. Kopf- und Fußzeile definieren                                  %%
%% 2. Counter anpassen: Überschrift, Seite, Gleichung                %%
%% 3. Kapitel (tex-Dateien) einfügen mit \input{}                    %%
%%(4. neue Kommandos können zu den bestehenden in der Präambel hinzu-%%
%%    gefügt werden)                                                 %%
%%%%%%%%%%%%%%%%%%%%%%%%%%%%%%%%%%%%%%%%%%%%%%%%%%%%%%%%%%%%%%%%%%%%%%%




%%%%%%%%%%%%%%%%%%%%%%%%%%%%%%%%%%%%%%%%%%%%%%%%%%%%%%%%%%%%%%%%%%%%%%%
%%%%%%%%%%%%%%%%%%%%%%%%%%%%%%%%%%%%%%%%%%%%%%%%%%%%%%%%%%%%%%%%%%%%%%%
%% Präambel                                                          %%
%%%%%%%%%%%%%%%%%%%%%%%%%%%%%%%%%%%%%%%%%%%%%%%%%%%%%%%%%%%%%%%%%%%%%%%
%%%%%%%%%%%%%%%%%%%%%%%%%%%%%%%%%%%%%%%%%%%%%%%%%%%%%%%%%%%%%%%%%%%%%%%

% !TeX root = LaTeX_recipes.tex

\documentclass[
a4paper,
11pt,
twoside,
]{scrartcl}

% \usepackage{atbegshi}% http://ctan.org/pkg/atbegshi
% \AtBeginDocument{\AtBeginShipoutNext{\AtBeginShipoutDiscard}}

% \newcommand*\cleartoleftpage{%
%   \clearpage
%   \ifodd\value{page}\hbox{}\newpage\fi
% }


\usepackage[T1]{fontenc}
\usepackage[utf8]{inputenc}
\usepackage{lmodern}
\usepackage[ngerman]{babel}

\usepackage{textcomp}
\usepackage{gensymb}

\usepackage{paracol}
\usepackage{supertabular}
% \usepackage{lipsum}

\usepackage{amssymb}                                % mathematische Symbole importieren
\usepackage{amsmath}                                % Mathematikumgebungen (align) importieren
\usepackage{nicefrac}
\usepackage{siunitx}[=v2]

\newcommand{\ca}{\acs{ca}\,}
\newcommand{\pck}{\acs{Pck}\,}
\newcommand{\msp}{\acs{Msp}\,}
\newcommand{\geh}{\acs{geh}\,}
\newcommand{\Ta}{\acs{Ta}\,}
\newcommand{\TL}{\acs{TL}}
\newcommand{\EL}{\acs{EL}}
\newcommand{\dmesser}{\(\varnothing\)}

\usepackage{xcolor}

%% Hyperlinks setzen, PDF bearbeiten %%%%%%%%%%%%%%%%%%%%%%%%%%%%%%%%%%
\usepackage[
		bookmarks=true,
		% bookmarksopen=true,							% Kapitelmarken in Acrobat anzeigen
		% bookmarksopenlevel=1,						% nu oberste Ebene der Kapitel anzeigen
		pagebackref=true,							% Links: aus Literatur in Text
		pdfpagelayout=TwoColumnRight,				% Anzeige: scollen und zweiseitigs
		breaklinks, 								% Allows link text to break across lines; This makes links on multiple lines into different PDF links to the same target
		hidelinks,									% Farbe und Rahmen der Links entfernen
		linktoc=all,									% Text, Seitenzahlen in TOC, LOF, LOT
		% Fit,										% Seite an Fenster anpassen
	]{hyperref}
\hypersetup{										% Einstellungen zum pdf
	%pdftoolbar=false,							% Acrobat toolbar aus
	% pdfmenubar=false,							% Acrobat menu aus
	pdftitle={Vegane Gerichte und Getränke},
	pdfsubject={Rezeptsammlung},
	pdfauthor={arwsl},
}

\usepackage{todonotes}

\usepackage[printonlyused]{acronym}					% Abkürzungsverzeichnis
\makeatletter										% \acs-Befehl neu definieren: \mbox entfernt, sodass Umbrüche möglich sind
\renewcommand*\AC@acs[1]{%
    \expandafter\AC@get\csname fn@#1\endcsname\@firstoftwo{#1}}
\makeatother


\usepackage[ngerman]{minitoc} % kleine TOC einfügen
\mtcsetfeature{secttoc}{after}{\clearpage}
% \mtcsetfeature{minitoc}{pagestyle}{\thispagestyle{empty}}


\usepackage{imakeidx}
\AtBeginDocument{\renewcommand{\indexname}{Stichwortverzeichnis\vspace{1cm}}}
\renewcommand{\seename}{siehe}
\renewcommand{\alsoname}{siehe auch}
\newcommand{\addtoidx}[1]{#1\index{#1}}
\newcommand{\addsubidx}[2]{#1\index{#1!#2}}
\makeindex[columns=2]


%%% Farbendefinitionen                                                %%
\definecolor{ostfaliaBlau}{HTML}{003A79}
\definecolor{PTBBlau}{HTML}{009CCF}
\definecolor{PTBRot}{HTML}{CF3300}

\definecolor{petrol}{HTML}{11697A}
\definecolor{sundown}{HTML}{DB6400}
\definecolor{sand}{HTML}{FFA62B}
\definecolor{ivory}{HTML}{F8F1F1}


\usepackage[nowarnings]{xcookybooky}											   % Rezepte typesetten

\setRecipeColors{
	recipename = petrol,
	ing = black,
	inghead = sundown,
	prep = black,
	prephead = sundown,
	suggestion = black,
	suggestionhead = petrol,
	separationgraph = petrol,
	hint = sand,
	hinthead = sundown,
	hintline = sundown,
	numeration = sundown,
}

\setRecipeLengths{
	pictureheight = 0.2\textheight,
	bigpicturewidth = 0.55\textwidth, % ratio: pw/ph = 11/4
	smallpicturewidth = 0.35\textwidth, % ratio: pw/ph = 7/4
	introductionwidth = 0.75\textwidth,
	preparationwidth = 0.55\textwidth,
	ingredientswidth = 0.35\textwidth,
}

\setRecipeSizes{
	recipename = \Huge,
	intro = \normalsize,
	ing = \normalsize,
	inghead = \Large,
	prephead = \Large,
	suggestion = \normalsize,
	hint = \large,
	hinthead = \Large,
}

% \usepackage{emerald}
% \setRecipenameFont{fwb}{T1}{m}{n}

\setRecipenameFont{\sfdefault}{T1}{b}{n}

\setHeadlines{
	inghead = Zutaten,
	prephead = Zubereitung,
	hinthead = Anmerkung,
	continuationhead = { },
	continuationfoot = { },
	portionvalue = Portion(en),
	calory = Energie,
}


% updated package implementations
\renewcommand{\step}
{%
    \stepcounter{step}%
    \lettrine
    [%
        lines=2,
        lhang=0,          % space into margin, value between 0 and 1
        loversize=0.1,   % enlarges the height of the capital
        slope=0em,
        findent=0.75em,      % gap between capital and intended text
        nindent=0em       % shifts all intended lines, begining with the second line
    ]{\thestep}{}%
}

\renewcommand{\sectionmark}[1]
{
	\markright{\MakeUppercase{\thesection.\ #1}}
}
\setlength{\evensidemargin}{0cm}
\setlength{\oddsidemargin}{0cm}
\setlength{\textheight}{23.0cm}
% lange Zutatenliste:
% https://stackoverflow.com/questions/64688876/using-xcookybooky-in-latex-i-have-many-ingredients-but-it-does-go-into-the-nex



%%%%%%%%%%%%%%%%%%%%%%%%%%%%%%%%%%%%%%%%%%%%%%%%%%%%%%%%%%%%%%%%%%%%%%%%
%%% Counter setzen                                                    %%
%%%%%%%%%%%%%%%%%%%%%%%%%%%%%%%%%%%%%%%%%%%%%%%%%%%%%%%%%%%%%%%%%%%%%%%%
%\setcounter{section}{1}												% starte mit anderer Überschrift-Nr. (+1)
\setcounter{page}{0}													% starte mit anderer Seitenzahl

% Rezepte Zählen
\newcounter{recipeCntr}
\setcounter{recipeCntr}{0}


\setcounter{secnumdepth}{0}
\setcounter{tocdepth}{3}
\renewcommand{\recipesection}[2][]
{
	\subsection[#1]{#2}
}

\newcommand{\sectionwithtoc}[1]{
	\section{#1}
	\secttoc
}

% Rezept auf gleicher Seite starten (single page)
\newenvironment{recipeSP}{\stepcounter{recipeCntr}\begin{recipe}}{\end{recipe}}
% Rezept auf neuer Seite starten (double page)
\newenvironment{recipeDP}{\clearpage\stepcounter{recipeCntr}\begin{recipe}}{\end{recipe}}

% Neue, nicht getestete Rezepte
\newenvironment{recipeSPToTest}{
	\stepcounter{recipeCntr}
	{noch testen!} \\
		\begin{tikzpicture}
			\filldraw [fill=sundown,draw=sundown] (0,0) rectangle (15,0.3);
		\end{tikzpicture}\\
	\begin{recipeSP}
}{
\end{recipeSP}
		\begin{tikzpicture}
			\filldraw [fill=sundown,draw=sundown] (0,0) rectangle (15,0.3);
		\end{tikzpicture}
}
\newenvironment{recipeDPToTest}{\clearpage\begin{recipeToTest}}{\end{recipeToTest}}

% \newcommand{\partwithtoc}[1]{
% 	\part{#1}
% 	\parttoc
% }

% \SI[parse-numbers = false]{\nicefrac{1}{2}}{\Ta} &

%%%%%%%%%%%%%%%%%%%%%%%%%%%%%%%%%%%%%%%%%%%%%%%%%%%%%%%%%%%%%%%%%%%%%%%%
%%%%%%%%%%%%%%%%%%%%%%%%%%%%%%%%%%%%%%%%%%%%%%%%%%%%%%%%%%%%%%%%%%%%%%%%
%%% Dokument                                                          %%
%%%%%%%%%%%%%%%%%%%%%%%%%%%%%%%%%%%%%%%%%%%%%%%%%%%%%%%%%%%%%%%%%%%%%%%%
%%%%%%%%%%%%%%%%%%%%%%%%%%%%%%%%%%%%%%%%%%%%%%%%%%%%%%%%%%%%%%%%%%%%%%%%


% \begin{RECIPE-UMGEBUNG}
%     [
%         preparationtime = {\SI{ZEIT}{\minute}},
%         bakingtime = {\SI{ZEIT}{\minute} bis \SI{ZEIT}{\minute}},
%         bakingtemperature = {\protect\bakingtemperature{fanoven=\SI{TEMPERATUR}{\celsius}}},
%         portion = {PORTIONEN/MENGE},
%         source = {QUELLE/HERAUSGEBER}
%     ]
%     {REZEPTNAME}
%
%     \graph
%         {
%             big=TEIL/KAPITEL/REZEPT/big.jpg,
%             small=TEIL/KAPITEL/REZEPT/small.jpg
%         }
%
%     \introduction{
%         EINLEITUNG
%     }
%
%     \ingredients{
%         \SI{1}{\ml} & ZUTAT 1 \\
%         \SI{2}{\g} & ZUTAT 2 \\
%         \\
%         \multicolumn{2}{l}{\textbf{ABSCHNITTÜBERSCHRIFT}} \\
%		  \SI[parse-numbers = false]{\nicefrac{1}{2}}{\Ta} & Öl \\
%         \SI{3}{\EL} & ZUTAT 3 \\
%         4 PRISEN & ZUTAT 4
%     }
%
%     \preparation{
%         \step ANTLEITUNG IN MEHREREN SCHRITTEN
%         \step NÄCHSTER SCHRITT...
%     }
%
%     \suggestion[TITEL EINES VORSCHLAGS]{
% 		VORSCHLAG (DURCH HORIZONTALE LINIE VOM REZEPT GETRENNT)
%     }
%
%     \hint{
%         HINWEIS (IN EINEM KASTEN UNTEN AUF DER SEITE)
%     }
%
% \end{RECIPE-UMGEBUNG}


\begin{document}
	% \titlehead{\includegraphics[width=0.6\columnwidth,right]{ostfalia_logo_rechts}}

\subject{\vspace{2cm}
	\large Rezeptsammlung
	\vspace{1cm}
	}
\title{Vegane Gerichte und Getränke}
\subtitle{Eine Sammlung der besten Ideen}

\author{\vspace{4cm}\\
	arwsl\\
	\vspace{2cm}}

\date{seit 2021}

% \publishers{\vspace{2cm}}



\maketitle
\thispagestyle{empty}
\cleardoublepage

	\cleardoubleemptypage

\thispagestyle{empty}

\section*{Vorwort}
\label{sec:Vorwort}

\todo{Vorwort einfügen?}

\begin{flushright}
	Wolfenbüttel, Januar 2021
\end{flushright}



\cleardoubleemptypage

	\pagestyle{plain}					% keine Kopf- und Fußzeilen ab hier (am Ende der tex wieder aufgehoben)
\pagenumbering{Roman}							% römische Seitennummerierung
\setcounter{page}{2}

\doparttoc
\dosecttoc
\tableofcontents								% Inhaltsverzeichnis

\vspace{2cm}
\addcontentsline{toc}{section}{Abkürzungsverzeichnis} % manuell ins Inhaltsverzeichins fügen
\section*{Abkürzungsverzeichnis}
\begin{acronym}[MMMMMMM]
	\acro{ca}[ca.]{cirka}
	\acro{EL}{Esslöffel}
	\acro{geh}[geh.]{gehäuft}
	\acro{Msp}[Msp.]{Messerspitze}
	\acro{Pck}[Pck.]{Päckchen}
	\acro{Ta}[Ta.]{Tasse}
	\acro{TL}{Teelöffel}
\end{acronym}


\cleardoublepage								% neue (rechte) Seite
\setcounter{page}{0}
\pagenumbering{arabic}							% arabische (normale) Seitenzahlen

\renewcommand\thesection{\arabic{section}}
\renewcommand\thesubsection{\thesection.\arabic{subsection}}

\pagestyle{fancy}							%Kopf- und Fußzeile wird ab hier wieder angezeigt


	\clearpage
	\cleardoublepage
	\part{Frühstück}
		\sectionwithtoc{Zerealien}
			\begin{recipeDP}
    [
        preparationtime = {\SI{10}{\minute}},
        % bakingtime = {\SI{12}{\minute} bis \SI{15}{\minute}},
        % bakingtemperature = {\protect\bakingtemperature{fanoven=\SI{180}{\celsius}}},
        portion = {2 Portionen},
        source = {veganheaven.de}
    ]
    {Porridge}

    \graph
        {
            big=Frühstück/Zerealien/Porridge/big.jpg,
            small=Frühstück/Zerealien/Porridge/small.jpg
        }

    \introduction{Porridge ist einfach das perfekte Frühstück! Gesund, super einfach zuzubereiten und in Kombination mit frischem Obst und Nüssen sooo lecker! }

    \ingredients{
        \SI{100}{\g} & grobe Haferflocken \\
        \SI{500}{\ml} & pflanzliche Milch \\
        1 & reife Banane \\
        \SI[parse-numbers = false]{\nicefrac{1}{4}}{\TL} & Salz \\
        \SI[parse-numbers = false]{\nicefrac{1}{2}}{\TL} & Zimt \\
        \SI[parse-numbers = false]{\nicefrac{1}{2}}{\TL} & Vanilleextrakt
    }

    \preparation{
        \step Die Banane in einem Topf mit einer Gabel zerdrücken bis ein gleichmäßiger Brei entstanden ist. Alle anderen Zutaten für das Porridge hinzu geben.
        \step Bei niedriger Hitze 5 Minuten unter ständigem Rühren aufkochen bis der Porridge die gewünschte Konsistent erreicht hat. Nach Belieben mit Obst, Nüssen oder Nussmus servieren.
    }

    % \suggestion[Title of Suggestion]{
	% 	Suggestion
    % }
    %
    % \hint{Hint}

\end{recipeDP}



	\clearpage
	\cleardoublepage
	\part{Vorspeisen und Beilagen}
		\sectionwithtoc{Salate} % und Dressing
			\begin{recipeDP}
    [
        preparationtime = {\SI{45}{\minute}},
        % bakingtime = {\SI{12}{\minute} bis \SI{15}{\minute}},
        % bakingtemperature = {\protect\bakingtemperature{fanoven=\SI{180}{\degree}}},
        portion = {4 Portionen},
        source = {köstlich vegetarisch-2/2016}
    ]
    {Kartoffel\index{Kartoffeln}-\addtoidx{Radieschen}-Salat mit \addtoidx{Tofu}}

    % \graph
    %     {
    %         small=Recipes/MainCourses/BBQChicken/Small.jpg,
    %         big=example-image
    %     }

    % \introduction{einleitung}

    \ingredients{
        \SI{200}{\g} & Tofu oder \addtoidx{Räuchertofu} \\
        \SI{6}{\EL} & milde Sojasoße \\
        \SI{800}{\g} & vorwiegend festkochende Kartoffeln \\
         & Salz \\
        \SI{4}{\EL} & Weißweinessig \\
        \SI{2}{\TL} & mittelscharfer Senf \\
         & schwarzer Pfeffer \\
        \SI{3}{\EL} & Rapsöl \\
        \SI{60}{\g} & Zwiebeln \\
        \SI{1}{Bund} & Radieschen \\
        \SI{1}{Bund} & Schnittlauch \\
        \SI{2}{\EL} & Olivenöl \\
        \SI{150}{\ml} & heiße Gemüsebrühe
    }

    \preparation{
        \step Den Tofu in \SI{5}{\mm} große Stücke schneiden , in einer Schüssel mit Sojasauce und \ca \SI{30}{\minute} marinieren. Kartoffeln waschen, große Exemplare halbieren und mit Schale in Salzwasser \ca \SI{20}{\minute} bis \SI{25}{\minute} kochen.

        \step Inzwischen für die Vinaigrette Essig mit Senf mischen und mit Salz und Pfeffer abschmecken. Öl unterrühren, Zwiebeln schälen und fein würfeln. Radieschen inklusive gut erhaltener Blätter putzen und waschen. Blätter abzupfen und in eine feine Streifen schneiden. Radieschen in feine Scheiben schneiden. Schnittlauch kalt abbrausen, trocken und in feine Röllchen schneiden.

        \step Tofuwürfel in einem Sieb abtropfen lassen, dabei die Sojasauce auffangen. Das Olivenöl in einer Pfanne erhitzen und den Tofu darin bei hoher Hitze \ca \SI{5}{\minute} knusprig braten. Mit Pfeffer würzen und mit der Sojasoße ablöschen. Dann einkochen lassen und zur Seite stellen.

        \step Kartoffeln abgießen, ausdampfen lassen, heiß pellen und in etwa \SI{1,5}{\cm} große Würfel schneiden. Kartoffeln, Zwiebeln und Tofu-Würfel mischen. Gemüsebrühe zugeben und untermischen, die Vinaigrette vorsichtig unterheben. Radieschen, Radieschenblätter und Schnittlauch unter den Salat heben. Mit Salz und Pfeffer abschmecken und zugedeckt 30 Minuten durchziehen lassen.
    }

    % \suggestion[Title of Suggestion]{
	% 	Suggestion
    % }
    %
    % \hint{Hint}

\end{recipeDP}

			\begin{recipeDP}
    [
        preparationtime = {\SI{30}{\minute}},
        % bakingtime = {\SI{12}{\minute} bis \SI{15}{\minute}},
        % bakingtemperature = {\protect\bakingtemperature{fanoven=\SI{180}{\degree}}},
        portion = {4 Portionen},
        source = {Sabine Schramm}
    ]
    {Asiatischer \addtoidx{Krautsalat}}

    % \graph
    %     {
    %         small=Recipes/MainCourses/BBQChicken/Small.jpg,
    %         big=example-image
    %     }

    % \introduction{einleitung}

    \ingredients{
        \ca \SI{2}{\kg} & Weiß-\index{Kohl!Weiß-} oder Spitzkohl\index{Kohl!Spitz-} \\
        \SI{1}{Bund} & Frühlingszwiebeln \\
        2 bis 3 \pck & \addtoidx{Instant-Nudeln} \\
        \\
        \multicolumn{2}{l}{\textbf{Dressing}}\\
        \SI[parse-numbers = false]{\nicefrac{1}{2}}{\Ta} & Öl \\
        \SI{1}{\TL} & Pfeffer \\
        \SI{1}{\TL} & Salz \\
        \SI{4}{\EL} & Essig \\
        \SI{4}{\EL} & Zucker \\
         & Gewürze der Instant-Nudeln
        \\
         & Sonnenblumenkerne \\
         & Mandelstifte
    }

    \preparation{
        \step Zuerst den Kohl und die Zwiebeln klein schneiden. Den Kohl am besten in schmale Streifen schneiden. Zusammen mit den Instant-Nudeln in einer großen Schüssel vermengen.

        \step Alle Zutaten für das Dressing gemeinsam in einem Topf erhitzen, bis sich alles gut aufgelöst hat. Das heiße Dressing dann über die Instant-Nudeln und den Kohl gießen, mit den Händen dann kräftig durchkneten. Solange kneten bis der Kohl und die Nudeln weich werden.

        \step Die Sonnenblumenkerne und die Mandelstifte zusammen in der Pfanne rösten. Beides zum Anrichten über den Salat geben.
    }

    % \suggestion[Title of Suggestion]{
	% 	Suggestion
    % }
    %
    % \hint{Hint}

\end{recipeDP}

			\begin{recipeDP}
    [
        preparationtime = {\SI{15}{\minute}},
        % bakingtime = {\SI{12}{\minute} bis \SI{15}{\minute}},
        % bakingtemperature = {\protect\bakingtemperature{fanoven=\SI{180}{\degree}}},
        % portion = {40 Stück},
        source = {Oma Jutta}
    ]
    {\addtoidx{Bohnensalat}}

    % \graph
    %     {
    %         small=Recipes/MainCourses/BBQChicken/Small.jpg,
    %         big=example-image
    %     }

    % \introduction{einleitung}

    \ingredients{
        \SI{5}{\EL} & Essig \\
        \SI{4}{\TL} & Zucker \\
        \SI[parse-numbers = false]{\nicefrac{1}{2}}{\TL} & Salz \\
        \SI{3}{\EL} & neutrales Öl \\
         & kleine Zwiebel \\
         & heißes Wasser \\
        \SI{1}{Weckglas} & grüne Bohnen\index{Bohnen!grüne}
    }

    \preparation{
        \step Zuerst den Essig, den Zucker, das Salz, das Öl, die Zwiebelwürfel, den Pfeffer und das Wasser gut vermischen. Dann die Bohnen hinein geben und gut unterrühren. Wenn möglich einen Tag ziehen lassen.
    }

    % \suggestion[Title of Suggestion]{
	% 	Suggestion
    % }
    %
    % \hint{Hint}

\end{recipeDP}

			\begin{recipeDP}
    [
        preparationtime = {\SI{30}{\minute}},
        % bakingtime = {\SI{12}{\minute} bis \SI{15}{\minute}},
        % bakingtemperature = {\protect\bakingtemperature{fanoven=\SI{180}{\celsius}}},
        portion = {4 Portionen},
        source = {veeatcookbake.com}
    ]
    {Rotkohlsalat\index{Kohl!Rot-}}

    \graph
        {
            big=Vorspeisen_und_Beilagen/Salate/Rotkohlsalat/big.jpg,
            small=Vorspeisen_und_Beilagen/Salate/Rotkohlsalat/small.jpg
        }

    % \introduction{einleitung}

    \ingredients{
        1 & kleiner Rotkohl \\
        2 & Schalotten \\
        \SI{4}{\EL} & Apfelindex\index{Apfel!-essig}\index{Essig!Apfel-} \\
        \SI{2}{\EL} & Wasser \\
        \SI{2}{\TL} & Salz \\
        \SI{1}{\TL} & Pfeffer \\
         & Kümmel (optional)
    }

    \preparation{
        \step Den Kohl fein schneiden, oder auch durch den Food Processor lassen. Die Zwiebel fein schneiden oder ebenfalls mit dem Rotkohl durch die Maschine. Den geschnittenen Kohl in eine ausreichend große, am besten verschließbare, Schüssel geben und die restlichen Zutaten dazu geben. Nun den Kohl für einige Minuten durchkneten bzw. massieren. So wird die Texture gebrochen und der Kohlsalat wird schön saftig und nicht so trocken.
        \step Nun den Salat für einige Stunden auf der Arbeitsfläche oder an einem kühlen Platz stehen lassen. Am besten über Nacht durchziehen lassen. Am nächsten Morgen nochmal kurz abschmecken und gegebenenfalls durchkneten.
    }

    % \suggestion[Title of Suggestion]{
	% 	Suggestion
    % }
    %
    % \hint{Hint}

\end{recipeDP}

	\clearpage
		\sectionwithtoc{Aufstriche und Dips}
			\begin{recipeDP}
    [
        preparationtime = {\SI{5}{\minute}},
        % bakingtime = {\SI{12}{\minute} bis \SI{15}{\minute}},
        % bakingtemperature = {\protect\bakingtemperature{fanoven=\SI{180}{\celsius}}},
        portion = {1 mittleres Glas},
        source = {kaffeeundcupcakes.de}
    ]
    {\addtoidx{Mayonnaise}}

    \graph
        {
            big=Vorspeisen_und_Beilagen/Aufstriche_und_Dips/Mayonnaise/big.jpg,
            small=Vorspeisen_und_Beilagen/Aufstriche_und_Dips/Mayonnaise/small.jpg
        }

    \introduction{Schnelles und einfaches Grundrezept für vegane Mayonnaise.}

    \ingredients{
        \SI{130}{\g} & Sojamilch\index{Milch!Soja-}, Raumtemperatur \\
        \SI{2}{\TL} & Branntweinessig \\
        \SI{200}{\g} & \addtoidx{Öl}, geschmacksneutral \\
        \SI{1}{\TL} & Salz \\
        \SI{1}{\TL} & mittelscharfer Senf
    }

    \preparation{
        \step Die Sojamilch (unbedingt auf Raumtemperatur!) in ein hohes schmales Mix-Gefäß gießen. Den Essig dazugeben und dann das Öl darauf gießen.
        \step Einen Stabmixer ganz bis auf den Boden des Gefäßes stellen und losmixen. Nach einiger Zeit (30-60 Sekunden) langsam den Stabmixer beim Mixen auf und ab bewegen. So lange mixen, bis die Mayonnaise angedickt hat und alles eine gleichmäßige, cremige Konsistenz angenommen hat.
        \step Die Mayonnaise mit Salz und Senf abschmecken. In einem verschließbaren Gefäß im Kühlschrank aufbewahren und innerhalb von 1 Woche aufbrauchen. Guten Appetit!
    }

    % \suggestion[Title of Suggestion]{
	% 	Suggestion
    % }
    %
    % \hint{Hint}

\end{recipeDP}

			\begin{recipeDP}
    [
        preparationtime = {\SI{30}{\minute}},
        % bakingtime = {\SI{12}{\minute} bis \SI{15}{\minute}},
        % bakingtemperature = {\protect\bakingtemperature{fanoven=\SI{180}{\celsius}}},
        portion = {etwa \SI{900}{\ml}},
        source = {Sedef Salmo, Kristiane Redecker}
    ]
    {\addtoidx{Hummus}}

    % \graph
    %     {
    %         big=example-image
    %         small=Small.jpg
    %     }

    % \introduction{einleitung}

    \ingredients{
        \SI{200}{\g} & \addtoidx{Kichererbsen}, getrocknet \\
        1 & Zitrone \\
        \SI{3}{\EL} & Olivenöl \\
        \\
        \SI{150}{\g} & \addtoidx{Tahin} \\
        3 Zehen & Knoblauch \\
        \SI{1,5}{\TL} & Kreuzkümmel \\
        \SI[parse-numbers = false]{\nicefrac{3}{4}}{\TL} & Piment \\
        \SI[parse-numbers = false]{\nicefrac{3}{4}}{\TL} & Pfeffer \\
        \SI{1,5}{\TL} & Salz \\
        \SI{6}{\EL} & Olivenöl \\
         & Wasser \\
        \\
        \SI{3}{\EL} & Olivenöl \\
         & Petersilie
    }

    \preparation{
        \step Am Vortag die trockenen Kichererbsen in etwa \SI[parse-numbers = false]{\nicefrac{1}{2}}{\l} Wasser einweichen.
        \step Wenn die Kichererbsen gut eingeweicht sind, für etwa eine Stunde kochen, bis sie ganz weich sind. Die weichen Kichererbsen dann mit \SI{3}{\EL} Olivenöl und dem Saft einer Zitrone pürieren.
        \step Dann das Tahin, den Knoblauch, Kreuzkümmel, Piment, Pfeffer, Salz und das restliche Olivenöl hinzugeben und gut verrühren. Nach Bedarf noch Wasser oder aufgefangenes Aquafaba hinzugeben, um eine glatte Konsistenz zu erhalten.
        \step Zum Anrichten mit Olivenöl beträufeln und mit Petersilie garnieren.
    }

    \suggestion[Dicke Bohnen Püree]{
		Statt der Kichererbsen lassen sich auch Dicke Bohnen (Bakla) verwenden: etwa \SI{45}{\minute} kochen und dann wie Kichererbsen zubereiten, nur statt Tahin \SI{3}{\EL} Olivenöl verwenden.
    }

    \hint{\SI{200}{\g} getrocknete Kichererbsen lassen sich auch durch etwa \SI{530}{\g} Kichererbsen aus der Dose ersetzen. Diese können direkt verarbeitet werden.}

\end{recipeDP}

			\begin{recipeDP}
    [
        preparationtime = {\SI{15}{\minute}},
        % bakingtime = {\SI{12}{\minute} bis \SI{15}{\minute}},
        % bakingtemperature = {\protect\bakingtemperature{fanoven=\SI{180}{\celsius}}},
        portion = {etwa \SI{400}{\ml}},
        % source = {}
    ]
    {\addtoidx{Salsa}}

    \graph
        {
            big=Vorspeisen_und_Beilagen/Aufstriche_und_Dips/Salsa/big.jpg,
            small=Vorspeisen_und_Beilagen/Aufstriche_und_Dips/Salsa/small.jpg
        }

    % \introduction{einleitung}

    \ingredients{
        \SI{170}{\g} & eingelegte Tomatenpaprika\index{Paprika!Tomaten-} \\
        \SI{120}{\g} & Kirschtomaten\index{Tomaten!Kirsch-} \\
        \SI{1}{\EL} & Ajvar \\
        \SI{1}{\EL} & Tomatenmark \\
        \SI[parse-numbers = false]{\nicefrac{1}{2}}{\EL} & Apfelessig \\
        1 & kleine Zwiebel \\
         & Salz \\
         & Pfeffer \\
         & Chilipulver
    }

    \preparation{
        \step Die Zwiebel, die eingelegte Paprika und die Tomaten fein würfeln und mit einander vermengen. Das Ajvar mit dem Tomatenmark und dem Essig hinzugeben und gut vermengen. Mit Salz und Pfeffer abschmecken und mit Chili nach Belieben würzen.
    }

    % \suggestion[Title of Suggestion]{
	% 	Suggestion
    % }
    %
    % \hint{Hint}

\end{recipeDP}



	\clearpage
	\cleardoublepage
	\part{Hauptgerichte}
		\sectionwithtoc{Ofengerichte}
			\begingroup
\makeatletter
\renewenvironment{recipe}[2][]
{% initialisation
    \setkeys{recipe}{preparationtime, bakingtime, bakingtemperature, portion, calory, source}
    \setkeys{picture}{small, big, smallpicturewidth=\xcb@smallpicturewidth, bigpicturewidth=\xcb@bigpicturewidth} % load the default values
    \def\xcb@hook@pregraph{}
    \def\xcb@hook@pretitle{}
    \def\xcb@introduction{}
    \def\xcb@hook@prepreparation{}
    \preparation{}
    \def\xcb@hook@postpreparation{}
    \def\xcb@hook@preingredients{}
    \ingredients{}
    \def\xcb@hook@postingredients{}
    \def\xcb@suggestion{}
    \def\xcb@hint{}

    \def\xcb@recipename{#2}
    \setkeys{recipe}{#1}  % reading the optional parameters

    \setcounter{xcb@newpagefoot}{1}
    \setcounter{xcb@newpagehead}{\value{page}}
}
{% this part is executed at \end{recipe}
%% FIRST BLOCK
    \xcb@hook@pregraph
    \if@twoside
        \ifodd\arabic{page}
            \begin{minipage}[T]{\xcb@picture@bigwidth}
                \xcb@picture@big
            \end{minipage}
            \hfill
            \begin{minipage}[T]{\xcb@picture@smallwidth}
                \xcb@picture@small
            \end{minipage}
        \else
            \begin{minipage}[T]{\xcb@picture@smallwidth}
                \xcb@picture@small
            \end{minipage}
            \hfill
            \begin{minipage}[T]{\xcb@picture@bigwidth}
                \xcb@picture@big
            \end{minipage}
        \fi
    \else
        \begin{minipage}[T]{\xcb@picture@bigwidth}
            \xcb@picture@big
        \end{minipage}
        \hfill
        \begin{minipage}[T]{\xcb@picture@smallwidth}
            \xcb@picture@small
        \end{minipage}
    \fi

%% SECOND BLOCK
    \xcb@hook@pretitle
    \recipesection[\normalsize\xcb@recipename]%
    {\hspace{-1em}\textcolor{\xcb@color@recipename}{\xcb@font@recipename\xcb@recipename}}
    \xcb@cmd@recipeoverview

    \xcb@introduction

%% THIRD BLOCK
    \columnratio{0.66}
    \begin{paracol}{2}
        \xcb@hook@prepreparation

        \xcb@preparation

        \xcb@hook@postpreparation

        \xcb@suggestion

        \vfill

        \xcb@cmd@wrapfill
        \xcb@hint
        \setcounter{xcb@newpagefoot}{0}
      \switchcolumn
            \xcb@hook@preingredients

            \xcb@ingredients

            \xcb@hook@postingredients
    \end{paracol}
}

\renewcommand*{\ingredients}[2][\empty]
{% The optional argument contains the number of lines
    \def\xcb@ingredientslines{#1}
    \def\xcb@ingredients
    {%
        \xcb@name@inghead
        \\[1em]
        {\xcb@fontsize@ing\color{\xcb@color@ing}
        \begin{supertabular}{r>{\raggedright\arraybackslash}p{3cm}}
            #2
        \end{supertabular}}
    }
}
\makeatother

\begin{recipeDP}
    [
        preparationtime = {\SI{30}{\minute}},
        bakingtime = {\SI{30}{\minute}},
        bakingtemperature = {\protect\bakingtemperature{topbottomheat=\SI{200}{\celsius}}},
        portion = {6 Portionen},
        source = {Michaela Vais}
    ]
    {\addtoidx{Linsen} \addtoidx{Moussaka}}

    \graph
        {
            big=Hauptgerichte/Auflaeufe/Linsen_Moussaka/big.jpg,
            small=Hauptgerichte/Auflaeufe/Linsen_Moussaka/small.jpg
        }

    \introduction{Leckere Linsen Moussaka mit Auberginen und Kartoffeln. Dieses beliebte griechische Gericht kann leicht ohne Fleisch zubereitet werden und schmeckt trotzdem super. Das Rezept ist vegan, glutenfrei und relativ leicht herzustellen.}

    \ingredients{
        \SI{1}{\kg} & \addtoidx{Kartoffeln} \\
        3 & große \addtoidx{Auberginen} \\
         & Olivenöl \\
         & Meersalz \\
         & Pfeffer \\
        \\
        \multicolumn{2}{l}{\textbf{Linsenmischung}}\\
        \SI{600}{\g} & gekochte Linsen \\
        \SI{450}{\g} & passierte Tomaten \\
        \SI{150}{\g} & gehackte Tomaten \\
        \SI{1}{\EL} & Olivenöl \\
        1 & große Zwiebel \\
        2 & Knoblauchzehen \\
        2 & Lorbeerblätter \\
        \SI{1}{\TL} & Thymian \\
        \SI{1}{\TL} & Oregano \\
        \SI{1}{\TL} & Paprika \\
        \SI{1}{\TL} & brauner Zucker \\
        \SI{1}{Prise} & Zimt \\
        & Meersalz \\
        & Pfeffer \\
        \\
        \multicolumn{2}{l}{\textbf{Béchamelsoße}}\\
        \SI{30}{\g} & vegane Butter \\
        \SI{480}{\ml} & pflanzl. Milch \\
        \SI{30}{\g} & Maisstärke \\
        \SI{2}{\EL} & Hefeflocken \\
        & Meersalz \\
        & Pfeffer \\
        1 Prise & Muskatnuss \\
        & optional: Veganer Käse \\
    }

    \preparation{
        \step Den Ofen auf \SI{200}{\celsius} vorheizen und zwei Backbleche mit Backpapier auslegen. Jede Aubergine der Länge nach in vier Stücke schneiden. Die Kartoffel in 1 cm dicke Scheiben schneiden. Alle Scheiben in einer Schicht auf dem Backblech anordnen und leicht mit etwas Olivenöl bepinseln, mit Salz und Pfeffer bestreuen. Im Ofen etwa \SI{20}{\minute} backen.

        \step In der Zwischenzeit \SI{1}{\EL} Olivenöl in einer Pfanne erhitzen. Zwiebel und Knoblauch für \ca \SI{4}{\minute} bis \SI{5}{\minute} anbraten. Tomatenpüree, gehackte Tomaten, alle Gewürze sowie Salz und Pfeffer dazugeben. Zum Schluss die gekochten Linsen hinzufügen und bei schwacher Hitze \ca \SI{5}{\minute} köcheln lassen.

        \step Für die Béchamelsoße die pflanzliche Milch in eine Pfanne geben. Die Maisstärke, Hefeflocken, Salz und Pfeffer dazugeben und verrühren. Die vegane Butter hinzugeben und die Mischung zum Kochen bringen. Bei geringer Hitze einige Minuten köcheln lassen, bis die Sauce andickt. Dabei immer wieder umrühren. Herdplatte ausmachen.

        \step Eine \SI{33}{\cm} mal \SI{23}{\cm} oder größere Backform einfetten und die Hälfte der Kartoffel- und Auberginenscheiben auf dem Boden der Backform verteilen. Die Linsen Mischung darüber geben, gefolgt von den restlichen Kartoffel- und Auberginenscheiben. Die Béchamelsoße darüber gießen und gleichmäßig verteilen. Veganen Käse nach Geschmack oben drauf verteilen.

        \step Im Ofen \ca \SI{30}{\minute} backen oder bis die Béchamelschicht goldbraun ist. Mit frischen Kräutern garnieren.
    }

    % \suggestion[]{
    %
    % }

    \hint{Als oberste Schicht lässt sich gut Kartoffelpüree verwenden, wenn die Béchamelsoße ersetzt werden soll. Dann wird der Auflauf zum Shepherd's Pie.}

\end{recipeDP}
\endgroup

			\begin{recipeDP}
    [
        preparationtime = {\SI{10}{\minute}},
        bakingtime = {\SI{50}{\minute}},
        bakingtemperature = {\protect\bakingtemperature{topbottomheat=\SI{220}{\celsius}}},
        portion = {4 Portionen},
        source = {Caitlin Shoemaker}
    ]
    {\addtoidx{Kichererbsen}-\addtoidx{Nudelauflauf}}

    \graph
        {
            big=Hauptgerichte/Ofengerichte/Kichererbsen-Nudelauflauf/big.jpg,
            small=Hauptgerichte/Ofengerichte/Kichererbsen-Nudelauflauf/small.jpg
        }

    \introduction{Nudelauflauf aus rohen Nudeln.}

    \ingredients{
        \multicolumn{2}{l}{\textbf{Cremige Soße}} \\
        \SI{115}{\g} & Cashewnüsse, übernacht eingeweicht \\
        \SI{840}{\ml} & Gemüsebrühe \\
        \SI{2}{\EL} & Hefeflocken \\
        \SI[parse-numbers = false]{\nicefrac{1}{2}}{\TL} & Paprikapulver \\
        \SI[parse-numbers = false]{\nicefrac{1}{2}}{\TL} & Knoblauchpulver \\
        \SI[parse-numbers = false]{\nicefrac{1}{2}}{\TL} & Pfeffer \\
        & Salbei \\
        & Thymian\\
        \\
        \multicolumn{2}{l}{\textbf{Auflauf}} \\
        \SI{250}{\g} & Nudeln \\
        1 Dose & Kichererbsen \\
        \SI[parse-numbers = false]{\nicefrac{1}{2}}{} & Zwiebel \\
        2 Stiele & Sellerie \\
        \SI{100}{\g} & Möhren, gewürfelt \\
        \SI{100}{\g} & Erbsen, gefroren
    }

    \preparation{
        \step Den Ofen auf \SI{220}{\celsius} Ober- und Unterhitze vorheizen und eine \SI{23}{\cm} mal \SI{33}{\cm} Auflaufform zurecht legen.

        \step Für die cremige Soße die eingeweichten und abgegossenen Cashews mit der Gemüsebrühe, den Hefeflocken und den Gewürzen in einem Standmixer zu einer homogenen Soße mixen. Etwa \SI{60}{\second} mixen, damit die Cashews gut aufgelöst werden.

        \step Die rohen Nudeln, Kichererbsen, die klein geschnittene Zwiebel zusammen mit klein geschnittenem Sellerie und den Möhrenstücken und den Erbsen in der Auflaufform vermischen. Dann die Soße darüber verteilen und besonders die Nudeln mit der Soße bedecken.

        \step Den Auflauf im Ofen für \SI{50}{\minute} backen. Nach etwa \SI{25}{\minute} den Auflauf abdecken, damit er nicht oberflächlich verbrennt. Während der Backzeit kann es helfen, einmal umzurühren, damit alle Nudeln gar werden.
    }

    \suggestion[Cashewnüsse einweichen]{
		Entweder die Cashews über Nacht in kaltes Wasser einlegen, oder alternativ für \SI{20}{\minute} in heißem Wasser stehen lassen. Andernfalls lassen sich die Nüsse auch - mit Wasser bedeckt - \SI{3}{\minute} in der Mikrowelle erhitzen, danach noch \SI{5}{\minute} stehen lassen.
    }

    \hint{Es empfehlen sich kurze Nudeln, wie Fusilli oder Farfalle.}

\end{recipeDP}

			\begin{recipeDP}
    [
        preparationtime = {\SI{15}{\minute}},
        bakingtime = {\SI{30}{\minute}},
        bakingtemperature = {\protect\bakingtemperature{fanoven=\SI{180}{\celsius}}},
        portion = {4 Portionen},
        source = {@elavegan}
    ]
    {\addtoidx{Curry}-\addtoidx{Blumenkohl}-\addtoidx{Auflauf}}

    \graph
        {
            big=Hauptgerichte/Ofengerichte/Curry-Blumenkohl-Auflauf/big.jpg,
            small=Hauptgerichte/Ofengerichte/Curry-Blumenkohl-Auflauf/small.jpg
        }

    % \introduction{
    %     EINLEITUNG
    % }

    \ingredients{
        \multicolumn{2}{l}{\textbf{Curry}} \\
        2 & mittelgroße Blumenkohle \\
        \SI{500}{\g} & gewürfelte Dosentomaten \\
        3 & Knoblauchzehen \\
        \SI{3}{\cm} & frischer Ingwer \\
        \SI{4}{\TL} & Currypulver \\
        \SI{1}{\TL} & Salz \\
        \SI{1}{\TL} & Zwiebelpulver \\
        \SI{1}{\TL} & Paprika \\
        \SI{1}{\TL} & Kreuzkümmel \\
        \SI{1}{\TL} & Pfeffer \\
        \\
        \multicolumn{2}{l}{\textbf{Kokosmilchsoße}} \\
        \SI{240}{\g} & \addtoidx{Kokosmilch} \\
        \SI{1}{\EL} & Maisstärke \\
        \SI{2}{\EL} & Hefeflocken \\
        \SI[parse-numbers = false]{\nicefrac{1}{2}}{\TL} & Salz \\
        \SI[parse-numbers = false]{\nicefrac{1}{2}}{\TL} & Paprikapulver \\
        \\
         & veganen Reibekäse
    }

    \preparation{
        \step Den Blumenkohl in mundgerechte Röschen aufteilen und in einen großen Topf geben. Die Röschen mit Salzwasser bedecken und zum Kochen bringen. Etwa 10 Minuten kochen lassen, bis der Blumenkohl gar ist (nicht zu lange kochen!), dann das Wasser abgießen.
        \step Den Ofen auf \SI{180}{\celsius} vorheizen. Die gehackten Tomaten, den frischen Knoblauch, den Ingwer und alle Gewürze (Curry, Salz, Zwiebelpulver, Knoblauchpulver, Paprika, gemahlener Kreuzkümmel, schwarzer Pfeffer) in eine ofenfeste Pfanne/Auflaufform geben. Anschließend umrühren und die gekochten Blumenkohlröschen hinzufügen.
        \step Alle Zutaten für die Kokosmilchsoße in eine mittelgroße Schüssel geben und mit einem Schneebesen verrühren. Die Mischung über den Blumenkohl gießen.
        \step Den Auflauf \SI{10}{\minute} im Ofen backen, dann umrühren. Mit veganem Käse bestreuen und weitere \SI{10}{\minute} backen. Mit gekochtem Reis oder Naan servieren!
    }

    % \suggestion[TITEL EINES VORSCHLAGS]{
	% 	VORSCHLAG (DURCH HORIZONTALE LINIE VOM REZEPT GETRENNT)
    % }

    \hint{
        Gut passt auch Brokkoli statt oder zum Blumenkohl. Dabei aber beachten, dass Brokkoli schneller gar ist, als Blumenkohl.
    }

\end{recipeDP}

	\clearpage
		\sectionwithtoc{Suppen und Eintöpfe}
			\begin{recipeDP}
    [
        preparationtime = {\SI{25}{\minute}},
        % bakingtime = {\SI{12}{\minute} bis \SI{15}{\minute}},
        % bakingtemperature = {\protect\bakingtemperature{fanoven=\SI{180}{\degree}}},
        portion = {4 Portionen},
        source = {chefkoch.de}
    ]
    {Bunter Kichererbseneintopf}

    \graph
        {
            big=Hauptgerichte/Suppen_und_Eintoepfe/Bunter_Kichererbseneintopf/big.jpg,
            % small=Recipes/MainCourses/BBQChicken/Small.jpg
        }

    % \introduction{einleitung}

    \ingredients{
        \SI{1}{Stange} & Lauch, groß \\
        2 & Paprikaschoten, gelb, rot \\
        2 & Knoblauchzehen \\
        \SI{800}{\g} & Dosentomaten \\
        \SI{2}{\EL} & Tomatenmark \\
        \SI{400}{\g} & \addtoidx{Kichererbsen}, aus der Dose \\
        \SI{1}{\l} & Gemüsebrühe \\
        \SI{225}{\g} & Blattspinat \\
        \SI{1}{\TL} & Paprikapulver, scharf \\
        \SI{1}{\TL} & Paprikapulver, mild \\
         & Salz \\
         & Pfeffer
    }

    \preparation{
        \step Das Gemüse waschen, den Lauch in dünne Ringe schneiden, die Paprikaschoten würfeln. Die Kichererbsen gut abspülen, abtropfen lassen.

        \step Das Olivenöl in einem großen Topf erhitzen, Lauchscheiben und Paprikawürfel darin unter Rühren \ca \SI{5}{\minute} scharf anbraten. Die Knoblauchzehen dazu pressen, das Tomatenmark dazu geben und beides kurz mit braten. Mit den etwas zerkleinerten Tomaten und der Gemüsebrühe auffüllen und die Kichererbsen sowie das Paprikapulver zufügen.

        \step Alles aufkochen und dann bei kleinerer Hitze ca. 5 Minuten kochen lassen. Den Blattspinat etwas zerkleinern, in die Suppe geben und nochmal \ca \SI{3}{\minute} kochen lassen. Alles mit Salz und Pfeffer abschmecken.
    }

    % \suggestion[Title of Suggestion]{
	% 	Suggestion
    % }
    %
    % \hint{Hint}

\end{recipeDP}

			\begin{recipeDP}
    [
        preparationtime = {\SI{45}{\minute}},
        bakingtime = {\SI{25}{\minute}},
        bakingtemperature = {\protect\bakingtemperature{fanoven=\SI{180}{\celsius}}},
        portion = {4 Portionen},
        source = {@petadeutschland}
    ]
    {Kürbisrisotto\index{Risotto}}

    \graph
        {
            big=Hauptgerichte/Suppen_und_Eintoepfe/Kuerbisrisotto/big.jpg,
            small=Hauptgerichte/Suppen_und_Eintoepfe/Kuerbisrisotto/small.jpg
        }

    % \introduction{einleitung}

    \ingredients{
        1 & rote Zwiebel \\
        1 & Knoblauchzehe \\
        \SI{2}{\EL} & Olivenöl \\
        \SI{2}{\Ta} & Risottoreis\index{Reis!Risotto-} \\
        \SI{1}{\Ta} & Weißwein \\
        \SI{4}{\Ta} & Gemüsebrühe \\
        \SI{1}{Dose} & Kürbispüree \\
        \SI{1}{\TL} & Ingwer, gehackt \\
        \SI{1}{Prise} & Zimt \\
        \SI{4}{\EL} & Rosmarin \\
        \SI{4}{\EL} & Basilikum, gehackt \\
        1 & Hokkaidokürbis\index{Kuerbis@Kürbis!Hokkaido-} \\
        \SI{1}{\EL} & Olivenöl (zum Backen) \\
        \SI{4}{\EL} & Walnüsse, gehackt
    }

    \preparation{
        \step Den Kürbis waschen, schneiden und in zwei Hälften teilen. Die Kürbiskerne entfernen. Eine Hälfte in kleine Würfel schneiden, die andere in lange Streifen. Die Schale ist essbar und muss nicht entfernt werden. Den Backofen auf \SI{180}{\celsius} vorheizen.

        \step Olivenöl in einem großen Topf erhitzen und Zwiebel und Knoblauch hinzugeben. Alles goldbraun anbraten und dann den Risottoreis hinzugeben. Das Ganze unter Rühren für \SI{5}{\minute} anrösten, danach die Kürbiswürfel hinzugeben. Dann nach und nach, unter ständigem Rühren, den Weißwein zum Reis dazu geben. Bei mittlerer Hitze kochen lassen. Danach langsam und unter ständigem Rühren die Gemüsebrühe hinzugeben. Reis aufkochen. Kürbispüree hinzufügen und noch fünf weitere Minuten kochen.

        \step Risotto vom Ofen nehmen und Ingwer, Zimt, Basilikum, die Hälfte des Rosmarins und die Hälfte der Walnüsse unterrühren. Das Risotto in eine Auflaufform geben, und über darüber die Kürbisstreifen. Diese mit dem restlichen Olivenöl bestreichen. Für \SI{25}{\minute}, oder bis die Kürbisstreifen leicht goldbraun gebrannt sind und der Reis weich ist, backen lassen. Das Risotto aus dem Ofen nehmen und den Rest der Walnüsse und des Rosmarins zurückgeben.
    }

    % \suggestion[Title of Suggestion]{
	% 	Suggestion
    % }
    %
    % \hint{Hint}

\end{recipeDP}

			\begin{recipeDP}
    [
        preparationtime = {\SI{45}{\minute}},
        % bakingtime = {\SI{25}{\minute}},
        % bakingtemperature = {\protect\bakingtemperature{fanoven=\SI{180}{\degree}}},
        portion = {3 Portionen},
        source = {Chefkoch: roddenberry}
    ]
    {Rote Linsen\index{Linsen!rote}-\addtoidx{Curry} mit \addtoidx{Süßkartoffeln}}

    \graph
        {
            big=Hauptgerichte/Suppen_und_Eintoepfe/Rote_Linsen_Curry/big.png,
            small=Hauptgerichte/Suppen_und_Eintoepfe/Rote_Linsen_Curry/small.png
        }

    \introduction{vegan, gesund, schnell und günstig}

    \ingredients{
        \SI{450}{\g} & Süßkartoffeln \\
        1 & rote Paprika \\
        \SI{200}{\g} & rote Linsen \\
        2 & Zwiebeln \\
        2 & Knoblauchzehen \\
        \SI{1}{\EL} & Olivenöl \\
        \SI{500}{\ml} & Kokosmilch \\
        \SI{250}{\ml} & Gemüsebrühe \\
        \SI{2}{\EL} & Tomatenmark \\
        \SI{2}{\TL} & Currypulver \\
        \SI{2}{\TL} & Kurkuma \\
        \SI{1}{\TL} & Garam Masala \\
         & Salz \\
         & Pfeffer
    }

    \preparation{
        \step Die Süßkartoffeln schälen und würfeln. Die Paprikaschote würfeln, den Knoblauch schälen und die Zwiebeln klein schneiden.
        \step Das Olivenöl in einem großen Topf erhitzen und die Zwiebeln glasig anschwitzen. Den Knoblauch durch die Presse in den Topf drücken, kurz mitbraten und dann Süßkartoffeln und Paprika in den Topf geben.
        \step Das Tomatenmark und die Gewürze dazugeben, kurz mitbraten, anschließend die Linsen hinzufügen. Nun alles mit Kokosmilch und Gemüsebrühe aufgießen und ca. 25 min. köcheln lassen.
    }

    % \suggestion[Title of Suggestion]{
	% 	Suggestion
    % }

    \hint{Das Curry schmeckt am nächsten Tag meist noch besser und lässt sich auch gut einfrieren.}

\end{recipeDP}

			\begin{recipeDP}
    [
        preparationtime = {\SI{45}{\minute}},
        % bakingtime = {\SI{12}{\minute} bis \SI{15}{\minute}},
        % bakingtemperature = {\protect\bakingtemperature{fanoven=\SI{180}{\celsius}}},
        portion = {4 Portionen},
        source = {@byanjushka}
    ]
    {Pilzsuppe}

    \graph
        {
            % big=example-image
            small=Hauptgerichte/Suppen_und_Eintoepfe/Pilzsuppe/small.jpg
        }

    % \introduction{einleitung}

    \ingredients{
        \multicolumn{2}{l}{\textbf{Suppe}} \\
        \SI{50}{\g} & getrocknete Pilze \\
        \SI{500}{\ml} & Wasser \\
        \\
        \SI{2}{\EL} & Öl \\
        2 & Knoblauchzehen \\
        1 & Zwiebel \\
        \SI{500}{\g} & \addtoidx{Champignons} \\
        \SI{500}{\ml} & Gemüsebrühe \\
        \SI{250}{\ml} & Sojasahne \\
         & Salz \\
         & Pfeffer \\
        \\
        \multicolumn{2}{l}{\textbf{dazu servieren}} \\
        \SI{500}{\g} & Nudeln \\
        \SI{250}{\g} & Champignons \\
         & frische Petersilie
    }

    \preparation{
        \step Zuerst die getrockneten Pilze kurz abspülen und dann für etwa \SI{30}{\minute} in \SI{500}{\ml} Wasser einweichen.
        \step Die eingeweichten Pilze abgießen und das Wasser auffangen. Dann in einem großen Topf das Öl erhitzen, den Knoblauch und die Zwiebeln dazugeben und für etwa \SI{3}{\minute} braten, bis die Zwiebeln glasig werden. Nun die grob geschnittenen und die eingeweichten Pilze in den Topf geben und etwa \SI{8}{\minute} braten.
        \step Die Gemüsebrühe und das Wasser der getrockneten Pilze hinzugeben, alles aufkochen und dann \SI{10}{\minute} bis \SI{15}{\minute} köcheln lassen. Danach alles mit einem Stabmixer pürieren, bis eine glatte Suppe entsteht. Die Sojasahne einrühren und mit Salz und Pfeffer würzen.
        \step Die Nudeln nach Packungsangabe kochen und die restlichen Pilze in einer Pfanne anbraten. Auch hier mit Salz und Pfeffer würzen.
        \step Zum servieren die Nudeln in eine Schale geben, die Suppe darüber gießen und mit den angebratenen Pilzen und Petersilie garnieren.
    }

    % \suggestion[Title of Suggestion]{
	% 	Suggestion
    % }

    \hint{Die getrockneten Pilz und das dazu benutzte Wasser können durch \SI{250}{\g} Pilze und etwa \SI{250}{\g} Gemüsebrühe ersetzt werden.}

\end{recipeDP}

			\begin{recipeDP}
    [
        preparationtime = {\SI{15}{\minute}},
        bakingtime = {\SI{35}{\minute},
        % bakingtemperature = {\protect\bakingtemperature{fanoven=\SI{180}{\celsius}}},
        portion = {6 Portionen},
        source = {rainbowplantlife.com}
    ]
    {Kritharaki-Suppe mit Kichererbsen}

    \graph
        {
            big=Hauptgerichte/Suppen_und_Eintoepfe/Kritharaki-Suppe/big.jpg,
            small=Hauptgerichte/Suppen_und_Eintoepfe/Kritharaki-Suppe/small.jpg
        }

    \introduction{Diese Kritharaki-Suppe ist eine herzhafte und einfache Suppe mit Kichererbsen, Kritharaki, Gewürzen und frischen Kräutern, die sich gut für einen Schnellkochtopf eignet.}

    \ingredients{
        \SI{2}{\EL} & natives Olivenöl extra \\
        1 & gelbe Zwiebel \\
        4 & Knoblauchzehen \\
        \SI{2}{\TL} & Paprikapulver \\
        \SI{2}{\TL} & Kreuzkümmel \\
        \SI{1}{\TL} & Chiliflocken \\
        \SI{4}{\EL} & Tomatenmark \\
        \SI{1}{\l} & Gemüsebrühe \\
        \SI{500}{\ml} & Wasser \\
        \SI{250}{\g} & Kritharaki \\
        \SI{450}{\g} & Kichererbsen (Dose) \\
        2 & Lorbeerblätter \\
        \SI{1}{\TL} & Rosmarin \\
        \SI{1,5}{\TL} & Oregano \\
        \SI{1,5}{\TL} & Salz \\
         & Pfeffer \\
         \SI{800}{\g} & Tomatenstücke (Dose) \\
         10 & sonnengetrocknete Tomaten \\
         \SI{3}{\TL} & Kapern \\
         & Petersilie \\
         & Zitronenabrieb
    }

    \preparation{
        \step Olivenöl in einem großen Topf erhitzen, dann die gewürfelte Zwiebel hinzugeben und für \SI{5}{\minute} braten, bis die Zwiebel leicht braun ist. Den gepressten Knoblauch dazugeben und nochmal für \SI{2}{\minute} braten.
        \step Paprikapulver, Kreuzkümmel, Chiliflocken und Tomatenmark hinzugeben. Ständig umrühren, damit nichts anbrennt und für \SI{2}{minute} erwärmen bis das Tomatenmark dunkler geworden ist. Danach mit einen Schuss der Gemüsebrühe ablöschen und den Topf frei kratzen. Wenn alles in der Brühe ausgelöst ist, die restliche Gemüsebrühe dazugeben. Das Wasser und die Kritharaki, die Kichererbsen, Lorbeerblätter, Rosmarin, Oregano, Salz, Pfeffer und die Tomaten aus der Dose hinzugeben. Gut miteinander verrühren und aufkochen.
        \step Wenn die Suppe einmal aufgekocht ist, auf mittler bis geringe Hitze stellen und die sonnengetrockneten Tomaten mit den Kapern und der Petersilie dazugeben. Mit Salz und Pfeffer abschmecken und mit frischer Petersilie und etwas Zitronenabrieb servieren.
    }

    % \suggestion[Title of Suggestion]{
	% 	Suggestion
    % }

    % \hint{Hint}

\end{recipeDP}

			% \begin{recipeDP}
    [
        preparationtime = {\SI{40}{\minute}},
        % bakingtime = {\SI{ZEIT}{\minute} bis \SI{ZEIT}{\minute}},
        % bakingtemperature = {\protect\bakingtemperature{fanoven=\SI{TEMPERATUR}{\celsius}}},
        portion = {4 Portionen},
        source = {@vegamelon}
    ]
    {Reis-Champignon-\addtoidx{Suppe}}

    \graph
        {
            % big=TEIL/KAPITEL/REZEPT/big.jpg,
            small=Hauptgerichte\Suppen_und_Eintoepfe\Reis-Champignon-Suppe\small.jpg
        }

    % \introduction{
    %     EINLEITUNG
    % }

    \ingredients{
        \SI{3}{\EL} & Olivenöl \\
        \SI{1}{\TL} & Thymian \\
        \SI[parse-numbers = false]{\nicefrac{1}{2}}{\TL} & Oregano \\
        2 & Schalotten \\
        2 Zehen & Knoblauch \\
        \SI{300}{\g} & \addtoidx{Reis} \\
        \SI{700}{\g} & \addtoidx{Champignons} \\
        \SI{700}{\ml} & Gemüsebrühe \\
        \SI{200}{\g} & Soja-Sahne
         & Tahin \\
         & Salz \\
         & Pfeffer \\
         & Zitronensaft
    }

    \preparation{
        \step Zuerst die Pilze säubern und in grobe Scheiben schneiden, die Schalotten fein würfeln. In einem großen Topf dann das Olivenöl erwärmen und die Schalotten dazugeben, den Knoblauch hinein pressen und für etwa \SI{5}{\minute} glasig dünsten. Danach die Champignons, den Reis und die Gewürze dazu geben, unterrühren und gemeinsam anbraten.
        \step Wenn die Pilze etwa Flüssigkeit abgegeben haben, die Gemüsebrühe hinein geben. Alles aufkochen und dann auf niedriger Stufe köcheln lassen (gelegentlich umrühren). Wenn der Reis gar ist, die Soja-Sahne dazu geben, nach Bedarf auch etwas Tahin. Die Suppe kann nach Belieben mit mehr Gemüsebrühe noch etwas verflüssigt werden. Abschließend mit Salz, Pfeffer und Zitronensaft abschmecken.
    }

    % \suggestion[TITEL EINES VORSCHLAGS]{
	% 	VORSCHLAG (DURCH HORIZONTALE LINIE VOM REZEPT GETRENNT)
    % }
    %
    % \hint{
    %     HINWEIS (IN EINEM KASTEN UNTEN AUF DER SEITE)
    % }

\end{recipeDP}

			\begin{recipeDP}
    [
        preparationtime = {\SI{40}{\minute}},
        % bakingtime = {\SI{ZEIT}{\minute} bis \SI{ZEIT}{\minute}},
        % bakingtemperature = {\protect\bakingtemperature{fanoven=\SI{TEMPERATUR}{\celsius}}},
        portion = {4 Portionen},
        source = {veganmiche.blog}
    ]
    {Matar Paneer}

    \graph
        {
            big=Hauptgerichte/Suppen_und_Eintoepfe/Matar_Paneer/big.jpg,
            small=Hauptgerichte/Suppen_und_Eintoepfe/Matar_Paneer/small.jpg
        }

    \introduction{
        Einfaches cremiges \addtoidx{Curry} mit \addtoidx{Tofu} und \addtoidx{Erbsen}
    }

    \ingredients{
        \SI{450}{\g} & Naturtofu \\
        \SI{200}{\g} & Tiefkühlerbsen \\
        \\
        2 & weiße Zwiebeln \\
        \SI{2}{\cm} & Ingwer \\
        5 & Knoblauchzehen \\
        2 & grüne Chilis \\
        \SI{100}{\ml} & Wasser \\
        \SI{1}{\TL} & Kreuzkümmel \\
        \SI{500}{\ml} & passierte Tomaten \\
        \SI{1}{\TL} & Koriander \\
        \SI{1}{\TL} & Kurkuma \\
        \SI{1}{\TL} & Currypulver \\
        \SI{1}{\TL} & Salz \\
        \SI{1}{\TL} & Pfeffer \\
        \\
        \SI{50}{\g} & Cashews \\
        \SI{60}{\ml} & Wasser \\
        \SI{300}{\g} & Reis oder Quinoa
    }

    \preparation{
        \step Zuerst die Cashews mit heißem Wasser übergießen und mindestens \SI{10}{\minute} einweichen lassen. Die Zwiebeln mit dem Ingwer, dem Knoblauch, den Chilis und \SI{100}{\ml} Wasser pürieren. Den Reis oder Quinoa in heißem Waser zubereiten.
        \step In einem großen Topf den Kreuzkümmel in etwas Öl erwärmen und nach einer Minute die Zwiebelsoße dazu geben. Zusammen \SI{5}{\minute} köcheln lassen und dann mit den passierten Tomaten ablöschen. Die Gewürze hinzugeben und weitere \SI{5}{\minute} köcheln lassen.
        \step Den Tofu in Würfel schneiden und mit den Erbsen in das Curry geben. Gemeinsam \SI{15}{\minute} köcheln lassen. Währenddessen die eingeweichten Cashews mit \SI{60}{\ml} Wasser zu einer cremigen Soße pürieren.
        \step Abschließend die Cashew-Soße einrühren und das Curry mit Reis oder Quinoa servieren.
    }

    % \suggestion[TITEL EINES VORSCHLAGS]{
	% 	VORSCHLAG (DURCH HORIZONTALE LINIE VOM REZEPT GETRENNT)
    % }

    % \hint{
    %     HINWEIS (IN EINEM KASTEN UNTEN AUF DER SEITE)
    % }

\end{recipeDP}

	\clearpage
		\sectionwithtoc{Nudel-Gerichte}
			\begin{recipeDP}
    [
        preparationtime = {\SI{10}{\minute}},
        % bakingtime = {\SI{12}{\minute} bis \SI{15}{\minute}},
        % bakingtemperature = {\protect\bakingtemperature{fanoven=\SI{180}{\degree}}},
        portion = {1 Portion},
        source = {@fitgreenmind}
    ]
    {Lazy Noodles}

    % \graph
    %     {
    %         small=Recipes/MainCourses/BBQChicken/Small.jpg,
    %         big=example-image
    %     }

    % \introduction{einleitung}

    \ingredients{
        \textbf{Soße}\\
        \SI{125}{\ml} & Wasser \\
        \SI{4}{\EL} & Soja Soße \\
        \SI{2}{\EL} & Süße (z.B. Ahornsirup) \\
        \SI{1}{\EL} & Weißweinessig oder Zitronensaft \\
        \SI[parse-numbers = false]{\nicefrac{1}{2}}{\EL} & Tahin \\
        \SI[parse-numbers = false]{\nicefrac{1}{2}}{\TL} & Chiliflocken \\
        \SI{2}{\EL} & Maisstärke \\
        \\
        \textbf{sonst} \\
        \SI{250}{\g} & gekochte \addtoidx{Nudeln} \\
         & Petersilie
    }

    \preparation{
        \step Alle Zutaten für die Soße in einem Mixer miteinander verquirlen, bis eine homogene Soße entsteht.

        \step Die fertige in einer Pfanne erhitzen, wenn sie anfängt anzudicken, die Nudeln hinzugeben. Verrühren bis alles mit dickflüssiger Soße bedeckt ist. Dann mit Petersilie anrichten.
    }

    % \suggestion[Title of Suggestion]{
	% 	Suggestion
    % }

    % \hint{Hint}

\end{recipeDP}

			\begin{recipeDP}
    [
        preparationtime = {\SI{45}{\minute}},
        % bakingtime = {\SI{12}{\minute} bis \SI{15}{\minute}},
        % bakingtemperature = {\protect\bakingtemperature{fanoven=\SI{180}{\degree}}},
        portion = {5 Portionen},
        source = {@cookingforpeanuts}
    ]
    {\addtoidx{Brokkoli}, \addtoidx{Kichererbsen} in heller Soße}

    \graph
        {
            big=Hauptgerichte/Nudelgerichte/Brokkoli_und_Kicherebsen/big.jpg
            small=Hauptgerichte/Nudelgerichte/Brokkoli_und_Kicherebsen/small.jpg
        }

    \introduction{Eine helle Soße, die gut zu Nudeln passt}

    \ingredients{
        \SI{150}{\g} & \addtoidx{Cashews} \\
         & Wasser \\
        \\
        \SI{300}{\g} & Brokkoli \\
        \SI{45}{\g} & \addtoidx{Hefeflocken} \\
        \SI{1}{\TL} & Knoblauchpulver \\
        \SI{1}{\TL} & Zwiebelpulver \\
        \SI{1}{\TL} & Pfeffer \\
        \SI{350}{\milli\l} & Wasser\\
        \SI{340}{\g} & gekochte Kichererbsen \\
        \\
         & Nudeln
    }

    \preparation{
        \step Zuerst die Cashews in kochendem Wasser einlegen, sodass die \SI{2}{\cm} bedeckt sind. Dann etwa \SI{20}{\minute} ziehen lassen.

        \step in der Zwischenzeit den Brokkoli vorbereiten: säubern und in kleine Stücke zerlegen. Dann in kochendem Wasser etwa \SI{3}{\minute} blanchieren. Dann unter kaltem Wasser abschrecken. In dem heißen Wasser des Brokkolis anschlie0end die Nudeln nach Bedarf kochen.

        \step Für die helle Soße die Cashew abgießen und mit den Hefeflocken, den Gewürzen und etwa \SI{350}{\ml} Wasser zusammen pürieren, bis eine glatte Soße entsteht. Die fertige Soße in einer großen Pfanne erhitzen, bis sie langsam eindickt. Dann Die Kichererbsen und den Brokkoli unterrühren. Mit Nudeln servieren.
    }

    % \suggestion[Title of Suggestion]{
	% 	Suggestion
    % }
    %
    \hint{Damit der Brokkoli beim Kochen seine Farbe behält, etwa einen halben Teelöffel Natron ins Kochwasser geben.}

\end{recipeDP}

			\begin{recipeDP}
    [
        preparationtime = {\SI{30}{\minute}},
        % bakingtime = {\SI{12}{\minute} bis \SI{15}{\minute}},
        % bakingtemperature = {\protect\bakingtemperature{fanoven=\SI{180}{\celsius}}},
        portion = {4 Portionen},
        source = {@myperfectgreens}
    ]
    {\addtoidx{Tortellini} in Tomatensoße}

    \graph
        {
            % big=example-image
            small=Hauptgerichte/Nudelgerichte/Tortellini_in_Tomatensoße/small.jpg
        }

    % \introduction{einleitung}

    \ingredients{
        \SI{500}{\g} & semifrische Tortellini \\
        \\
        \SI{250}{\g} & Shiitake-\addtoidx{Pilze} \\
        \SI{200}{\g} & \addtoidx{Brokkoli} \\
        1 & rote Zwiebel \\
        1 & Knoblauchzehe \\
        \SI{1}{\EL} & Olivenöl \\
        \SI{3}{\EL} & Tomatenmark \\
        \SI{100}{\g} & Cherry-Tomaten \\
        \SI{1}{\TL} & Paprikapulver \\
        \SI{2}{\EL} & \addtoidx{Hefeflocken} \\
        \SI[parse-numbers = false]{\nicefrac{1}{2}}{\TL} & Salz \\
        \SI[parse-numbers = false]{\nicefrac{1}{2}}{\TL} & Pfeffer \\
        \SI{1}{\TL} & Chiliflocken \\
         & Basilikum
    }

    \preparation{
        \step Die Zwiebel und den Knoblauch fein würfeln und für eine Minute in einer großen Pfanne mit dem Olivenöl, Paprikapulver und Chili anbraten. Dann die in Scheiben geschnittenen Pilze mit einem Schuss Wasser hinzugeben und für \SI{3}{\minute} braten. Währenddessen die Tortellini nach Packungsangabe kochen.
        \step Die Brokkolistücke und Cherry-Tomaten in die Pfanne geben und mit Salz und Pfeffer würzen. Auf mittlerer Stufe köcheln lassen, bis die Tomaten weich werden.
        \step Die Tomaten dann (nach etwa \SI{2}{\minute}) mit einem Löffel zerdrücken, das Tomatenmark mit den Hefeflocken und den fertigen Tortellini hinzugeben. Wenn die Soße noch etwas flüssiger werden soll, mit etwas Wasser verdünnen. Schließlich etwa \SI{2}{\minute} köcheln, bis die Soße die gewünschte Konsistenz erreicht.
        \step Mit Basilikum und Hefeflocken servieren.
    }

    % \suggestion[Title of Suggestion]{
	% 	Suggestion
    % }

    \hint{Statt der Tortellini lassen sich auch andere Nudeln verwenden. Einfach gekocht in die Soße geben und servieren.}

\end{recipeDP}

			\begin{recipeDP}
    [
        preparationtime = {\SI{20}{\minute}},
        % bakingtime = {\SI{12}{\minute} bis \SI{15}{\minute}},
        % bakingtemperature = {\protect\bakingtemperature{fanoven=\SI{180}{\celsius}}},
        portion = {4 Portionen},
        source = {@healthygirlkitchen}
    ]
    {Italienischer Nudeltopf}

    \graph
        {
            % big=example-image,
            small=Hauptgerichte/Nudelgerichte/Italienischer_Nudeltopf/small.jpg
        }

    % \introduction{einleitung}

    \ingredients{
        \SI{500}{\g} & \addtoidx{Nudeln} \\
        \SI{1,2}{\l} & Wasser \\
        \SI{600}{\ml} & passierte Tomaten\index{Tomaten!passiert} \\
        \SI{90}{\g} & \addtoidx{Oliven} \\
        \SI{30}{\g} & sonnengetrocknete Tomaten\index{Tomaten!sonnengetrocknet} \\
        \SI{2}{\EL} & Kapern \\
        \SI{2}{\TL} & Basilikum \\
        \SI{1}{\TL} & Oregano \\
        \SI{1}{\TL} & Knoblauch \\
        \SI{100}{\g} & Spinat \\
         & Pfeffer \\
         & Salz \\
         & Chiliflocken
    }

    \preparation{
        \step Alle Zutaten in einem Topf oder einer großen Pfanne zusammen aufkochen. Mit Salz, Pfeffer und Chiliflocken würzen. Nach dem Aufkochen auf mittlerer Hitze etwa \SI{10}{\minute} weiter köcheln lassen. Währenddessen regelmäßig umrühren damit nichts anbrennt.
    }

    % \suggestion[Title of Suggestion]{
	% 	Suggestion
    % }

    % \hint{Hint}

\end{recipeDP}

			\begin{recipeDP}
    [
        preparationtime = {\SI{30}{\minute}},
        % bakingtime = {\SI{ZEIT}{\minute} bis \SI{ZEIT}{\minute}},
        % bakingtemperature = {\protect\bakingtemperature{fanoven=\SI{TEMPERATUR}{\celsius}}},
        portion = {4 Portionen},
        source = {@cookingforpeanuts}
    ]
    {Tofu-Tomaten-\addtoidx{Nudeln}}

    \graph
        {
            % big=TEIL/KAPITEL/REZEPT/big.jpg,
            small=Hauptgerichte/Nudelgerichte/Tofu_Tomaten_Nudeln/small.jpg
        }

    % \introduction{
    %     EINLEITUNG
    % }

    \ingredients{
        \SI{500}{\g} & Nudeln \\
        \SI{400}{\g} & Naturtofu\index{Tofu!Natur-} \\
        \SI{70}{\ml} & Olivenöl \\
        2 & Knoblauchzehen \\
        \SI{20}{\g} & Hefeflocken \\
        \SI{2}{\EL} & weiße \addtoidx{Miso}-Paste \\
        \SI{180}{\g} & sonnengetrocknete Tomaten\index{Tomaten!sonnengetrocknet} \\
        & Zitronensaft \\
        & Chiliflocken \\
    }

    \preparation{
        \step Die Nudeln nach Packungsanleitung kochen, bis sie gar sind. Von dem Kochwasser \SI{400}{\ml} aufheben, den rest abgießen.
        \step Während die Nudeln kochen, mit einem Mixer den Tofu mit dem Olivenöl, dem Knoblauch, den Hefeflocken und der Miso-Paste vermischen. Das Nudelwasser kann dann eingerührt werden, zunächst nur \SI{200}{\ml}. Die Soße soll cremig werden, nach Bedarf bs zu \SI{400}{\ml} dazugeben.
        \step Mit einem Teigschaber die Soße in einen Topf geben und auf mittlerer Stufe erhitzen. Währenddessen die sonnengetrockneten Tomaten etwas vom Öl befreien und grob zerschneiden. Wenn die Soße warm ist, die Nudeln und die Tomatenstücke hinzugeben und vermengen.
        \step Zum servieren mit Zitronensaft beträufeln und mit Chiliflocken würzen.
    }

    % \suggestion[TITEL EINES VORSCHLAGS]{
	% 	VORSCHLAG (DURCH HORIZONTALE LINIE VOM REZEPT GETRENNT)
    % }
    %
    % \hint{
    %     HINWEIS (IN EINEM KASTEN UNTEN AUF DER SEITE)
    % }

\end{recipeDP}

			\begin{recipeDP}
    [
        preparationtime = {\SI{20}{\minute}},
        bakingtime = {\SI{30}{\minute} bis \SI{40}{\minute}},
        bakingtemperature = {\protect\bakingtemperature{fanoven=\SI{180}{\celsius}}},
        portion = {4 Portionen},
        source = {@healthygirlkitchen}
    ]
    {\addtoidx{Nudeln} mit gebackenem \addtoidx{Hummus}}

    \graph
        {
            % big=TEIL/KAPITEL/REZEPT/big.jpg,
            small=Hauptgerichte/Nudelgerichte/Nudeln_mit_gebackenem_Hummus/small.jpg
        }

    % \introduction{
    %     EINLEITUNG
    % }

    \ingredients{
        \SI{300}{\g} & Hummus \\
        \SI{500}{\g} & Nudeln \\
        \SI{150}{\g} & Cherry-Tomaten\index{Tomaten!Cherry-} \\
        1 & Zucchini \\
        \SI{250}{\g} & Champignons \\
        \SI{2}{\TL} & Knoblauch \\
        \SI{50}{\ml} & Weißwein \\
        \SI{50}{\ml} & Gemüsebrühe \\
        \SI{1}{\TL} & Salz \\
        \SI{3}{\EL} & Olivenöl \\
        \SI{1}{\TL} & Oregano \\
        \SI{1}{\TL} & Thymian
    }

    \preparation{
        \step Den Hummus in einer großen Auflaufform mittig platzieren. Drum herum das klein geschnittene Gemüse verteilen. Mit dem Knoblauch, dem Weißwein, die Gemüsebrühe, dem Öl und den Gewürzen vermischen.
        \step Die Auflaufform für \SI{30}{\minute} bis \SI{40}{\minute} bei \SI{180}{\celsius} im Backofen garen. Wenn die letzten \SI{10}{\minute} anbrechen, die Nudeln nach Packungsanleitung zubereiten. Abschließend die Nudeln in die fertige Auflaufform geben und servieren.
    }

    % \suggestion[TITEL EINES VORSCHLAGS]{
	% 	VORSCHLAG (DURCH HORIZONTALE LINIE VOM REZEPT GETRENNT)
    % }
    %
    % \hint{
    %     HINWEIS (IN EINEM KASTEN UNTEN AUF DER SEITE)
    % }

\end{recipeDP}

			\begin{recipeDP}
    [
        preparationtime = {\SI{60}{\minute}},
        % bakingtime = {\SI{ZEIT}{\minute} bis \SI{ZEIT}{\minute}},
        % bakingtemperature = {\protect\bakingtemperature{fanoven=\SI{TEMPERATUR}{\celsius}}},
        portion = {4 Portionen},
        source = {@thevegansara}
    ]
    {Mac \'n\' Cheese \addtoidx{Chowder}}

    \graph
        {
            % big=TEIL/KAPITEL/REZEPT/big.jpg,
            small=Hauptgerichte\Nudelgerichte\Mac_n_Cheese_Chowder\small.jpg
        }

    % \introduction{
    %     EINLEITUNG
    % }

    \ingredients{
        \SI{500}{\5} & \addtoidx{Nudeln} \\
        \\
        \multicolumn{2}{l}{\textbf{Cashew-Creme:}} \\
        \SI{120}{\g} & Cashewkerne \\
        \SI{475}{\ml} & Gemüsebrühe \\
        \SI{150}{\g} & eingelegte, geröstete Paprika\index{Paprika!geröstet} \\
        \SI{4}{\EL} & \addtoidx{Hefeflocken} \\
        \\
        \multicolumn{2}{l}{\textbf{Suppe:}} \\
        \SI{5}{\EL} & Butter \\
        2 & Zwiebeln \\
        \SI{5}{\EL} & Mehl \\
        \SI{4}{\EL} & Senf \\
        \\
        \SI{1}{\l} & Gemüsebrühe \\
        4 Stangen & \addtoidx{Sellerie} \\
        2 & Möhren \\
        \SI{500}{\g} & \addtoidx{Brokkoli} \\
        \\
        \SI{700}{\ml} & Milch \\
        \SI{2}{\EL} & Essig \\
         & Salz \\
         & Pfeffer
    }

    \preparation{
        \step Für die Cashew-Creme zuerst die Cashews, mit der Brühe, den Paprikastücken und Hefeflocken in einem Mixer pürieren bis die Soße cremig ist. Dann beiseite stellen. Das Gemüse säubern und klein schneiden, den Brokkoli in kleine Röschen zerteilen.
        \step In einem sehr großen Topf die Butter schmelzen und die kleingeschnittenen Zwiebeln etwa \SI{5}{\minute} darin andünsten. Wenn die Zwiebeln glasig sind, den Senf hinzu geben und vorsichtig das Mehl mit einem Schneebesen einrühren. Die Mehlschwitze auf mittlerer Stufe etwa \SI{2}{\minute} erwärmen. Nebenher die Nudeln nach Packungsangabe zubereiten und dann das Wasser abgießen.
        \step Die Mehlschwitze mit etwas von der Brühe cremig rühren. Wenn keine Klümpchen mehr übrig sind, den Rest der Gemüsebrühe dazugeben und das klein geschnittene Gemüse darin aufkochen und etwa \SI{20}{\minute} köcheln.
        \step Abschließend die Milch, die Cashew-Creme und den Essig hinzugeben und mit Salz, Pfeffer und Hefeflocken abschmecken. Die Nudeln im dem Chowder servieren.
    }

    % \suggestion[TITEL EINES VORSCHLAGS]{
	% 	VORSCHLAG (DURCH HORIZONTALE LINIE VOM REZEPT GETRENNT)
    % }
    %
    % \hint{
    %     HINWEIS (IN EINEM KASTEN UNTEN AUF DER SEITE)
    % }

\end{recipeDP}

	\clearpage
		\sectionwithtoc{Pfannen-Gerichte}
			\begin{recipe}
    [
        preparationtime = {\SI{40}{\minute}},
        % bakingtime = {\SI{12}{\minute} bis \SI{15}{\minute}},
        % bakingtemperature = {\protect\bakingtemperature{fanoven=\SI{180}{\degree}}},
        portion = {4 Portionen},
        source = {@playitisvegan}
    ]
    {\addtoidx{Tofu} in scharfer Erdnusssoße}

    % \graph
    %     {
    %         small=Recipes/MainCourses/BBQChicken/Small.jpg,
    %         big=example-image
    %     }

    % \introduction{einleitung}

    \ingredients{
        \SI{450}{\g} & Naturtofu \\
        \SI{2}{\EL} & Öl \\
        2 & Schalotten \\
        \SI{2}{Zehen} & Knoblauch \\
        \SI{2}{\EL} & Sojasoße \\
        \SI{2}{\EL} & Ketchup \\
        \SI{60}{\ml} & \addtoidx{Erdnussbutter} \\
        \SI{1}{\EL} & Ahornsirup \\
        \SI{250}{\ml} & Gemüsebrühe \\
        \SI{1}{\EL} & Sriracha-Soße
    }

    \preparation{
        \step In einer großen Pfanne das Öl erhitzen und die fein gehackten Schalotten und den Knoblauch hinzufügen. Alles 2 Minuten lang glasig braten.

        \step Die Sojasauce, den Ketchup, die Erdnussbutter, den Ahornsirup, die Gemüsebrühe und die Sriracha-Soße einrühren und gut miteinander vermixen. In diese Soße den gewürfelten Tofu geben und gut unterheben. Die Soße aufkochen und eindicken lassen.

        \step Den Tofu auf Reis servieren und mit Petersilie garnieren.
    }

    % \suggestion[Title of Suggestion]{
	% 	Suggestion
    % }
    %
    % \hint{Hint}

\end{recipe}

			\begin{recipeDP}
    [
        preparationtime = {\SI{45}{\minute}},
        % bakingtime = {\SI{12}{\minute} bis \SI{15}{\minute}},
        % bakingtemperature = {\protect\bakingtemperature{fanoven=\SI{180}{\degree}}},
        portion = {10 Stück},
        source = {@vegangotgame.co}
    ]
    {Kartoffel-\addtoidx{Zucchini}-\addtoidx{Bratlinge}}

    \graph
        {
            small=Hauptgerichte/Pfannengerichte/Kartoffel-Zucchini-Bratlinge/small.jpg,
            % big=example-image
        }

    % \introduction{einleitung}

    \ingredients{
    1 & Zucchini \\
    \SI{150}{\g} & \addtoidx{Kartoffeln} \\
    1 & kleine Zwiebel \\
    \SI{40}{\g} & rote Paprika, gewürfelt \\
    \SI{30}{\g} & Grünkohl\index{Gruenkohl@Grünkohl}, zerkleinert \\
    \SI{65}{\g} & Mehl \\
    \SI[parse-numbers = false]{\nicefrac{1}{2}}{\TL} & Salz \\
    \SI[parse-numbers = false]{\nicefrac{1}{2}}{\TL} & Knoblauchpulver \\
    \SI[parse-numbers = false]{\nicefrac{1}{2}}{\TL} & versch. Kräuter \\
     & Pfeffer
    }

    \preparation{
        \step Die Zucchini und die geschälten Kartoffeln raspeln und mit einem sauberen Tuch möglichst viel Flüssigkeit ausdrücken. Die fein gewürfelte Paprika und Zwiebel sowie den klein geschnittenen Grünkohl mit den Raspeln zusammen vermengen.

        \step Salz, Pfeffer, Knoblauch, Kräuter und Mehl einrühren, bis eine homogene Masse entsteht. Wenn der Teig zu feucht ist, noch etwa Mehl hinzu geben, sodass der Teig gut zusammen hält. Den Teig dann in \SI{5}{\cm} bis \SI{8}{\cm} Bratlinge formen.

        \step Die Bratlinge entweder frittieren oder in einer heißen Pfanne mit etwas Öl ausbraten, jede Seite etwa \SI{4}{\minute}.
    }

    % \suggestion[Title of Suggestion]{
	% 	Suggestion
    % }
    %
    % \hint{Hint}

\end{recipeDP}

			\begin{recipeDP}
    [
        preparationtime = {\SI{50}{\minute}},
        % bakingtime = {\SI{12}{\minute} bis \SI{15}{\minute}},
        % bakingtemperature = {\protect\bakingtemperature{fanoven=\SI{180}{\degree}}},
        portion = {6 Burger},
        source = {myprotein.com}
    ]
    {Würziger Kichererbsenburger\index{Kichererbsen}}

    \graph
        {
            big=Hauptgerichte/Pfannengerichte/Wuerziger_Kichererbsenburger/big.png,
            small=Hauptgerichte/Pfannengerichte/Wuerziger_Kichererbsenburger/small.jpg
        }

    % \introduction{einleitung}

    \ingredients{
        \multicolumn{2}{l}{\textbf{Burger}\index{Burger}} \\
        \SI{600}{\g} & Kichererbsen (aus der Dose) \\
        \SI{50}{\g} & Vollkornmehl \\
        \SI{60}{\g} & frischer Koriander \\
        1 & Zwiebel \\
        4 & Knoblauchzehen \\
        \SI{50}{\g} & getrocknete Tomaten \\
        \SI{1}{\TL} & Kumin \\
        \SI[parse-numbers = false]{\nicefrac{1}{2}}{\TL} & Kurkuma \\
        \SI{1}{\TL} & Salz \\
        \SI{1}{\TL} & Cayenne-Pfeffer \\
        \SI{2}{\TL} & Hummus \\
         & Öl \\
        \\
        \multicolumn{2}{l}{\textbf{Sweet Chili Mayo}} \\
        \SI{3}{\TL} & \addtoidx{Mayonnaise} \\
        \SI{3}{\TL} & \addtoidx{Sweet Chili Sauce} \\
        \\
        6 & Vollkorn-Burgerbrötchen \\
        \SI[parse-numbers = false]{\nicefrac{1}{2}}{} & Lila Kohl \\
        1 Handvoll & Spinat
    }

    \preparation{
        \step Tropfe zunächst die Kichererbsen ab. Gib sie anschließend in deine Küchenmaschine, zusammen mit dem Mehl, Koriander, der Zwiebel, dem Knoblauch, den Tomaten, dem Cayenne Pfeffer, dem Kumin, dem Kurkuma, Salz und Hummus.
        \step Mixe alles gut durch, bis du einen homogenen Teig herausbekommst. Gib etwas mehr Hummus hinzu, wenn der Teig zu krümelig wird und mixe noch einmal. Gib den Teig anschließend auf einen Teller und decke ihn mit Frischhaltefolie ab. Stelle ihn für 30 Minuten in den Kühlschrank.
        \step Sobald die Mischung etwas fester geworden ist, teilst du sie in 6 gleich große Teile auf. Verleihe ihnen die Form eines Burger Patties, indem du sie mit der Hand formst. Erhitze eine antihaft-beschichtete Pfanne auf mittlerer Hitze mit etwas Öl. Sobald die Pfanne heiß ist, gibst du die Burger Patties hinein. Brate sie für je 3 Minuten von jeder Seite und sei vorsichtig beim Wenden!
        \step Während die Patties braten, kannst du die Sweet Chili Mayonnaise zubereiten, indem du die vegane Mayo mit der Sweet Chili Sauce mischt.
        \step Toaste die Burgerbrötchen, ehe du die Patties mit ein wenig Sweet Chili Mayo dazu gibst. Garniere nach Belieben mit Kohl, Salat und etwas mehr Sweet Chili Mayo.
    }

    % \suggestion[Title of Suggestion]{
	% 	Suggestion
    % }
    %
    % \hint{Hint}

\end{recipeDP}

	\clearpage
		\sectionwithtoc{Mehrtopf-Gerichte}
			\begin{recipeDP}
    [
        preparationtime = {\SI{45}{\minute}},
        % bakingtime = {\SI{12}{\minute} bis \SI{15}{\minute}},
        % bakingtemperature = {\protect\bakingtemperature{fanoven=\SI{180}{\celsius}}},
        portion = {4 Portionen},
        source = {@rabbitandwolves}
    ]
    {\addtoidx{Frikassee} und Blumenkohlbrei\index{Brei|Blumenkohl-}}

    \graph
        {
            % big=example-image
            small=Hauptgerichte/Mehrtopfgerichte/Frikassee_und_Blumenkohlbrei/small.png
        }

    % \introduction{einleitung}

    \ingredients{
        \multicolumn{2}{l}{\textbf{Frikassee}} \\
        \SI{3}{\EL} & Olivenöl \\
        1 & Zwiebel \\
        2 & Knoblauchzehen \\
        4 Stangen & \addtoidx{Sellerie} \\
        5 & Möhren \\
        \SI{500}{\g} & \addtoidx{Champignons} \\
        \SI{1}{\TL} & Thymian \\
        \SI[parse-numbers = false]{\nicefrac{1}{2}}{\TL} & Salbei \\
        \SI{1}{\TL} & Rosmarin \\
        \SI[parse-numbers = false]{\nicefrac{1}{2}}{\TL} & Muskatnuss \\
        \SI{55}{\g} & Butter \\
        \SI{35}{\g} & Mehl \\
        \SI{120}{\ml} & Weißwein \\
        \SI{500}{\ml} & Gemüsebrühe \\
        \SI{400}{\ml} & Kokosmilch \\
        Salz \\
        Pfeffer \\
        \\
        \multicolumn{2}{l}{\textbf{Blumenkohlbrei}} \\
        1 großer & \addtoidx{Blumenkohl} \\
        \SI{80}{\ml} & Mandelmilch \\
        \SI{2}{\EL} & Butter \\
        Salz \\
        Pfeffer \\
        Muskatnuss
    }

    \preparation{
        \step Zuerst die Zwiebel und den Sellerie würfeln, die Möhren in grobe Scheiben schneiden und die Pilze grob schneiden. Auch den Blumenkohl in grobe Röschen schneiden.
        \step Dann das Olivenöl in einem großen Topf bei mittlerer Hitze erhitzen. Die Zwiebel hinzugeben, den Knoblauch hinein pressen und den Sellerie mit hinzugeben. Das zusammen etwa \SI{3}{\minute} dünsten, bis die Zwiebel etwas glasig wird. Währenddessen die Blumenkohl-Röschen in einem großen Topf mit heißem, gesalzenem Wasser kochen (\SI{10}{\minute} oder bis ganz gar).
        \step Sind die Zwiebeln glasig, die Möhren und Pilze dazu geben und weitere \SI{3}{\minute} dünsten. Dann mit Salz und Pfeffer würzen, und wenn die Pilze (nach wiederum \SI{5}{\minute}) braun werden, mit den restlichen Gewürzen vermengen.
        \step Danach die Butter über dem Gemüse zerlassen und gut verrühren. Das Mehl gut darüber verteilen und klümpchenfrei verrühren.
        \step Als nächstes den Wein dazugeben und \SI{3}{\minute} köcheln lassen. Dann die Gemüsebrühe und die Kokosmilch einrühren und alles auf niedriger Hitze \SI{10}{\minute} bis \SI{15}{\minute} köcheln lassen, dabei immer wieder gut umrühren.
        \step Ist der Blumenkohl gar, wird das Kochwasser abgegossen und die Milch zu dem Blumenkohl in den Topf gegeben. Beides mit dem Stabmixer pürieren, bis ein glatter Brei entsteht und mit den Gewürzen abschmecken.
    }

    % \suggestion[Title of Suggestion]{
	% 	Suggestion
    % }

    % \hint{Hint}

\end{recipeDP}

	\clearpage
		\sectionwithtoc{Fingerfood}
			\begin{recipeDP}
    [
        preparationtime = {\SI{25}{\minute}},
        % bakingtime = {\SI{12}{\minute} bis \SI{15}{\minute}},
        % bakingtemperature = {\protect\bakingtemperature{fanoven=\SI{180}{\celsius}}},
        % portion = {40 Stück},
        source = {deutschlandistvegan.de}
    ]
    {\addtoidx{Tortilla}-\addtoidx{Couscous}-Röllchen}

    \graph
        {
            % big=example-image
            small=Hauptgerichte/Fingerfood/Tortilla-Couscous-Röllchen/small.jpg
        }

    \introduction{Leckere Röllchen gefüllt mit Couscous und Spinat dürfen auf eurem nächsten Party-Buffet für veganes Fingerfood nicht fehlen!}

    \ingredients{
        \SI{120}{\g} & Couscous \\
        \SI{300}{\ml} & kochendes Wasser \\
        1 Handvoll & \addtoidx{Spinat} \\
        1 Zweig & Petersilie \\
        \SI{2}{\EL} & Zitronensaft \\
        \SI{1}{\TL} & Salz \\
        \SI[parse-numbers = false]{\nicefrac{1}{2}}{\TL} & Chili-Flocken \\
         & Pfeffer \\
        \\
        4 bis 5 & Tortilla-Wraps
    }

    \preparation{
        \step Den Couscous mit dem kochenden Wasser übergießen und ziehen lassen, bis das Wasser komplett aufgesaugt ist.
        \step Die restlichen Zutaten mit einem Mixer fein pürieren und unter den Couscous rühren.
        \step Die Couscous-Füllung gleichmäßig auf die Tortilla-Wraps streichen. Dann die Wraps zusammenrollen und fingerdicke Scheiben runterschneiden.
    }

    % \suggestion[Title of Suggestion]{
	% 	Suggestion
    % }

    \hint{Am besten mit kleinen Spießen und einer Soße (etwa Barbecuesoße) servieren.}

\end{recipeDP}



	\clearpage
	\cleardoublepage
	\part{Nachspeisen und Snacks}
		\sectionwithtoc{Nachspeisen}
			\begin{recipeDP}
    [
        preparationtime = {\SI{30}{\minute}},
        % bakingtime = {\SI{12}{\minute} bis \SI{15}{\minute}},
        % bakingtemperature = {\protect\bakingtemperature{fanoven=\SI{180}{\degree}}},
        portion = {6 Stück},
        source = {minimalistbaker.com}
    ]
    {Glutenfreie \addtoidx{Waffeln}}

    % \graph
    %     {
    %         small=Recipes/MainCourses/BBQChicken/Small.jpg,
    %         big=example-image
    %     }

    % \introduction{einleitung}

    \ingredients{
        \SI{300}{\ml} & Mandelmilch \\
        \SI{1}{\TL} & Apfelessig \\
        \SI{60}{\ml} & Öl \\
        \SI{60}{\ml} & Ahornsirup oder andere Süße \\
        \SI{50}{\g} & (glutenfreie) Haferflocken \\
        \SI{280}{\g} & glutenfreies Mehl \\
        \SI[parse-numbers = false]{1\nicefrac{1}{2}}{\TL} & Backpulver \\
        \SI{1}{Prise} & Meersalz \\
        \\
        \multicolumn{2}{l}{\textbf{optional:}} \\
        \SI{1}{\TL} & Vanilleextrakt \\
        \SI[parse-numbers = false]{\nicefrac{1}{2}}{\TL} & Zimt \\
        \SI{7}{\g} & Leinsamenmehl \\
        \SI{45}{\g} & vegane Schokoladenstücke \\
        \SI{38}{\g} & Früchte
    }

    \preparation{
        \step Zuerst die Mandelmilch und den Essig in einer kleinen Schüssel miteinander verrühren, dann ein paar Minuten zum aktivieren stehen lassen. Anschließend das Öl und den Sirup hinzugeben und beiseite stellen.

        \step Die trockenen Zutaten in eine große Schüssel geben und gut miteinander vermischen. Dann die feuchten Zutaten in die Trockenen mixen, bis alles gut miteinander vermengt ist. Wenn der Teig etwas zu dickflüssig ist etwas Milch hinzugeben andernfalls etwas Mehl.

        \step Den Teig fünf bis zehn Minuten stehen lassen und dann das Waffeleisen vorheizen. Wenn das Waffeleisen heiß ist, gegebenenfalls einfetten und dann den Teig darin ausbraten. Die fertigen Waffeln am besten sofort mit den Toppings servieren.
    }

    % \suggestion[Title of Suggestion]{
	% 	Suggestion
    % }
    %
    \hint{Grundsätzlich ist Hafer glutenfrei, kleine Mengen Gluten können nur bei der Verarbeitung oder dem Transport des Hafers in die Haferflocken geraten.}

\end{recipeDP}

	\clearpage
		\sectionwithtoc{Snacks}
			\begin{recipeDP}
    [
        preparationtime = {\SI{20}{\minute}},
        % bakingtime = {\SI{12}{\minute} bis \SI{15}{\minute}},
        % bakingtemperature = {\protect\bakingtemperature{fanoven=\SI{180}{\degree}}},
        portion = {30 Stück},
        source = {@healthygirlkitchen}
    ]
    {Snickers-\addtoidx{Datteln}}

    \graph
        {
            big=Nachspeisen_und_Snacks/Snacks/Snickers-Datteln/big.jpg,
            small=Nachspeisen_und_Snacks/Snacks/Snickers-Datteln/small.jpg
        }

    % \introduction{einleitung}

    \ingredients{
        30 & Datteln, entsteint \\
        \SI{100}{\g} & Schokolade \\
         & Erdnussbutter \\
         & Erdnüsse \\
         & Meersalz
    }

    \preparation{
        \step Zuerst Blech mit Backpapier vorbereiten. Die entsteinten Datteln mit Erdnussbutter füllen und auf dem Backpapier bereit legen. Die Schokolade über einem heißen Wasserbad schmelzen und die Datteln mit Schokolade überziehen oder dekorieren. Mit ein oder zwei Erdnüssen und etwas Meersalz garnieren. Für mindestens \SI{30}{\minute} im Kühlschrank kühlen, bis die Schokolade fest ist.
    }

    % \suggestion[Title of Suggestion]{
	% 	Suggestion
    % }
    %
    % \hint{Hint}

\end{recipeDP}



	\clearpage
	\cleardoublepage
	\part{Gebäck}
		\sectionwithtoc{Brote}
			\begin{recipeDP}
    [
        preparationtime = {\SI{10}{\minute}},
        bakingtime = {\SI{40}{\minute}},
        bakingtemperature = {\protect\bakingtemperature{fanoven=\SI{200}{\celsius}}},
        portion = {1 Kastenform},
        source = {bbc goodfood}
    ]
    {Bananenbrot}

    % \graph
    %     {
    %         small=Recipes/MainCourses/BBQChicken/Small.jpg,
    %         big=example-image
    %     }

    % \introduction{einleitung}

    \ingredients{
        3 & große \addtoidx{Bananen} \\
        \SI{75}{\ml} & Öl \\
        \SI{100}{\g} & brauner Zucker \\
        \SI{225}{\g} & Weizenmehl \\
        \SI{3}{\geh \TL} & Backpulver \\
        \SI{3}{\TL} & Zimt oder Gewürzmischung\\
         & getrocknete Früchte
    }

    \preparation{
        \step Sen Ofen auf \SI{200}{\celsius} vorheizen. Die Bananen mit einer Gabel zerstampfen und dann Öl und den Zucker hinzufügen und verrühren.

        \step Das Weizenmehl, Backpulver und den Zimt oder die Gewürze hinzufügen und gut miteinander verrühren. An dieser Stelle können optional getrocknete Früchte unter gerührt werden.

        \step Dann den Teig in eine Kastenform füllen. Für \SI{20}{\minute} backen und dann nachsehen, ob das Bananenbrot mit Folie abgedeckt werden muss. Weitere \SI{20}{\minute} backen, bis die Garprobe erfolgreich ist. Etwas abkühlen lassen vor dem Servieren.
    }

    % \suggestion[Title of Suggestion]{
	% 	Suggestion
    % }

    \hint{Gut dazu passt: Erdnussbutter}

\end{recipeDP}

			\begin{recipeDP}
    [
        preparationtime = {\SI{25}{\minute}},
        bakingtime = {\SI{40}{\minute} bis \SI{50}{\minute}},
        bakingtemperature = {\protect\bakingtemperature{fanoven=\SI{175}{\degree}}},
        portion = {1 Baguette},
        source = {Sabine Schramm}
    ]
    {Mediterranes \addtoidx{Baguette}}

    % \graph
    %     {
    %         small=Recipes/MainCourses/BBQChicken/Small.jpg,
    %         big=example-image
    %     }

    % \introduction{einleitung}

    \ingredients{
        \SI{1}{\EL} & Chiasamen \\
        \SI{3}{\EL} & Wasser \\
        \SI{200}{\g} & Sojajoghurt \\
        \SI{130}{\g} & Dinkelvollkornmehl \\
        \SI{40}{\g} & Haferkleie \\
        \SI{1}{\EL} & Tomatenmark \\
        \SI{100}{\g} & Pepperoni \\
        \SI{1}{\TL} & Salz \\
        \SI{1}{\TL} & Thymian \\
        \SI{1}{\TL} & Basilikum \\
        \SI{1}{\pck} & Backpulver
    }

    \preparation{
        \step Zuerst mit den Chiasamen und dem Wasser ein veganes Ei vorbereiten: beides miteinander verrühren und etwa \SI{10}{\minute} lang stehen lassen. Währenddessen die Peperoni in Ringe schneiden. Den Ofen auf \SI{175}{\degree} Umluft vorheizen.

        \step Für das Baguette dann den Quark mit dem veganen Ei verrühren, dann alle anderen Zutaten hinzugeben. Gut miteinander verkneten, bis ein homogener Teig entsteht. Den Teig zu einem Baguette formen. Da der Teig sehr klebrig ist, helfen nasse Hände beim formen. Das Baguette auf einem mit Backpapier bedeckten Backblech für \SI{40}{\minute} bis \SI{50}{\minute} backen.

    }

    % \suggestion[Title of Suggestion]{
	% 	Suggestion
    % }
    %
    % \hint{Hint}

\end{recipeDP}

	\clearpage
		\sectionwithtoc{Brötchen}
	\clearpage
		\sectionwithtoc{Kekse}
			\begin{recipeDP}
    [
        preparationtime = {\SI{15}{\minute}},
        bakingtime = {\SI{12}{\minute} bis \SI{15}{\minute}},
        bakingtemperature = {\protect\bakingtemperature{fanoven=\SI{180}{\celsius}}},
        portion = {40 Stück},
        source = {Stina Spiegelberg}
    ]
    {Fruchtige Johannisbeerlebkuchen\index{Lebkuchen}}

    % \graph
    %     {
    %         small=Recipes/MainCourses/BBQChicken/Small.jpg,
    %         big=example-image
    %     }

    % \introduction{einleitung}

    \ingredients{
        \multicolumn{2}{l}{\textbf{Teig}} \\
        \SI{80}{\g} & Dinkelvollkornmehl \\
        \SI{100}{\g} & blanchierte, gemahlene Mandeln \\
        \SI{150}{\g} & gemischte, gemahlene Nüsse \\
        \SI{250}{\g} & Marzipanrohmasse \\
        \SI{180}{\g} & Rohrohrzucker \\
        \SI{1}{\pck} & Vanillezucker \\
        \SI[parse-numbers = false]{\nicefrac{1}{2}}{\TL} & Backpulver \\
        \SI{4}{\EL} & dunkle \addtoidx{Johannisbeermarmelade} \\
        \SI{60}{\ml} & Wasser \\
        etwa 40 & Backoblaten mit \SI{5}{\cm} \dmesser \\
        \\
        \multicolumn{2}{l}{\textbf{Dekor}} \\
        \SI{200}{\g} & Zartbitterkuvertüre \\
         & bunte Zuckerstreusel
    }

    \preparation{
        \step Mehl, Mandeln und Nüsse in eine große Rührschüssel geben, die Marzipanrohmasse mit einer groben Reibe in die trockenen Zutaten reibe. Dabei mehrmals rühren, damit das Marzipan nicht verklebt. Zucker, Backpulver Marmelade und Wasser zugeben und mit den Knethaken zu einem Teig verarbeiten. Es dürfen noch kleine Marzipanstückchen zu sehen sein.

        \step Den Teig kalt stellen, etwa über Nacht im Kühlschrank oder einige Stunden im Tiefkühler. Der Teig lässt sich dann einfacher verarbeiten und klebt nicht so sehr.

        \step Den Ofen auf \SI{180}{\celsius} vorheizen. Für die Zubereitung der Lebkuchen jeweils etwa \SI{1}{\TL} Teig auf eine Backoblate geben und mit den Fingern andrücken und bis an de Rand verteilen. Die Lebkuchen dann etwa \SI{12}{\minute} bis \SI{15}{\minute} backen, bis sie leicht braun werden. Dabei öfter in den Ofen schauen, damit nichts anbrennt.

        \step Die Lebkuchen über Nacht auskühlen lassen. Anschließend die Zartbitterkuvertüre im Wasserbad erwärmen und die Lebkuchen damit bestreichen. Mit den Zuckerstreuseln dann verzieren.
    }

    \suggestion[Zuckergussglasur]{
        Sehr gut passt auch eine Glasur aus etwas Zitronensaft und Puderzucker anstelle der Zartbitterkuvertüre. Dazu einige Tropfen Zitronensaft mit etwa \SI{100}{\g} Puderzucker verrühren, bis eine dickflüssige Glasur entsteht. Dann die Lebkuchen damit einstreichen.
    }

    \hint{Für das Original ``Elisenlebkuchen'' die Johannisbeermarmelade durch Aprikosenmarmelade ersetzen, je \SI{1}{\msp} gemahlenen Zimt und Koriander hinzufügen. Die Lebkuchen mit Kuvertüre und blanchierten Mandelhälften verzieren.}

\end{recipeDP}

			\begin{recipe}
    [
        preparationtime = {\SI{20}{\minute}},
        bakingtime = {\SI{15}{\minute}},
        bakingtemperature = {\protect\bakingtemperature{topbottomheat=\SI{190}{\degree}}},
        portion = {50 Stück},
        source = {Stina Spiegelberg}
    ]
    {Rosmarin-\addtoidx{Heidesand}}

    % \graph
    %     {
    %         small=Recipes/MainCourses/BBQChicken/Small.jpg,
    %         big=example-image
    %     }

    % \introduction{einleitung}

    \ingredients{
        \SI{180}{\g} & Zucker \\
        \SI{1}{\pck} & Vanillezucker \\
        \SI{200}{\g} & vegane Butter, zimmerwarm \\
        \SI{300}{\g} & Weizenmehl (Typ 405) \\
        \SI[parse-numbers = false]{\nicefrac{1}{2}}{\TL} & Fleur de Sel \\
        \SI{2}{\TL} & \addtoidx{Rosmarin}, gehackt \\
        1 & Zitrone, abgeriebene Schale \\
        \SI{4}{\EL} & Pflanzendrink
    }

    \preparation{
        \step \SI{150}{\g} Zucker, Vanillezucker und vegane Butter in einer großen Schüssel mit dem Schneebesen schaumig schlagen. Mehl, Salz, Rosmarin, Zitronenschale und Pflanzendrink zugeben und zu einem glatten Teig kneten. Den Teig in Frischhaltefolie wickeln und eine Stunde kalt stellen.

        \step Den Backofen auf \SI{190}{\degree} Ober-/Unterhitze vorheizen.

        \step Den Teig halbieren. Die Arbeitsfläche mit Mehl bestäuben und die Teighälften jeweils zu etaw \SI{40}{\cm} langen Broten formen. Den restlichen Zucker auf die Arbeitsfläche streuen und die Brote darin wälzen. Erneut in Frischhaltefolie wickeln und 30 Minuten kalt stellen.

        \step Von den Broten etwa \SI{1}{\cm} dicke Stücke abschneiden die Plätzchen auf einem mit Backpapier ausgelegten Backblech \ca 15 Minuten backen. Auskühlen lassen und luftdicht aufbewahren.
    }

    % \suggestion[Title of Suggestion]{
	% 	Suggestion
    % }

    \hint{Für die klassischen Heidesand-Plätzchen das Fleur de Sel durch eine Prise Meersalz ersetzen, den Rosmarin weglassen und nur den Abrieb einer halben Zitrone verwenden}

\end{recipe}

			\begin{recipeDP}
    [
        preparationtime = {\SI{25}{\minute}},
        bakingtime = {\SI{10}{\minute} bis \SI{12}{\minute}},
        bakingtemperature = {\protect\bakingtemperature{topbottomheat=\SI{180}{\celsius}}},
        portion = {70 Stück},
        source = {Stina Spiegelberg}
    ]
    {Lussekatter}

    % \graph
    %     {
    %         small=Recipes/MainCourses/BBQChicken/Small.jpg,
    %         big=example-image
    %     }

    \introduction{
        Lussekatter werden in Schweden am 13. Dezember zum Santa-Lucia-Fest, dem Lichterfest, gereicht. Übersetzt bedeutet der Name des leckeren Safrangebäcks ``Lucia-Katzen''.
    }

    \ingredients{
        \SI{300}{\g} & Weizenmehl (Typ 550) \\
        \SI{50}{\g} & blanchierte, gemahlene Mandeln \\
        \SI{100}{\g} & Rohrohrzucker \\
        \SI{1}{\msp} & Vanillepulver \\
        \SI{1}{\msp} & Ingwerpulver \\
        \SI{1}{Prise} & gemahlene Safranfäden\index{Safran} \\
        \nicefrac{1}{4} & Zitrone, abgeriebene Schale \\
        \SI{200}{\g} & vegane Butter, zimmerwarm \\
        \SI{3}{\EL} & Pflanzendrink \\
        \SI{30}{\g} & getrocknete Cranberrys
    }

    \preparation{
        \step Mehl, Mandeln, Zucker, Vanille, Gewürze und Orangenschale in einer Rührschüssel mischen. Mit veganer Butter und Pflanzendrink zu einem glatten Teig verkneten. Den Teig in Frischhaltefolie wickeln und eine Stunde kalt stellen.

        \step Den Backofen auf \SI{180}{\celsius} Ober-/Unterhitze vorheizen.

        \step Den Teig auf einer bemehlten Arbeitsfläche zu langen Teigwürsten formen und gleich große, \ca \SI{10}{\cm} lange Stücke abschneiden. Diese Röllchen an beiden Enden in entgegengesetzter Richtung einrollen, sodass ein ``S'' entsteht. Oben und unten jeweils eine halbe Cranberry in das ``S'' drücken.

        \step Plätzchen auf ein mit Backpapier ausgelegtes Backblech legen und \SI{10}{\minute} bis \SI{12}{\minute} backen. Aus dem Ofen nehmen und vollständig auskühlen lassen.
    }

    % \suggestion[Title of Suggestion]{
	% 	Suggestion
    % }

    % \hint{Hint}

\end{recipeDP}

			\begin{recipe}
    [
        preparationtime = {\SI{30}{\minute}},
        bakingtime = {\SI{15}{\minute}},
        bakingtemperature = {\protect\bakingtemperature{topbottomheat=\SI{160}{\degree}}},
        % portion = {40 Stück},
        source = {@petadeutschland}
    ]
    {Kokosmakronen\index{Kokos!-makronen}}

    % \graph
    %     {
    %         small=Recipes/MainCourses/BBQChicken/Small.jpg,
    %         big=example-image
    %     }

    % \introduction{einleitung}

    \ingredients{
        \SI{70}{\ml} & Hafermilch \\
        \SI{200}{\g} & Puderzucker \\
        \SI{200}{\g} & Kokosraspeln\index{Kokos!-raspel} \\
        \SI{1}{\msp} & Backpulver \\
        \\
        \SI{100}{\g} & Zartbitterschokolade
    }

    \preparation{
        \step Die Hafermilch in eine Schüssel geben. Zucker nach und nach mit einem Schneebesen kräftig unterrühren, bis er sich komplett aufgelöst hat. Kokosraspeln gut mit dem Backpulver verrühren und sehr zügig unterheben.

        \step Den Backofen auf \SI{160}{\degree} vorheizen. Ein Backblech mit Backpapier auslegen und mit Hilfe von zwei Teelöffeln kleine Makronen auf das Blech setzen. Kokosmakronen für \SI{15}{\minute} backen, bis sie leicht braun werden.

        \step Kokosmakronen gut abkühlen lassen, sonst fallen sie auseinander. Kuvertüre klein hacken, über dem Wasserbad schmelzen und die Kokosmakronen damit nach Belieben verzieren.
    }

    % \suggestion[Title of Suggestion]{
	% 	Suggestion
    % }
    %
    % \hint{Hint}

\end{recipe}

	\clearpage
		\sectionwithtoc{Kuchen}
			\begin{recipe}
    [
        preparationtime = {\SI{20}{\minute}},
        bakingtime = {\SI{40}{\minute}},
        bakingtemperature = {\protect\bakingtemperature{fanoven=\SI{160}{\celsius}}},
        portion = {1 Backform 20 x 20 cm},
        source = {Stina Spiegelberg}
    ]
    {\addtoidx{Lebkuchen}-Brownies}

    % \graph
    %     {
    %         small=Recipes/MainCourses/BBQChicken/Small.jpg,
    %         big=example-image
    %     }

    % \introduction{einleitung}

    \ingredients{
        \multicolumn{2}{l}{\textbf{Rührteig}} \\
        \SI{200}{\g} & Marzipanrohmasse \\
        \SI{40}{\ml} & Öl \\
        \SI{250}{\g} & Weizenmehl (Typ 405) \\
        \SI{60}{\g} & Rohrohrzucker \\
        \SI{1}{\pck} & Vanillezucker \\
        \SI{200}{\g} & gemahlene Haselnüsse \\
        \SI{1}{\TL} & Lebkuchengewürz \\
        \SI{2}{\EL} & Kakao \\
        \SI{1}{Prise} & Salz \\
        \SI{1}{\geh\TL} & Hirschhornsalz \\
        \SI{50}{\g} & Orangeat \\
        \SI{50}{\g} & Aprikosenmarmelade \\
        \SI{300}{\ml} & stilles Wasser \\
         & Fett zum Einfetten \\
        \\
        \multicolumn{2}{l}{\textbf{Dekor}} \\
        \SI{60}{\g} & Halbbitterschokolade \\
        \SI{2}{\EL} & Hafer Cuisine \\
        \SI{4}{\EL} & Puderzucker
    }

    \preparation{
        \step Den Backofen auf \SI{160}{\celsius} vorheizen.

        \step Die Marzipanrohmasse mit dem Öl kurz im Mixer verrühren. Mehl, Zucker, Vanillezucker, Haselnüsse, Lebkuchengewürz, Kakao, Salz und Hirschhornsalz in einer Rührschüssel kurz mischen. Das Orangeat fein hacken, mit der Marzipanmasse, der Marmelade und dem Wasser dazugeben und mit dem Schneebesen zu einem glatten Teig rühren. In die gefettete Backform geben und ca. 40 Minuten backen. Mit einem Holzstäbchen die Garprobe machen und abkühlen lassen.

        \step Die Schokolade hacken. Die Pflanzensahne erwärmen, über die Schokolade gießen und so lange rühren, bis die Schokolade vollständig geschmolzen ist. Dann den Puderzucker einsieben. Die Mischung auf dem Kuchen verteilen und auskühlen lassen.
    }

    % \suggestion[Title of Suggestion]{
    %     Suggestion
    % }

    % \hint{Hint}

\end{recipe}

			\begin{recipeDP}
    [
        preparationtime = {\SI{20}{\minute}},
        bakingtime = {\SI{35}{\minute}},
        bakingtemperature = {\protect\bakingtemperature{fanoven=\SI{160}{\celsius}}},
        portion = {12 Stück},
        source = {Dr. Oetker}
    ]
    {\addtoidx{Apfel}-Knusper-Muffins}

    \graph
        {
            small=Gebaeck/Kuchen/Apfel-Knusper-Muffins/small.jpg,
            % big=example-image
        }

    % \introduction{einleitung}

    \ingredients{
        \multicolumn{2}{l}{\textbf{\addtoidx{Streuselteig}}} \\
        \SI{60}{\g} & Margarine \\
        \SI{50}{\g} & kernige Haferflocken \\
        \SI{50}{\g} & Weizenmehl \\
        \SI{60}{\g} & Rohrzucker \\
        \\
        \multicolumn{2}{l}{\textbf{Muffinteig}} \\
        \ca \SI{150}{\g} & Apfel \\
        \SI{1}{\pck} & Backpulver \\
        \SI{1}{\TL} & Zimt \\
        \SI{125}{\g} & Rohrzucker \\
        \SI{125}{\ml} & Sonnenblumenöl \\
        \SI{250}{\ml} & Sojadrink
    }

    \preparation{
        \step Zuerst eine Muffinform mit Papierförmchen vorbereiten und den Backofen auf \SI{160}{\celsius} Umluft vorheizen.

        \step Für den Streuselteig alle Zutaten in eine Rührschüssel geben und mit den Rührstäben eines Handrührgerätes auf niedrigster Stufe zu Streuseln verarbeiten.

        \step Für den Muffinteig den Apfel schälen vierteln und in kleine Würfel schneiden. Mehl und Backpulver in einer Rührschüssel mischen. Übrige Zutaten hinzufügen und alles mit den Rührstäben des Mixers kurz auf niedrigster, dann auf höchster Stufe \ca \SI{2}{\minute} zu einem glatten Teig verarbeiten. Dann die Apfelstückchen unterheben und den Teig mit Hilfe von zwei Esslöffeln in die Muffinförmchen geben.

        \step Die Streusel darauf verteilen und die Form auf dem Rost in den Backofen schieben. Auf mittlerem Einschub etwa \SI{35}{\minute} backen. Anschließend die Muffins mit Hilfe von zwei Gabeln aus der Form nehmen und auf einem Kuchenrost erkalten lassen.
    }

    % \suggestion[Title of Suggestion]{
	% 	Suggestion
    % }
    %
    \hint{Das einfache Rezept lässt sich auch in einer runden Backform backen. Dann unter Umständen abdecken und die Backzeit verlängern. Garprobe machen!}

\end{recipeDP}

			\begin{recipeDP}
    [
        preparationtime = {\SI{30}{\minute}},
        bakingtime = {\SI{40}{\minute}},
        bakingtemperature = {\protect\bakingtemperature{fanoven=\SI{175}{\celsius}}},
        portion = {1 Blech},
        source = {Sabine Schramm}
    ]
    {\addtoidx{Mohnkuchen}}

    % \graph
    %     {
    %         big=TEIL/KAPITEL/REZEPT/big.jpg,
    %         small=TEIL/KAPITEL/REZEPT/small.jpg
    %     }

    % \introduction{
    %     EINLEITUNG
    % }

    \ingredients{
        \multicolumn{2}{l}{\textbf{Teig}} \\
        \SI{600}{\g} & Mehl \\
        \SI{300}{\g} & Margarine \\
        \SI{150}{\g} & Zucker \\
        \SI{6}{\EL} & Sojamilch \\
        \\
        \multicolumn{2}{l}{\textbf{Mohnfüllung}} \\
        \SI{1}{\l} & Sojamilch \\
        \SI{250}{\g} & Backmohn \\
        \SI{100}{\g} & \addtoidx{Mohn} \\
        \SI{160}{\g} & \addtoidx{Dinkelgrieß} \\
        \SI{100}{\g} & Zucker \\
        \SI{100}{\g} & Butter
    }

    \preparation{
        \step Die trockenen Zutaten des Teigs vermengen, dann mit der Butter und der Sojamilch zu Streuseln verarbeiten. Eine Hälfte der Streusel auf einem mit Backpapier vorbereiteten Backblech verteilen. Die Streusel zu einem Boden festdrücken. Die andere Hälfte der Streusel beiseite stellen.
        \step Für die Füllung Sojamilch und Zucker aufkochen, dann den Grieß einrühren und alles weitere \SI{2}{\minute} köcheln lassen. Den Ofen auf \SI{175}{\celsius} vorheizen.
        \step Anschließend den Mohn und den Backmohn dazu geben und gut unterrühren. Die Butter einrühren und die Füllung \SI{5}{\minute} stehen lassen zum abkühlen.
        \step Die fertige, noch heiße Füllung gleichmäßig auf dem Boden verteilen. Die Streusel darüber verteilen und den Kuchen im vorgeheizten Backofen bei \SI{175}{\celsius} etwa \SI{40}{\minute} backen.
    }

    % \suggestion[TITEL EINES VORSCHLAGS]{
	% 	VORSCHLAG (DURCH HORIZONTALE LINIE VOM REZEPT GETRENNT)
    % }

    \hint{
        Für eine \SI{24}{\cm}-Springform die Angaben im Rezept einmal halbieren.
    }

\end{recipeDP}



	\clearpage
	\cleardoublepage
	\part{Getränke}
		\sectionwithtoc{Alkoholische Getränke}
			\begin{recipe}
    [
        preparationtime = {\SI{5}{\minute}},
        % bakingtime = {\SI{12}{\minute} bis \SI{15}{\minute}},
        % bakingtemperature = {\protect\bakingtemperature{fanoven=\SI{180}{\degree}}},
        portion = {1 Portion},
        source = {Malibu}
    ]
    {\addtoidx{Malibu} Pina Colada}

    % \graph
    %     {
    %         small=Recipes/MainCourses/BBQChicken/Small.jpg,
    %         big=example-image
    %     }

    % \introduction{einleitung}

    \ingredients{
        \SI{100}{\ml} & Malibu \\
        \SI{50}{\ml} & \addtoidx{Kokosnuss}-Creme \\
        \SI{50}{\ml} & \addtoidx{Ananassaft}
    }

    \preparation{
        \step Fülle einen Shaker mit Eiswürfeln. Gib Malibu und Kokosnuss Creme dazu. Mixe anschließend alles gut durch und seihe es in ein Highball Glas ab. Mit etwas Ananas garnieren.
    }

    % \suggestion[Title of Suggestion]{
	% 	Suggestion
    % }
    %
    % \hint{Hint}

\end{recipe}

			\begin{recipeDP}
    [
        preparationtime = {\SI{5}{\minute}},
        % bakingtime = {\SI{12}{\minute} bis \SI{15}{\minute}},
        % bakingtemperature = {\protect\bakingtemperature{fanoven=\SI{180}{\degree}}},
        portion = {1 Portion},
        source = {Malibu}
    ]
    {\addtoidx{Malibu} Sun Splash}

    % \graph
    %     {
    %         small=Recipes/MainCourses/BBQChicken/Small.jpg,
    %         big=example-image
    %     }

    % \introduction{einleitung}

    \ingredients{
        \SI{50}{\ml} & Malibu \\
        \SI{50}{\ml} & \addtoidx{Ananassaft} \\
        \SI{50}{\ml} & \addtoidx{Grapefruitsaft}
    }

    \preparation{
        \step Fülle die gekühlten Zutaten mit Eiswürfeln in einem Glas und verpasst dem Drink mit einer Scheibe Ananas und einer Spalte Grapefruit den letzten Schliff.
    }

    % \suggestion[Title of Suggestion]{
	% 	Suggestion
    % }
    %
    % \hint{Hint}

\end{recipeDP}

	\clearpage
		\sectionwithtoc{Nicht-alkoholische Getränke}


	\clearpage
	\pagestyle{plain}
	\addcontentsline{toc}{part}{Stichwortverzeichnis}
	\indexprologue{Insgesamt\:\therecipeCntr\:Rezepte enthalten die folgende Begriffe. Es werden nur markante Zutaten oder Schlagworte referenziert, die Entscheidung, welche Inhalte referenziert werden, ist dabei eher subjektiv; erfolgt aber nach Maßgaben der Relevanz und der Interessanz.}
	\printindex

\end{document}
