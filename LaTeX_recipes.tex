%%%%%%%%%%%%%%%%%%%%%%%%%%%%%%%%%%%%%%%%%%%%%%%%%%%%%%%%%%%%%%%%%%%%%%%
%% Diese tex-Datei kompilieren, Vorgaben können so übernommen werden %%
%% Diese tex-Datei anpassen:                                         %%
%% 1. Kopf- und Fußzeile definieren                                  %%
%% 2. Counter anpassen: Überschrift, Seite, Gleichung                %%
%% 3. Kapitel (tex-Dateien) einfügen mit \input{}                    %%
%%(4. neue Kommandos können zu den bestehenden in der Präambel hinzu-%%
%%    gefügt werden)                                                 %%
%%%%%%%%%%%%%%%%%%%%%%%%%%%%%%%%%%%%%%%%%%%%%%%%%%%%%%%%%%%%%%%%%%%%%%%




%%%%%%%%%%%%%%%%%%%%%%%%%%%%%%%%%%%%%%%%%%%%%%%%%%%%%%%%%%%%%%%%%%%%%%%
%%%%%%%%%%%%%%%%%%%%%%%%%%%%%%%%%%%%%%%%%%%%%%%%%%%%%%%%%%%%%%%%%%%%%%%
%% Präambel                                                          %%
%%%%%%%%%%%%%%%%%%%%%%%%%%%%%%%%%%%%%%%%%%%%%%%%%%%%%%%%%%%%%%%%%%%%%%%
%%%%%%%%%%%%%%%%%%%%%%%%%%%%%%%%%%%%%%%%%%%%%%%%%%%%%%%%%%%%%%%%%%%%%%%

% !TeX root = LaTeX_recipes.tex

\documentclass[
a4paper,
11pt,
twoside,
]{scrartcl}

% \usepackage{atbegshi}% http://ctan.org/pkg/atbegshi
% \AtBeginDocument{\AtBeginShipoutNext{\AtBeginShipoutDiscard}}

% \newcommand*\cleartoleftpage{%
%   \clearpage
%   \ifodd\value{page}\hbox{}\newpage\fi
% }


\usepackage[T1]{fontenc}
\usepackage[utf8]{inputenc}
\usepackage{lmodern}
\usepackage[ngerman]{babel}

\usepackage{textcomp}
\usepackage{gensymb}

\usepackage{paracol}
\usepackage{supertabular}
% \usepackage{lipsum}

\usepackage{amssymb}                                % mathematische Symbole importieren
\usepackage{amsmath}                                % Mathematikumgebungen (align) importieren
\usepackage{nicefrac}
\usepackage{siunitx}[=v2]

\newcommand{\ca}{\acs{ca}\,}
\newcommand{\pck}{\acs{Pck}\,}
\newcommand{\msp}{\acs{Msp}\,}
\newcommand{\geh}{\acs{geh}\,}
\newcommand{\Ta}{\acs{Ta}\,}
\newcommand{\TL}{\acs{TL}}
\newcommand{\EL}{\acs{EL}}
\newcommand{\dmesser}{\(\varnothing\)}

\usepackage{xcolor}

%% Hyperlinks setzen, PDF bearbeiten %%%%%%%%%%%%%%%%%%%%%%%%%%%%%%%%%%
\usepackage[
		bookmarks=true,
		% bookmarksopen=true,							% Kapitelmarken in Acrobat anzeigen
		% bookmarksopenlevel=1,						% nu oberste Ebene der Kapitel anzeigen
		pagebackref=true,							% Links: aus Literatur in Text
		pdfpagelayout=TwoColumnRight,				% Anzeige: scollen und zweiseitigs
		breaklinks, 								% Allows link text to break across lines; This makes links on multiple lines into different PDF links to the same target
		hidelinks,									% Farbe und Rahmen der Links entfernen
		linktoc=all,									% Text, Seitenzahlen in TOC, LOF, LOT
		% Fit,										% Seite an Fenster anpassen
	]{hyperref}
\hypersetup{										% Einstellungen zum pdf
	%pdftoolbar=false,							% Acrobat toolbar aus
	% pdfmenubar=false,							% Acrobat menu aus
	pdftitle={Vegane Gerichte und Getränke},
	pdfsubject={Rezeptsammlung},
	pdfauthor={arwsl},
}

\usepackage{todonotes}

\usepackage[printonlyused]{acronym}					% Abkürzungsverzeichnis
\makeatletter										% \acs-Befehl neu definieren: \mbox entfernt, sodass Umbrüche möglich sind
\renewcommand*\AC@acs[1]{%
    \expandafter\AC@get\csname fn@#1\endcsname\@firstoftwo{#1}}
\makeatother


\usepackage[ngerman]{minitoc} % kleine TOC einfügen
\mtcsetfeature{secttoc}{after}{\clearpage}
% \mtcsetfeature{minitoc}{pagestyle}{\thispagestyle{empty}}


\usepackage{imakeidx}
\AtBeginDocument{\renewcommand{\indexname}{Stichwortverzeichnis\vspace{1cm}}}
\renewcommand{\seename}{siehe}
\renewcommand{\alsoname}{siehe auch}
\newcommand{\addtoidx}[1]{#1\index{#1}}
\newcommand{\addsubidx}[2]{#1\index{#1!#2}}
\makeindex[columns=2]


%%% Farbendefinitionen                                                %%
\definecolor{ostfaliaBlau}{HTML}{003A79}
\definecolor{PTBBlau}{HTML}{009CCF}
\definecolor{PTBRot}{HTML}{CF3300}

\definecolor{petrol}{HTML}{11697A}
\definecolor{sundown}{HTML}{DB6400}
\definecolor{sand}{HTML}{FFA62B}
\definecolor{ivory}{HTML}{F8F1F1}


\usepackage[nowarnings]{xcookybooky}											   % Rezepte typesetten

\setRecipeColors{
	recipename = petrol,
	ing = black,
	inghead = sundown,
	prep = black,
	prephead = sundown,
	suggestion = black,
	suggestionhead = petrol,
	separationgraph = petrol,
	hint = sand,
	hinthead = sundown,
	hintline = sand,
	numeration = sundown,
}

\setRecipeLengths{
	pictureheight = 0.2\textheight,
	bigpicturewidth = 0.55\textwidth,
	smallpicturewidth = 0.35\textwidth,
	introductionwidth = 0.9\textwidth,
	preparationwidth = 0.55\textwidth,
	ingredientswidth = 0.35\textwidth,
}

\setRecipeSizes{
	recipename = \Huge,
	intro = \normalsize,
	ing = \normalsize,
	inghead = \Large,
	prephead = \Large,
	suggestion = \normalsize,
	hint = \large,
	hinthead = \Large,
}

% \usepackage{emerald}
% \setRecipenameFont{fwb}{T1}{m}{n}

\setRecipenameFont{\sfdefault}{T1}{b}{n}

\setHeadlines{
	inghead = Zutaten,
	prephead = Zubereitung,
	hinthead = Anmerkungen,
	continuationhead = { },
	continuationfoot = { },
	portionvalue = Portion(en),
	calory = Energie,
}


% updated package implementations
\renewcommand{\step}
{%
    \stepcounter{step}%
    \lettrine
    [%
        lines=2,
        lhang=0,          % space into margin, value between 0 and 1
        loversize=0.1,   % enlarges the height of the capital
        slope=0em,
        findent=0.75em,      % gap between capital and intended text
        nindent=0em       % shifts all intended lines, begining with the second line
    ]{\thestep}{}%
}

% lange Zutatenliste:
% https://stackoverflow.com/questions/64688876/using-xcookybooky-in-latex-i-have-many-ingredients-but-it-does-go-into-the-nex



%%%%%%%%%%%%%%%%%%%%%%%%%%%%%%%%%%%%%%%%%%%%%%%%%%%%%%%%%%%%%%%%%%%%%%%
%% Kopf- und Fußzeile                                                %%
%%%%%%%%%%%%%%%%%%%%%%%%%%%%%%%%%%%%%%%%%%%%%%%%%%%%%%%%%%%%%%%%%%%%%%%
% \usepackage[automark]{scrlayer-scrpage}									% eigene Kopf-/Fusszeile
%% Kopfzeile: links auf ungeraden, rechts auf geraden Seiten:
%\ihead{}
%% dazu: immer zentriert (Labor und Versuch):
%\chead{}
%% dazu: rechts auf ungeraden, links auf geraden Seiten
%\ohead{}
%% Fußzeile: links auf ungeraden, rechts auf geraden Seiten:
%\ifoot{}
%% dazu: rechts auf ungeraden, links auf geraden Seiten:
%\ofoot{\pagemark}


%%%%%%%%%%%%%%%%%%%%%%%%%%%%%%%%%%%%%%%%%%%%%%%%%%%%%%%%%%%%%%%%%%%%%%%%
%%% Counter setzen                                                    %%
%%%%%%%%%%%%%%%%%%%%%%%%%%%%%%%%%%%%%%%%%%%%%%%%%%%%%%%%%%%%%%%%%%%%%%%%
%\setcounter{section}{1}												% starte mit anderer Überschrift-Nr. (+1)
\setcounter{page}{0}													% starte mit anderer Seitenzahl
%\setcounter{equation}{0}												% starte mit anderer Gleichungsnummer

\setcounter{secnumdepth}{1}
\renewcommand{\subsectionmark}[1]
{
}

% \renewcommand{\section}[1]{
% 	\addtocounter{section}{1}
% 	\clearpage
% 	\pagestyle{empty}
% 	\vspace{0.4\textheight}
% 	{\Huge #1} %\section{#1}
% 	\clearpage
% 	\pagestyle{fancy}
% }


%%%%%%%%%%%%%%%%%%%%%%%%%%%%%%%%%%%%%%%%%%%%%%%%%%%%%%%%%%%%%%%%%%%%%%%%
%%%%%%%%%%%%%%%%%%%%%%%%%%%%%%%%%%%%%%%%%%%%%%%%%%%%%%%%%%%%%%%%%%%%%%%%
%%% Dokument                                                          %%
%%%%%%%%%%%%%%%%%%%%%%%%%%%%%%%%%%%%%%%%%%%%%%%%%%%%%%%%%%%%%%%%%%%%%%%%
%%%%%%%%%%%%%%%%%%%%%%%%%%%%%%%%%%%%%%%%%%%%%%%%%%%%%%%%%%%%%%%%%%%%%%%%
\begin{document}

	% \titlehead{\includegraphics[width=0.6\columnwidth,right]{ostfalia_logo_rechts}}

\subject{\vspace{2cm}
	\large Rezeptsammlung
	\vspace{1cm}
	}
\title{Vegane Gerichte und Getränke}
\subtitle{Eine Sammlung der besten Ideen}

\author{\vspace{4cm}\\
	arwsl\\
	\vspace{2cm}}

\date{seit 2021}

% \publishers{\vspace{2cm}}



\maketitle
\thispagestyle{empty}
\cleardoublepage


	\cleardoubleemptypage

\thispagestyle{empty}

\section*{Vorwort}
\label{sec:Vorwort}

\todo{Vorwort einfügen?}

\begin{flushright}
	Wolfenbüttel, Januar 2021
\end{flushright}



\cleardoubleemptypage

	\pagestyle{plain}					% keine Kopf- und Fußzeilen ab hier (am Ende der tex wieder aufgehoben)
\pagenumbering{Roman}							% römische Seitennummerierung
\setcounter{page}{2}

\doparttoc
\dosecttoc
\tableofcontents								% Inhaltsverzeichnis

\vspace{2cm}
\addcontentsline{toc}{section}{Abkürzungsverzeichnis} % manuell ins Inhaltsverzeichins fügen
\section*{Abkürzungsverzeichnis}
\begin{acronym}[MMMMMMM]
	\acro{ca}[ca.]{cirka}
	\acro{EL}{Esslöffel}
	\acro{geh}[geh.]{gehäuft}
	\acro{Msp}[Msp.]{Messerspitze}
	\acro{Pck}[Pck.]{Päckchen}
	\acro{Ta}[Ta.]{Tasse}
	\acro{TL}{Teelöffel}
\end{acronym}


\cleardoublepage								% neue (rechte) Seite
\setcounter{page}{0}
\pagenumbering{arabic}							% arabische (normale) Seitenzahlen

\renewcommand\thesection{\arabic{section}}
\renewcommand\thesubsection{\thesection.\arabic{subsection}}

\pagestyle{fancy}							%Kopf- und Fußzeile wird ab hier wieder angezeigt


	% \section{Aufläufe}

\begin{recipe}
[%
    preparationtime = {\unit[1]{h}},
    bakingtime={\unit[20-25]{min}},
    bakingtemperature={\protect\bakingtemperature{
        fanoven=\unit[175]{\textcelcius}}},
    portion = {\portion{6}},
	calory = 20 kcal,
]
{Spinach Lasagne}

\introduction{einleitung \lipsum[2]}
\suggestion[Titel des Vorschlags]{vorschlag ...}
\hint{Hinweis...}
% \graph
%     {
%         %small=Recipes/MainCourses/BBQChicken/Small.jpg,
%         big=example-image
%     }

\ingredients
    {
        \textbf{Sauce}\\
        2 & chopped onions\\
        \unit[2]{tbsp} & Olive oil\\
        3 & chopped cloves of garlic\\
        \unit[2]{tsps} & dried thyme\\
        4 & grated carrots\\
        3 & grated celery leaves\\
        1 & grated squash\\
        1 & grated aubergine\\
        \unit[250]{g} & mushrooms\\
        \unit[1]{dL} & tomato paste\\
        \unit[1]{can} & chopped tomatoes\\
        To taste & salt and pepper\\
        \\
        \textbf{Spinach}\\
        \unit[500]{g} & frozen spinach\\
        \unit[1]{dL} & cream\\
        1 & onion\\
        1 & clove of garlic\\
        \unit[2]{tbsp} & oil\\
        To taste & salt \& pepper \& fresh basil\\
        % \\
        % \textbf{Mornay}\\
        % \unit[2]{tbsp} & butter\\
        % \unit[2]{tbsp} & flour\\
        % \unit[3]{dL} & milk\\
        % \unit[125]{g} & fresh mozzarella\\
        % To taste & salt \& pepper \& nutmeg\\
        % \\
        % \unit[250]{g} & Lasagne plates\\
        % Enough & mozzarella\\
    }

    \preparation{
      \step
	   test und andere sachen

      \step
	  \lipsum[5]

      \step
	  \lipsum[4]

      \step
	  \lipsum[2]

      \step \lipsum[6]

      \step \lipsum[2]
    }

\end{recipe}






\end{document}
