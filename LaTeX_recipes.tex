%%%%%%%%%%%%%%%%%%%%%%%%%%%%%%%%%%%%%%%%%%%%%%%%%%%%%%%%%%%%%%%%%%%%%%%
%% Diese tex-Datei kompilieren, Vorgaben können so übernommen werden %%
%% Diese tex-Datei anpassen:                                         %%
%% 1. Kopf- und Fußzeile definieren                                  %%
%% 2. Counter anpassen: Überschrift, Seite, Gleichung                %%
%% 3. Kapitel (tex-Dateien) einfügen mit \input{}                    %%
%%(4. neue Kommandos können zu den bestehenden in der Präambel hinzu-%%
%%    gefügt werden)                                                 %%
%%%%%%%%%%%%%%%%%%%%%%%%%%%%%%%%%%%%%%%%%%%%%%%%%%%%%%%%%%%%%%%%%%%%%%%




%%%%%%%%%%%%%%%%%%%%%%%%%%%%%%%%%%%%%%%%%%%%%%%%%%%%%%%%%%%%%%%%%%%%%%%
%%%%%%%%%%%%%%%%%%%%%%%%%%%%%%%%%%%%%%%%%%%%%%%%%%%%%%%%%%%%%%%%%%%%%%%
%% Präambel                                                          %%
%%%%%%%%%%%%%%%%%%%%%%%%%%%%%%%%%%%%%%%%%%%%%%%%%%%%%%%%%%%%%%%%%%%%%%%
%%%%%%%%%%%%%%%%%%%%%%%%%%%%%%%%%%%%%%%%%%%%%%%%%%%%%%%%%%%%%%%%%%%%%%%

% !TeX root = LaTeX_recipes.tex

\documentclass[
a4paper,
11pt,
twoside,
]{scrartcl}

% \usepackage{atbegshi}% http://ctan.org/pkg/atbegshi
% \AtBeginDocument{\AtBeginShipoutNext{\AtBeginShipoutDiscard}}

% \newcommand*\cleartoleftpage{%
%   \clearpage
%   \ifodd\value{page}\hbox{}\newpage\fi
% }


\usepackage[T1]{fontenc}
\usepackage[utf8]{inputenc}
\usepackage{lmodern}
\usepackage[ngerman]{babel}

\usepackage{textcomp}
\usepackage{gensymb}

\usepackage{paracol}
\usepackage{supertabular}
% \usepackage{lipsum}

\usepackage{amssymb}                                % mathematische Symbole importieren
\usepackage{amsmath}                                % Mathematikumgebungen (align) importieren
\usepackage{nicefrac}
\usepackage{siunitx}[=v2]

\newcommand{\ca}{\acs{ca}\,}
\newcommand{\pck}{\acs{Pck}\,}
\newcommand{\msp}{\acs{Msp}\,}
\newcommand{\geh}{\acs{geh}\,}
\newcommand{\Ta}{\acs{Ta}\,}
\newcommand{\TL}{\acs{TL}}
\newcommand{\EL}{\acs{EL}}
\newcommand{\dmesser}{\(\varnothing\)}

\usepackage{xcolor}

%% Hyperlinks setzen, PDF bearbeiten %%%%%%%%%%%%%%%%%%%%%%%%%%%%%%%%%%
\usepackage[
		bookmarks=true,
		% bookmarksopen=true,							% Kapitelmarken in Acrobat anzeigen
		% bookmarksopenlevel=1,						% nu oberste Ebene der Kapitel anzeigen
		pagebackref=true,							% Links: aus Literatur in Text
		pdfpagelayout=TwoColumnRight,				% Anzeige: scollen und zweiseitigs
		breaklinks, 								% Allows link text to break across lines; This makes links on multiple lines into different PDF links to the same target
		hidelinks,									% Farbe und Rahmen der Links entfernen
		linktoc=all,									% Text, Seitenzahlen in TOC, LOF, LOT
		% Fit,										% Seite an Fenster anpassen
	]{hyperref}
\hypersetup{										% Einstellungen zum pdf
	%pdftoolbar=false,							% Acrobat toolbar aus
	% pdfmenubar=false,							% Acrobat menu aus
	pdftitle={Vegane Gerichte und Getränke},
	pdfsubject={Rezeptsammlung},
	pdfauthor={arwsl},
}

\usepackage{todonotes}

\usepackage[printonlyused]{acronym}					% Abkürzungsverzeichnis
\makeatletter										% \acs-Befehl neu definieren: \mbox entfernt, sodass Umbrüche möglich sind
\renewcommand*\AC@acs[1]{%
    \expandafter\AC@get\csname fn@#1\endcsname\@firstoftwo{#1}}
\makeatother


\usepackage[ngerman]{minitoc} % kleine TOC einfügen
\mtcsetfeature{secttoc}{after}{\clearpage}
% \mtcsetfeature{minitoc}{pagestyle}{\thispagestyle{empty}}


\usepackage{imakeidx}
\AtBeginDocument{\renewcommand{\indexname}{Stichwortverzeichnis\vspace{1cm}}}
\renewcommand{\seename}{siehe}
\renewcommand{\alsoname}{siehe auch}
\newcommand{\addtoidx}[1]{#1\index{#1}}
\newcommand{\addsubidx}[2]{#1\index{#1!#2}}
\makeindex[columns=2]


%%% Farbendefinitionen                                                %%
\definecolor{ostfaliaBlau}{HTML}{003A79}
\definecolor{PTBBlau}{HTML}{009CCF}
\definecolor{PTBRot}{HTML}{CF3300}

\definecolor{petrol}{HTML}{11697A}
\definecolor{sundown}{HTML}{DB6400}
\definecolor{sand}{HTML}{FFA62B}
\definecolor{ivory}{HTML}{F8F1F1}


\usepackage[nowarnings]{xcookybooky}											   % Rezepte typesetten

\setRecipeColors{
	recipename = petrol,
	ing = black,
	inghead = sundown,
	prep = black,
	prephead = sundown,
	suggestion = black,
	suggestionhead = petrol,
	separationgraph = petrol,
	hint = sand,
	hinthead = sundown,
	hintline = sundown,
	numeration = sundown,
}

\setRecipeLengths{
	pictureheight = 0.2\textheight,
	bigpicturewidth = 0.55\textwidth, % ratio: pw/ph = 11/4
	smallpicturewidth = 0.35\textwidth, % ratio: pw/ph = 7/4
	introductionwidth = 0.75\textwidth,
	preparationwidth = 0.55\textwidth,
	ingredientswidth = 0.35\textwidth,
}

\setRecipeSizes{
	recipename = \Huge,
	intro = \normalsize,
	ing = \normalsize,
	inghead = \Large,
	prephead = \Large,
	suggestion = \normalsize,
	hint = \large,
	hinthead = \Large,
}

% \usepackage{emerald}
% \setRecipenameFont{fwb}{T1}{m}{n}

\setRecipenameFont{\sfdefault}{T1}{b}{n}

\setHeadlines{
	inghead = Zutaten,
	prephead = Zubereitung,
	hinthead = Anmerkungen,
	continuationhead = { },
	continuationfoot = { },
	portionvalue = Portion(en),
	calory = Energie,
}


% updated package implementations
\renewcommand{\step}
{%
    \stepcounter{step}%
    \lettrine
    [%
        lines=2,
        lhang=0,          % space into margin, value between 0 and 1
        loversize=0.1,   % enlarges the height of the capital
        slope=0em,
        findent=0.75em,      % gap between capital and intended text
        nindent=0em       % shifts all intended lines, begining with the second line
    ]{\thestep}{}%
}

\renewcommand{\sectionmark}[1]
{
	\markright{\MakeUppercase{\thechapter.\ #1}}
}

% lange Zutatenliste:
% https://stackoverflow.com/questions/64688876/using-xcookybooky-in-latex-i-have-many-ingredients-but-it-does-go-into-the-nex



%%%%%%%%%%%%%%%%%%%%%%%%%%%%%%%%%%%%%%%%%%%%%%%%%%%%%%%%%%%%%%%%%%%%%%%
%% Kopf- und Fußzeile                                                %%
%%%%%%%%%%%%%%%%%%%%%%%%%%%%%%%%%%%%%%%%%%%%%%%%%%%%%%%%%%%%%%%%%%%%%%%
% \usepackage[automark]{scrlayer-scrpage}									% eigene Kopf-/Fusszeile
%% Kopfzeile: links auf ungeraden, rechts auf geraden Seiten:
%\ihead{}
%% dazu: immer zentriert (Labor und Versuch):
%\chead{}
%% dazu: rechts auf ungeraden, links auf geraden Seiten
%\ohead{}
%% Fußzeile: links auf ungeraden, rechts auf geraden Seiten:
%\ifoot{}
%% dazu: rechts auf ungeraden, links auf geraden Seiten:
%\ofoot{\pagemark}


%%%%%%%%%%%%%%%%%%%%%%%%%%%%%%%%%%%%%%%%%%%%%%%%%%%%%%%%%%%%%%%%%%%%%%%%
%%% Counter setzen                                                    %%
%%%%%%%%%%%%%%%%%%%%%%%%%%%%%%%%%%%%%%%%%%%%%%%%%%%%%%%%%%%%%%%%%%%%%%%%
%\setcounter{section}{1}												% starte mit anderer Überschrift-Nr. (+1)
\setcounter{page}{0}													% starte mit anderer Seitenzahl
%\setcounter{equation}{0}												% starte mit anderer Gleichungsnummer

\setcounter{secnumdepth}{0}
\setcounter{tocdepth}{3}
\renewcommand{\recipesection}[2][]
{
	\section[#1]{#2}
}

\newcommand{\chapterwithtoc}[1]{
	\chapter{#1}
	\minitoc
}
% \mtcselectlanguage


%%%%%%%%%%%%%%%%%%%%%%%%%%%%%%%%%%%%%%%%%%%%%%%%%%%%%%%%%%%%%%%%%%%%%%%%
%%%%%%%%%%%%%%%%%%%%%%%%%%%%%%%%%%%%%%%%%%%%%%%%%%%%%%%%%%%%%%%%%%%%%%%%
%%% Dokument                                                          %%
%%%%%%%%%%%%%%%%%%%%%%%%%%%%%%%%%%%%%%%%%%%%%%%%%%%%%%%%%%%%%%%%%%%%%%%%
%%%%%%%%%%%%%%%%%%%%%%%%%%%%%%%%%%%%%%%%%%%%%%%%%%%%%%%%%%%%%%%%%%%%%%%%
\begin{document}

	% \titlehead{\includegraphics[width=0.6\columnwidth,right]{ostfalia_logo_rechts}}

\subject{\vspace{2cm}
	\large Rezeptsammlung
	\vspace{1cm}
	}
\title{Vegane Gerichte und Getränke}
\subtitle{Eine Sammlung der besten Ideen}

\author{\vspace{4cm}\\
	arwsl\\
	\vspace{2cm}}

\date{seit 2021}

% \publishers{\vspace{2cm}}



\maketitle
\thispagestyle{empty}
\cleardoublepage


	\cleardoubleemptypage

\thispagestyle{empty}

\section*{Vorwort}
\label{sec:Vorwort}

\todo{Vorwort einfügen?}

\begin{flushright}
	Wolfenbüttel, Januar 2021
\end{flushright}



\cleardoubleemptypage

	\pagestyle{plain}					% keine Kopf- und Fußzeilen ab hier (am Ende der tex wieder aufgehoben)
\pagenumbering{Roman}							% römische Seitennummerierung
\setcounter{page}{2}

\doparttoc
\dosecttoc
\tableofcontents								% Inhaltsverzeichnis

\vspace{2cm}
\addcontentsline{toc}{section}{Abkürzungsverzeichnis} % manuell ins Inhaltsverzeichins fügen
\section*{Abkürzungsverzeichnis}
\begin{acronym}[MMMMMMM]
	\acro{ca}[ca.]{cirka}
	\acro{EL}{Esslöffel}
	\acro{geh}[geh.]{gehäuft}
	\acro{Msp}[Msp.]{Messerspitze}
	\acro{Pck}[Pck.]{Päckchen}
	\acro{Ta}[Ta.]{Tasse}
	\acro{TL}{Teelöffel}
\end{acronym}


\cleardoublepage								% neue (rechte) Seite
\setcounter{page}{0}
\pagenumbering{arabic}							% arabische (normale) Seitenzahlen

\renewcommand\thesection{\arabic{section}}
\renewcommand\thesubsection{\thesection.\arabic{subsection}}

\pagestyle{fancy}							%Kopf- und Fußzeile wird ab hier wieder angezeigt



	\part{Salate}

	\part{Hauptgerichte}
		\chapterwithtoc{Eintopf-Gerichte}
			\begin{recipeDP}
    [
        preparationtime = {\SI{25}{\minute}},
        % bakingtime = {\SI{12}{\minute} bis \SI{15}{\minute}},
        % bakingtemperature = {\protect\bakingtemperature{fanoven=\SI{180}{\degree}}},
        portion = {4 Portionen},
        source = {chefkoch.de}
    ]
    {Bunter Kichererbseneintopf}

    \graph
        {
            % small=Recipes/MainCourses/BBQChicken/Small.jpg,
            big=Hauptgerichte/Eintoepfe/Bunter_Kichererbseneintopf/big.jpg
        }

    % \introduction{einleitung}

    \ingredients{
        \SI{1}{Stange} & Lauch, groß \\
        2 & Paprikaschoten, gelb, rot \\
        2 & Knoblauchzehen \\
        \SI{800}{\g} & Dosentomaten \\
        \SI{2}{\EL} & Tomatenmark \\
        \SI{400}{\g} & \addtoidx{Kichererbsen}, aus der Dose \\
        \SI{1}{\l} & Gemüsebrühe \\
        \SI{225}{\g} & Blattspinat \\
        \SI{1}{\TL} & Paprikapulver, scharf \\
        \SI{1}{\TL} & Paprikapulver, mild \\
         & Salz \\
         & Pfeffer
    }

    \preparation{
        \step Das Gemüse waschen, den Lauch in dünne Ringe schneiden, die Paprikaschoten würfeln. Die Kichererbsen gut abspülen, abtropfen lassen.

        \step Das Olivenöl in einem großen Topf erhitzen, Lauchscheiben und Paprikawürfel darin unter Rühren \ca \SI{5}{\minute} scharf anbraten. Die Knoblauchzehen dazu pressen, das Tomatenmark dazu geben und beides kurz mit braten. Mit den etwas zerkleinerten Tomaten und der Gemüsebrühe auffüllen und die Kichererbsen sowie das Paprikapulver zufügen.

        \step Alles aufkochen und dann bei kleinerer Hitze ca. 5 Minuten kochen lassen. Den Blattspinat etwas zerkleinern, in die Suppe geben und nochmal \ca \SI{3}{\minute} kochen lassen. Alles mit Salz und Pfeffer abschmecken.
    }

    % \suggestion[Title of Suggestion]{
	% 	Suggestion
    % }
    %
    % \hint{Hint}

\end{recipeDP}

		\chapterwithtoc{Nudel-Gerichte}
			\begin{recipeDP}
    [
        preparationtime = {\SI{10}{\minute}},
        % bakingtime = {\SI{12}{\minute} bis \SI{15}{\minute}},
        % bakingtemperature = {\protect\bakingtemperature{fanoven=\SI{180}{\degree}}},
        portion = {1 Portion},
        source = {@fitgreenmind}
    ]
    {Lazy Noodles}

    % \graph
    %     {
    %         small=Recipes/MainCourses/BBQChicken/Small.jpg,
    %         big=example-image
    %     }

    % \introduction{einleitung}

    \ingredients{
        \textbf{Soße}\\
        \SI{125}{\ml} & Wasser \\
        \SI{4}{\EL} & Soja Soße \\
        \SI{2}{\EL} & Süße (z.B. Ahornsirup) \\
        \SI{1}{\EL} & Weißweinessig oder Zitronensaft \\
        \SI[parse-numbers = false]{\nicefrac{1}{2}}{\EL} & Tahin \\
        \SI[parse-numbers = false]{\nicefrac{1}{2}}{\TL} & Chiliflocken \\
        \SI{2}{\EL} & Maisstärke \\
        \\
        \textbf{sonst} \\
        \SI{250}{\g} & gekochte \addtoidx{Nudeln} \\
         & Petersilie
    }

    \preparation{
        \step Alle Zutaten für die Soße in einem Mixer miteinander verquirlen, bis eine homogene Soße entsteht.

        \step Die fertige in einer Pfanne erhitzen, wenn sie anfängt anzudicken, die Nudeln hinzugeben. Verrühren bis alles mit dickflüssiger Soße bedeckt ist. Dann mit Petersilie anrichten.
    }

    % \suggestion[Title of Suggestion]{
	% 	Suggestion
    % }

    % \hint{Hint}

\end{recipeDP}


	\part{Nachspeisen}

	\part{Gebäck}
		\chapterwithtoc{Brote}
			\begin{recipeDP}
    [
        preparationtime = {\SI{10}{\minute}},
        bakingtime = {\SI{40}{\minute}},
        bakingtemperature = {\protect\bakingtemperature{fanoven=\SI{200}{\celsius}}},
        portion = {1 Kastenform},
        source = {bbc goodfood}
    ]
    {Bananenbrot}

    % \graph
    %     {
    %         small=Recipes/MainCourses/BBQChicken/Small.jpg,
    %         big=example-image
    %     }

    % \introduction{einleitung}

    \ingredients{
        3 & große \addtoidx{Bananen} \\
        \SI{75}{\ml} & Öl \\
        \SI{100}{\g} & brauner Zucker \\
        \SI{225}{\g} & Weizenmehl \\
        \SI{3}{\geh \TL} & Backpulver \\
        \SI{3}{\TL} & Zimt oder Gewürzmischung\\
         & getrocknete Früchte
    }

    \preparation{
        \step Sen Ofen auf \SI{200}{\celsius} vorheizen. Die Bananen mit einer Gabel zerstampfen und dann Öl und den Zucker hinzufügen und verrühren.

        \step Das Weizenmehl, Backpulver und den Zimt oder die Gewürze hinzufügen und gut miteinander verrühren. An dieser Stelle können optional getrocknete Früchte unter gerührt werden.

        \step Dann den Teig in eine Kastenform füllen. Für \SI{20}{\minute} backen und dann nachsehen, ob das Bananenbrot mit Folie abgedeckt werden muss. Weitere \SI{20}{\minute} backen, bis die Garprobe erfolgreich ist. Etwas abkühlen lassen vor dem Servieren.
    }

    % \suggestion[Title of Suggestion]{
	% 	Suggestion
    % }

    \hint{Gut dazu passt: Erdnussbutter}

\end{recipeDP}

			\begin{recipeDP}
    [
        preparationtime = {\SI{25}{\minute}},
        bakingtime = {\SI{40}{\minute} bis \SI{50}{\minute}},
        bakingtemperature = {\protect\bakingtemperature{fanoven=\SI{175}{\degree}}},
        portion = {1 Baguette},
        source = {Sabine Schramm}
    ]
    {Mediterranes \addtoidx{Baguette}}

    % \graph
    %     {
    %         small=Recipes/MainCourses/BBQChicken/Small.jpg,
    %         big=example-image
    %     }

    % \introduction{einleitung}

    \ingredients{
        \SI{1}{\EL} & Chiasamen \\
        \SI{3}{\EL} & Wasser \\
        \SI{200}{\g} & Sojajoghurt \\
        \SI{130}{\g} & Dinkelvollkornmehl \\
        \SI{40}{\g} & Haferkleie \\
        \SI{1}{\EL} & Tomatenmark \\
        \SI{100}{\g} & Pepperoni \\
        \SI{1}{\TL} & Salz \\
        \SI{1}{\TL} & Thymian \\
        \SI{1}{\TL} & Basilikum \\
        \SI{1}{\pck} & Backpulver
    }

    \preparation{
        \step Zuerst mit den Chiasamen und dem Wasser ein veganes Ei vorbereiten: beides miteinander verrühren und etwa \SI{10}{\minute} lang stehen lassen. Währenddessen die Peperoni in Ringe schneiden. Den Ofen auf \SI{175}{\degree} Umluft vorheizen.

        \step Für das Baguette dann den Quark mit dem veganen Ei verrühren, dann alle anderen Zutaten hinzugeben. Gut miteinander verkneten, bis ein homogener Teig entsteht. Den Teig zu einem Baguette formen. Da der Teig sehr klebrig ist, helfen nasse Hände beim formen. Das Baguette auf einem mit Backpapier bedeckten Backblech für \SI{40}{\minute} bis \SI{50}{\minute} backen.

    }

    % \suggestion[Title of Suggestion]{
	% 	Suggestion
    % }
    %
    % \hint{Hint}

\end{recipeDP}

		\chapterwithtoc{Brötchen}
		\chapterwithtoc{Kekse}
			\begin{recipeDP}
    [
        preparationtime = {\SI{15}{\minute}},
        bakingtime = {\SI{12}{\minute} bis \SI{15}{\minute}},
        bakingtemperature = {\protect\bakingtemperature{fanoven=\SI{180}{\celsius}}},
        portion = {40 Stück},
        source = {Stina Spiegelberg}
    ]
    {Fruchtige Johannisbeerlebkuchen\index{Lebkuchen}}

    % \graph
    %     {
    %         small=Recipes/MainCourses/BBQChicken/Small.jpg,
    %         big=example-image
    %     }

    % \introduction{einleitung}

    \ingredients{
        \multicolumn{2}{l}{\textbf{Teig}} \\
        \SI{80}{\g} & Dinkelvollkornmehl \\
        \SI{100}{\g} & blanchierte, gemahlene Mandeln \\
        \SI{150}{\g} & gemischte, gemahlene Nüsse \\
        \SI{250}{\g} & Marzipanrohmasse \\
        \SI{180}{\g} & Rohrohrzucker \\
        \SI{1}{\pck} & Vanillezucker \\
        \SI[parse-numbers = false]{\nicefrac{1}{2}}{\TL} & Backpulver \\
        \SI{4}{\EL} & dunkle \addtoidx{Johannisbeermarmelade} \\
        \SI{60}{\ml} & Wasser \\
        etwa 40 & Backoblaten mit \SI{5}{\cm} \dmesser \\
        \\
        \multicolumn{2}{l}{\textbf{Dekor}} \\
        \SI{200}{\g} & Zartbitterkuvertüre \\
         & bunte Zuckerstreusel
    }

    \preparation{
        \step Mehl, Mandeln und Nüsse in eine große Rührschüssel geben, die Marzipanrohmasse mit einer groben Reibe in die trockenen Zutaten reibe. Dabei mehrmals rühren, damit das Marzipan nicht verklebt. Zucker, Backpulver Marmelade und Wasser zugeben und mit den Knethaken zu einem Teig verarbeiten. Es dürfen noch kleine Marzipanstückchen zu sehen sein.

        \step Den Teig kalt stellen, etwa über Nacht im Kühlschrank oder einige Stunden im Tiefkühler. Der Teig lässt sich dann einfacher verarbeiten und klebt nicht so sehr.

        \step Den Ofen auf \SI{180}{\celsius} vorheizen. Für die Zubereitung der Lebkuchen jeweils etwa \SI{1}{\TL} Teig auf eine Backoblate geben und mit den Fingern andrücken und bis an de Rand verteilen. Die Lebkuchen dann etwa \SI{12}{\minute} bis \SI{15}{\minute} backen, bis sie leicht braun werden. Dabei öfter in den Ofen schauen, damit nichts anbrennt.

        \step Die Lebkuchen über Nacht auskühlen lassen. Anschließend die Zartbitterkuvertüre im Wasserbad erwärmen und die Lebkuchen damit bestreichen. Mit den Zuckerstreuseln dann verzieren.
    }

    \suggestion[Zuckergussglasur]{
        Sehr gut passt auch eine Glasur aus etwas Zitronensaft und Puderzucker anstelle der Zartbitterkuvertüre. Dazu einige Tropfen Zitronensaft mit etwa \SI{100}{\g} Puderzucker verrühren, bis eine dickflüssige Glasur entsteht. Dann die Lebkuchen damit einstreichen.
    }

    \hint{Für das Original ``Elisenlebkuchen'' die Johannisbeermarmelade durch Aprikosenmarmelade ersetzen, je \SI{1}{\msp} gemahlenen Zimt und Koriander hinzufügen. Die Lebkuchen mit Kuvertüre und blanchierten Mandelhälften verzieren.}

\end{recipeDP}

			\begin{recipe}
    [
        preparationtime = {\SI{20}{\minute}},
        bakingtime = {\SI{15}{\minute}},
        bakingtemperature = {\protect\bakingtemperature{topbottomheat=\SI{190}{\degree}}},
        portion = {50 Stück},
        source = {Stina Spiegelberg}
    ]
    {Rosmarin-\addtoidx{Heidesand}}

    % \graph
    %     {
    %         small=Recipes/MainCourses/BBQChicken/Small.jpg,
    %         big=example-image
    %     }

    % \introduction{einleitung}

    \ingredients{
        \SI{180}{\g} & Zucker \\
        \SI{1}{\pck} & Vanillezucker \\
        \SI{200}{\g} & vegane Butter, zimmerwarm \\
        \SI{300}{\g} & Weizenmehl (Typ 405) \\
        \SI[parse-numbers = false]{\nicefrac{1}{2}}{\TL} & Fleur de Sel \\
        \SI{2}{\TL} & \addtoidx{Rosmarin}, gehackt \\
        1 & Zitrone, abgeriebene Schale \\
        \SI{4}{\EL} & Pflanzendrink
    }

    \preparation{
        \step \SI{150}{\g} Zucker, Vanillezucker und vegane Butter in einer großen Schüssel mit dem Schneebesen schaumig schlagen. Mehl, Salz, Rosmarin, Zitronenschale und Pflanzendrink zugeben und zu einem glatten Teig kneten. Den Teig in Frischhaltefolie wickeln und eine Stunde kalt stellen.

        \step Den Backofen auf \SI{190}{\degree} Ober-/Unterhitze vorheizen.

        \step Den Teig halbieren. Die Arbeitsfläche mit Mehl bestäuben und die Teighälften jeweils zu etaw \SI{40}{\cm} langen Broten formen. Den restlichen Zucker auf die Arbeitsfläche streuen und die Brote darin wälzen. Erneut in Frischhaltefolie wickeln und 30 Minuten kalt stellen.

        \step Von den Broten etwa \SI{1}{\cm} dicke Stücke abschneiden die Plätzchen auf einem mit Backpapier ausgelegten Backblech \ca 15 Minuten backen. Auskühlen lassen und luftdicht aufbewahren.
    }

    % \suggestion[Title of Suggestion]{
	% 	Suggestion
    % }

    \hint{Für die klassischen Heidesand-Plätzchen das Fleur de Sel durch eine Prise Meersalz ersetzen, den Rosmarin weglassen und nur den Abrieb einer halben Zitrone verwenden}

\end{recipe}

			\begin{recipeDP}
    [
        preparationtime = {\SI{25}{\minute}},
        bakingtime = {\SI{10}{\minute} bis \SI{12}{\minute}},
        bakingtemperature = {\protect\bakingtemperature{topbottomheat=\SI{180}{\celsius}}},
        portion = {70 Stück},
        source = {Stina Spiegelberg}
    ]
    {Lussekatter}

    % \graph
    %     {
    %         small=Recipes/MainCourses/BBQChicken/Small.jpg,
    %         big=example-image
    %     }

    \introduction{
        Lussekatter werden in Schweden am 13. Dezember zum Santa-Lucia-Fest, dem Lichterfest, gereicht. Übersetzt bedeutet der Name des leckeren Safrangebäcks ``Lucia-Katzen''.
    }

    \ingredients{
        \SI{300}{\g} & Weizenmehl (Typ 550) \\
        \SI{50}{\g} & blanchierte, gemahlene Mandeln \\
        \SI{100}{\g} & Rohrohrzucker \\
        \SI{1}{\msp} & Vanillepulver \\
        \SI{1}{\msp} & Ingwerpulver \\
        \SI{1}{Prise} & gemahlene Safranfäden\index{Safran} \\
        \nicefrac{1}{4} & Zitrone, abgeriebene Schale \\
        \SI{200}{\g} & vegane Butter, zimmerwarm \\
        \SI{3}{\EL} & Pflanzendrink \\
        \SI{30}{\g} & getrocknete Cranberrys
    }

    \preparation{
        \step Mehl, Mandeln, Zucker, Vanille, Gewürze und Orangenschale in einer Rührschüssel mischen. Mit veganer Butter und Pflanzendrink zu einem glatten Teig verkneten. Den Teig in Frischhaltefolie wickeln und eine Stunde kalt stellen.

        \step Den Backofen auf \SI{180}{\celsius} Ober-/Unterhitze vorheizen.

        \step Den Teig auf einer bemehlten Arbeitsfläche zu langen Teigwürsten formen und gleich große, \ca \SI{10}{\cm} lange Stücke abschneiden. Diese Röllchen an beiden Enden in entgegengesetzter Richtung einrollen, sodass ein ``S'' entsteht. Oben und unten jeweils eine halbe Cranberry in das ``S'' drücken.

        \step Plätzchen auf ein mit Backpapier ausgelegtes Backblech legen und \SI{10}{\minute} bis \SI{12}{\minute} backen. Aus dem Ofen nehmen und vollständig auskühlen lassen.
    }

    % \suggestion[Title of Suggestion]{
	% 	Suggestion
    % }

    % \hint{Hint}

\end{recipeDP}

		\chapterwithtoc{Kuchen}
			\begin{recipe}
    [
        preparationtime = {\SI{20}{\minute}},
        bakingtime = {\SI{40}{\minute}},
        bakingtemperature = {\protect\bakingtemperature{fanoven=\SI{160}{\celsius}}},
        portion = {1 Backform 20 x 20 cm},
        source = {Stina Spiegelberg}
    ]
    {\addtoidx{Lebkuchen}-Brownies}

    % \graph
    %     {
    %         small=Recipes/MainCourses/BBQChicken/Small.jpg,
    %         big=example-image
    %     }

    % \introduction{einleitung}

    \ingredients{
        \multicolumn{2}{l}{\textbf{Rührteig}} \\
        \SI{200}{\g} & Marzipanrohmasse \\
        \SI{40}{\ml} & Öl \\
        \SI{250}{\g} & Weizenmehl (Typ 405) \\
        \SI{60}{\g} & Rohrohrzucker \\
        \SI{1}{\pck} & Vanillezucker \\
        \SI{200}{\g} & gemahlene Haselnüsse \\
        \SI{1}{\TL} & Lebkuchengewürz \\
        \SI{2}{\EL} & Kakao \\
        \SI{1}{Prise} & Salz \\
        \SI{1}{\geh\TL} & Hirschhornsalz \\
        \SI{50}{\g} & Orangeat \\
        \SI{50}{\g} & Aprikosenmarmelade \\
        \SI{300}{\ml} & stilles Wasser \\
         & Fett zum Einfetten \\
        \\
        \multicolumn{2}{l}{\textbf{Dekor}} \\
        \SI{60}{\g} & Halbbitterschokolade \\
        \SI{2}{\EL} & Hafer Cuisine \\
        \SI{4}{\EL} & Puderzucker
    }

    \preparation{
        \step Den Backofen auf \SI{160}{\celsius} vorheizen.

        \step Die Marzipanrohmasse mit dem Öl kurz im Mixer verrühren. Mehl, Zucker, Vanillezucker, Haselnüsse, Lebkuchengewürz, Kakao, Salz und Hirschhornsalz in einer Rührschüssel kurz mischen. Das Orangeat fein hacken, mit der Marzipanmasse, der Marmelade und dem Wasser dazugeben und mit dem Schneebesen zu einem glatten Teig rühren. In die gefettete Backform geben und ca. 40 Minuten backen. Mit einem Holzstäbchen die Garprobe machen und abkühlen lassen.

        \step Die Schokolade hacken. Die Pflanzensahne erwärmen, über die Schokolade gießen und so lange rühren, bis die Schokolade vollständig geschmolzen ist. Dann den Puderzucker einsieben. Die Mischung auf dem Kuchen verteilen und auskühlen lassen.
    }

    % \suggestion[Title of Suggestion]{
    %     Suggestion
    % }

    % \hint{Hint}

\end{recipe}


	\part{Getränke}
		\chapterwithtoc{alkoholische Getränke}

		% \begin{recipe}
		%     [
		%         preparationtime = {\SI{15}{\minute}},
		%         bakingtime = {\SI{12}{\minute} bis \SI{15}{\minute}},
		%         bakingtemperature = {\protect\bakingtemperature{fanoven=\SI{180}{\degree}}},
		%         portion = {40 Stück},
		%         source = {Stina Spiegelberg}
		%     ]
		%     {Name}
		%
		%     % \graph
		%     %     {
		%     %         small=Recipes/MainCourses/BBQChicken/Small.jpg,
		%     %         big=example-image
		%     %     }
		%
		%     % \introduction{einleitung}
		%
		%     \ingredients{
		%
		%     }
		%
		%     \preparation{
		%         \step
		%     }
		%
		%     \suggestion[Title of Suggestion]{
		% 		Suggestion
		%     }
		%
		%     \hint{Hint}
		%
		% \end{recipe}

\end{document}
