\begin{recipe}
    [
        preparationtime = {\SI{30}{\minute}},
        % bakingtime = {\SI{12}{\minute} bis \SI{15}{\minute}},
        % bakingtemperature = {\protect\bakingtemperature{fanoven=\SI{180}{\degree}}},
        portion = {6 Stück},
        source = {minimalistbaker.com}
    ]
    {Glutenfreie \addtoidx{Waffeln}}

    % \graph
    %     {
    %         small=Recipes/MainCourses/BBQChicken/Small.jpg,
    %         big=example-image
    %     }

    % \introduction{einleitung}

    \ingredients{
        \SI{300}{\ml} & Mandelmilch \\
        \SI{1}{\TL} & Apfelessig \\
        \SI{60}{\ml} & Öl \\
        \SI{60}{\ml} & Ahornsirup oder andere Süße \\
        \SI{50}{\g} & (glutenfreie) Haferflocken \\
        \SI{280}{\g} & glutenfreis Mehl \\
        \SI[parse-numbers = false]{1\nicefrac{1}{2}}{\TL} & Backpulver \\
        \SI{1}{Prise} & Meersalz \\
        \\
        \multicolumn{2}{l}{\textbf{optional:}} \\
        \SI{1}{\TL} & Vanilleextrakt \\
        \SI[parse-numbers = false]{\nicefrac{1}{2}}{\TL} & Zimt \\
        \SI{7}{\g} & Leinsamenmehl \\
        \SI{45}{\g} & vegane Schokoladenstücke \\
        \SI{38}{\g} & Früchte
    }

    \preparation{
        \step Zuerst die Mandelmilch und den Essig in einer kleinen Schüssel miteinander verrühren, dann ein paar Minuten zum aktivieren stehen lassen. Anschließend das Öl und den Sirup hinzugeben und beiseite stellen.

        \step Die trockenen Zutaten in eine große Schüssel geben und gut miteinander vermischen. Dann die feuchten Zutaten in die Trockenen mixen, bis alles gut miteinander vermengt ist. Wenn der Teig etwas zu dickflüssig ist etwas Milch hinzugeben andernfalls etwas Mehl.

        \step Den Teig fünf bis zehn Minuten stehen lassen und dann das Waffeleisen vorheizen. Wenn das Waffeleisen heiß ist, gegebenenfalls einfetten und dann den Teig darin ausbraten. Die fertigen Waffeln am besten sofort mit den Toppings servieren.
    }

    % \suggestion[Title of Suggestion]{
	% 	Suggestion
    % }
    %
    \hint{Grundsätzlich ist Hafer glutenfrei, kleine Mengen Gluten können nur bei der Verarbeitung oder dem Transport des Hafers in die Haferflocken geraten.}

\end{recipe}
