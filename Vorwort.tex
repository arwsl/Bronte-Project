% \cleardoubleemptypage
\thispagestyle{empty}
\setcounter{page}{0}

\section*{Vorwort}
\label{sec:Vorwort}

Ein Projekt auf unbestimmte Zeit zu terminieren ist vielleicht nicht die beste Idee, um tatsächlich ein fertiges Produkt zu erhalten, womöglich ist aber genau das auch Kern dieses Projektes: Denn dieses Kochbuch kann als fortlaufende, sich ständig verbessernde Enzyklopädie widerspiegeln, welche interessanten, leckeren und praktikablen Rezepte mein Mahlzeiten-Portfolio füllen. Alle Inhalte sind von erprobt und konnten nur so ihren Weg in diese Sammlung gefunden.

Die Idee, ein Kochbuch zu entwickeln ist schon lange eine Idee von mir, mit der Kombination aus \texttt{git} und \LaTeX\:hält mich nun aber nichts mehr davon ab, diese Idee auch wirklich umzusetzen. Das Prinzip der \emph{continuous integration} unterstützt dabei besonders die fortlaufende Entwicklung, an der sich jeder gerne beteiligen darf! Dafür einfach die \href{https://github.com/arwsl/Bronte-Project}{README.md} im Repository berücksichtigen. Da wird eine kurze Einführung in die Struktur des Dokuments, die Eigenarten und Eigenschaften des \texttt{xcookybooky}-Pakets sowie das Setzen der Rezepte in verschiedenen \texttt{recipe}-Umgebungen geliefert.

Als schwierigster Teil in der Entwicklung dieses Kochbuchs stellte sich das konzipieren der Kapitelstruktur heraus. Und auch wenn in Zukunft neuartige Rezepte nicht in die aktuelle Struktur passen mögen, ist die Struktur im Moment auf jeden Fall in der Lage, in sechs Teilen alle Rezepte zu sortieren. Bis dahin ist zu Glück der schwierigste Teil geschafft. Das weitere Testen und Einpflegen neuer Gerichte gehört da nun wirklich eher zum gemütlichen Teil, zumal ja auch das Essen dazu gehört!

Allen, die ausschließlich konsumierend Teil haben und sich nicht an der Weiterentwicklung beteiligen, wünsche ich viel Spaß beim Ausprobieren, Genießen und Weiterempfehlen!

\vspace{1.5cm}
\hspace{1cm} arwsl
\begin{flushright}
	\hfill Wolfenbüttel, April 2021
\end{flushright}



\clearpage
