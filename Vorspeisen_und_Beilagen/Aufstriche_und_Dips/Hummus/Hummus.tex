\begin{recipeDP}
    [
        preparationtime = {\SI{30}{\minute}},
        % bakingtime = {\SI{12}{\minute} bis \SI{15}{\minute}},
        % bakingtemperature = {\protect\bakingtemperature{fanoven=\SI{180}{\celsius}}},
        portion = {etwa \SI{900}{\ml}},
        source = {Sedef Salmo, Kristiane Redecker}
    ]
    {\addtoidx{Hummus}}

    % \graph
    %     {
    %         big=example-image
    %         small=Small.jpg
    %     }

    % \introduction{einleitung}

    \ingredients{
        \SI{200}{\g} & \addtoidx{Kichererbsen}, getrocknet \\
        1 & Zitrone \\
        \SI{3}{\EL} & Olivenöl \\
        \\
        \SI{150}{\g} & \addtoidx{Tahin} \\
        3 Zehen & Knoblauch \\
        \SI{1,5}{\TL} & Kreuzkümmel \\
        \SI[parse-numbers = false]{\nicefrac{3}{4}}{\TL} & Piment \\
        \SI[parse-numbers = false]{\nicefrac{3}{4}}{\TL} & Pfeffer \\
        \SI{1,5}{\TL} & Salz \\
        \SI{6}{\EL} & Olivenöl \\
         & Wasser \\
        \\
        \SI{3}{\EL} & Olivenöl \\
         & Petersilie
    }

    \preparation{
        \step Am Vortag die trockenen Kichererbsen in etwa \SI[parse-numbers = false]{\nicefrac{1}{2}}{\l} Wasser einweichen.
        \step Wenn die Kichererbsen gut eingeweicht sind, für etwa eine Stunde kochen, bis sie ganz weich sind. Die weichen Kichererbsen dann mit \SI{3}{\EL} Olivenöl und dem Saft einer Zitrone pürieren.
        \step Dann das Tahin, den Knoblauch, Kreuzkümmel, Piment, Pfeffer, Salz und das restliche Olivenöl hinzugeben und gut verrühren. Nach Bedarf noch Wasser oder aufgefangenes Aquafaba hinzugeben, um eine glatte Konsistenz zu erhalten.
        \step Zum Anrichten mit Olivenöl beträufeln und mit Petersilie garnieren.
    }

    \suggestion[Dicke Bohnen Püree]{
		Statt der Kichererbsen lassen sich auch Dicke Bohnen (Bakla) verwenden: etwa \SI{45}{\minute} kochen und dann wie Kichererbsen zubereiten, nur statt Tahin \SI{3}{\EL} Olivenöl verwenden.
    }

    \hint{\SI{200}{\g} getrocknete Kichererbsen lassen sich auch durch etwa \SI{530}{\g} Kichererbsen aus der Dose ersetzen. Diese können direkt verarbeitet werden.}

\end{recipeDP}
