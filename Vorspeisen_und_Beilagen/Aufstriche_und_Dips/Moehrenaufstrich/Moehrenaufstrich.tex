\begin{recipeDP}
    [
        preparationtime = {\SI{30}{\minute}},
        % bakingtime = {\SI{ZEIT}{\minute} bis \SI{ZEIT}{\minute}},
        % bakingtemperature = {\protect\bakingtemperature{fanoven=\SI{TEMPERATUR}{\celsius}}},
        % portion = {PORTIONEN/MENGE},
        source = {rezeptwelt.de}
    ]
    {Möhrenaufstrich}

    \graph
        {
            big=Vorspeisen_und_Beilagen/Aufstriche_und_Dips/Moehrenaufstrich/big.jpg,
            % small=TEIL/KAPITEL/REZEPT/small.jpg
        }

    % \introduction{
    %     EINLEITUNG
    % }

    \ingredients{
        1 & kleine Zwiebel \\
        \SI{20}{\g} & Margarine \\
        \SI{250}{\g} & \addtoidx{Möhren} \\
        \SI{80}{\g} & Margarine \\
        \SI{140}{\g} & Tomatenmark \\
        \SI{1}{\TL} & Oregano \\
        \SI{1}{\TL} & Thymian \\
        \SI{1}{\TL} & Salz \\
    }

    \preparation{
        \step Die Zwiebel sehr klein schneiden und dann mit \SI{20}{\g} Margarine kurz in einer Pfanne andünsten.
        \step Die Möhren schälen und in kleine Stücke schneiden.
        Dann die restliche Margarine, das Tomatenmark sowie Orgeano, Thymian und Salz gemeinsam mit den Möhren in einer Pfanne auf mittlerer Stufe garen.
        Nach etwa 10 Minuten aus der Pfanne nehmen und alles zusammen (mit den Zwiebeln) in einem hohen Gefäß pürieren bis ein streichfähiger Dip entsteht.
    }

    % \suggestion[TITEL EINES VORSCHLAGS]{
	% 	VORSCHLAG (DURCH HORIZONTALE LINIE VOM REZEPT GETRENNT)
    % }

    % \hint{
    %     HINWEIS (IN EINEM KASTEN UNTEN AUF DER SEITE)
    % }

\end{recipeDP}