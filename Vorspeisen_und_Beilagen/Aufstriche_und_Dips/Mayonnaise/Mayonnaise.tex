\begin{recipeDP}
    [
        preparationtime = {\SI{5}{\minute}},
        % bakingtime = {\SI{12}{\minute} bis \SI{15}{\minute}},
        % bakingtemperature = {\protect\bakingtemperature{fanoven=\SI{180}{\celsius}}},
        portion = {1 mittleres Glas},
        source = {kaffeeundcupcakes.de}
    ]
    {\addtoidx{Mayonnaise}}

    \graph
        {
            big=Vorspeisen_und_Beilagen/Aufstriche_und_Dips/Mayonnaise/big.jpg,
            small=Vorspeisen_und_Beilagen/Aufstriche_und_Dips/Mayonnaise/small.jpg
        }

    \introduction{Schnelles und einfaches Grundrezept für vegane Mayonnaise.}

    \ingredients{
        \SI{130}{\g} & Sojamilch\index{Milch!Soja-}, Raumtemperatur \\
        \SI{2}{\TL} & Branntweinessig \\
        \SI{200}{\g} & \addtoidx{Öl}, geschmacksneutral \\
        \SI{1}{\TL} & Salz \\
        \SI{1}{\TL} & mittelscharfer Senf
    }

    \preparation{
        \step Die Sojamilch (unbedingt auf Raumtemperatur!) in ein hohes schmales Mix-Gefäß gießen. Den Essig dazugeben und dann das Öl darauf gießen.
        \step Einen Stabmixer ganz bis auf den Boden des Gefäßes stellen und losmixen. Nach einiger Zeit (30-60 Sekunden) langsam den Stabmixer beim Mixen auf und ab bewegen. So lange mixen, bis die Mayonnaise angedickt hat und alles eine gleichmäßige, cremige Konsistenz angenommen hat.
        \step Die Mayonnaise mit Salz und Senf abschmecken. In einem verschließbaren Gefäß im Kühlschrank aufbewahren und innerhalb von 1 Woche aufbrauchen. Guten Appetit!
    }

    % \suggestion[Title of Suggestion]{
	% 	Suggestion
    % }
    %
    % \hint{Hint}

\end{recipeDP}
