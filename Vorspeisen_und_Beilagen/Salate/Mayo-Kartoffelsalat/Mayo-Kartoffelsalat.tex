\begin{recipeDP}
    [
        preparationtime = {\SI{20}{\minute}},
        bakingtime = {\SI{40}{\minute}},
        bakingtemperature = {\protect\bakingtemperature{fanoven=\SI{200}{\celsius}}},
        portion = {4 bis 6 Portionen},
        source = {zuckerjagdwurst.com}
    ]
    {Kartoffelsalat mit Mayonnaise}

    \graph
        {
            big=Vorspeisen_und_Beilagen/Salate/Mayo-Kartoffelsalat/big.jpg,
            small=Vorspeisen_und_Beilagen/Salate/Mayo-Kartoffelsalat/small.jpg
        }

    % \introduction{
    %     Der beste Kartoffelsalat mit cremigem Mayo-Dressing, knusprig gerösteten Kartoffeln und Gurkenrelish -- lauwarm serviert ein Genuss.
    % }

    \ingredients{
        \multicolumn{2}{l}{\textbf{Kartoffeln}} \\
        \SI{800}{\g} & kleine \addtoidx{Kartoffeln} \\
        \SI{1}{\EL} & Olivenöl \\
        & Salz \\
        & Pfeffer \\
        \\
        \multicolumn{2}{l}{\textbf{Dressing}} \\
        1 & rote Zwiebel \\
        1 & Gurke \\
        5 & Gewürzgurken \\
        \SI{10}{\g} & Dill \\
        \SI{250}{\g} & vegane Mayonnaise\index{Mayonnaise} \\
        \SI{100}{\g} & \addtoidx{Gurkenrelish} \\
        \SI{2}{\TL} & Senf \\
        \SI{1}{\TL} & Zwiebelpulver \\
        \SI{1}{\TL} & Knoblauchpulver \\
        & Salz \\
        & Pfeffer \\
        \\
        \SI{10}{\g} & Schnittlauch
    }

    \preparation{
        \step Den Backofen auf \SI{200}{\celsius} (Umluft) vorheizen.
        Die Kartoffeln gründlich waschen, halbieren und auf einem Backblech verteilen.
        Mit Olivenöl beträufeln und mit Salz und Pfeffer würzen.
        Die Kartoffeln 30 bis 40 Minuten backen, bis sie gar, gebräunt und leicht knusprig sind.
        Während der Backzeit ein- bis zweimal wenden.
        \step Währenddessen für das Dressing die Zwiebel schälen und klein würfeln, die Gurke waschen und klein schneiden und die Gewürzgurken in kleine Scheiben schneiden.
        Dill und Schnittlauch waschen und fein hacken.
        Zwiebel, Gurke, Gewürzgurke, Dill, vegane Mayonnaise, Gurkenrelish, Senf, Zwiebel- und Knoblauchpulver in einer großen Schüssel vermengen.
        Mit Salz und Pfeffer abschmecken.
        \step Die Kartoffeln nach dem Backen 5 bis 10 Minuten abkühlen lassen.
        Im Anschluss in die Schüssel zum Dressing geben und vermengen.
        Kartoffelsalat mit gehacktem Schnittlauch bestreuen und noch lauwarm oder abgekühlt servieren.
    }

    % \suggestion[TITEL EINES VORSCHLAGS]{
	% 	VORSCHLAG (DURCH HORIZONTALE LINIE VOM REZEPT GETRENNT)
    % }

    \hint{
        \textbf{Lauwarm genießen:} Der Kartoffelsalat schmeckt lauwarm besonders gut, kann aber auch kalt serviert werden.
    }

\end{recipeDP}
