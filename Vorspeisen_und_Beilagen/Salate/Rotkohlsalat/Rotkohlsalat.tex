\begin{recipeDP}
    [
        preparationtime = {\SI{30}{\minute}},
        % bakingtime = {\SI{12}{\minute} bis \SI{15}{\minute}},
        % bakingtemperature = {\protect\bakingtemperature{fanoven=\SI{180}{\celsius}}},
        portion = {4 Portionen},
        source = {veeatcookbake.com}
    ]
    {Rotkohlsalat\index{Kohl!Rot-}}

    \graph
        {
            big=Vorspeisen_und_Beilagen/Salate/Rotkohlsalat/big.jpg,
            small=Vorspeisen_und_Beilagen/Salate/Rotkohlsalat/small.jpg
        }

    % \introduction{einleitung}

    \ingredients{
        1 & kleiner Rotkohl \\
        2 & Schalotten \\
        \SI{4}{\EL} & Apfelindex\index{Apfel!-essig}\index{Essig!Apfel-} \\
        \SI{2}{\EL} & Wasser \\
        \SI{2}{\TL} & Salz \\
        \SI{1}{\TL} & Pfeffer \\
         & Kümmel (optional)
    }

    \preparation{
        \step Den Kohl fein schneiden, oder auch durch den Food Processor lassen. Die Zwiebel fein schneiden oder ebenfalls mit dem Rotkohl durch die Maschine. Den geschnittenen Kohl in eine ausreichend große, am besten verschließbare, Schüssel geben und die restlichen Zutaten dazu geben. Nun den Kohl für einige Minuten durchkneten bzw. massieren. So wird die Texture gebrochen und der Kohlsalat wird schön saftig und nicht so trocken.
        \step Nun den Salat für einige Stunden auf der Arbeitsfläche oder an einem kühlen Platz stehen lassen. Am besten über Nacht durchziehen lassen. Am nächsten Morgen nochmal kurz abschmecken und gegebenenfalls durchkneten.
    }

    % \suggestion[Title of Suggestion]{
	% 	Suggestion
    % }
    %
    % \hint{Hint}

\end{recipeDP}
