\begin{recipeDP}
    [
        preparationtime = {\SI{35}{\minute}},
        % bakingtime = {\SI{ZEIT}{\minute} bis \SI{ZEIT}{\minute}},
        % bakingtemperature = {\protect\bakingtemperature{fanoven=\SI{TEMPERATUR}{\celsius}}},
        portion = {4},
        source = {veganlovlie.com}
    ]
    {\addtoidx{Tempeh}}

    \graph
        {
            big=Vorspeisen_und_Beilagen/Beilagen/DIY_Tempeh/big.jpg,
            small=Vorspeisen_und_Beilagen/Beilagen/DIY_Tempeh/small.jpg
        }

    \introduction{
        Sojabohnen mit dem Schnellkochtopf gekocht gehen super schnell! Getrocknete Sojabohnen werden nach dem Kochen etwa 2,4 mal so schwer wie vorher, sie lassen sich also durch die entsprechende Menge gekochter Bohnen ersetzen.
    }

    \ingredients{
        \SI{375}{\g} & getrocknete \addtoidx{Sojabohnen} (\SI{900}{\g} gekochte Sojabohnen) \\
        \SI{4}{\EL} & Apfelessig \\
        \SI{1}{\TL} & Tempeh Starter \\
    }

    \preparation{
        \step Die Sojabohnen in reichlich Wasser mindestens 8 Stunden (besser über Nacht im Kühlschrank) einweichen lassen.
        \step Dann, nach dem Einweichen die Sojabohnen abgießen und gründlich abspülen.
        In einem Schnellkochtopf auf 1. Stufe für \SI{18}{\minute} kochen (ein wenig Öl soll die Schaumbildung verringern).
        Dann den Druck erst durch natürliches Abkühlen und dann lagsam auch mit dem Ventil ablassen.
        \step Nach dem Kochen, die Sojabohnen abgießen, abtropfen und abkühlen lassen (bis auf \SI{35}{\celsius}).
        In zwei Glasschüsseln (zu je einem Liter) die Sojabohnen verteilen und mit dem Apfelessig vermischen.
        Dann gleichmäßig die Rhizopus oligosporus Kultur (der Tempeh Starter) verteilen und unterheben.
        Die Sojabohnen fest in die Form drücken und an einem warmen Ort für etwa 36 bis 48 Stunden stehen lassen.
        Wenn alle Zwischenräume mit Pilzkultur verschlossen sind, ist der Tempeh fertig und kann eingefroren oder direkt verarbeitet werden.
        (Kleine schwarze Punkte sind völlig normal - nur bei grünem oder rötlichen Schimmel sofort entsorgen!)
        \step Der Tempeh muss vor dem Verzehr gekocht werden: etwa gedünstet, gekocht oder mariniert und gebraten.
        Im Kühlschrank hält er sich etwa eine Woche, im Tiefkühler mehrere Monate.
    }

    % \suggestion[TITEL EINES VORSCHLAGS]{
	% 	VORSCHLAG (DURCH HORIZONTALE LINIE VOM REZEPT GETRENNT)
    % }

    % \hint{
    %     HINWEIS (IN EINEM KASTEN UNTEN AUF DER SEITE)
    % }

\end{recipeDP}