\begin{recipe}
    [
        preparationtime = {\SI{15}{\minute}},
        % bakingtime = {\SI{12}{\minute} bis \SI{15}{\minute}},
        % bakingtemperature = {\protect\bakingtemperature{fanoven=\SI{180}{\degree}}},
        % portion = {40 Stück},
        source = {Oma Jutta}
    ]
    {\addtoidx{Bohnensalat}}

    % \graph
    %     {
    %         small=Recipes/MainCourses/BBQChicken/Small.jpg,
    %         big=example-image
    %     }

    % \introduction{einleitung}

    \ingredients{
        \SI{5}{\EL} & Essig \\
        \SI{4}{\TL} & Zucker \\
        \SI[parse-numbers = false]{\nicefrac{1}{2}}{\TL} & Salz \\
        \SI{3}{\EL} & neutrales Öl \\
         & kleine Zwiebel \\
         & heißes Wasser \\
        \SI{1}{Weckglas} & grüne Bohnen\index{Bohnen!grüne}
    }

    \preparation{
        \step Zuerst den Essig, den Zucker, das Salz, das Öl, die Zwiebelwürfel, den Pfeffer und das Wasser gut vermischen. Dann die Bohnen hinein geben und gut unterrühren. Wenn möglich einen Tag ziehen lassen.
    }

    % \suggestion[Title of Suggestion]{
	% 	Suggestion
    % }
    %
    % \hint{Hint}

\end{recipe}
