\begin{recipeDP}
    [
        preparationtime = {\SI{30}{\minute}},
        bakingtime = {\SI{20}{\minute}},
        bakingtemperature = {\protect\bakingtemperature{topbottomheat=\SI{180}{\celsius}}},
        portion = {12 Stück},
        source = {zuckerjagdwurst.com}
    ]
    {Zimtschnecken}

    \graph
        {
            big=Gebaeck/Broetchen/Zimtschnecken/big.jpg,
            small=Gebaeck/Broetchen/Zimtschnecken/small.jpg
        }

    \introduction{
        Fluffige vegane Zimtschnecken mit zimtiger Füllung – das perfekte Gebäck für die kalte Jahreszeit.
    }

    \ingredients{
        \multicolumn{2}{l}{\textbf{Hefeteig}} \\
        \SI{250}{\ml} & pflanzliche Milch \\
        \SI{10}{\g} & frische \addtoidx{Hefe} \\
        \SI{500}{\g} & Weizenmehl (Type 550) \\
        \SI{80}{\g} & Zucker \\
        \SI{8}{\g} & Salz \\
        \SI{100}{\g} & vegane Butter \\
        \\
        \multicolumn{2}{l}{\textbf{Füllung}} \\
        \SI{150}{\g} & vegane Butter \\
        \SI{100}{\g} & brauner Zucker\index{Zucker!braun} \\
        \SI{4}{\EL} & \addtoidx{Zimt} \\
        1 Prise & Salz \\
        \\
        \multicolumn{2}{l}{\textbf{Zum Bestreichen}} \\
        \SI{100}{\ml} & pflanzliche Milch
    }

    \preparation{
        \step Die pflanzliche Milch lauwarm erwärmen, einen Esslöffel Zucker einrühren und die Hefe darin auflösen.
        Die Mischung für 15 Minuten stehen lassen, bis sie sichtbar Blasen schlägt.
        \step Mehl, restlichen Zucker, Salz und die Hefemilch in einer großen Schüssel 8 Minuten lang zu einem glatten Teig verkneten.
        Dann zu einer runden Kugel formen, in die Schüssel legen, mit einem feuchten Tuch abdecken und für 30 Minuten bei Raumtemperatur ruhen lassen.
        \step Die vegane Butter würfeln und 5 Minuten lang unter den Teig kneten, bis er wieder ganz glatt und weich ist.
        Die Teigkugel erneut abdecken und für weitere 90 Minuten ruhen lassen.
        \step Für die Füllung vegane Butter (weich) mit braunem Zucker, Zimt und Salz glattrühren.
        \step Den Ofen auf \SI{180}{\celsius} (Ober-/Unterhitze) vorheizen und eine rechteckige Auflaufform einfetten.
        Den Hefeteig auf die bemehlte Arbeitsfläche geben, circa 5 Millimeter dick zu einem Rechteck ausrollen.
        Die Füllung auf den Teig geben und vorsichtig gleichmäßig verteilen.
        Dann den Teig von der längeren Seite her aufrollen und in 12 gleich dicke Scheiben schneiden.
        \step Die Schnecken mit etwas Abstand zueinander in die Form legen, die Form nochmals mit einem feuchten Tuch abdecken und für 30 Minuten ruhen lassen.
        Dünn mit pflanzlicher Milch bestreichen und für circa 20 Minuten bei \SI{180}{\celsius} (Ober-/Unterhitze) backen.
        Nach dem Backen noch heiß mit etwas pflanzlicher Milch bestreichen und abkühlen lassen.
    }

    % \suggestion[TITEL EINES VORSCHLAGS]{
	% 	VORSCHLAG (DURCH HORIZONTALE LINIE VOM REZEPT GETRENNT)
    % }

    \hint{
        \begin{itemize}
            \item \textbf{Milchtemperatur:} Nur lauwarm, nicht zu warm, sonst stirbt die Hefe ab.
            \item \textbf{Zeit zum Gehen:} Der Teig braucht Zeit, ideal über Nacht im Kühlschrank.
            \item \textbf{Butter für die Füllung:} Weiche, streichfähige Butter verwenden – geschmolzene Butter fließt heraus.
            \item \textbf{Schneiden:} Küchengarn unter die Rolle schieben und in entgegengesetzte Richtungen ziehen, statt mit dem Messer zu schneiden.
            \item \textbf{Zweites Gehen:} 30 Minuten in der Form gehen lassen für fluffige Schnecken.
        \end{itemize}
    }

\end{recipeDP}
