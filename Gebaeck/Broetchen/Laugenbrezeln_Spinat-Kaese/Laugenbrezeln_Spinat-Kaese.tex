\begin{recipeDP}
    [
        preparationtime = {\SI{60}{\minute}},
        bakingtime = {\SI{15}{\minute} bis \SI{20}{\minute}},
        bakingtemperature = {\protect\bakingtemperature{topbottomheat=\SI{180}{\celsius}}},
        portion = {16 Brezeln},
        source = {byanjushka.com}
    ]
    {Laugenbrezeln mit Spinat-Käse}

    \graph
        {
            big=Gebaeck/Broetchen/Laugenbrezeln_Spinat-Kaese/big.jpg,
            small=Gebaeck/Broetchen/Laugenbrezeln_Spinat-Kaese/small.jpg
        }

    % \introduction{
    %     EINLEITUNG
    % }

    \ingredients{
        \multicolumn{2}{l}{\textbf{Laugenbrezeln}} \\
        \SI{500}{\g} & Mehl \\
        \SI{1}{\TL} & Zucker \\
        \SI[parse-numbers = false]{\nicefrac{1}{2}}{\TL} & Salz \\
        \SI{125}{\ml} & Wasser \\
        \SI{25}{\g} & Margarine \\
        \SI{1}{\pck} & \addtoidx{Hefe} \\
        \\
        \multicolumn{2}{l}{\textbf{Spinat-Käse Füllung}} \\
		\SI{100}{\g} & Feta \\
		\SI{100}{\g} & Reibekäse \\
		\SI{100}{\g} & Frischkäse \\
		\SI{100}{\g} & TK Spinat aufgetaut und ausgedrückt \\
        1 & Knoblauchzehe \\
         & Salz \\
         & Pfeffer \\
        \\
        \multicolumn{2}{l}{\textbf{Lauge}} \\
        \SI{1}{\l} & Wasser \\
        \SI{30}{\g} & \addtoidx{Natron} \\
    }

    \preparation{
        \step Mehl, Zucker und Salz in eine große Schüssel geben und vermischen.
        Das Wasser mit der Margarine lauwarm erwärmen und die Hefe darin auflösen.
        Hefe-Gemisch in die Mehlmischung einrühren und etwa 10 Minuten zu einem elastischen Teig verkneten.
        Die Schüssel mit einem feuchten Tuch abdecken und für etwa 45 Minuten an einen warmen Ort stellen.
        \step Ein Backblech mit Backpapier auslegen.
        Den Hefeteig in etwa 16 gleich große Kugeln teilen (ca. 50-60 g pro Kugel). Auf dem Backblech platzieren und weitere 30 Minuten abgedeckt an einem warmen Ort ruhen lassen.
        \step Währenddessen die Füllung vorbereiten:
        Alle Zutaten dafür miteinander vermischen und nach Geschmack würzen.
        \step Nach der Ruhezeit je ein Teigbällchen zu einem Strang ausrollen (die Enden dabei dünner werden lassen) und den Strang mittig flach ausrollen.
        Etwas Füllung auf den Strang verteilen und erneut aufrollen, sodass die Füllung von dem Teig umschlossen ist.
        Zu Brezeln formen.
        Ofen auf 180 Grad Ober- und Unterhitze vorheizen.
        \step 1 Liter Wasser in einem Topf erhitzen.
        Sobald das Wasser kocht, vorsichtig das Natron hinzugeben und die Temperatur auf mittlere Stufe stellen.
        Die Brezeln 30 Sekunden in der Laugeköcheln lassen und herausnehmen.
        Abtropfen lassen und auf das Backblech geben.
        Im Ofen etwa 15 bis 20 Minuten gold-braun backen und herausnehmen.
    }

    % \suggestion[TITEL EINES VORSCHLAGS]{
	% 	VORSCHLAG (DURCH HORIZONTALE LINIE VOM REZEPT GETRENNT)
    % }

    % \hint{
    %     HINWEIS (IN EINEM KASTEN UNTEN AUF DER SEITE)
    % }

\end{recipeDP}