\begin{recipeDP}
    [
        preparationtime = {\SI{10}{\minute}},
        bakingtime = {\SI{15}{\minute} bis \SI{20}{\minute}},
        bakingtemperature = {\protect\bakingtemperature{fanoven=\SI{200}{\celsius}}},
        portion = {10 Brezeln},
        source = {chefkoch.de; @alphaht}
    ]
    {Tigerbrezeln}

    \graph
        {
            big=Gebaeck/Broetchen/Tigerbrezeln/big.jpg
            % small=TEIL/KAPITEL/REZEPT/small.jpg
        }

    % \introduction{
    %     EINLEITUNG
    % }

    \ingredients{
        10 & Laugenbrezeln (TK)\\
        \\
        \SI{50}{\g} & Kokosfett (Palmin) \\
        \SI{35}{\g} & Pflanzenöl \\
        \SI{10}{\g} & Meersalzflocken \\
        \SI{8}{\g} & schwarzer Pfeffer (geschrotet) \\
        \SI{12}{\g} & Maisstärke
    }

    \preparation{
        \step Das Fett zerlassen und etwas abkühlen lassen. 
        Dann die restlichen Zutaten untermischen und alles gut verrühren. 
        \step Die Paste am besten auf Teiglinge nach dem Gehen (Brezen nach dem Laugen) oder antauen streichen und die Teiglinge sofort (nach Packungsanleitung) backen.
        Etwa 15 Minuten bei \SI{200}{\celsius} Umluft backen.
    }

    % \suggestion[TITEL EINES VORSCHLAGS]{
	% 	VORSCHLAG (DURCH HORIZONTALE LINIE VOM REZEPT GETRENNT)
    % }

    \hint{
        Im Kühlschrank wird die Paste sehr fest, daher vor der Anwendung etwas temperieren. Sie sollte allerdings auch nicht zu flüssig sein, denn dann setzen sich Salz und Pfeffer unten ab und man streicht mehr Fett als Gewürze auf. 
        Am besten ist es, wenn die Paste eine cremige Konsistenz hat.
        Vor der Verwendung immer gut durchrühren und gleichmäßig mit einem Pinsel auftragen.
    }

\end{recipeDP}