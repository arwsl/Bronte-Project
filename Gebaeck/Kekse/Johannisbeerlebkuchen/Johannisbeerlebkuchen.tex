\begin{recipeDP}
    [
        preparationtime = {\SI{15}{\minute}},
        bakingtime = {\SI{12}{\minute} bis \SI{15}{\minute}},
        bakingtemperature = {\protect\bakingtemperature{fanoven=\SI{180}{\celsius}}},
        portion = {40 Stück},
        source = {Stina Spiegelberg}
    ]
    {Fruchtige Johannisbeerlebkuchen\index{Lebkuchen}}

    % \graph
    %     {
    %         small=Recipes/MainCourses/BBQChicken/Small.jpg,
    %         big=example-image
    %     }

    % \introduction{einleitung}

    \ingredients{
        \multicolumn{2}{l}{\textbf{Teig}} \\
        \SI{80}{\g} & Dinkelvollkornmehl \\
        \SI{100}{\g} & blanchierte, gemahlene Mandeln \\
        \SI{150}{\g} & gemischte, gemahlene Nüsse \\
        \SI{250}{\g} & Marzipanrohmasse \\
        \SI{180}{\g} & Rohrohrzucker \\
        \SI{1}{\pck} & Vanillezucker \\
        \SI[parse-numbers = false]{\nicefrac{1}{2}}{\TL} & Backpulver \\
        \SI{4}{\EL} & dunkle \index{Johannisbeermarmelade} \\
        \SI{60}{\ml} & Wasser \\
        etwa 40 & Backoblaten mit \SI{5}{\cm} \dmesser \\
        \\
        \multicolumn{2}{l}{\textbf{Dekor}} \\
        \SI{200}{\g} & Zartbitterkuvertüre \\
         & bunte Zuckerstreusel
    }

    \preparation{
        \step Mehl, Mandeln und Nüsse in eine große Rührschüssel geben, die Marzipanrohmasse mit einer groben Reibe in die trockenen Zutaten reibe. Dabei mehrmals rühren, damit das Marzipan nicht verklebt. Zucker, Backpulver Marmelade und Wasser zugeben und mit den Knethaken zu einem Teig verarbeiten. Es dürfen noch kleine Marzipanstückchen zu sehen sein.

        \step Den Teig kalt stellen, etwa über Nacht im Kühlschrank oder einige Stunden im Tiefkühler. Der Teig lässt sich dann einfacher verarbeiten und klebt nicht so sehr.

        \step Den Ofen auf \SI{180}{\celsius} vorheizen. Für die Zubereitung der Lebkuchen jeweils etwa \SI{1}{\TL} Teig auf eine Backoblate geben und mit den Fingern andrücken und bis an de Rand verteilen. Die Lebkuchen dann etwa \SI{12}{\minute} bis \SI{15}{\minute} backen, bis sie leicht braun werden. Dabei öfter in den Ofen schauen, damit nichts anbrennt.

        \step Die Lebkuchen über Nacht auskühlen lassen. Anschließend die Zartbitterkuvertüre im Wasserbad erwärmen und die Lebkuchen damit bestreichen. Mit den Zuckerstreuseln dann verzieren.
    }

    \suggestion[Zuckergussglasur]{
        Sehr gut passt auch eine Glasur aus etwas Zitronensaft und Puderzucker anstelle der Zartbitterkuvertüre. Dazu einige Tropfen Zitronensaft mit etwa \SI{100}{\g} Puderzucker verrühren, bis eine dickflüssige Glasur entsteht. Dann die Lebkuchen damit einstreichen.
    }

    \hint{Für das Original ``Elisenlebkuchen'' die Johannisbeermarmelade durch Aprikosenmarmelade ersetzen, je \SI{1}{\msp} gemahlenen Zimt und Koriander hinzufügen. Die Lebkuchen mit Kuvertüre und blanchierten Mandelhälften verzieren.}

\end{recipeDP}
