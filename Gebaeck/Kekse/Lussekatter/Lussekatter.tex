\begin{recipeDP}
    [
        preparationtime = {\SI{25}{\minute}},
        bakingtime = {\SI{10}{\minute} bis \SI{12}{\minute}},
        bakingtemperature = {\protect\bakingtemperature{topbottomheat=\SI{180}{\degree}}},
        portion = {70 Stück},
        source = {Stina Spiegelberg}
    ]
    {Lussekatter}

    % \graph
    %     {
    %         small=Recipes/MainCourses/BBQChicken/Small.jpg,
    %         big=example-image
    %     }

    \introduction{
        Lussekatter werden in Schweden am 13. Dezember zum Santa-Lucia-Fest, dem Lichterfest, gereicht. Übersetzt bedeutet der Name des leckeren Safrangebäcks ``Lucia-Katzen''.
    }

    \ingredients{
        \SI{300}{\g} & Weizenmehl (Typ 550) \\
        \SI{50}{\g} & blanchierte, gemahlene Mandeln \\
        \SI{100}{\g} & Rohrohrzucker \\
        \SI{1}{\msp} & Vanillepulver \\
        \SI{1}{\msp} & Ingwerpulver \\
        \SI{1}{Prise} & gemahlene Safranfäden\index{Safran} \\
        \nicefrac{1}{4} & Zitrone, abgeriebene Schale \\
        \SI{200}{\g} & vegane Butter, zimmerwarm \\
        \SI{3}{\EL} & Pflanzendrink \\
        \SI{30}{\g} & getrocknete Cranberrys
    }

    \preparation{
        \step Mehl, Mandeln, Zucker, Vanille, Gewürze und Orangenschale in einer Rührschüssel mischen. Mit veganer Butter und Pflanzendrink zu einem glatten Teig verkneten. Den Teig in Frischhaltefolie wickeln und eine Stunde kalt stellen.

        \step Den Backofen auf \SI{180}{\degree} Ober-/Unterhitze vorheizen.

        \step Den Teig auf einer bemehlten Arbeitsfläche zu langen Teigwürsten formen und gleich große, \ca \SI{10}{\cm} lange Stücke abschneiden. Diese Röllchen an beiden Enden in entgegengesetzter Richtung einrollen, sodass ein ``S'' entsteht. Oben und unten jeweils eine halbe Cranberry in das ``S'' drücken.

        \step Plätzchen auf ein mit Backpapier ausgelegtes Backblech legen und \SI{10}{\minute} bis \SI{12}{\minute} backen. Aus dem Ofen nehmen und vollständig auskühlen lassen.
    }

    % \suggestion[Title of Suggestion]{
	% 	Suggestion
    % }

    % \hint{Hint}

\end{recipeDP}
