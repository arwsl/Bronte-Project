\begin{recipeDP}
    [
        preparationtime = {\SI{25}{\minute}},
        bakingtime = {\SI{10}{\minute} bis \SI{12}{\minute}},
        bakingtemperature = {\protect\bakingtemperature{fanoven=\SI{180}{\celsius}}},
        portion = {13 Stück},
        source = {@annibacktvegan}
    ]
    {Mandelhörnchen}

    \graph
        {
            big=Gebaeck/Kekse/Mandelhoernchen/big.jpg,
            small=Gebaeck/Kekse/Mandelhoernchen/small.jpg
        }

    % \introduction{
    %     EINLEITUNG
    % }

    \ingredients{
        \SI{200}{\g} & \addtoidx{Marzipan!-rohmasse} \\
        \SI{100}{\g} & gemahlene Mandeln \index{Mandeln!gemahlen} \\
        \SI{100}{\g} & Weizenmehl \\
        \SI{80}{\g} & Zucker \\
        \SI{100}{\g} & Sojamilch \\
        \SI{20}{\g} & Speisestärke \\
        \SI[parse-numbers = false]{\nicefrac{1}{2}}{\TL} & Backpulver \\
        1 \pck & Vanillezucker \\
        1 Prise & Salz \\
    	\nicefrac{1}{2} Fläschchen & \addtoidx{Bittermandelaroma} \\
        \SI{100}{\g} & Mandelblätter\\
        \SI{200}{\g} & Zartbitterkuvertüre\\
    }

    \preparation{
        \step Das Marzipan klein schneiden und mit dme Zucker, dem Vanillezucker und der Sojamilch verkneten - am besten mit den Händen.
        Das Marzipan muss so fein werden, dass keine Stücke mehr spürbar sind.
        \step Anschließend Mehl, gemahlene Mandeln, Backpulver, Salz und Speisestärke dazu geben und zu einem Teig verkneten (dieser soll sehr klebrig sein).
        \step Nun die Mandelblätter auf der Arbeitsfläche verteilen.
        Einen Esslöffel der Teigmasse auf die Mandelblätter geben und auf den Mandelblättern zu einem Strang rollen.
        Mit befeuchteten Händen lässt es sich einfacher rollen.
        Den Strang dann zu einem Hörnchen formen und auf ein mit Backpapier belegtes Blech legen.
        Die gesamte Masse zu etwa 10 bis 14 Hörnchen verarbeiten.
        \step Abschließend bei \SI{180}{\celsius} für \SI{10}{\minute} bis \SI{12}{\minute} backen.
        Während die Hörnchen abkühlen, die Kuvertüre über einem heißen Wasserbad zerlassen und dann die Enden der Mandelhörnchen mit Kuvertüre bedecken.

    }

    % \suggestion[TITEL EINES VORSCHLAGS]{
	% 	VORSCHLAG (DURCH HORIZONTALE LINIE VOM REZEPT GETRENNT)
    % }

    % \hint{
    %     HINWEIS (IN EINEM KASTEN UNTEN AUF DER SEITE)
    % }

\end{recipeDP}