\begin{recipeDP}
    [
        preparationtime = {\SI{30}{\minute}},
        bakingtime = {\SI{15}{\minute}},
        bakingtemperature = {\protect\bakingtemperature{topbottomheat=\SI{160}{\celsius}}},
        % portion = {40 Stück},
        source = {@petadeutschland}
    ]
    {Kokosmakronen\index{Kokos!-makronen}}

    \graph
        {
    %         small=Recipes/MainCourses/BBQChicken/Small.jpg,
            big=Gebaeck/Kekse/Kokosmakronen/big.jpg
        }

    % \introduction{einleitung}

    \ingredients{
        \SI{70}{\ml} & Hafermilch \\
        \SI{200}{\g} & Puderzucker \\
        \SI{200}{\g} & Kokosraspeln\index{Kokos!-raspel} \\
        \SI{1}{\msp} & Backpulver \\
        \\
        \SI{100}{\g} & Zartbitterschokolade
    }

    \preparation{
        \step Die Hafermilch in eine Schüssel geben. Zucker nach und nach mit einem Schneebesen kräftig unterrühren, bis er sich komplett aufgelöst hat. Kokosraspeln gut mit dem Backpulver verrühren und sehr zügig unterheben.

        \step Den Backofen auf \SI{160}{\celsius} vorheizen. Ein Backblech mit Backpapier auslegen und mit Hilfe von zwei Teelöffeln kleine Makronen auf das Blech setzen. Kokosmakronen für \SI{15}{\minute} backen, bis sie leicht braun werden.

        \step Kokosmakronen gut abkühlen lassen, sonst fallen sie auseinander. Kuvertüre klein hacken, über dem Wasserbad schmelzen und die Kokosmakronen damit nach Belieben verzieren.
    }

    % \suggestion[Title of Suggestion]{
	% 	Suggestion
    % }
    %
    % \hint{Hint}

\end{recipeDP}
