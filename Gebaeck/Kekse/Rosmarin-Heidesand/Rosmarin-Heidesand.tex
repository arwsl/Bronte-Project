\begin{recipeDP}
    [
        preparationtime = {\SI{20}{\minute}},
        bakingtime = {\SI{15}{\minute}},
        bakingtemperature = {\protect\bakingtemperature{topbottomheat=\SI{190}{\celsius}}},
        portion = {50 Stück},
        source = {Stina Spiegelberg}
    ]
    {Rosmarin-\addtoidx{Heidesand}}

    % \graph
    %     {
    %         small=Recipes/MainCourses/BBQChicken/Small.jpg,
    %         big=example-image
    %     }

    % \introduction{einleitung}

    \ingredients{
        \SI{180}{\g} & Zucker \\
        \SI{1}{\pck} & Vanillezucker \\
        \SI{200}{\g} & vegane Butter, zimmerwarm \\
        \SI{300}{\g} & Weizenmehl (Typ 405) \\
        \SI[parse-numbers = false]{\nicefrac{1}{2}}{\TL} & Fleur de Sel \\
        \SI{2}{\TL} & \addtoidx{Rosmarin}, gehackt \\
        1 & Zitrone, abgeriebene Schale \\
        \SI{4}{\EL} & Pflanzendrink
    }

    \preparation{
        \step \SI{150}{\g} Zucker, Vanillezucker und vegane Butter in einer großen Schüssel mit dem Schneebesen schaumig schlagen. Mehl, Salz, Rosmarin, Zitronenschale und Pflanzendrink zugeben und zu einem glatten Teig kneten. Den Teig in Frischhaltefolie wickeln und eine Stunde kalt stellen.

        \step Den Backofen auf \SI{190}{\celsius} Ober-/Unterhitze vorheizen.

        \step Den Teig halbieren. Die Arbeitsfläche mit Mehl bestäuben und die Teighälften jeweils zu etwa \SI{40}{\cm} langen Broten formen. Den restlichen Zucker auf die Arbeitsfläche streuen und die Brote darin wälzen. Erneut in Frischhaltefolie wickeln und 30 Minuten kalt stellen.

        \step Von den Broten etwa \SI{1}{\cm} dicke Stücke abschneiden die Plätzchen auf einem mit Backpapier ausgelegten Backblech \ca 15 Minuten backen. Auskühlen lassen und luftdicht aufbewahren.
    }

    % \suggestion[Title of Suggestion]{
	% 	Suggestion
    % }

    \hint{Für die klassischen Heidesand-Plätzchen das Fleur de Sel durch eine Prise Meersalz ersetzen, den Rosmarin weglassen und nur den Abrieb einer halben Zitrone verwenden}

\end{recipeDP}
