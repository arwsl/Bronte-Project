\begin{recipe}
    [
        preparationtime = {\SI{25}{\minute}},
        bakingtime = {\SI{40}{\minute} bis \SI{50}{\minute}},
        bakingtemperature = {\protect\bakingtemperature{fanoven=\SI{175}{\degree}}},
        portion = {1 Baguette},
        source = {Sabine Schramm}
    ]
    {Mediterranes Baguette}

    % \graph
    %     {
    %         small=Recipes/MainCourses/BBQChicken/Small.jpg,
    %         big=example-image
    %     }

    % \introduction{einleitung}

    \ingredients{
        \SI{1}{\EL} & Chiasamen \\
        \SI{3}{\EL} & Wasser \\
        \SI{200}{\g} & Sojajoghurt \\
        \SI{130}{\g} & Dinkelvollkornmehl \\
        \SI{40}{\g} & Haferkleie \\
        \SI{1}{\EL} & Tomatenmark \\
        \SI{100}{\g} & Pepperoni \\
        \SI{1}{\TL} & Salz \\
        \SI{1}{\TL} & Thymian \\
        \SI{1}{\TL} & Basilikum \\
        \SI{1}{\pck} & Backpulver
    }

    \preparation{
        \step Zuerst mit den Chiasamen und dem Wasser ein veganes Ei vorbereiten: beides miteinander verrühren und etwa \SI{10}{\minute} lang stehen lassen. Währenddessen die Peperoni in Ringe schneiden. Den Ofen auf \SI{175}{\degree} Umluft vorheizen.

        \step Für das Baguette dann den Quark mit dem veganen Ei verrühren, dann alle anderen Zutaten hinzugeben. Gut miteinander verkneten, bis ein homogener Teig entsteht. Den Teig zu einem Baguette formen. Da der Teig sehr klebrig ist, helfen nasse Hände beim formen. Das Baguette auf einem mit Backpapier bedeckten Backblech für \SI{40}{\minute} bis \SI{50}{\minute} backen.

    }

    % \suggestion[Title of Suggestion]{
	% 	Suggestion
    % }
    %
    % \hint{Hint}

\end{recipe}
