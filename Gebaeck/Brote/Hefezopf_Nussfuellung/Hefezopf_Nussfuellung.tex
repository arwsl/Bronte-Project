\begin{recipeDP}
    [
        preparationtime = {\SI{40}{\minute}},
        bakingtime = {\SI{20}{\minute}},
        bakingtemperature = {\protect\bakingtemperature{fanoven=\SI{180}{\celsius}}},
        portion = {4 mittelgroße Kringel},
        source = {zuckerjagdwurst.com}
    ]
    {Hefezopf \index{Hefe!-zopf} mit Nussfüllung}

    \graph
        {
            big=Gebaeck/Brote/Hefezopf_Nussfuellung/big.jpg,
            small=Gebaeck/Brote/Hefezopf_Nussfuellung/small.jpg
        }

    % \introduction{
    %     EINLEITUNG
    % }

    \ingredients{
        \multicolumn{2}{l}{\textbf{Hefezopf}} \\
        \SI{250}{\ml} & Hafermilch \\
        \SI{500}{\g} & Weizenmehl \\
        \SI{50}{\g} & Zucker \\
        \SI{1}{\pck} & Trockenhefe \\
        \SI{75}{\g} & Margarine \\
        1 Prise & Salz \\
        \\
        \multicolumn{2}{l}{\textbf{Nussfüllung}} \\
        \SI{100}{\ml} & Hafermilch \\
        \SI{100}{\g} & Rohrohrzucker \\
        \SI{1}{\TL} & \addtoidx{Zimt} \\
        \SI{200}{\g} & gehackte Mandeln \index{Mandeln!gehackt} \\
        \SI{100}{\g} & gemahlene Haselnüsse \index{Haselnüsse!gemahlen} \\
        \SI[parse-numbers = false]{\nicefrac{1}{2}}{} & Zitrone \\
        \\
        \multicolumn{2}{l}{\textbf{zum Bestreichen}} \\
        \SI{4}{\EL} & Hafermilch \\
        \SI{1}{\EL} & \addtoidx{Agavendicksaft} \\
        \\
        \SI{100}{\g} & Puderzucker \\
        \SI{2}{\EL} & Hafermilch \\
    }

    \preparation{
        \step Das Mehl in eine große Schüssel geben und mit dem Zucker, dem Salz und der Hefe vermengen.
        Die Margarine mit der Milch erwärmen, bis sie lauwarm ist.
        Alles zu einem Teig verkneten und eine Stunde an einem warmen Ort stehen lassen.
        \step Für die Füllung die Milch in einem kleinen Topf erhitzen und Zucker und Zimt einrühren.
        Vom Herd nehmen und die gehackten und gemahlenen Nüsse sowie den Abrieb der Zitrone unterrühren.
        \step Den Hefeteig in vier teilen und jede der Teigkugeln nach und nach zu einem großen Oval ausrollen.
        Die Füllung auf den vier Ovalen gleichmäßig verstreichen.
        Jedes Oval nun von der längeren Seite aufrollen, sodass eine lange Rolle entsteht.
        Jede Rolle längs mit einem scharfen Messer halbieren und nach oben aufklappen, sodass alle Schichten nach oben zeigen.
        Die zwei entstandenen Teigstränge umeinanderlegen und zu einem Kreis formen.
        Die Enden gut miteinander verbinden.
        Mit jedem Teig so vorgehen und die vier Hefezöpfe nochmals 30 Minuten gehen lassen, bevor sie gebacken werden.
        \step Den Ofen auf \SI{180}{\celsius} (Umluft) vorheizen. Milch und Agavendicksaft vermengen und die Hefezöpfe damit bestreichen. Die Hefezöpfe bei \SI{180}{\celsius} (Umluft) auf mittlerer Stufe etwa 20 Minuten backen.
        \step Für den Zuckerguss die Milch mit Puderzucker verrühren und auf den abgekühlten Hefezöpfen verteilen.
    }

    % \suggestion[TITEL EINES VORSCHLAGS]{
	% 	VORSCHLAG (DURCH HORIZONTALE LINIE VOM REZEPT GETRENNT)
    % }

    % \hint{
    %     HINWEIS (IN EINEM KASTEN UNTEN AUF DER SEITE)
    % }

\end{recipeDP}
