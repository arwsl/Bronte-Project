\begin{recipeDP}
    [
        preparationtime = {\SI{15}{\minute}},
        bakingtime = {\SI{70}{\minute}},
        bakingtemperature = {\protect\bakingtemperature{fanoven=\SI{180}{\celsius}}},
        portion = {1 Kastenform (25 cm)},
        source = {zuckerjagdwurst.com},
    ]
    {Zucchinibrot mit Feta}

    \graph
        {
            big=Gebaeck/Brote/Zucchinibrot_mit_Feta/big.jpg,
            small=Gebaeck/Brote/Zucchinibrot_mit_Feta/small.jpg
        }

    % \introduction{
    %     EINLEITUNG
    % }

    \ingredients{
        \SI{400}{\g} & Weizenmehl (Typ 405) \\
        \SI{2}{\TL} & Backpulver \\
        \SI{2}{\TL} & Salz \\
        \SI{100}{\g} & Nüsse (z.B. Pekannüsse\index{Nüsse!Pekan-}) \\
        \SI{120}{\g} & \addtoidx{Oliven} \\
        \SI{200}{\g} & \addtoidx{Zucchini} \\
        \nicefrac{1}{2} & Zitrone (Abrieb) \\
        \SI{400}{\ml} & Pflanzendrink \\
        \SI{1}{\EL} & Olivenöl \\
        \SI{150}{\g} & Feta
    }

    \preparation{
        \step Den Ofen auf \SI{180}{\celsius} (Umluft) vorheizen.
        Mehl, Backpulver und Salz in eine große Schüssel geben.
        Die Nüsse klein hacken und ebenfalls dazugeben und kurz vermischen.
        \step Die Oliven halbieren, die Zucchini grob reiben, die Zitronenschale fein abreiben und zusammen mit der pflanzlichen Milch und dem Olivenöl in die Schüssel geben und gut verrühren.
        Zuletzt den veganen Feta in kleine Stücke schneiden oder mit den Händen zerbröseln, zu den anderen Zutaten hinzugeben und vorsichtig unterheben.
        \step Den Teig in eine eingefettete Form geben und 60 bis 70 Minuten bei \SI{180}{\celsius} (Umluft) backen.
        Mit einem Stäbchen testen, ob das Brot durch ist --- dafür einen Holzspieß schräg in das Brot stecken und herausziehen.
        Wenn keine Teigreste daran kleben bleiben, ist das Brot fertig.
        (Durch den veganen Feta können aber durchaus Reste am Spieß zu sehen sein können.)
        Das Zucchinibrot aus dem Ofen nehmen und komplett abkühlen lassen.
    }

    % \suggestion[TITEL EINES VORSCHLAGS]{
	% 	VORSCHLAG (DURCH HORIZONTALE LINIE VOM REZEPT GETRENNT)
    % }

    % \hint{
    %     HINWEIS (IN EINEM KASTEN UNTEN AUF DER SEITE)
    % }

\end{recipeDP}