\begin{recipeDP}
    [
        preparationtime = {\SI{10}{\minute}},
        bakingtime = {\SI{40}{\minute}},
        bakingtemperature = {\protect\bakingtemperature{fanoven=\SI{200}{\celsius}}},
        portion = {1 Kastenform},
        source = {bbc goodfood}
    ]
    {Bananenbrot}

    % \graph
    %     {
    %         small=Recipes/MainCourses/BBQChicken/Small.jpg,
    %         big=example-image
    %     }

    % \introduction{einleitung}

    \ingredients{
        3 & große \addtoidx{Bananen} \\
        \SI{75}{\ml} & Öl \\
        \SI{100}{\g} & brauner Zucker \\
        \SI{225}{\g} & Weizenmehl \\
        \SI{3}{\geh \TL} & Backpulver \\
        \SI{3}{\TL} & Zimt oder Gewürzmischung\\
	\SI{60}{\ml} & Sojamilch \\
         & getrocknete Früchte
    }

    \preparation{
        \step Sen Ofen auf \SI{200}{\celsius} vorheizen. Die Bananen mit einer Gabel zerstampfen und dann Öl und den Zucker hinzufügen und verrühren.

        \step Das Weizenmehl, Backpulver und den Zimt oder die Gewürze hinzufügen und gut miteinander verrühren. Die Sojamilch unterrühren, sodass der Teig eine cremige Konsistenz erhält. An dieser Stelle können optional getrocknete Früchte unter gerührt werden, da sollten aber insgesamt etwa \SI{100}{\ml} Sojamilch verwendet werden.

        \step Dann den Teig in eine Kastenform füllen. Für \SI{20}{\minute} backen und dann nachsehen, ob das Bananenbrot mit Folie abgedeckt werden muss. Weitere \SI{20}{\minute} backen, bis die Garprobe erfolgreich ist. Etwas abkühlen lassen vor dem Servieren.
    }

    % \suggestion[Title of Suggestion]{
	% 	Suggestion
    % }

    \hint{Gut dazu passt: Erdnussbutter}

\end{recipeDP}
