\begin{recipeDP}
    [
        preparationtime = {\SI{20}{\minute}},
        bakingtime = {\SI{35}{\minute}},
        bakingtemperature = {\protect\bakingtemperature{fanoven=\SI{160}{\celsius}}},
        portion = {12 Stück},
        source = {Dr. Oetker}
    ]
    {\addtoidx{Apfel}-Knusper-Muffins}

    \graph
        {
            small=Gebaeck/Kuchen/Apfel-Knusper-Muffins/small.jpg,
            % big=example-image
        }

    % \introduction{einleitung}

    \ingredients{
        \multicolumn{2}{l}{\textbf{\addtoidx{Streuselteig}}} \\
        \SI{60}{\g} & Margarine \\
        \SI{50}{\g} & kernige Haferflocken \\
        \SI{50}{\g} & Weizenmehl \\
        \SI{60}{\g} & Rohrzucker \\
        \\
        \multicolumn{2}{l}{\textbf{Muffinteig}} \\
        \ca \SI{150}{\g} & Apfel \\
        \SI{1}{\pck} & Backpulver \\
        \SI{1}{\TL} & Zimt \\
        \SI{125}{\g} & Rohrzucker \\
        \SI{125}{\ml} & Sonnenblumenöl \\
        \SI{250}{\ml} & Sojadrink
    }

    \preparation{
        \step Zuerst eine Muffinform mit Papierförmchen vorbereiten und den Backofen auf \SI{160}{\celsius} Umluft vorheizen.

        \step Für den Streuselteig alle Zutaten in eine Rührschüssel geben und mit den Rührstäben eines Handrührgerätes auf niedrigster Stufe zu Streuseln verarbeiten.

        \step Für den Muffinteig den Apfel schälen vierteln und in kleine Würfel schneiden. Mehl und Backpulver in einer Rührschüssel mischen. Übrige Zutaten hinzufügen und alles mit den Rührstäben des Mixers kurz auf niedrigster, dann auf höchster Stufe \ca \SI{2}{\minute} zu einem glatten Teig verarbeiten. Dann die Apfelstückchen unterheben und den Teig mit Hilfe von zwei Esslöffeln in die Muffinförmchen geben.

        \step Die Streusel darauf verteilen und die Form auf dem Rost in den Backofen schieben. Auf mittlerem Einschub etwa \SI{35}{\minute} backen. Anschließend die Muffins mit Hilfe von zwei Gabeln aus der Form nehmen und auf einem Kuchenrost erkalten lassen.
    }

    % \suggestion[Title of Suggestion]{
	% 	Suggestion
    % }
    %
    \hint{Das einfache Rezept lässt sich auch in einer runden Backform backen. Dann unter Umständen abdecken und die Backzeit verlängern. Garprobe machen!}

\end{recipeDP}
