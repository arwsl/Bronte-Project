\begin{recipeDP}
    [
        preparationtime = {\SI{20}{\minute}},
        bakingtime = {\SI{45}{\minute}},
        bakingtemperature = {\protect\bakingtemperature{fanoven=\SI{180}{\celsius}}},
        portion = {1 Kastenform},
        source = {@elavegan}
    ]
    {Ananaskuchen\index{Ananas}}

    \graph
        {
            big=Gebaeck/Kuchen/Ananaskuchen/big.jpg,
            small=Gebaeck/Kuchen/Ananaskuchen/small.jpg
        }

    % \introduction{
    %     EINLEITUNG
    % }

    \ingredients{
        \multicolumn{2}{l}{\textbf{Teig}} \\
        \SI{160}{\g} & \addtoidx{Reismehl} \\
        \SI{90}{\g} & Haferflocken \\
        \SI{75}{\g} & Kokosraspeln \\
        \SI{2}{\EL} & Leinsamen \\
        \SI[parse-numbers = false]{1 \nicefrac{1}{2}}{\TL} & Backpulver \\
        \SI[parse-numbers = false]{\nicefrac{1}{2}}{\TL} & Natron \\
        \SI{1}{\pck} & Vanillezucker \\
        \\
        \SI{350}{\g} & Ananas \\
        \SI{100}{\ml} & \addtoidx{Kokosmilch} \\
        \SI{75}{\g} & Ahornsirup \\
        \\
        \multicolumn{2}{l}{\textbf{Glasur}} \\
        \SI{50}{\g} & Puderzucker \\
        \SI{1}{\EL} & Kokosmilch \\
		\SI[parse-numbers = false]{\nicefrac{1}{2}}{\EL} & Zitronensaft \\
    }

    \preparation{
        \step 1/2 kleine/mittlere Ananas (geschält) in grobe Stücke schneiden und in einem Mixer zu Püree pürieren. Alternativ lässt sich auch Ananas aus der Dose verwenden, die nur sehr gut mit Wasser abspülen und damit etwas vom Zucker befreien. Das Püree beiseite stellen.
        \step Den Ofen auf \SI{180}{\celsius} vorheizen und eine 23-cm-Kastenform mit Backpapier auslegen.
        \step Alle trockenen Zutaten in einer Küchenmaschine oder einem Mixer zerkleinern. Die feuchten zu den trockenen Zutaten geben und erneut pulsieren, bis alles vermischt ist. Die Mischung in die Form füllen und im Ofen ca. \SI{45}{\minute} backen, oder bis ein Zahnstocher sauber herauskommt.
        \step Das Brot abkühlen lassen, dann die Zutaten für die Glasur in einer Schale mit einem Schneebesen verrühren. Den Guss über das Brot streichen und trocknen lassen.
    }

    % \suggestion[TITEL EINES VORSCHLAGS]{
	% 	VORSCHLAG (DURCH HORIZONTALE LINIE VOM REZEPT GETRENNT)
    % }

    \hint{
        Statt Reismehl lässt sich auch normales verwenden (wenn es nicht glutenfrei sein muss). Die Kokosraspeln lassen sich auch durch die gleiche Menge gemahlener Mandeln oder Nüsse ersetzen. Der Ahornsirup kann auch durch jedes andere flüssige Süßungsmittel ersetzt werden.
    }

\end{recipeDP}
