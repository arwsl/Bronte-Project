\begin{recipeDP}
    [
        preparationtime = {\SI{25}{\minute}},
        bakingtime = {\SI{40}{\minute} bis \SI{55}{\minute}},
        bakingtemperature = {\protect\bakingtemperature{fanoven=\SI{160}{\celsius}/\SI{180}{\celsius}}},
        portion = {24er Springform},
        source = {knusperstuebchen.net}
    ]
    {Marzipankuchen mit Apfelmus}

    \graph
        {
            big=Gebaeck/Kuchen/Marzipankuchen_mit_Apfelmus/big.jpg,
            small=Gebaeck/Kuchen/Marzipankuchen_mit_Apfelmus/small.jpg
        }

    % \introduction{
    %     EINLEITUNG
    % }

    \ingredients{
        \multicolumn{2}{l}{\textbf{Boden}} \\
        \SI{120}{\g} & \addtoidx{Marzipan} \\
        \SI{140}{\g} & Margarine \\
        \SI{100}{\g} & Zucker \\
        \SI{1}{\pck} & Vanillezucker \\
        \SI{3}{\EL} & Leinsamenschrot \\
        \SI{3}{\EL} & Wasser \\
        \SI{230}{\g} & Mehl \\
        \SI{2}{\TL} & Backpulver \\
        1 Prise & Salz \\
        \SI{50}{\ml} & Sojamilch \\
        \SI{70}{\ml} & Sahne \\
        \SI{200}{\g} & gehackte Mandeln \\
        \SI{1}{\TL} & Margarine \\
        \SI{3}{\EL} & Zucker \\
        \\
        \multicolumn{2}{l}{\textbf{Apfelmus-Füllung}} \\
        3 - 4 & Äpfel\index{Apfel} \\
        \SI{2}{\EL} & Wasser \\
        \SI{40}{\g} & Zucker \\
        1 Prise & \addtoidx{Zimt} \\
        \SI{1}{\EL} & Mehl \\
        \SI{1}{\EL} & Margarine \\
        \\
        \multicolumn{2}{l}{\textbf{Streusel}} \\
        \SI{50}{\g} & Marzipan \\
        \SI{80}{\g} & Mehl \\
        \SI{50}{\g} & Zucker \\
        \SI{40}{\g} & Margarine
    }

    \preparation{
        \step Für das Apfelmus: Äpfel mit dem Wasser, dem Zucker und Zimt in einen Topf geben und zum Kochen bringen.
        Solange Kochen bis die Äpfel sehr weich sind. Nun leicht mit einer Gabel zerdrücken.
        Nochmals aufkochen lassen und unter Rühren Mehl in die Masse sieben, weiter rühren bis die Masse gebunden ist, Butter unterrühren und beiseite stellen.
        \step ür den Boden: Ofen auf \SI{160}{\celsius} Umluft vorheizen.
        Zunächst 3EL Zucker gemeinsam mit 1TL Butter in einer Pfanne schmelzen.
        Die gehackten Mandeln hinzugeben und leicht anrösten, sie sollten von allen Seiten mit Karamell überzogen sein.
        Von der Herdplatte nehmen.
        Nochmals rühren und in eine Schüssel zum Abkühlen geben.
        \step Aus Leinsamen und Wasser einen Eiersatz ansetzen.
        Nun Marzipan und Butter in eine Schüssel geben und cremig rühren.
        Zucker und Vanillezucker hinzugeben, weiter rühren.
        Nun nach und nach die gequollenen Leinsamen unterrühren.
        Mehl, Backpulver und Prise Salz mischen.
        Milch und Sahne mischen und die flüssigen und die trockenen Zutaten abwechselnd in den Teig rühren.
        Zum Schluss etwa \SI{150}{\g} der vorbereiteten, gehackten Mandeln unterheben.
        Teig in eine gefetteteSpringform geben und gleichmäßig verteilen.
        Etwa 35 Minuten backen. Stäbchenprobe machen,der Boden sollte gerade so durchgebacken sein.
        \step Währenddessen für die Streusel: Alle Zutaten sowie die restlichen gehackten Mandeln in eine Schüssel geben und zu einer krümeligen Masse verrühren, notfalls mehr Mehl hinzugeben.
        Sobald der Teig gebacken ist, diesen kurz aus dem Ofen nehmen, das Apfelmus auf dem gesamten Teig verteilen und mit den Streuseln bestreuen. Nochmals bei \SI{180}{\celsius} etwa 10 bis 15 Minuten backen bis die Streusel goldbraun sind.
    }

    % \suggestion[TITEL EINES VORSCHLAGS]{
	% 	VORSCHLAG (DURCH HORIZONTALE LINIE VOM REZEPT GETRENNT)
    % }

    \hint{
        Statt der frischen Äpfel können auch 600 ml Apfelmus aus dem Glas verwendet werden.
    }

\end{recipeDP}