\begin{recipeDP}
    [
        preparationtime = {\SI{25}{\minute}},
        bakingtime = {\SI{35}{\minute} bis \SI{40}{\minute}},
        bakingtemperature = {\protect\bakingtemperature{topbottomheat=\SI{180}{\celsius}}},
        portion = {20 x 20 cm Backform},
        source = {elavegan.com}
    ]
    {Dattelkuchen}

    \graph
        {
            big=Gebaeck/Kuchen/Dattelkuchen/big.png,
            small=Gebaeck/Kuchen/Dattelkuchen/small.png
        }

    % \introduction{
    %     EINLEITUNG
    % }

    \ingredients{
        \SI{250}{\g} & Medjool-Datteln\\
        \SI{360}{\ml} & Hafermilch\\
        \SI{170}{\g} & Äpfel\\
        \SI{60}{\g} & Tahini\\
        \SI{1}{\pck} & Vanillezucker\\
        \SI{180}{\g} & Hafermehl\\
        \SI{60}{\g} & gemahlene Haselnüsse\\
        \SI{2}{\TL} & Zimt\\
        \SI[parse-numbers = false]{1\nicefrac{1}{2}}{\TL} & Backpulver \\
        \SI[parse-numbers = false]{\nicefrac{1}{4}}{\TL} & Salz \\
         & gehackte Haselnüsse
    }

    \preparation{
        \step Die Milch in einem Topf bei mittlerer Hitze erhitzen, bis sie leicht köchelt. In der Zwischenzeit die Datteln entsteinen und in eine Schüssel geben. Die heiße Milch darüber gießen und etwa 10 Minuten einweichen lassen.
        \step Den Apfel grob reiben.
        Die eingeweichten Datteln mit der Milch, Tahini und Vanillezucker in einem Mixer (oder mit einem Pürierstab) cremig pürieren. Gleichzeitig den Ofen auf \SI{180}{\celsius} (Ober-/Unterhitze) vorheizen und eine quadratische Backform (20 cm) mit Backpapier auslegen.
        \step In einer großen Schüssel das Dattelpüree, den geriebenen Apfel, die gemahlenen Haselnüsse, das Hafermehl, Zimt, Backpulver und Salz gut vermengen.
        Den Teig gleichmäßig in die vorbereitete Form geben, glatt streichen und mit gehackten Haselnüssen bestreuen.
        \step Den Dattelkuchen 35 bis 40 Minuten backen, bis ein Zahnstocher sauber herauskommt. Anschließend auf einem Gitter vollständig auskühlen lassen und in Stücke schneiden.
    }

    % \suggestion[TITEL EINES VORSCHLAGS]{
	% 	VORSCHLAG (DURCH HORIZONTALE LINIE VOM REZEPT GETRENNT)
    % }

    % \hint{
    %     HINWEIS (IN EINEM KASTEN UNTEN AUF DER SEITE)
    % }

\end{recipeDP}