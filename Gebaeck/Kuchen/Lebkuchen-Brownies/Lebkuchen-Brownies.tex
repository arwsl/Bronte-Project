\begin{recipe}
    [
        preparationtime = {\SI{20}{\minute}},
        bakingtime = {\SI{40}{\minute}},
        bakingtemperature = {\protect\bakingtemperature{fanoven=\SI{160}{\celsius}}},
        portion = {1 Backform 20 x 20 cm},
        source = {Stina Spiegelberg}
    ]
    {\addtoidx{Lebkuchen}-Brownies}

    % \graph
    %     {
    %         small=Recipes/MainCourses/BBQChicken/Small.jpg,
    %         big=example-image
    %     }

    % \introduction{einleitung}

    \ingredients{
        \textbf{Rührteig} \\
        \SI{200}{\g} & Marzipanrohmasse \\
        \SI{40}{\ml} & Öl \\
        \SI{250}{\g} & Weizenmehl (Typ 405) \\
        \SI{60}{\g} & Rohrohrzucker \\
        \SI{1}{\pck} & Vanillezucker \\
        \SI{200}{\g} & gemahlene Haselnüsse \\
        \SI{1}{\TL} & Lebkuchengewürz \\
        \SI{2}{\EL} & Kakao \\
        \SI{1}{Prise} & Salz \\
        \SI{1}{\geh\TL} & Hirschhornsalz \\
        \SI{50}{\g} & Orangeat \\
        \SI{50}{\g} & Aprikosenmarmelade \\
        \SI{300}{\ml} & stilles Wasser \\
         & Fett zum Einfetten \\
        \\
        \textbf{Dekor} \\
        \SI{60}{\g} & Halbbitterschokolade \\
        \SI{2}{\EL} & Hafer Cuisine \\
        \SI{4}{\EL} & Puderzucker
    }

    \preparation{
        \step Den Backofen auf \SI{160}{\celsius} vorheizen.

        \step Die Marzipanrohmasse mit dem Öl kurz im Mixer verrühren. Mehl, Zucker, Vanillezucker, Haselnüsse, Lebkuchengewürz, Kakao, Salz und Hirschhornsalz in einer Rührschüssel kurz mischen. Das Orangeat fein hacken, mit der Marzipanmasse, der Marmelade und dem Wasser dazugeben und mit dem Schneebesen zu einem glatten Teig rühren. In die gefettete Backform geben und ca. 40 Minuten backen. Mit einem Holzstäbchen die Garprobe machen und abkühlen lassen.

        \step Die Schokolade hacken. Die Pflanzensahne erwärmen, über die Schokolade gießen und so lange rühren, bis die Schokolade vollständig geschmolzen ist. Dann den Puderzucker einsieben. Die Mischung auf dem Kuchen verteilen und auskühlen lassen.
    }

    % \suggestion[Title of Suggestion]{
    %     Suggestion
    % }

    % \hint{Hint}

\end{recipe}
