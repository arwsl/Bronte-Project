\begin{recipeDP}
    [
        preparationtime = {\SI{15}{\minute}},
        bakingtime = {\SI{20}{\minute} bis \SI{30}{\minute}},
        bakingtemperature = {\protect\bakingtemperature{topbottomheat=\SI{175}{\celsius}}},
        portion = {9 Stück},
        source = {@healthygirlkitchen}
    ]
    {REZEPTNAME}

    \graph
        {
            big=Gebaeck/Kuchen/Funfetti-Muffins/big.jpg,
            small=Gebaeck/Kuchen/Funfetti-Muffins/small.jpg
        }

    % \introduction{
    %     EINLEITUNG
    % }

    \ingredients{
        \SI{100}{\g} & Haferflocken \\
        \SI{25}{\g} & gemahlene Mandeln \\
        \SI{30}{\g} & Mehl \\
        \SI{15}{\g} & Zucker \\
        1 & reife Banane \\
        \SI{180}{\ml} & Mandelmilch \\
        \SI{1}{\Pck} & Backpulver \\
        \SI{1}{\Pck} & Vanillezucker \\
        \SI{50}{\g} & bunte Zuckerstreusel \\
        \\
        \SI{20}{\g} & Puderzucker \\
         & Zitronensaft
    }

    \preparation{
        \step Zuerst den Ofen auf \SI{175}{\celsius} vorheizen. Für den Teig dann die Haferflocken in einem Mixer zerhäckseln, bis Hafermehl entsteht.
        \step Die Banane mit der Mandelmilch, dem Zucker und dem Vanillezucker in einem Mixer pürieren, bis eine glatte Flüssigkeit entsteht. In einer Rührschüssel dann das Hafermehl mit den gemahlenen Mandeln, dem Mehl und dem Backpulver vermengen. Die vorbereitete Flüssigkeit mit dem Handrührgerät einrühren und abschließend die Zuckerstreusel unterheben.
        \step Den Teig in neun Muffinformen füllen. Die einzelnen Formen sollten etwa zu drei Vierteln gefüllt sein. Die Muffins bei Ober- und Unterhitze für etwa \SI{20}{\minute} Bachen (Garprobe mit einem Zahnstocher nicht vergessen).
        \step Nach dem Auskühlen die Muffins mit einem Zuckerguss aus Puderzucker und etwas Zitronensaft bestreichen. Nach Belieben mit weiteren Zuckerstreuseln verzieren.
    }

    % \suggestion[TITEL EINES VORSCHLAGS]{
	% 	VORSCHLAG (DURCH HORIZONTALE LINIE VOM REZEPT GETRENNT)
    % }
    %
    \hint{
        Mit dem doppelten Rezept lässt sich auch eine runde Kuchenform füllen. Die Backzeit beträgt dann etwa \SI{45}{\minute} (Stäbchenprobe nicht vergessen).
    }

\end{recipeDP}
