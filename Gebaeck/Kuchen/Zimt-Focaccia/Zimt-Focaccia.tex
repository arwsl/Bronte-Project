\begin{recipeDP}
    [
        preparationtime = {\SI{30}{\minute}},
        bakingtime = {\SI{20}{\minute}},
        bakingtemperature = {\protect\bakingtemperature{topbottomheat=\SI{220}{\celsius}}},
        portion = {20 x 20 cm Form},
        source = {sallys-blog.de}
    ]
    {Zimt-Focaccia}

    \graph
        {
            big=Gebaeck/Kuchen/Zimt-Focaccia/big.jpg,
            small=Gebaeck/Kuchen/Zimt-Focaccia/small.jpg
        }

    \introduction{
        Dieses Focaccia lässst sich ganz einfach ohne Maschine zubereiten!
    }

    \ingredients{
        \multicolumn{2}{l}{\textbf{Vorteig}} \\
        \SI{300}{\ml} & Wasser \\
        \SI{5}{\g} & frische Hefe \\
        \SI{1}{\TL} & Zucker \\
        \SI{100}{\g} & Mehl \\
        \\
        \multicolumn{2}{l}{\textbf{Teig}} \\
        \SI{300}{\g} & Mehl \\
        \SI{1}{\TL} & Salz \\
        \SI{15}{\g} & Olivenöl \\
        \\
        \multicolumn{2}{l}{\textbf{Topping}} \\
        \SI{1}{\EL} & Olivenöl \\
        \SI{75}{\g} & Zucker \\
        \SI{1}{\TL} & Zimt \\
        \SI{2}{\EL} & Butter \\
         & Apfel, nach Bedarf
    }

    \preparation{
        \step Am Vortag den Vorteig zubereiten: Dafür alle Zutaten des Vorteiges zusammen geben und glatten rühren.
        Den Teig eine halbe Stunde ruhen lassen.
        \step Danach die Zutaten des Teiges dazu geben und zu einem glatten Teig rühren (der Teig soll recht flüssig sein).
        Den Teig luftdicht verschließen und über Nacht in den Kühlschrank geben.
        \step Am nächsten Tag das Focaccia auf eine beölte Arbeitsfläche geben, auseinander ziehen und kurz ruhen lassen.
        Für das Topping Zucker und Zimt vermengen und eine Hälfte auf dem flachen Teig verteilen.
        Dann den Teig von allen Seiten Falten, sodass eine glatte Unterseite entsteht.
        Dabei mehr vom Zimt-Zucker einstreuen.
        Am Ende mit der Naht nach unten in eine eingeölte 20 mal 20 Zentimeter Backform geben und von oben nochmal mit dem restlichen Topping bestreuen.
        Mit der geschmolzenen Butter beträufeln und mit den Fingerspitzen in den Teig drücken.
        Nochmal für \SI{20}{\minute} gehen lassen.
        Dann bei \SI{220}{\celsius} für \SI{20}{\minute} backen.


    }

    % \suggestion[Apfeltopping]{
	% 	Das süße Focaccia lässt sich auch gut vor dem Backen mit kleinen Apfelstücken bedecken.
    % }

    \hint{
        Anstatt den Vorteig eine Nacht im Kühlschrank stehen zu lassen, kann man die Hefe auch auf \SI{15}{\g} verdreifachen und einen Esslöffel Essig hinzu geben.
        Dann sind nur noch 2 Stunden Ruhezeit nötig.
    }

\end{recipeDP}