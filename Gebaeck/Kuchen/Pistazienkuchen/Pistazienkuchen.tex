\begin{recipeDP}
    [
        preparationtime = {\SI{15}{\minute}},
        bakingtime = {\SI{60}{\minute}},
        bakingtemperature = {\protect\bakingtemperature{topbottomheat=\SI{180}{\celsius}}},
        portion = {1 Kastenform},
        source = {@ginamarie.grimm}
    ]
    {Pistazienkuchen}

    \graph
        {
            big=Gebaeck/Kuchen/Pistazienkuchen/big.jpg,
            small=Gebaeck/Kuchen/Pistazienkuchen/small.jpg
        }

    % \introduction{
    %     EINLEITUNG
    % }

    \ingredients{
        \SI{160}{\g} & vegane Butter \\
        \SI{120}{\g} & Rohrzucker \\
        \SI{50}{\g} & \addtoidx{Agavendicksaft} \\
        \SI{70}{\g} & \addtoidx{Pistazien} \\
        \SI{100}{\g} & Blattspinat \\
        \SI{1}{\TL} & Apfelessig \\
        \SI{250}{\ml} & Sojamilch \\
        \SI{250}{\g} & Mehl \\
        \SI[parse-numbers = false]{\nicefrac{1}{2}}{\TL} & Natron \\
        \SI[parse-numbers = false]{1\nicefrac{1}{2}}{\TL} & Backpulver \\
        1 & Zitrone (Saft und Schale) \\
        1 Prise & Salz \\
        \\
        \multicolumn{2}{l}{\textbf{Topping}} \\
        \SI{2}{\EL} & Puderzucker \\
        \SI{1}{\TL} & Zitronensaft \\
    }

    \preparation{
        \step Zuerst die Sojamilch mit dem Spinat, dem Agavendicksaft, dem Apfelessig und den Pistazien pürieren.
        Dann in einer großen Schüssel das Mehl mit dem Zucker, den Natron und dem Backpulver vermischen.
        Die pürierte Pistazienmasse hineinrühren, die Butter zerlassen und auch hinein geben.
        Zuletzt noch Saft und den Schalenabrieb der Zitrone untermischen und, wenn alles zu einem glatten Teig vermengt wurde, in eine Kastenform mit Backpapier füllen.
        \step Im vorgeheizten Backofen bei \SI{180}{\celsius} für etwa \SI{60}{\minute} backen.
    }

    % \suggestion[TITEL EINES VORSCHLAGS]{
	% 	VORSCHLAG (DURCH HORIZONTALE LINIE VOM REZEPT GETRENNT)
    % }

    % \hint{
    %     HINWEIS (IN EINEM KASTEN UNTEN AUF DER SEITE)
    % }

\end{recipeDP}