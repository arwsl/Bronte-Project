\begin{recipeDP}
    [
        preparationtime = {\SI{30}{\minute}},
        bakingtime = {\SI{45}{\minute} bis \SI{50}{\minute}},
        bakingtemperature = {\protect\bakingtemperature{topbottomheat=\SI{180}{\celsius}}},
        portion = {1 Kastenform},
        source = {biancazapatka.com, namelymarly.com}
    ]
    {Kokosnusskuchen}

    \graph
        {
            big=Gebaeck/Kuchen/Kokosnusskuchen/big.jpg,
            small=Gebaeck/Kuchen/Kokosnusskuchen/small.jpg
        }

    % \introduction{
    %     EINLEITUNG
    % }

    \ingredients{
        \multicolumn{2}{l}{\textbf{Teig}} \\
        \SI{300}{\g} & Mehl \\
        \SI{50}{\g} & Kokosraspel \index{Kokos!-raspel} \\
        \SI{52}{\TL} & Backpulver \\
        \SI{1}{\TL} & Natron \\
        \SI{100}{\g} & Rohrohrzucker \index{Zucker!Rohr-}\\
        1 Prise & Salz \\
        \SI{400}{\ml} & Kokosmilch \\
        1 & Bio Zitrone \\
        \SI{1}{\pck} & Vanillezucker \\
        \\
        \multicolumn{2}{l}{\textbf{Topping}} \\
		\SI{150}{\g} & \addtoidx{Cashewkerne} \\
        \SI{85}{\g} & Agavendicksaft \\
        \SI{35}{\g} & Kokosnussraspel \\
        \SI{1}{\TL} & Kokosöl
    }

    \preparation{
        \step Den Backofen auf 180 °C Ober-/ Unterhitze vorheizen.
        Eine 25-cm Kastenform leicht einfetten und mit Backpapier auslegen.
        Die Cashews mit heißem Wasser übergießen und darin ziehen lassen. 

        \step Mehl, Kokosraspel, Backpulver, Natron, Zucker und Salz in einer großen Schüssel vermischen.
        Kokosmilch, Zitronensaft, Zitronenschale und Vanilleextrakt hinzugegeben und alles kurz mit einem Schneebesen zu einem Teig verrühren.
        Den Teig in die vorbereitete Backform füllen und den Kuchen etwa 45-50 Minuten backen, bis ein Stäbchen sauber herauskommt (gegebenenfalls nach einiger Zeit mit Backpapier abdecken, damit der Kuchen nicht zu dunkel wird).

        \step Anschließend den Kuchen \ca 15 Minuten in der Form ruhen lassen, dann vorsichtig herausheben und auf einem Kuchengitter zu Ende abkühlen lassen.

        \step Für das Topping die Cashews abgießen und zum Beispiel mit einem Stabmixer - gemeinsam mit den anderen Zutaten für das Topping - zerkleinern.
        Je nach Geschmack die die Masse feiner oder gröber lassen.
        Wenn die Creme streichfähig ist, kann sie auf dem Kuchen verteilt werden.
    }

    % \suggestion[TITEL EINES VORSCHLAGS]{
	% 	VORSCHLAG (DURCH HORIZONTALE LINIE VOM REZEPT GETRENNT)
    % }

    \hint{
        Das Topping macht sich auch gut als Füllung im Kuchen.
        Dazu den Kuchen einmal aufschneiden, und dann die Füllung hineingeben.
    }

\end{recipeDP}