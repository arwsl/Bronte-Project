\begin{recipeDP}
    [
        preparationtime = {\SI{60}{\minute}},
        bakingtime = {\SI{45}{\minute}},
        bakingtemperature = {\protect\bakingtemperature{fanoven=\SI{170}{\celsius}}},
        portion = {1 Kuchen (\SI{26}{\cm})},
        source = {tinesveganebackstube.de}
    ]
    {Marzipantorte\index{Marzipan}}

    \graph
        {
            big=Gebaeck/Kuchen/Marzipantorte/big.jpg,
            small=Gebaeck/Kuchen/Marzipantorte/small.jpg
        }

    % \introduction{
    %     EINLEITUNG
    % }

    \ingredients{
        \multicolumn{2}{l}{\textbf{Teig}} \\
        \SI{350}{\g} & Mehl \\
        \SI{1}{\pck} & Backpulver \\
        \SI{75}{\g} & Zucker \\
        \SI{1}{\pck} & Vanillezucker \\
        \SI{200}{\g} & Marzipanrohmasse \\
        \SI{125}{\g} & \addtoidx{Haselnüsse}, gemahlen \\
        \SI{125}{\g} & Margarine \\
        \SI{220}{\ml} & Sprudelwasser \\
        \\
        \multicolumn{2}{l}{\textbf{Vanillecreme}} \\
        \SI{250}{\ml} & Kochsahne \\
        \SI{100}{\g} & Marzipanrohmasse  \\
        \SI{1}{\pck} & Vanillepuddingpulver \\
        \SI{1}{\pck} & Vanillezucker \\
        \SI{400}{\ml} & \addtoidx{Schlagsahne} \\
        \SI{2}{\pck} & Sahnesteif
        \\
        1 & Marzipandecke
    }

    \preparation{
        \step Die Zutaten für den Teig so kurz wie möglich miteinander vermischen. Veganer Kuchenteig wird durch zu langes Rühren gummiartig.
        \step Den Kuchen in eine eingefettete Springform geben und bei \SI{170}{\celsius} Umluft für \SI{45}{\minute} backen. Die Stäbchenprobe nicht vergessen!
        \step Während der Kuchen im Ofen ist, die Füllung vorbereiten: Die Zutaten für die Vanillecreme in einen Topf geben und die Masse kurz aufkochen lassen. Ein paar Sekunden köcheln lassen und dann schnell vom Herd nehmen. Zum schnelleren Abkühlen am besten danach in eine andere Schüssel umfüllen. In der Zwischenzeit die Sahne mit Sahnesteif aufschlagen und sobald der Pudding komplett abgekühlt ist, hebt man diesen unter die Schlagsahne.
        \step Wenn der Kuchen fertig gebacken ist, im Ofen abkühlen lassen. Abgekühlt aus der Form  lösen und in zwei bis drei Böden schneiden. Zwischen die Böden die Puddingcreme geben und rundum mit der Creme eindecken. Als letztes darauf die Marzipandecke legen. Am besten schmeckt die Torte am nächsten Tag, wenn alles gut durchgezogen ist.
    }

    \suggestion{
		Zum Dekorieren mit Marzipandecke überziehen, und mit Schokolade, Sahne und Haselnusskrokant verzieren. Anstatt einer fertigen Marzipandecke lassen sich auch etwa \SI{250}{\g} Marzipanrohmasse auf ausreichend Puderzucker ausrollen.
    }

    % \hint{
    %
    % }

\end{recipeDP}
