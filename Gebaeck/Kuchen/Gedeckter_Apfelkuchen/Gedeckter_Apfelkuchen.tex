\begin{recipeDP}
    [
        preparationtime = {\SI{60}{\minute}},
        bakingtime = {\SI{25}{\minute}},
        bakingtemperature = {\protect\bakingtemperature{fanoven=\SI{180}{\celsius}}},
        % portion = {PORTIONEN/MENGE},
        source = {zuckerjagdwurst.com}
    ]
    {Gedeckter Apfelkuchen vom Blech}

    \graph
        {
            big=Gebaeck/Kuchen/Gedeckter_Apfelkuchen/big.jpg,
            small=Gebaeck/Kuchen/Gedeckter_Apfelkuchen/small.jpg
        }

    \introduction{
        Für einen Kuchen vom Blech mit \SI{21}{\cm} x \SI{25}{\cm}.
    }

    \ingredients{
        \multicolumn{2}{l}{\textbf{Mürbeteig}} \\
        \SI{500}{\g} & Weizenmehl (Typ 405) \\
        \SI{5}{\EL} & Zucker \\
        1 Prise & Salz \\
        \SI{250}{\g} & (kalte) vegane Butter\\
        etwa \SI[]{100}{\ml} & kaltes Wasser \\
        \\
        \multicolumn{2}{l}{\textbf{Füllung}} \\
		\SI[]{1}{\kg} & Äpfel \\
        \SI[]{700}{\g} & Apfelmus \\
        \SI[]{2}{\EL} & Rosinen \\
        \SI[parse-numbers = false]{\nicefrac{1}{2}}{\TL} & Zimt \\
        \\
        \multicolumn{2}{l}{\textbf{Zuckerglasur}} \\
        \SI[]{200}{\g} & Puderzucker \\
        etwa \SI[]{3}{\EL} & Zitronensaft
    }

    \preparation{
        \step Weizenmehl, Zucker und Salz vermengen.
        Kalte vegane Butter in Flocken darübergeben und mit kalten Händen in die Mehlmischung einarbeiten, bis eine fein-krümelige Mischung entsteht.
        Das kalte Wasser dazugeben und verkneten, bis ihr einen geschmeidigen Teig habt.
        Nach Bedarf mehr oder weniger Wasser verwenden. Der Teig sollte jedoch am Ende nicht kleben.
        Den Mürbeteig mindestens \SI{20}{\minute} im Kühlschrank ruhen lassen.

        \step In der Zwischenzeit Äpfel schälen, entkernen und in kleine Würfel schneiden.
        Apfelmus und die geschnittenen Äpfel in einen Topf geben und etwa \SI[]{15}{\minute} bei mittlerer Hitze köcheln lassen, bis die Apfelstückchen weich sind.
        Rosinen und Zimt dazugeben und weitere \SI[]{5}{\minute} köcheln lassen.
        Die Füllung danach etwas abkühlen lassen.

        \step Den Backofen auf \SI{180}{\celsius} (Umluft) vorheizen.
        Zwei Drittel des Teigs direkt auf Backpapier zu einem Rechteck ausrollen, dabei sollte Boden und Rand der Form bedeckt sein.
        Das Backpapier mitsamt des Teigs vorsichtig in die Backform legen und die Apfelfüllung darauf verteilen.
        Den restlichen Mürbeteig ausrollen, bis er über die Füllung passt.
        Den Teig über die Apfelfüllung legen und andrücken.

        \step Den gedeckten Apfelkuchen bei \SI{180}{\celsius} etwa \SI[]{25}{\minute} backen, oder bis der Teig oben goldbraun ist.
        Aus dem Ofen nehmen und entweder abkühlen lassen.
        Für die Glasur Puderzucker nach und nach mit Wasser oder Zitronensaft verrühren, bis ein dickflüssiger Zuckerguss entsteht.
        Die Glasur auf der Oberseite des Kuchens verstreichen und kurz trocknen lassen.
    }

    % \suggestion[TITEL EINES VORSCHLAGS]{
	% 	VORSCHLAG (DURCH HORIZONTALE LINIE VOM REZEPT GETRENNT)
    % }

    % \hint{
    %     HINWEIS (IN EINEM KASTEN UNTEN AUF DER SEITE)
    % }

\end{recipeDP}