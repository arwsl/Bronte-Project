\begin{recipeDP}
    [
        preparationtime = {\SI{25}{\minute}},
        bakingtime = {\SI{25}{\minute}},
        bakingtemperature = {\protect\bakingtemperature{topbottomheat=\SI{180}{\celsius}}},
        portion = {1 20er Ring-Kuchen},
        source = {dinkelliebe.de}
    ]
    {Pistazien-Himbeer-Kuchen}

    \graph
        {
            big=Gebaeck/Kuchen/Pistazien-Himbeer-Kuchen/big.jpg,
            small=Gebaeck/Kuchen/Pistazien-Himbeer-Kuchen/small.jpg
        }

    % \introduction{
    %     EINLEITUNG
    % }

    \ingredients{
        \multicolumn{2}{l}{\textbf{Rührteig}} \\
        \SI{120}{\g} & Butter (z. B. Alsan) \\
        \SI{80}{\g} & Rohrohrzucker \\
        \SI{1}{\EL} & Leinsamenschrot \\
        \SI{4}{\EL} & Wasser \\
        \SI{40}{\ml} & Pflanzenmilch \\
        \SI{100}{\g} & Dinkelmehl \\
        \SI{100}{\g} & \addtoidx{Pistazien} \\
        \SI{10}{\g} & Backpulver \\
        \SI{1}{\pck} & Vanillezucker \\
        1 Prise & Salz \\
        \SI{150}{\g} & \addtoidx{Himbeeren} \\
        \\
        \multicolumn{2}{l}{\textbf{Frosting}} \\
        \SI{100}{\g} & \addtoidx{Schlagsahne} \\
        \SI{1}{\pck} & Sahnesteif \\
        \SI{100}{\g} & Frischkäse \\
        \SI{15}{\g} & Limettensaft \\
        \SI{50}{\g} & Puderzucker \\
        \SI{1}{\pck} & Vanillezucker \\
        \\
         & Pistazien \\
         & Limettenscheiben \\
         & Himbeeren
    }

    \preparation{
        \step Die Pistazien schälen, mahlen (oder im Mixer fein häckseln) und beiseite stellen.
        Den Backofen auf 180 Grad Ober-/Unterhitze vorheizen.
        \step Die Butter und den Zucker mit dem Mixer cremig rühren.
        Wasser und Leinsamen zu einem Leinei verrührren.
        Wenn die Masse angedickt ist, unterrühren.
        Danach die Pflanzenmilch zugeben und unterrühren.
        Die trockenen Zutaten mischen und löffelweise dazugeben.
        Alles nur kurz vermixen bis der Teig glatt ist!
        \step Nun die Himbeeren vorsichtig unter den Teig heben.
        Die Ring-Backform fetten, den Teig einfüllen und glatt streichen.
        \step Auf der mittleren Schiene etwa 25 Minuten backen (Stäbchenprobe nicht vergessen!).
        Etwa 10 Minuten in der Form abkühlen lassen, erst dann stürzen, andernfalls kann er zerbrechen.
        Den Kuchen nach dem Stürzen vollständig auskühlen lassen.
        \step Für das Frosting die Sahne und das Sahnesteif mit dem Mixer steif schlagen.
        Separat Frischkäse, Limettensaft, Zucker und Vanille mischen und die steif geschlagene Sahne unterheben.
        Bis zur Verwendung kalt stellen.
        \step Ist der Kuchen abgekühlt, das Frosting auf den Kuchen geben und verteilen.
        Den Kuchen mit den gehackten Pistazien, den geviertelten Limettenscheiben und den Himbeeren dekorieren und genießen.
    }

    % \suggestion[TITEL EINES VORSCHLAGS]{
	% 	VORSCHLAG (DURCH HORIZONTALE LINIE VOM REZEPT GETRENNT)
    % }

    % \hint{
    %     HINWEIS (IN EINEM KASTEN UNTEN AUF DER SEITE)
    % }

\end{recipeDP}