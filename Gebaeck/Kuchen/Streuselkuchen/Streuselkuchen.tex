\begin{recipeDP}
    [
        preparationtime = {\SI{20}{\minute}},
        bakingtime = {\SI{35}{\minute} bis \SI{40}{\minute}},
        bakingtemperature = {\protect\bakingtemperature{topbottomheat=\SI{180}{\celsius}}},
        portion = {9 Portionen},
        source = {elavegan.com}
    ]
    {\addtoidx{Streuselkuchen}}

    \graph
        {
            big=Gebaeck/Kuchen/Streuselkuchen/big.jpg,
            small=Gebaeck/Kuchen/Streuselkuchen/small.jpg
        }

    % \introduction{
    %     EINLEITUNG
    % }

    \ingredients{
        \multicolumn{2}{l}{\textbf{Trockene Zutaten}} \\
        \SI{90}{\g} & Hafermehl\index{Mehl!Hafer-} \\
        \SI{80}{\g} & Weizenmehl \\
        \SI[]{16}{\g} & Stärke \\
        \SI[]{50}{\g} & Zucker \\
        \SI[parse-numbers = false]{1 \nicefrac{1}{2}}{\TL} & Backpulver \\
        \SI[parse-numbers = false]{\nicefrac{1}{4}}{\TL} & Natron \\
        \SI[parse-numbers = false]{\nicefrac{1}{2}}{\TL} & Salz \\
        \\
        \\
        \multicolumn{2}{l}{\textbf{feuchte Zutaten}} \\
        \SI[]{180}{\g} & Sojamilch \\
        \SI[]{160}{\g} & \addtoidx{Apfelmus} ungesüßt \\
        \SI[]{30}{\g} & Öl \\
        \SI[]{1}{\EL} & Apfelessig \\
        \SI[]{1}{\pck} & Vanillezucker \\
        \\
        \multicolumn{2}{l}{\textbf{Streusel}} \\
        \SI[]{80}{\g} & Weizenmehl \\
        \SI[]{60}{\g} & gemahlene Mandeln\index{Mandeln!gemahlene} \\
        \SI[]{20}{\g} & Zucker \\
        \SI[]{28}{\g} & Öl \\
        \SI[]{30}{\g} & Rübensirup \\
		\SI[parse-numbers = false]{\nicefrac{1}{2}}{\TL} & Zimt \\
        \SI[parse-numbers = false]{\nicefrac{1}{4}}{\TL} & Salz
    }

    \preparation{
        \step Eine Backform (\ca 15 x 25 Zentimeter) mit Backpapier auslegen. Für die Zimt-Streusel alle trockenen Zutaten für die Streusel in eine Schüssel geben, mit einem Schneebesen vermischen und dann die feuchten Zutaten hinzugeben.
        Mit den Fingern alles gut durchkneten, bis die Mischung leicht bröselig ist.
        Kühl stellen und den Backofen auf \SI[]{180}{\degree} vorheizen.

        \step Alle trockenen Zutaten für den Rührteig in eine große Rührschüssel geben und mit einem Schneebesen vermischen.
        Die trockenen Zutaten können auch in einer Küchenmaschine für einige Sekunden lang gemixt werden.
        Als Nächstes die feuchten Zutaten hinzufügen und mit einem Schneebesen umrühren.

        \step Etwa die Hälfte des Teigs in die ausgekleidete Backform gießen.
        Dann die Hälfte der Zimtstreusel darauf verteilen.
        Den restlichen Teig auf die Streusel gießen und dann schließlich die restlichen Streusel obendrauf geben.
        Etwa \SI[]{35}{\minute} bis \SI[]{40}{\minute} im Ofen backen oder bis ein Zahnstocher aus der Mitte des Krümelkuchens fast sauber herauskommt. Vollständig abkühlen lassen, dann optional mit Zuckerguss beträufeln.
        Der Streuselkuchen lässt sich im Kühlschrank bis zu 6 Tage, im Tiefkühlschrank bis zu 2 Monate lagern.
    }

    % \suggestion[TITEL EINES VORSCHLAGS]{
	% 	VORSCHLAG (DURCH HORIZONTALE LINIE VOM REZEPT GETRENNT)
    % }

    % \hint{
        
    % }

\end{recipeDP}