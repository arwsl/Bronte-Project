\begin{recipeDP}
    [
        preparationtime = {\SI{20}{\minute}},
        bakingtime = {\SI{60}{\minute}},
        bakingtemperature = {\protect\bakingtemperature{fanoven=\SI{180}{\celsius}}},
        portion = {1 Kastenform},
        source = {zuckerjagdwurst.com}
    ]
    {Orangen-Olivenöl-Kuchen}

    \graph
        {
            big=Gebaeck/Kuchen/Orangen-Olivenoel-Kuchen/big.jpg,
            small=Gebaeck/Kuchen/Orangen-Olivenoel-Kuchen/small.jpg
        }

    % \introduction{
    %     EINLEITUNG
    % }

    \ingredients{
        \multicolumn{2}{l}{\textbf{Teig}} \\
        \SI{400}{\g} & Mehl \\
        \SI{100}{\g} & Zucker \\
        \SI{1}{\TL} & Backpulver \\
        1 Prise & Salz \\
        \SI{150}{\ml} & \addtoidx{Oliven!-öl} \\
        \SI{200}{\ml} & Sojamilch \\
        1 & Bio Orange \\
        \SI{1}{\pck} & Vanillezucker \\
        \SI{100}{\g} & \addtoidx{Cashews} \\
        \\
        \multicolumn{2}{l}{\textbf{Glasur}} \\
        \SI{75}{\g} & Puderzucker \\
         & Orangensaft \\
         & Orangenabrieb \\
        \\
         & Margarine \\
         & Mehl
    }

    \preparation{
        \step Den Backofen auf 180°C Umluft vorheizen.
        Die Cashews hacken und beiseite legen.
        \step Zunächst Mehl, Zuckerm, Vanillezucker, Backpulver und Salz in einer großen Schüssel mischen.
        Olivenöl, pflanzliche Milch, den Saft einer Orange in einem Messbecher verrühren und danach zu den trockenen Zutaten geben.
        Alle Zutaten vorsichtig zu einem glatten, dickflüssigen Teig vermengen – per Hand gerade soweit verrühren, dass kein Mehl mehr zu sehen ist.
        Zum Schluss die gehackten Cashews untermengen.
        \step Eine Kastenform einfetten und mit Mehl ausklopfen.
        Den Teig in die Backform füllen, glatt streichen und bei 180°C \ca 50 bis 60 Minuten backen (bis die Stäbchenprobe gelingt).
        \step Den Kuchen abkühlen lassen – am besten erst 15 Minuten in der Form, danach stürzen und auf einem Kuchengitter komplett abkühlen lassen.
        Für die Glasur Puderzucker mit so viel Orangensaft verrühren, dass ein zähflüssiger Guss entsteht.
        Den Abrieb der Orange dazugeben und verrühren. Die Glasur über dem Kuchen verteilen und trocknen lassen.
    }

    % \suggestion[TITEL EINES VORSCHLAGS]{
	% 	VORSCHLAG (DURCH HORIZONTALE LINIE VOM REZEPT GETRENNT)
    % }

    % \hint{
    %     HINWEIS (IN EINEM KASTEN UNTEN AUF DER SEITE)
    % }

\end{recipeDP}