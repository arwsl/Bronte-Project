\begin{recipeDP}
    [
        preparationtime = {\SI{30}{\minute}},
        bakingtime = {\SI{40}{\minute}},
        bakingtemperature = {\protect\bakingtemperature{fanoven=\SI{175}{\celsius}}},
        portion = {1 Blech},
        source = {Sabine Schramm}
    ]
    {\addtoidx{Mohnkuchen}}

    % \graph
    %     {
    %         big=TEIL/KAPITEL/REZEPT/big.jpg,
    %         small=TEIL/KAPITEL/REZEPT/small.jpg
    %     }

    % \introduction{
    %     EINLEITUNG
    % }

    \ingredients{
        \multicolumn{2}{l}{\textbf{Teig}} \\
        \SI{600}{\g} & Mehl \\
        \SI{300}{\g} & Margarine \\
        \SI{150}{\g} & Zucker \\
        \SI{6}{\EL} & Sojamilch \\
        \\
        \multicolumn{2}{l}{\textbf{Mohnfüllung}} \\
        \SI{1}{\l} & Sojamilch \\
        \SI{250}{\g} & Backmohn \\
        \SI{100}{\g} & \addtoidx{Mohn} \\
        \SI{160}{\g} & \addtoidx{Dinkelgrieß} \\
        \SI{100}{\g} & Zucker \\
        \SI{100}{\g} & Butter
    }

    \preparation{
        \step Die trockenen Zutaten des Teigs vermengen, dann mit der Butter und der Sojamilch zu Streuseln verarbeiten. Eine Hälfte der Streusel auf einem mit Backpapier vorbereiteten Backblech verteilen. Die Streusel zu einem Boden festdrücken. Die andere Hälfte der Streusel beiseite stellen.
        \step Für die Füllung Sojamilch und Zucker aufkochen, dann den Grieß einrühren und alles weitere \SI{2}{\minute} köcheln lassen. Den Ofen auf \SI{175}{\celsius} vorheizen.
        \step Anschließend den Mohn und den Backmohn dazu geben und gut unterrühren. Die Butter einrühren und die Füllung \SI{5}{\minute} stehen lassen zum abkühlen.
        \step Die fertige, noch heiße Füllung gleichmäßig auf dem Boden verteilen. Die Streusel darüber verteilen und den Kuchen im vorgeheizten Backofen bei \SI{175}{\celsius} etwa \SI{40}{\minute} backen.
    }

    % \suggestion[TITEL EINES VORSCHLAGS]{
	% 	VORSCHLAG (DURCH HORIZONTALE LINIE VOM REZEPT GETRENNT)
    % }

    \hint{
        Für eine \SI{24}{\cm}-Springform die Angaben im Rezept einmal halbieren.
    }

\end{recipeDP}
