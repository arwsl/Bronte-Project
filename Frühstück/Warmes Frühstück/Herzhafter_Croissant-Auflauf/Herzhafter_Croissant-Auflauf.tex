\begin{recipeDP}
    [
        preparationtime = {\SI{25}{\minute}},
        bakingtime = {\SI{25}{\minute}},
        bakingtemperature = {\protect\bakingtemperature{topbottomheat=\SI{200}{\celsius}}},
        portion = {4 Portionen},
        source = {zuckerjagdwurst.com}
    ]
    {Herzhafter Croissant-Auflauf}

    \graph
        {
            big=Frühstück/Warmes Frühstück/Herzhafter_Croissant-Auflauf/big.jpg,
            small=Frühstück/Warmes Frühstück/Herzhafter_Croissant-Auflauf/small.jpg
        }

    % \introduction{
    %     EINLEITUNG
    % }

    \ingredients{
        \multicolumn{2}{l}{\textbf{Für die Croissants}} \\
        5 & vegane Tiefkühlcroissants\index{Croissant} \\
        \SI{2}{\EL} & Leinsamen \\
        \SI{2}{\EL} & heller Sesam\index{Sesam!hell} \\
        \SI{2}{\EL} & Sonnenblumenkerne \\
         & Hafersahne \\
        \\
        \multicolumn{2}{l}{\textbf{Für den Auflauf}} \\
        2 & Zwiebeln \\
        4 & Knoblauchzehen \\
        \SI{250}{\g} & braune Champignons \\
        \SI{100}{\g} & Spinat \\
        \SI{150}{\g} & Cocktailtomaten \index{Tomaten!Cocktail-} \\
        \SI{200}{\g} & Räuchertofu \index{Tofu!geräuchert} \\
        \SI{1}{\EL} & Sojasoße \\
        \SI{150}{\g} & Fetaersatz \index{Feta} \\
         & Salz \\
         & Pfeffer \\
         & Öl \\
        \\
        \multicolumn{2}{l}{\textbf{Für die Soße}} \\
        \SI{200}{\g} & Seitentofu \index{Tofu!Seiden-} \\
        \SI{100}{\g} & Joghurt \\
        \SI{2}{\EL} & Kichererbsenmehl \\
        \SI{5}{\EL} & \addtoidx{Hefeflocken} \\
        \SI{1}{\TL} & \addtoidx{Kala Namak} \\
        \SI{2}{\TL} & Senf \\
         & Muskat \\
         & Pfeffer
    }

    \preparation{
        \step Die veganen Croissants nach Packungsanleitung auftauen und ruhen lassen.
        Leinsamen, Sesam und Sonnenblumenkerne in einer kleinen Schüssel vermengen. Die Oberseite der Croissants nach dem Auftauen mit etwas Hafersahne bestreichen und in den Toppings wälzen.
        Anschließend in eine gefettete Auflaufform setzen.
        \step Während die Croissants auftauen, Zwiebel und Knoblauch fein würfeln.
        Die Pilze vierteln und den Blattspinat grob hacken.
        Cocktailtomaten halbieren oder vierteln und den Räuchertofu mit den Händen in kleinere Stücke reißen.
        \step Pflanzenöl in einer großen Pfanne erhitzen und den Räuchertofu mit einem Schuss Sojasoße anbraten, bis er leicht knusprig ist.
        Danach aus der Pfanne nehmen und beiseitestellen.
        Erneut etwas Pflanzenöl in die Pfanne geben und die Pilze 4 bis 5 Minuten anbraten, bis sie ihre Feuchtigkeit verloren haben und leicht bräunen.
        Zwiebel und Knoblauch dazugeben und weitere 3 bis 4 Minuten anbraten.
        Den gehackten Spinat dazugeben und nur kurz mit erwärmen, bis er zusammenfällt.
        Die Mischung mit Salz und Pfeffer würzen und danach ebenfalls beiseitestellen.
        \step Für die Soße Seidentofu, Joghurt, Kichererbsenmehl, Stärke und Hefeflocken in einen Mixbecher geben und glatt pürieren.
        Danach mit Kala Namak, Senf, Muskat und Pfeffer abschmecken.
        \step Die Spinatmischung mit den Cocktailtomaten, gebratenem Räuchertofu und der Soße vermengen und alles rund um die Croissants in der Auflaufform verteilen.
        Den veganen Feta mit den Händen zerbröseln und den Auflauf damit toppen.
        \step Den Auflauf bei \SI{200}{\celsius} (Ober-/Unterhitze) 20 bis 25 Minuten backen, bis die Croissants goldbraun gebacken sind.
    }

    % \suggestion[TITEL EINES VORSCHLAGS]{
	% 	VORSCHLAG (DURCH HORIZONTALE LINIE VOM REZEPT GETRENNT)
    % }

    \hint{
        Anstatt der Tiefkühlcroissants können auch frische Croissants verwendet werden.
        Die Backzeit muss dadurch eventuell etwas verkürzt werden.
    }

\end{recipeDP}