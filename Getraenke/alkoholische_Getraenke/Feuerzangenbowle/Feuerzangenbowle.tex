\begin{recipeDP}
    [
        preparationtime = {\SI{20}{\minute}},
        % bakingtime = {\SI{ZEIT}{\minute} bis \SI{ZEIT}{\minute}},
        % bakingtemperature = {\protect\bakingtemperature{fanoven=\SI{TEMPERATUR}{\celsius}}},
        portion = {4 Personen},
        source = {chefkoch.de}
    ]
    {Feuerzangenbowle}

    \graph
        {
            big=Getraenke/alkoholische_Getraenke/Feuerzangenbowle/big.jpg,
            small=Getraenke/alkoholische_Getraenke/Feuerzangenbowle/small.jpg
        }

    % \introduction{
    %     EINLEITUNG
    % }

    \ingredients{
        \SI{2}{\l} & \addtoidx{Rotwein} (Spätburgunder) \\
        \SI{500}{\ml} & Orangensaft \\
        1 & Orange, unbehandelt \\
        1 & Zitrone, unbehandelt \\
        1 Stange & Zimt \\
        6 & Gewürznelken \\
        4 & Sternanis \\
        1 & \addtoidx{Zuckerhut} \\
        \SI{350}{\ml} & Rum, \SI{54}{\percent}
    }

    \preparation{
        \step Die Orange in Scheiben schneiden. Die Schale der Zitrone möglichst am Stück dünn abschneiden, beides beiseite legen. Den Saft der Zitrone auspressen.
        \step Den Rotwein in einem Topf erhitzen. Den gesamten Fruchtsaft durch ein Sieb gießen, zum Wein geben und mit erhitzen (nicht kochen!). Die Gewürze in der heißen Flüssigkeit ziehen lassen.
        \step Die Weinmischung in in eine Feuerzangenbowle umfüllen und weiter warm halten. Die Orangenscheiben in die Bowle geben und mit der Zitronenspirale verzieren.

        Eine Feuerzange mit dem Zuckerhut über den Bowletopf legen und den Zuckerhut mit etwas erwärmtem Rum beträufeln. Ein wenig Rum in eine Kelle geben, mit langem Streichholz anzünden und brennend über den Zuckerhut gießen. Restlichen Rum zunächst in die Kelle gießen, dann über den brennenden Zuckerhut laufen lassen. Achtung, die Rumflasche nie direkt an die offene Flamme halten!
    }

    % \suggestion[TITEL EINES VORSCHLAGS]{
	% 	VORSCHLAG (DURCH HORIZONTALE LINIE VOM REZEPT GETRENNT)
    % }

    % \hint{
    %     HINWEIS (IN EINEM KASTEN UNTEN AUF DER SEITE)
    % }

\end{recipeDP}
