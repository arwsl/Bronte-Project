\begin{recipeDP}
    [
        % preparationtime = {\SI{ZEIT}{\minute}},
        % bakingtime = {\SI{ZEIT}{\minute} bis \SI{ZEIT}{\minute}},
        % bakingtemperature = {\protect\bakingtemperature{fanoven=\SI{TEMPERATUR}{\celsius}}},
        portion = {1 Drink},
        source = {gutekueche.at}
    ]
    {Sport-Cocktail}

    \graph
        {
            big=Getraenke/alkoholfreie_Getraenke/Sport-Cocktail/big.jpg,
            small=Getraenke/alkoholfreie_Getraenke/Sport-Cocktail/small.jpg
        }

    % \introduction{
    %     EINLEITUNG
    % }

    \ingredients{
        \SI{10}{\ml} & Zitronensaft \index {Zitronen!-saft} \\
        \SI{80}{\ml} & Grapefruitsaft \\
        \SI{10}{\ml} & Ananassaft \index{Ananas!-saft} \\
        \SI{10}{\ml} & Mandelsirup \index{Mandelsirup}
    }

    \preparation{
        \step Den Shaker mit 5 Eiswüfeln füllen.
        Alle Säfte und den Sirup dazugeben und kräftig schütteln.
        \step In ein gekühltes Glas einige Eiswüfel geben und den Drink über ein Barsieb dazugießen.
        Mit einer Limettenscheibe oder einer Ananasspalte garnieren.
    }

    % \suggestion[TITEL EINES VORSCHLAGS]{
	% 	VORSCHLAG (DURCH HORIZONTALE LINIE VOM REZEPT GETRENNT)
    % }

    \hint{
        Limettenscheiben können dem Cocktail auch untergerührt werden.
    }

\end{recipeDP}