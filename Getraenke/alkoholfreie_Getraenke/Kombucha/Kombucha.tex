\begin{recipeDP}
    [
        preparationtime = {\SI{25}{\minute}},
        bakingtime = {\SI{10}{\day} bis \SI{16}{\day}},
        % bakingtemperature = {\protect\bakingtemperature{fanoven=\SI{TEMPERATUR}{\celsius}}},
        portion = {\SI{4,5}{\l}},
        source = {wellness-drinks.de}
    ]
    {Kombucha}

    \graph
        {
            big=Getraenke/alkoholfreie_Getraenke/Kombucha/big.jpg,
            small=Getraenke/alkoholfreie_Getraenke/Kombucha/small.jpg
        }

    % \introduction{
    %     EINLEITUNG
    % }

    \ingredients{
        2 & Kombucha Pilz (Scoby) \\
        \SI[]{500}{\ml} & Ansatzflüssigkeit \\
        \SI{4}{\l} & Wasser \\
        \SI{14}{\g} & Tee \\
        \SI{360}{\g} & \addtoidx{Rohrohrzucker}
    }

    \preparation{
        \step Das Wasser aufkochen, in einem sehr großen Gärgefäß kurz abkühlen lassen und den Tee hinein geben.
        Insgesamt rund \SI{15}{\minute} ziehen lassen, dann den Tee herausnehmen.
        \step Den Rohrohrzucker in den Tee rühren, bis er sich aufgelöst hat.
        Dann den Tee vollständig abkühlen lassen.
        \step Den Scoby zusammen mit der Ansatzflüssigkeit in den zimmerwarmen Tee geben.
        Das Gärgefäß mit einem sauberen Küchentuch verschließen und alles an einem warmen Platz (mindestens \SI{21}{\degree}) unbewegt stehen lassen.
        Nach 10 etwa den Kombucha probieren und eventuell noch länger fermentieren lassen.
        \step Hat der Kombucha die gewünschte Säure erreicht, kann dieser umgefüllt werden.
        Am besten kalt stellen, damit die Fermentation abgeschlossen ist.
    }

    % \suggestion[TITEL EINES VORSCHLAGS]{
	% 	VORSCHLAG (DURCH HORIZONTALE LINIE VOM REZEPT GETRENNT)
    % }

    % \hint{
    %     HINWEIS (IN EINEM KASTEN UNTEN AUF DER SEITE)
    % }

\end{recipeDP}