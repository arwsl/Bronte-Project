\begin{recipeDP}
    [
        preparationtime = {\SI{20}{\minute}},
        bakingtime = {5 Tage},
        % bakingtemperature = {\protect\bakingtemperature{fanoven=\SI{TEMPERATUR}{\celsius}}},
        % portion = {PORTIONEN/MENGE},
        source = {chefkoch.de: kangoo}
    ]
    {Holundersirup}

    \graph
        {
            big=Getraenke/alkoholfreie_Getraenke/Holundersirup/big.jpg,
            small=Getraenke/alkoholfreie_Getraenke/Holundersirup/small.jpg
        }

    % \introduction{
    %     EINLEITUNG
    % }

    \ingredients{
        10 & Holunderblütendolden\index{Holunder} \\
        \SI{1,5}{\l} & Wasser \\
        \SI{1}{\kg} & Zucker \\
        1 & Biozitrone \\
        \SI{30}{\g} & \addtoidx{Zitronensäure} \\
    }

    \preparation{
        \step Die Blütenstängel abschneiden, von Insekten befreien und in einen 5-Liter-Topf oder anderes Gefäß geben.
        Dabei die Blüten nicht waschen!
        \step Das Wasser und den Zucker aufkochen, bis der Zucker sich aufgelöst hat.
        Die Zitronen in Scheiben schneiden und hinzufügen.
        Eine Tasse von der Zuckerlösung abnehmen und die Zitronensäure darin auflösen.
        Anschließend wieder zurückgeben zur restlichen Zuckerlösung.
        Jetzt gut verrühren und vorsichtig über die Holunderblüten gießen. 5 Tage zugedeckt stehen lassen.
        \step Den fast fertigen Sirup filtrieren und noch einmal aufkochen lassen.
        In saubere Flaschen abfüllen und im Keller lagern.
        Hält sich etwa bis zur nächsten Holunderblüten-Ernte.
    }

    % \suggestion[TITEL EINES VORSCHLAGS]{
	% 	VORSCHLAG (DURCH HORIZONTALE LINIE VOM REZEPT GETRENNT)
    % }

    \hint{
        Mit Mineralwasser verdünnt erhält man ein sehr erfrischendes Getränk. 
    }

\end{recipeDP}