\begin{recipe}
    [
        preparationtime = {\SI{10}{\minute}},
        bakingtime = {\SI{50}{\minute}},
        bakingtemperature = {\protect\bakingtemperature{topbottomheat=\SI{220}{\degree}}},
        portion = {4 Portionen},
        source = {Caitlin Shoemaker}
    ]
    {\addtoidx{Kichererbsen}-\addtoidx{Nudelauflauf}}

    \graph
        {
            big=Hauptgerichte/Auflaeufe/Kichererbsen-Nudelauflauf/big.jpg,
            small=Hauptgerichte/Auflaeufe/Kichererbsen-Nudelauflauf/small.jpg
        }

    \introduction{Nudelauflauf mit rohen Nudeln.}

    \ingredients{
        \multicolumn{2}{l}{\textbf{Cremige Soße}} \\
        \SI{115}{\g} & Cashewnüsse, übernacht eingeweicht \\
        \SI{840}{\ml} & Gemüsebrühe \\
        \SI{2}{\EL} & Hefeflocken \\
        \SI[parse-numbers = false]{\nicefrac{1}{2}}{\TL} & Paprikapulver \\
        \SI[parse-numbers = false]{\nicefrac{1}{2}}{\TL} & Knoblauchpulver \\
        \SI[parse-numbers = false]{\nicefrac{1}{2}}{\TL} & Pfeffer \\
        & Salbei \\
        & Thymian\\
        \\
        \multicolumn{2}{l}{\textbf{Auflauf}} \\
        \SI{250}{\g} & Nudeln \\
        1 Dose & Kichererbsen \\
        \SI[parse-numbers = false]{\nicefrac{1}{2}}{} & Zwiebel \\
        2 Stiele & Sellerie \\
        \SI{100}{\g} & Möhren, gewürfelt \\
        \SI{100}{\g} & Erbsen, gefroren
    }

    \preparation{
        \step Den Ofen auf \SI{220}{\degree} Ober- und Unterhitze vorheizen und eine \SI{23}{\cm} mal \SI{33}{\cm} Auflaufform zurecht legen.

        \step Für die cremige Soße die eingeweichten und abgegossenen Cashews mit der Gemüsebrühe, den Hefeflocken und den Gewürzen in einem Standmixer zu einer homogenen Soße mixen. Etwa \SI{60}{\second} mixen, damit die Cashews gut aufgelöst werden.

        \step Die rohen Nudeln, Kichererbsen, die klein geschnittene Zwiebel zusammen mit klein geschnittenem Sellerie und den Möhrenstücken und den Erbsen in der Auflaufform vermischen. Dann die Soße darüber verteilen und besonders die Nudeln mit der Soße bedecken.

        \step Den Auflauf im Ofen für \SI{50}{\minute} backen. Nach etwa \SI{25}{\minute} den Auflauf abdecken, damit er nicht oberflächlich verbrennt. Während der Backzeit kann es helfen, einmal umzurühren, damit alle Nudeln gar werden.
    }

    \suggestion[Cashewnüsse einweichen]{
		Entweder die Cashews über Nacht in kaltes Wasser einlegen, oder alternativ für \SI{20}{\minute} in heißem Wasser stehen lassen. Andernfalls lassen sich die Nüsse auch - mit Wasser bedeckt - \SI{3}{\minute} in der Mikrowelle erhitzen, danach noch \SI{5}{\minute} stehen lassen.
    }

    \hint{Es empfehlen sich kurze Nudeln, wie Fusilli oder Farfalle.}

\end{recipe}
