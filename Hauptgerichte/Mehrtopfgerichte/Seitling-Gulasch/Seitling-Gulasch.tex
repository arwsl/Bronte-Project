\begin{recipeDP}
    [
        preparationtime = {\SI{25}{\minute}},
        bakingtime = {\SI{90}{\minute} bis \SI{180}{\minute}},
        bakingtemperature = {\protect\bakingtemperature{topbottomheat=\SI{200}{\celsius}}},
        portion = {4 Portionen},
        source = {vegan-masterclass.de}
    ]
    {Seitling-Gulasch}

    \graph
        {
            big=Hauptgerichte/Mehrtopfgerichte/Seitling-Gulasch/big.jpg,
            small=Hauptgerichte/Mehrtopfgerichte/Seitling-Gulasch/small.jpg
        }

    \introduction{
        Kräftiges veganes Gulasch aus Zwiebeln, Kräuterseitlingen, Kartoffeln und Pulled BBQ; mehrfaches Deglacieren sorgt für tiefe Röstaromen und einen würzigen Schmorcharakter.
    }

    \ingredients{
        \SI{8}{\EL} & Bratöl \\
        \SI{450}{\g} & Zwiebeln \\
        \SI{250}{\g} & festkochende \addtoidx{Kartoffeln} \\
        \SI{450}{\g} & \addtoidx{Kräuterseitlinge} \\
        \SI{200}{\g} & Pulled BBQ \\
        \\
        \SI{300}{\ml} & trockener Rotwein \\
        \SI{500}{\ml} & Gemüsebrühe \\
        \\
        \SI{4}{\g} & Steinpilze, getrocknet\index{Pilze!Stein-} \\
        \SI{1}{\EL} & Majoran, getrocknet \\
        \SI{1}{\TL} & Kümmel \\
        \SI{1}{\EL} & Paprika-Tomatenmark \\
        \SI{2}{\EL} & Paprika, edelsüß \\
        \SI[parse-numbers = false]{\nicefrac{1}{2}}{\TL} & Paprika, geräuchert \\
        1 Prise & Paprikapulver, scharf \\
        \SI{1}{\TL} & Salz \\
        \SI{1}{\EL} & Zitronenschale \\
         & Pfeffer
    }

    \preparation{
        \step Den Backofen auf \SI{200}{\celsius} (Ober- und Unterhitze) vorheizen.
        \step Zwiebeln schälen und in Ringe schneiden.
        Kartoffeln schälen und würfeln.
        Kräuterseitlinge grob rupfen (nicht schneiden, da die rupfige Oberfläche die Soße besser aufnimmt).
        Knoblauch schälen und kleinschneiden.
        Von einer unbehandelten Bio-Zitrone die Schale fein abreiben.
        \step In einem schweren Schmortopf das Bratöl erhitzen und darin die Zwiebelringe mit einer Prise Salz schmoren.
        Eine Handvoll des Pulled BBQ, eine Handvoll gerupfte Kräuterseitlinge und ein paar Kartoffelwürfel hinzugeben und für 5 Minuten bei hoher Temperatur schön braun anbraten.
        \step Getrocknete Steinpilze, Kümmel und Knoblauch dazugeben und alles für weitere 5 Minuten braten.
        Mit einem Schluck Rotwein ablöschen, um die Röstaromen zu lösen, diesen verkochen lassen.
        Wenn sich wieder Röstaromen am Topfboden bilden, erneut mit etwas Rotwein ablöschen. Dieser Vorgang nennt sich Deglacieren und sollte 3 bis 5 Mal wiederholt werden.
        \step In der Mitte des Bräters etwas Platz machen und nochmal 2 Esslöffel Bratöl dazugeben.
        Das Paprika-Tomatenmark, alle weiteren Gewürze (Majoran, edelsüßes Paprikapulver, geräuchertes Paprikapulver, scharfes Paprikapulver) und den Zitronenabrieb ins Öl geben und kurz anrösten.
        \step Die restlichen Pilze, Kartoffeln und Pulled BBQ dazugeben und mit der Gemüsebrühe und dem restlichen Rotwein aufgießen.
        \step Den Schmortopf in den vorgeheizten Ofen schieben und das Gulasch 1,5 bis 3 Stunden mit geschlossenem Deckel schmoren. Alle 30 Minuten durchrühren und bei Bedarf etwas Gemüsebrühe nachgießen.
        \step Zum Schluss mit einem festen Schneebesen oder Kartoffelstampfer die Kartoffeln zerdrücken, um die restliche Flüssigkeit zu binden. Final mit Salz und Pfeffer abschmecken.
    }

    % \suggestion[TITEL EINES VORSCHLAGS]{
	% 	VORSCHLAG (DURCH HORIZONTALE LINIE VOM REZEPT GETRENNT)
    % }

    \hint{
        \begin{itemize}
            \item \textbf{Mehrfaches Deglacieren:} Dies ist die Geheimwaffe für Geschmack. Anrösten lassen, mit einem Schluck Rotwein ablöschen, verkochen lassen, wieder anrösten --- mindestens 5 Mal wiederholen. Jedes Mal lösen sich neue Aromen vom Topfboden.
            \item \textbf{Pilze rupfen statt schneiden:} Kräuterseitlinge per Hand zerrupfen. Dadurch entsteht eine strukturierte Oberfläche, die Soße besser aufnimmt. 
            \item \textbf{Fett als Geschmacksträger:} Ein klassisches Gulasch ist keine Light-Version. Hochwertige Bratöle wie Rapsöl bringen nicht nur Geschmack, sondern auch wertvolle ungesättigte Fettsäuren.
            \item \textbf{Alternative ohne Pulled BBQ:} Das Rezept funktioniert auch nur mit Pilzen oder anderen veganen Fleischalternativen.
            \item \textbf{Alkoholfrei:} Statt Rotwein eine Mischung aus Gemüsebrühe und einem Schuss Balsamico oder Aroniasaft verwenden (Traubensaft ist zu süß).
            \item \textbf{Ohne Schmortopf:} Einfach einen herkömmlichen Topf nehmen und das Gulasch bei kleiner Hitze auf dem Herd langsam schmoren lassen.
            \item \textbf{Beilagen:} Passt perfekt zu Klößen, Brot zum Tunken oder cremiger Polenta.
        \end{itemize}        
    }

\end{recipeDP}