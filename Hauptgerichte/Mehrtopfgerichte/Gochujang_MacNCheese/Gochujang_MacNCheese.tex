\begin{recipeDP}
    [
        preparationtime = {\SI{50}{\minute}},
        % bakingtime = {\SI{ZEIT}{\minute} bis \SI{ZEIT}{\minute}},
        % bakingtemperature = {\protect\bakingtemperature{fanoven=\SI{TEMPERATUR}{\celsius}}},
        portion = {4 Portionen}
        source = {@vegan\_punks}
    ]
    {Gochujang Tofu mit Mac \& Cheese}

    \graph
        {
            big=Hauptgerichte/Mehrtopfgerichte/Gochujang_MacNCheese/big.jpg,
            small=Hauptgerichte/Mehrtopfgerichte/Gochujang_MacNCheese/small.jpg
        }

    % \introduction{
    %     EINLEITUNG
    % }

    \ingredients{
        \multicolumn{2}{l}{\textbf{Mac \& Cheese}} \\
        \SI{500}{\g} & Makkaroni \\
        \SI{400}{\g} & Seidentofu \index{Tofu!Seiden-} \\
        \SI{3}{\EL} & Butter \\
        \SI{2}{\EL} & Mehl \\
        \SI{500}{\ml} & Pflanzenmilch \\
        \SI{40}{\g} & \addtoidx{Hefeflocken} \\
        \SI{2}{\TL} & Apfelessig \\
        \SI{1}{\TL} & \addtoidx{Misopaste} \\
        \SI[parse-numbers = false]{\nicefrac{1}{2}}{\TL} & Kurkuma \\
        \SI[parse-numbers = false]{\nicefrac{1}{2}}{\TL} & Knoblauch \\
         & Salz \\
         & Pfeffer \\
        \\
        \multicolumn{2}{l}{\textbf{Gochujang Tofu}} \\
        \SI{400}{\g} & fester Naturtofu \index{Tofu!Natur-} \\
        \SI{2}{\EL} & Speisestärke \\
        \SI{1}{\EL} & Sesamöl \\
        \SI{2}{\EL} & \addtoidx{Gochujang-Paste} \\
        \SI{2}{\EL} & helle Sojasoße \\
        \SI{2}{\EL} & Agavendicksaft \\
		  \SI[parse-numbers = false]{\nicefrac{1}{2}}{\TL} & Knoblauch \\
        \SI{3}{\EL} & Wasser
    }

    \preparation{
        \step Die Nudeln in leicht gesalzenem Wasser al dente kochen.
        Währenddessen den Tofu in einer großen Schüsel in grobe Stücke zerreißen und in der Speisestärke wenden.
        Dann in einer Pfanne den Tofu mit etwas Öl von allen Seiten goldbraun anbraten.
        Danach kurz aus der Pfanne nehmen.
        \step Die Zutaten für die Soße (Gochujang, Sojasoße, Agavendicksaft, Knoblauch, Wasser) mit einander verrühren.
        Die Soße in der gleichen Pfanne kurz eindicken lassen und dann den Tofu darin wenden.
        \step Für die Nudeln die Butter in einem großen Topf zerlassen und dann mit dem Mehl zu einer Mehlschwitze verrühren.
        Die Misopaste mit unterrühren und dann nach und nach die Pflanzenmilch einrühren, sodass keine Klumpen entstehen.
        Wenn die ganze Milch eingerührt ist, die Soße etwas eindicken lassen.
        Die weiteren Zutaten (Hefeflocken, Apfelessig, Kurkuma, Knoblauch, Salz und Pfeffer) hinein geben und kurz köcheln lassen.
        \step Die Nudeln mit der Käsesoße verrühren und mit dem Tofu garniert servieren.
    }

    % \suggestion[TITEL EINES VORSCHLAGS]{
	% 	VORSCHLAG (DURCH HORIZONTALE LINIE VOM REZEPT GETRENNT)
    % }

    % \hint{
    %     HINWEIS (IN EINEM KASTEN UNTEN AUF DER SEITE)
    % }

\end{recipeDP}
