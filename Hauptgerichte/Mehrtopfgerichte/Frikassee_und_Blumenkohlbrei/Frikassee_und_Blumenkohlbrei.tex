\begin{recipeDP}
    [
        preparationtime = {\SI{45}{\minute}},
        % bakingtime = {\SI{12}{\minute} bis \SI{15}{\minute}},
        % bakingtemperature = {\protect\bakingtemperature{fanoven=\SI{180}{\celsius}}},
        portion = {4 Portionen},
        source = {@rabbitandwolves}
    ]
    {\addtoidx{Frikassee} und Blumenkohlbrei\index{Brei|Blumenkohl-}}

    \graph
        {
            % big=example-image
            small=Hauptgerichte/Mehrtopfgerichte/Frikassee_und_Blumenkohlbrei/small.png
        }

    % \introduction{einleitung}

    \ingredients{
        \multicolumn{2}{l}{\textbf{Frikassee}} \\
        \SI{3}{\EL} & Olivenöl \\
        1 & Zwiebel \\
        2 & Knoblauchzehen \\
        4 Stangen & \addtoidx{Sellerie} \\
        5 & Möhren \\
        \SI{500}{\g} & \addtoidx{Champignons} \\
        \SI{1}{\TL} & Thymian \\
        \SI[parse-numbers = false]{\nicefrac{1}{2}}{\TL} & Salbei \\
        \SI{1}{\TL} & Rosmarin \\
        \SI[parse-numbers = false]{\nicefrac{1}{2}}{\TL} & Muskatnuss \\
        \SI{55}{\g} & Butter \\
        \SI{35}{\g} & Mehl \\
        \SI{120}{\ml} & Weißwein \\
        \SI{500}{\ml} & Gemüsebrühe \\
        \SI{400}{\ml} & Kokosmilch \\
        Salz \\
        Pfeffer \\
        \\
        \multicolumn{2}{l}{\textbf{Blumenkohlbrei}} \\
        1 großer & \addtoidx{Blumenkohl} \\
        \SI{80}{\ml} & Mandelmilch \\
        \SI{2}{\EL} & Butter \\
        Salz \\
        Pfeffer \\
        Muskatnuss
    }

    \preparation{
        \step Zuerst die Zwiebel und den Sellerie würfeln, die Möhren in grobe Scheiben schneiden und die Pilze grob schneiden. Auch den Blumenkohl in grobe Röschen schneiden.
        \step Dann das Olivenöl in einem großen Topf bei mittlerer Hitze erhitzen. Die Zwiebel hinzugeben, den Knoblauch hinein pressen und den Sellerie mit hinzugeben. Das zusammen etwa \SI{3}{\minute} dünsten, bis die Zwiebel etwas glasig wird. Währenddessen die Blumenkohl-Röschen in einem großen Topf mit heißem, gesalzenem Wasser kochen (\SI{10}{\minute} oder bis ganz gar).
        \step Sind die Zwiebeln glasig, die Möhren und Pilze dazu geben und weitere \SI{3}{\minute} dünsten. Dann mit Salz und Pfeffer würzen, und wenn die Pilze (nach wiederum \SI{5}{\minute}) braun werden, mit den restlichen Gewürzen vermengen.
        \step Danach die Butter über dem Gemüse zerlassen und gut verrühren. Das Mehl gut darüber verteilen und klümpchenfrei verrühren.
        \step Als nächstes den Wein dazugeben und \SI{3}{\minute} köcheln lassen. Dann die Gemüsebrühe und die Kokosmilch einrühren und alles auf niedriger Hitze \SI{10}{\minute} bis \SI{15}{\minute} köcheln lassen, dabei immer wieder gut umrühren.
        \step Ist der Blumenkohl gar, wird das Kochwasser abgegossen und die Milch zu dem Blumenkohl in den Topf gegeben. Beides mit dem Stabmixer pürieren, bis ein glatter Brei entsteht und mit den Gewürzen abschmecken.
    }

    % \suggestion[Title of Suggestion]{
	% 	Suggestion
    % }

    % \hint{Hint}

\end{recipeDP}
