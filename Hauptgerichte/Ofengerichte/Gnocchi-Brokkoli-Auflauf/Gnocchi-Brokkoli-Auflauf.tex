\begin{recipeDP}
    [
        preparationtime = {\SI{25}{\minute}},
        bakingtime = {\SI{15}{\minute} bis \SI{20}{\minute}},
        bakingtemperature = {\protect\bakingtemperature{topbottomheat=\SI{180}{\celsius}}},
        portion = {4 Portionen},
        source = {elavegan.com}
    ]
    {Gnocchi-Brokkoli-Auflauf}

    \graph
        {
            big=Hauptgerichte/Ofengerichte/Gnocchi-Brokkoli-Auflauf/big.jpg,
            small=Hauptgerichte/Ofengerichte/Gnocchi-Brokkoli-Auflauf/small.jpg
        }

    % \introduction{
    %     EINLEITUNG
    % }

    \ingredients{
        1 & \addtoidx{Brokkoli} \\
        \SI{2}{\EL} & Olivenöl \\
        3 & Knoblauchzehen \\
        2 & rote Paprika \\
        \SI[parse-numbers = false]{\nicefrac{1}{2}}{} & Zwiebel \\
        \SI{300}{\ml} & Gemüsebrühe \\
        \SI{400}{\g} & gehackte Tomaten \\
        \SI{200}{\ml} & Sojasahne \\
        \SI{500}{\g} & vorgekochte \addtoidx{Gnocchi} \\
         & Reibekäse \\
         & Salz \\
         & Pfefer
    }

    \preparation{
        \step Den Brokkoli in mundgerechte Röschen teilen und in einen Topf geben.
        Mit Wasser bedecken und zusammen mit etwas Salz zum Kochen bringen.
        4 Minuten kochen lassen, danach das Wasser abgießen.
        Währenddessen das Olivenöl in einem mittelgroßen Topf erhitzen und die gewürfelten Paprikaschoten mit der Zwiebel und dem Knoblauch hinzufügen und mehrere Minuten anbraten, bis sich Röstaromen entwickeln.
        \step Anschließend mit der Gemüsebrühe ablöschen, die gehackten Tomaten hinzufügen und etwa 15 Minuten köcheln lassen, bis das Gemüse weich ist und die Flüssigkeit etwas reduziert ist.
        Den Ofen auf 180 Grad Celsius (Ober- Unterhitze) vorheizen.
        \step Die Gemüsemischung (jedoch nicht den Brokkoli) mit einem Pürierstab fein pürieren und mit der Pflanzencreme vermischen.
        Mit Salz und Pfeffer abschmecken.
        \step Die Gnocchi (aus dem Kühlregal) in eine Auflaufform geben und mit den Brokkoli-Röschen vermischen.
        Die pürierte Gemüsesoße darüber gießen und die Form in den Ofen schieben.
        10 Minuten backen, dann mit veganem Käse bestreuen, weitere 15 Minuten im Ofen backen bis der Käse goldbraun ist.

    }

    % \suggestion[TITEL EINES VORSCHLAGS]{
	% 	VORSCHLAG (DURCH HORIZONTALE LINIE VOM REZEPT GETRENNT)
    % }

    % \hint{
    %     HINWEIS (IN EINEM KASTEN UNTEN AUF DER SEITE)
    % }

\end{recipeDP}