\begin{recipeDP}
    [
        preparationtime = {\SI{45}{\minute}},
        bakingtime = {\SI{40}{\minute}},
        bakingtemperature = {\protect\bakingtemperature{fanoven=\SI{180}{\celsius}}},
        portion = {4 Portionen},
        source = {zuckerjagdwurst.com}
    ]
    {Lasagne mit Béchamelsoße}

    \graph
        {
            big=Hauptgerichte/Ofengerichte/Lasagne/big.jpg,
            small=Hauptgerichte/Ofengerichte/Lasagne/small.jpg
        }

    % \introduction{
    %     EINLEITUNG
    % }

    \ingredients{
        \multicolumn{2}{l}{\textbf{Hack-Füllung}} \\
        \SI{150}{\g} & Sojagranulat \\
        \SI{1}{\l} & Gemüsebrühe \\
        1 & Zwiebel \\
        4 & Möhren \\
        \SI{2}{\EL} & Sojasoße \\
        \SI{4}{\EL} & Tomatenmark \\
        \SI{1}{\EL} & Agavendicksaft \\
        \SI{100}{\ml} & Rotwein \\
        \SI{600}{\g} & gestückelte Tomaten \\
        \SI[parse-numbers = false]{\nicefrac{1}{2}}{\TL} & Basilikum \\
        \SI[parse-numbers = false]{\nicefrac{1}{2}}{\TL} & Oregano \\
        \SI[parse-numbers = false]{\nicefrac{1}{2}}{\TL} & Cayennepfeffer \\
        \SI[parse-numbers = false]{\nicefrac{1}{2}}{\TL} & Paprikapulver \\
         & Pflanzenöl \\
         & Salz \\
         & Pfeffer \\
        \\
        \multicolumn{2}{l}{\textbf{Béchamelsoße}} \\
        \SI{3}{\EL} & Weizenmehl \\
        \SI{3}{\EL} & Margarine \\
        \SI{250}{\EL} & Hafermilch \\
        \SI{1}{\EL} & Hefeflocken \\
         & Salz \\
         & Pfeffer \\
         & Muskat \\
         \\
         & Lasagneplatten \\
         & veganer Käse
    }

    \preparation{
        \step Für die Tomaten-Hack-Schicht getrocknetes Sojagranulat in einer hitzefesten Schüssel oder einem Topf mit heißer Gemüsebrühe übergießen und \ca 10 bis 15~Minuten einweichen lassen. Anschließend abgießen, so viel Flüssigkeit wie möglich aus dem Granulat herausdrücken und großzügig mit Salz und Pfeffer würzen. Alternativ lässt sich dieser Schritt auch überspringen, wenn man stattdessen Hackfleisch-Ersatz verwendet, der nur noch angebraten werden muss.
        \step Während das Sojagranulat einweicht, Zwiebel schälen und fein würfeln. Möhre raspeln oder ebenfalls in kleine Würfel schneiden.
        \step Pflanzenöl in einer Pfanne über mittlerer bis hoher Hitze erwärmen und das Sojagranulat darin \ca 5~Minuten scharf anbraten, damit es gebräunt wird. Sojasauce dazugeben und weitere 5~Minuten anbraten. Zwiebel und Möhre in die Pfanne geben und alles gemeinsam über mittlerer Hitze \ca 4 bis 5~Minuten anbraten. Danach Tomatenmark und Agavendicksaft dazugeben und etwa 3~Minuten köcheln lassen.
        \step Mit Rotwein ablöschen und 5~Minuten einköcheln lassen, bevor die gestückelten Tomaten dazukommen. Ketchup und Gewürze dazugeben, mit Salz und Pfeffer abschmecken und die Soße auf kleiner Hitze köcheln lassen, bis die Béchamelsauce fertig ist.
        \step Für die Béchamelsauce vegane Butter in einem kleinen Topf schmelzen lassen und Mehl einrühren. Eine Minute über kleiner Hitze anschwitzen, danach die pflanzliche Milch langsam dazugeben und immer schön rühren, damit sich keine Klumpen bilden. Mit Hefeflocken, Salz und Pfeffer abschmecken.
        \step Backofen auf \SI{180}{\celsius} vorheizen. Es wird Zeit für die Auflaufform: Mit einer dünnen Schicht Tomate-Sojahack beginnen, dann trockene Lasagneplatten darüberlegen, anschließend wieder eine Schicht Tomaten-Sojahack-Sauce und dann das ganze mit einer Schicht Béchamelsauce bedecken. Jetzt wieder Lasagneplatten, Tomaten-Sojahack-Sauce, Béchamelsauce und immer so weiter, bis alles aufgebraucht ist. Oben mit einer letzten Schicht Béchamelsauce aufhören. Optional jetzt noch veganen Käse darüberstreuen.
        \step Die Lasagne bei \SI{180}{\celsius} \ca 40~Minuten backen, oder eben bis die Lasagneplatten weich sind. Danach aus dem Backofen nehmen und die Lasagne vor dem Anschneiden noch ca. 5 Minuten stehen und anziehen lassen.
    }

    % \suggestion[TITEL EINES VORSCHLAGS]{
	% 	VORSCHLAG (DURCH HORIZONTALE LINIE VOM REZEPT GETRENNT)
    % }
    %
    % \hint{
    %     HINWEIS (IN EINEM KASTEN UNTEN AUF DER SEITE)
    % }

\end{recipeDP}
