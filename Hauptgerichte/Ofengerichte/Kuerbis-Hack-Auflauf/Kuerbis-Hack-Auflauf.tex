\begin{recipeDP}
    [
        preparationtime = {\SI{20}{\minute}},
        bakingtime = {\SI{25}{\minute}},
        bakingtemperature = {\protect\bakingtemperature{fanoven=\SI{180}{\celsius}}},
        portion = {4 Portionen},
        source = {einfachkochen.de}
    ]
    {\addtoidx{Kürbis}-Hack-Auflauf}

    \graph
        {
            big=Hauptgerichte/Ofengerichte/Kuerbis-Hack-Auflauf/big.jpg,
            small=Hauptgerichte/Ofengerichte/Kuerbis-Hack-Auflauf/small.jpg
        }

    % \introduction{
    %     EINLEITUNG
    % }

    \ingredients{
        2 & Zwiebeln \\
        2 & Knoblauchzehen \\
        1 & Hokaido-Kürbis \\
        1 Zweig & Rosmarin \\
        \SI{2}{\EL} & Öl \\
        \SI{400}{\g} & \addtoidx{Soja-Hack} \\
        \SI{2}{\EL} & Tomatenmark \\
        & Petersilie \\
        & Salz \\
        \SI{400}{\ml} & Gemüsebrühe \\
        \SI{400}{\ml} & Wasser \\
        \SI{300}{\ml} & Kochsahne \\
         & Pfeffer \\
         & Muskatnuss \\
        \SI{500}{\g} & Fussili-Nudeln \\
        \SI{100}{\g} & Streukäse
    }

    \preparation{
        \step Zwiebeln und Knoblauch abziehen und fein würfeln.
        Kürbis sowie Kräuter waschen und abtropfen lassen.
        Kürbis entkernen und in 2 cm große Stücke schneiden.
        Kräuternadeln abzupfen und hacken.
        \step Backofen auf 180 Grad Umluft vorheizen und eine große Auflaufform mit Öl ausreiben.
        Öl in einer Pfanne erhitzen, Hack mit Zwiebel- und Knoblauchwürfeln zugeben und 5 Minuten bei mittlerer bis hoher Hitze braten.
        Tomatenmark, Kürbisstücke und Rosmarin in die Pfanne dazugeben und alles mit Salz würzen.
        Etwa eine Minute weiter braten und anschließend alles mit Gemüsebrühe und Wasser ablöschen.
        Anschließend Sahne einrühren, die Sauce aufkochen und mit Salz, Pfeffer und Muskatnuss kräftig würzen - die Sauce sollte etwas überwürzt schmecken, da die Nudeln beim Garen später noch Salz aufnehmen.
        \step Die ungekochten Nudeln in der Auflaufform verteilen, Kürbis-Hackfleischsoße zugeben und alles miteinander vermengen. Mit Käse bestreuen und im vorgeheizten Ofen etwa 25 Minuten goldgelb backen.
        Kürbis-Hackfleisch-Auflauf aus dem Ofen nehmen, mit der gehackten Petersilie bestreuen und servieren.
    }

    % \suggestion[TITEL EINES VORSCHLAGS]{
	% 	VORSCHLAG (DURCH HORIZONTALE LINIE VOM REZEPT GETRENNT)
    % }

    % \hint{
    %     HINWEIS (IN EINEM KASTEN UNTEN AUF DER SEITE)
    % }

\end{recipeDP}