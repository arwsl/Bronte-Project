\begin{recipeDP}
    [
        preparationtime = {\SI{40}{\minute}},
        bakingtime = {\SI{30}{\minute}},
        bakingtemperature = {\protect\bakingtemperature{topbottomheat=\SI{200}{\celsius}}},
        portion = {12 Enchiladas},
        source = {elavegan.com}
    ]
    {Linsen Enchiladas}

    \graph
        {
            big=Hauptgerichte/Ofengerichte/Linsen_Enchiladas/big.jpg,
            small=Hauptgerichte/Ofengerichte/Linsen_Enchiladas/small.jpg
        }

    % \introduction{
    %     EINLEITUNG
    % }

    \ingredients{
        12 &  \addtoidx{Tortillas} \\
        \SI{200}{\g} & Reibekäse \\
        \\
        \multicolumn{2}{l}{\textbf{Füllung}} \\
		\SI{200}{\g} & trockene Berglinsen \index{Linsen!Berg-} \\
        \SI{600}{\ml} & Gemüsebrühe \\
        \SI{70}{\g} & Sonnenblumenkerne \\
        \SI{120}{\g} & Haferflocken \\
        2 & Paprika \\
        3 & Möhren \\
        1 & Zwiebel \\
        2 & Knoblauchzehen \\
        \SI{2}{\EL} & geschrotete Leinsamen \index{Leinsamen!geschrotet} \\
        \SI[parse-numbers = false]{\nicefrac{1}{2}}{\EL} & Zwiebelpulver \\
        \SI[parse-numbers = false]{\nicefrac{1}{2}}{\EL} & Knoblauchpulver \\
        \SI{2}{\TL} & Oregano \\
        \SI{2}{\TL} & Kreuzkümmel \\
        \SI{1}{\TL} & Paprikapulver \\
        2 & Chilischoten \\
         & Salz \\
         & Pfeffer \\
        \SI{2}{\EL} & Öl \\
        \\
        \multicolumn{2}{l}{\textbf{Soße}} \\
		\SI{2}{\EL} & Öl \\
        \SI{2}{\EL} & Mehl \\
        \SI{1000}{\g} & Tomaten Passata \\
        \SI[parse-numbers = false]{\nicefrac{1}{2}}{\EL} & Chilipulver \\
        \SI{1}{\TL} & Zwiebelpulver \\
        \SI{1}{\TL} & Knoblauchpulver \\
        \SI{1}{\TL} & Kreuzkümmel \\
        \SI[parse-numbers = false]{\nicefrac{1}{4}}{\TL} & Cayennepfeffer \\
        & Salz \\
        & Pfeffer
    }

    \preparation{
        \step Die Linsen abspülen und ein einem Mitelgroßen Topf mit der Gemüsebrühe aufkochen.
        Mit Deckel für etwa 25 Minuten köcheln lassen (oder bis sie weich sind).
        Dann abgießen und zur Seite Stellen.

        \step Währen die Linsen kochen, die Soße vorbereiten, indem das Öl in einem mittelgroßen Topf erhitzt wird.
        Die Gewürze und das Mehl unterrühren und für \SI{2}{\minute} anbraten.
        Dann die Tomatensoße hinzu geben und köcheln lassen bis die Soße andickt.

        \step Für die Füllung die Zwiebel und den Knoblauch fein Hacken, die Paprika und Tomate klein schneiden und die Möhren grob raspeln.
        Öl in einer großen Pfanne erhitzen und die Zweibel und den Knoblauch anbraten.
        Nach etwa 4 Minuten Paprika, Tomate, die Möhren und die kleingeschnittene Chilischoten dazu geben.
        gemeinsam für 7 Minuten anbraten.

        \step Die trockenen Zutaten in einer Küchenmaschine mixen, bis alles klein ist.
        Dann dieses Mehl in die Gemüsepfanne geben und vermischen, sodass eine formbare Füllung entsteht.

        \step Den Ofen auf \SI{200}{\celsius} vorheizen und eine Auflaufformen einfetten.
        Die Tortillas dann mit etwa 2 gehäuften Esslöffeln Füllung vorbereiten und aufrollen.
        Die Tortillas in die Form geben und schließlich mit der Tomatensoße bedecken.
        Mit dem Reibekäse garnieren.
        Die Formen in den Ofen geben und für 25 bis 30 Minute backen.
    }

    % \suggestion[TITEL EINES VORSCHLAGS]{
	% 	VORSCHLAG (DURCH HORIZONTALE LINIE VOM REZEPT GETRENNT)
    % }

    % \hint{
    %     HINWEIS (IN EINEM KASTEN UNTEN AUF DER SEITE)
    % }

\end{recipeDP}