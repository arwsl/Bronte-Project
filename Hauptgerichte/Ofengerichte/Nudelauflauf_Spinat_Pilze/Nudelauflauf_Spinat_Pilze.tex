\begin{recipeDP}
    [
        preparationtime = {\SI{20}{\minute}},
        bakingtime = {\SI{20}{\minute} bis \SI{25}{\minute}},
        bakingtemperature = {\protect\bakingtemperature{fanoven=\SI{180}{\celsius}}},
        portion = {4 Portionen},
        source = {biancazapatka.com}
    ]
    {Nudelauflauf mit Spinat und Pilzen}

    \graph{
            big=Hauptgerichte/Ofengerichte/Nudelauflauf_Spinat_Pilze/big.jpg,
            small=Hauptgerichte/Ofengerichte/Nudelauflauf_Spinat_Pilze/small.jpg
    }

    % \introduction{
    %     EINLEITUNG
    % }

    \ingredients{
        \SI{500}{\g} & Nudeln \\
        \SI{2}{\EL} & Margarine \\
        1 & Zwiebel \\
        2 & Knoblauchzehen \\
        \SI{250}{\g} & \addtoidx{Champignons} \\
        \SI{1}{\EL} & Mehl \\
        \SI{360}{\ml} & Gemüsebrühe \\
        \SI{150}{\g} & Kochcreme \\
        \SI{100}{\g} & Babyspinat \index{Spinat!Baby-} \\
        \SI[parse-numbers = false]{1\nicefrac{1}{2}}{\TL} & ital. Kräuter \\
         & Salz \\
         & Pfeffer \\
         & Muskat \\
        \\
         & Hirtenkäse \\
         & Reibekäse
    }

    \preparation{
        \step Die Margraine in einer großen Pfanne zerlassen und die Zwiebeln 2-3 Minuten glasig braten.
        Dann den Knoblauch hineinpressen und die Champignons in Scheiben geschnitten hinzugeben und weitere 5 Minuten anbraten. 
        Dann mit dem Mehl bestreuen und die Gemüsebrühe eingießen.
        Mit italienischen Kräutern, etwas Salz, Pfeffer und Muskat würzen und etwa 5 Minuten köcheln lassen.
        \step Währenddessen die Nudeln in kochendem Salzwasser sehr bissfest kochen (2-3 Minuten weniger als auf der Verpackung angegeben) und den Ofen auf 180 °C Umluft vorheizen.
        \step Die Kochcreme unter die Sauce rühren, nochmal abschmecken und bei Bedarf nachwürzen.
        Dann den Spinat untermischen.
        \step Die Nudeln abgießen und mit der Soße vermengen.
        Alles in eine Auflaufform füllen und nach Geschmack mit Hirtenkäse und veganem Käse bestreuen.
        Im vorgeheizten Ofen 20 bis 25 Minuten goldbraun backen.
    }

    % \suggestion[TITEL EINES VORSCHLAGS]{
	% 	VORSCHLAG (DURCH HORIZONTALE LINIE VOM REZEPT GETRENNT)
    % }

    % \hint{
    %     HINWEIS (IN EINEM KASTEN UNTEN AUF DER SEITE)
    % }

\end{recipeDP}