\begin{recipeDP}
    [
        preparationtime = {\SI{15}{\minute}},
        bakingtime = {\SI{30}{\minute}},
        bakingtemperature = {\protect\bakingtemperature{fanoven=\SI{180}{\celsius}}},
        portion = {4 Portionen},
        source = {@elavegan}
    ]
    {\addtoidx{Curry}-\addtoidx{Blumenkohl}-\addtoidx{Auflauf}}

    \graph
        {
            big=Hauptgerichte/Ofengerichte/Curry-Blumenkohl-Auflauf/big.jpg,
            small=Hauptgerichte/Ofengerichte/Curry-Blumenkohl-Auflauf/small.jpg
        }

    % \introduction{
    %     EINLEITUNG
    % }

    \ingredients{
        \multicolumn{2}{l}{\textbf{Curry}} \\
        2 & mittelgroße Blumenkohle \\
        \SI{500}{\g} & gewürfelte Dosentomaten \\
        3 & Knoblauchzehen \\
        \SI{3}{\cm} & frischer Ingwer \\
        \SI{4}{\TL} & Currypulver \\
        \SI{1}{\TL} & Salz \\
        \SI{1}{\TL} & Zwiebelpulver \\
        \SI{1}{\TL} & Paprika \\
        \SI{1}{\TL} & Kreuzkümmel \\
        \SI{1}{\TL} & Pfeffer \\
        \\
        \multicolumn{2}{l}{\textbf{Kokosmilchsoße}} \\
        \SI{240}{\g} & \addtoidx{Kokosmilch} \\
        \SI{1}{\EL} & Maisstärke \\
        \SI{2}{\EL} & Hefeflocken \\
        \SI[parse-numbers = false]{\nicefrac{1}{2}}{\TL} & Salz \\
        \SI[parse-numbers = false]{\nicefrac{1}{2}}{\TL} & Paprikapulver \\
        \\
         & veganen Reibekäse
    }

    \preparation{
        \step Den Blumenkohl in mundgerechte Röschen aufteilen und in einen großen Topf geben. Die Röschen mit Salzwasser bedecken und zum Kochen bringen. Etwa 10 Minuten kochen lassen, bis der Blumenkohl gar ist (nicht zu lange kochen!), dann das Wasser abgießen.
        \step Den Ofen auf \SI{180}{\celsius} vorheizen. Die gehackten Tomaten, den frischen Knoblauch, den Ingwer und alle Gewürze (Curry, Salz, Zwiebelpulver, Knoblauchpulver, Paprika, gemahlener Kreuzkümmel, schwarzer Pfeffer) in eine ofenfeste Pfanne/Auflaufform geben. Anschließend umrühren und die gekochten Blumenkohlröschen hinzufügen.
        \step Alle Zutaten für die Kokosmilchsoße in eine mittelgroße Schüssel geben und mit einem Schneebesen verrühren. Die Mischung über den Blumenkohl gießen.
        \step Den Auflauf \SI{10}{\minute} im Ofen backen, dann umrühren. Mit veganem Käse bestreuen und weitere \SI{10}{\minute} backen. Mit gekochtem Reis oder Naan servieren!
    }

    % \suggestion[TITEL EINES VORSCHLAGS]{
	% 	VORSCHLAG (DURCH HORIZONTALE LINIE VOM REZEPT GETRENNT)
    % }

    \hint{
        Gut passt auch Brokkoli statt oder zum Blumenkohl. Dabei aber beachten, dass Brokkoli schneller gar ist, als Blumenkohl.
    }

\end{recipeDP}
