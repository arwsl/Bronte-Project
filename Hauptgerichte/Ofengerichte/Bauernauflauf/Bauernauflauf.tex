\begin{recipeDP}
    [
        preparationtime = {\SI{40}{\minute}},
        bakingtime = {\SI{35}{\minute}},
        bakingtemperature = {\protect\bakingtemperature{topbottomheat=\SI{180}{\celsius}}},
        portion = {4 Portionen},
        source = {Marlon Howe}
    ]
    {\addtoidx{Bauernauflauf}}

    % \graph
    %     {
    %         big=TEIL/KAPITEL/REZEPT/big.jpg,
    %         small=TEIL/KAPITEL/REZEPT/small.jpg
    %     }

    % \introduction{
    %     EINLEITUNG
    % }

    \ingredients{
        \SI{600}{\g} & \addtoidx{Süßkartoffeln} \\
        4 & große Möhren \\
        \SI{1}{\TL} & Salz \\
        \SI{1}{\EL} & Zucker \\
        \SI{1}{\EL} & Butter \\
        \\
        \SI{400}{\g} & frische Bobby-Bohnen \index{Bohnen!Bobby-} \\
        \\
        \SI{250}{\g} & \addtoidx{Sojagranulat} \\
        \SI{2}{\EL} & Tomatenmark \\
        2 & Zwiebeln \\
        \SI{1}{\TL} & Salz \\
        \SI{1}{\EL} & Zucker \\
        \SI{200}{\ml} & passierte Tomaten \\
        \SI{100}{\g} & \addtoidx{Reibekäse}
    }

    \preparation{
        \step Für den Süßkartoffel-Möhren-Brei zuerst die Süßkartoffeln und Möhren schälen. Süßkartoffeln in mittelgroße Würfel teilen und Möhren grob würfeln. Salz, Zucker und Butter in einem mittelgroßen Topf mit Wasser zum Kochen bringen, dann Süßkartoffeln und Karotten darin \ca 20 Minuten gar kochen. Dann das Wasser abgießen und das Gemüse mithilfe eines Kartoffelstampfers zu Brei verarbeiten und mit Muskat würzen.
        \step Die Bohnen in einem weiterem Topf je nach gewünschter Bissfestigkeit etwa 12 bis 15 Minuten abkochen. Anschließend die Bohnen mit kaltem Wasser abschrecken.
        \step Das Sojagranulat in einem anderen Topf aufkochen und 10 Minuten quellen lassen. Währenddessen die Zwiebeln fein würfeln und parallel in einer Pfanne mit etwas Salz und Zucker anschwitzen. Anschließend das Sojagranulat in einem Sieb abgießen und möglichst viel Flüssigkeit ausdrücken. Dann mit in die Pfanne geben und auf hoher Stufe anbraten. Tomatenmark und passierte Tomaten hinzugeben.
        \step Zuletzt den Ofen auf \SI{180}{\celsius} Ober- und Unterhitze aufheizen. Dann eine große Auflaufform befüllen: Dazu die Auflaufform mit einer feinen Schicht Soja-Hack auslegen. Die Hälfte des Hacks für eine weitere Schicht aufbewahren. Die Bohnen auf die erste Schicht Soja-Hack geben und das restliche Hack darauf verteilen. Den Brei auf der zweiten Hackschicht verteilen. Abschließend je nach Belieben mit Käse bestreuen und für 35 Minuten im Ofen backen.
    }

    % \suggestion[TITEL EINES VORSCHLAGS]{
	% 	VORSCHLAG (DURCH HORIZONTALE LINIE VOM REZEPT GETRENNT)
    % }
    %
    % \hint{
    %     HINWEIS (IN EINEM KASTEN UNTEN AUF DER SEITE)
    % }

\end{recipeDP}
