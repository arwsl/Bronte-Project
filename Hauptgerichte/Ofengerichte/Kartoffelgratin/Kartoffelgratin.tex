\begin{recipeDP}
    [
        preparationtime = {\SI{20}{\minute}},
        bakingtime = {\SI{60}{\minute} bis \SI{70}{\minute}},
        bakingtemperature = {\protect\bakingtemperature{fanoven=\SI{200}{\celsius}}},
        portion = {4 Portionen},
        source = {@elavegan}
    ]
    {\addtoidx{Kartoffelgratin}}

    \graph
        {
            big=Hauptgerichte/Ofengerichte/Kartoffelgratin/big.jpg,
            small=Hauptgerichte/Ofengerichte/Kartoffelgratin/small.jpg
        }

    \introduction{
        Veganes Kartoffelgratin was super cremig, sämig und geschmackvoll ist! Das Rezept ist pflanzlich, von Natur aus glutenfrei und sehr einfach zuzubereiten. Es ist das perfekte Gericht für die Feiertage oder für ein Abendessen am Wochenende!
    }

    \ingredients{
        \SI{1,4}{\kg} & (geschälte) Kartoffeln \\
        1 & mittelgroße Zwiebel \\
        \\
        \multicolumn{2}{l}{\textbf{Cashewsoße}} \\
        \SI{400}{\ml} & Sojamilch \\
        \SI{400}{\ml} & Gemüsebrühe \\
        \SI{150}{\ml} & \addtoidx{Cashews} \\
        \SI{2}{\EL} & Kartoffelstärke \\
        \SI[parse-numbers = false]{\nicefrac{1}{2}}{\TL} & Paprika \\
         & Pfeffer \\
         & Salz \\
        \SI{1}{\TL} & Zwiebelpulver \\
        \SI{30}{\g} & Hefeflocken \\
        2 & Knoblauchzehen
    }

    \preparation{
        \step Die Cashewnüsse \ca 15 Minuten in kochendem Wasser einweichen und dann das Wasser abschütten.
        \step In der Zwischenzeit die Kartoffeln schälen und in dünne Scheiben schneiden (\ca \SI{3}{\mm} dick). Die Zwiebel schälen und in dünne Scheiben schneiden. Den Ofen auf \SI{200}{\celsius} vorheizen und eine 23 x 33 cm große Auflaufform einfetten. Beiseite stellen.
        \step Für die Cashewsoße alle Saucen-Zutaten in einen Mixer geben. Etwa ein bis zwei Minuten lang (oder bis die Sauce klumpenfrei ist) pürieren. Die Soße wird anfangs sehr dünn sein, aber keine Sorge, denn sie wird beim Backen andicken. Soße abschmecken und bei Bedarf mehr Salz/Pfeffer/Gewürze dazugeben.
        \step Die Hälfte der Kartoffeln in die vorbereitete Auflaufform geben und die Hälfte der Sauce darüber gießen. Nun die in Scheiben geschnittene Zwiebel darauf verteilen und die noch vorhandenen Kartoffelscheiben obendrauf verteilen. Die restliche Sauce darüber gießen.
        \step Etwa 60-70 Minuten lang backen. Die Backzeit hängt von der Dicke der Kartoffelscheiben ab. Gegebenenfalls ist der Auflauf schon früher fertig. Die Kartoffeln sollten zart und goldbraun sein.
    }

    % \suggestion[TITEL EINES VORSCHLAGS]{
	% 	VORSCHLAG (DURCH HORIZONTALE LINIE VOM REZEPT GETRENNT)
    % }
    %
    % \hint{
    %     HINWEIS (IN EINEM KASTEN UNTEN AUF DER SEITE)
    % }

\end{recipeDP}
