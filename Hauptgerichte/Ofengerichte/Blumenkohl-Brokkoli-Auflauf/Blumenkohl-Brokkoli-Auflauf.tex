\begin{recipeDP}
    [
        preparationtime = {\SI{45}{\minute}},
        bakingtime = {\SI{25}{\minute}},
        bakingtemperature = {\protect\bakingtemperature{topbottomheat=\SI{180}{\celsius}}},
        portion = {1 Auflaufform (30 x 40 cm)},
        source = {zuckerjagdwurst.com}
    ]
    {Blumenkohl-Brokkoli-Auflauf mit Hack}

    \graph
        {
            big=Hauptgerichte/Ofengerichte/Blumenkohl-Brokkoli-Auflauf/big.jpg,
            small=Hauptgerichte/Ofengerichte/Blumenkohl-Brokkoli-Auflauf/small.jpg
        }

    \introduction{
        Cremiger Auflauf mit Blumenkohl, Brokkoli, Kartoffeln, veganem Hack und einer würzigen Béchamel-Soße, überbacken mit knuspriger Käsekruste.
    }

    \ingredients{
        \multicolumn{2}{l}{\textbf{Auflauf}} \\
        \SI{1000}{\g} & \addtoidx{Blumenkohl} \\
        \SI{500}{\g} & \addtoidx{Brokkoli} \\
        \SI{500}{\g} & festkochende \addtoidx{Kartoffeln} \\
        2 & Zwiebeln \\
        2 & Knoblauchzehen \\
        \SI{25}{\g} & Schnittlauch \\
        \SI{250}{\g} & veganes Hack \\
        \SI{150}{\g} & Sonnenblumenkerne \\
        \SI{300}{\g} & \addtoidx{Reibekäse} \\
        \SI{50}{\g} & Panko Paniermehl \\
        & Pflanzenöl \\
        \\
        \multicolumn{2}{l}{\textbf{Béchamel-Soße}} \\
        \SI{50}{\g} & vegane Butter \\
        \SI{20}{\g} & Weizenmehl (Type 405) \\
        \SI{500}{\ml} & pflanzliche Milch \\
        \SI{4}{\EL} & Hefeflocken \\
        \SI{100}{\g} & veganer Reibekäse \\
        & Muskat \\
        & Salz \\
        & Pfeffer \\
        \\
        \multicolumn{2}{l}{\textbf{Salat}} \\
        \SI{200}{\g} & Feldsalat \\
        \SI{3}{\EL} & Olivenöl \\
        \SI{2}{\EL} & Balsamico-Essig \\
        \SI{1}{\EL} & Ahornsirup \\
        & Salz \\
        & Pfeffer
    }

    \preparation{
        \step Blumenkohl und Brokkoli in kleine Röschen schneiden, den Strunk in 1 Zentimeter große Würfel schneiden.
        Die Blätter ebenfalls in kleine Stücke schneiden.
        Kartoffeln schälen und ebenfalls würfeln.
        Zwiebeln und Knoblauchzehen schälen und fein würfeln.
        Schnittlauch in feine Röllchen schneiden.
        \step Die Kartoffelwürfel in einem großen Topf mit reichlich gesalzenem Wasser bedecken, zum Kochen bringen und circa 5 Minuten kochen lassen.
        Nach zwei Minuten der Kochzeit Blumenkohl- und Brokkolistücke dazugeben und die restlichen circa 3 Minuten zusammen köcheln.
        Das Gemüse durch ein Sieb abgießen und beiseitestellen.
        Parallel dazu Pflanzenöl in einer Pfanne erhitzen und Zwiebel und Knoblauch darin anschwitzen, bis die Zwiebelwürfel glasig sind.
        Veganes Hack dazugeben und alles braten, bis das vegane Hack gut gebräunt ist.
        Mit Salz und Pfeffer abschmecken.
        \step Sonnenblumenkerne ohne Fett in einer Pfanne rösten.
        Einen Esslöffel davon beiseitestellen für den Salat.
        \step Für die Béchamel-Soße vegane Butter in einem Topf zum Schmelzen bringen.
        Dann Mehl dazu rühren und alles mit pflanzlicher Milch aufgießen.
        Hefeflocken, Muskat, Salz und Pfeffer dazugeben und für eine Minute kräftig aufkochen lassen.
        Dann veganen Käse unterrühren, bis er geschmolzen ist.
        \step Den Ofen auf \SI{180}{\celsius} (Ober-/Unterhitze) vorheizen und eine rechteckige Auflaufform (30 mal 40 Zentimeter) mit veganer Butter oder Pflanzenöl fetten.
        Die Blumenkohl-Brokkoli-Kartoffel-Mischung, gebratenes veganes Hack, geröstete Sonnenblumenkerne und den Großteil vom Schnittlauch in der Form verteilen, die vegane Béchamel darauf verteilen und alles vermengen.
        \step Veganen Reibekäse und Panko Paniermehl miteinander vermischen und den Auflauf damit bestreuen, bis das Gemüse bedeckt ist.
        Den Auflauf circa 25 Minuten bei \SI{180}{\celsius} (Ober-/Unterhitze) backen, bis sich eine goldbraune Käsekruste gebildet hat und das Gemüse bissfest, aber nicht zu weich ist.
        \step Während der Auflauf backt, Feldsalat waschen, abtropfen lassen, grob auseinander zupfen und in eine Schüssel geben.
        Mit den restlichen gerösteten Sonnenblumenkernen, Olivenöl, Balsamico-Essig, Ahornsirup, Salz und Pfeffer mischen.
        Den Salat zum gebackenen Auflauf servieren und den Auflauf mit dem restlichen Schnittlauch bestreuen.
    }

    % \suggestion[TITEL EINES VORSCHLAGS]{
	% 	VORSCHLAG (DURCH HORIZONTALE LINIE VOM REZEPT GETRENNT)
    % }

    \hint{
        \textbf{Vorbereitung:} Der Auflauf lässt sich gut vorbereiten und vor dem Backen ein paar Stunden im Kühlschrank aufbewahren.
    }

\end{recipeDP}
