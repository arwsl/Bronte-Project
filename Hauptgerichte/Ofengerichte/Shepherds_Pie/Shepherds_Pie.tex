\begin{recipeDP}
    [
        preparationtime = {\SI{40}{\minute}},
        bakingtime = {\SI{20}{\minute}},
        bakingtemperature = {\protect\bakingtemperature{topbottomheat=\SI{200}{\celsius}}},
        portion = {4 Portionen},
        source = {KptnCook}
    ]
    {\addtoidx{Shepherd's Pie}}

    \graph
        {
            big=Hauptgerichte/Ofengerichte/Shepherds_Pie/big.jpg,
            small=Hauptgerichte/Ofengerichte/Shepherds_Pie/small.jpg
        }

    % \introduction{
    %     EINLEITUNG
    % }

    \ingredients{
        3 Stangen & \addtoidx{Sellerie} \\
        1 & große Zucchini \\
        \SI{20}{\g} & Tomatenmark \\
        \SI{1}{\TL} & Thymian \\
        4 & Möhren \\
        2 & Zwiebeln \\
        \SI{60}{\g} & Margarine \\
        \SI{1}{\kilo\g} & Kartoffeln \\
        \SI{400}{\g} & Süßkartoffeln \\
        \SI{250}{\g} & Linsen (Berg- oder grüne) \index{Linsen!Berg-} \\
        2 Prisen & Muskatnuss \\
         & Salz \\
         & Pfeffer \\
        4 & Knoblauchzehen \\
        \SI{1}{\l} & Gemüsebrühe \\
        4 & Lorbeerblätter \\
         & Bulgur
    }

    \preparation{
        \step Zuerst die Zwiebeln schälen und fein würfeln und den Knoblauch fein zerkleinern. Dann den Sellerie und die Zucchini würfeln, die Möhren der Länge nach halbieren und in Scheiben schneiden.
        \step In einem großen Topf etwas Öl erwärmen und die Zwiebeln und den Knoblauch kurz anschwitzen. Dann Sellerie, Zucchini und Möhren dazu geben und etwa \SI{5}{\minute} garen. Danach das Tomatenmark dazu geben und kurz anbraten.
        \step Anschließend die Linsen, Gemüsebrühe, Thymian und die Lorbeerblätter dazu geben. Alles zusammen für \SI{20}{\minute} köcheln lassen. Währenddessen die Kartoffeln und Süßkartoffeln schälen und in Würfel schneiden. In kochendem Salzwasser für etwa \SI{15}{\minute} kochen, bis sie weich sind (ist am Ende noch zu viel Flüssigkeit übrig, etwas Bulgur untermischen und kurz ziehen lassen). Den Ofen auf \SI{200}{\celsius} vorheizen.
        \step Wenn die Kartoffeln gar sind, abgießen und mit einem Kartoffelstampfer zu Brei verarbeiten. Die Margarine und etwas Muskatnuss darunter mischen.
        \step In einer großen Auflaufform die Linsenmischung verteilen und darauf den Kartoffel-Süßkartoffeln-Brei geben. Den Brei eben verteilen und den Auflauf dann für \SI{20}{\minute} backen.
    }

    % \suggestion[TITEL EINES VORSCHLAGS]{
	% 	VORSCHLAG (DURCH HORIZONTALE LINIE VOM REZEPT GETRENNT)
    % }
    %
    % \hint{
    %     HINWEIS (IN EINEM KASTEN UNTEN AUF DER SEITE)
    % }

\end{recipeDP}
