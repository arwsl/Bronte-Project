\begin{recipeDP}
    [
        preparationtime = {\SI{45}{\minute}},
        bakingtime = {\SI{20}{\minute}},
        bakingtemperature = {\protect\bakingtemperature{topbottomheat=\SI{200}{\celsius}}},
        portion = {4 Portionen},
        source = {kptncook.com}
    ]
    {Shepherd's Pie}

    \graph
        {
            big=Hauptgerichte/Ofengerichte/Shepherds_Pie/big.jpg,
            small=Hauptgerichte/Ofengerichte/Shepherds_Pie/small.jpg
        }

    % \introduction{
    %     EINLEITUNG
    % }

    \ingredients{
        4 & Staudensellerie \index{Sellerie}\\
        \SI{20}{\g} & Tomatenmark \\
        \SI{1}{\TL} & Thymian \\
        4 & Möhren \\
        2 & Zwiebeln \\
        \SI{60}{\g} & Margarine \\
        \SI{1}{\kg} & \index{Kartoffeln} \\
        \SI{300}{\g} & Tellerlinsen \index{Linsen!Teller-} \\
        \SI[parse-numbers = false]{\nicefrac{1}{2}}{\TL} & Muskat \\
        4 & Knoblauchzehen \\
        \SI{1}{\l} & Gemüsebrühe \\
        4 & Lorbeerblätter \\
         & Salz \\
         & Pfeffer
    }

    \preparation{
        \step Die Zwiebeln schälen und fein würfeln.
        Auch den Knoblauch klein hacken und den Sellerie würfeln.
        Die Möhren der länge nach halbieren und in Scheiben schneiden.
        \step In einem großen Topf die Margarine zerlassen, Zwiebeln und Knoblauch darin anschwitzen und anschließent mit dem Sellerie und den Möhren zusammen für 5 Minuten dünsten.
        Dann das Tomatenmark dazugeben und weiter andünsten.
        \step Im nächsten Schritt die Linsen mit der Gemüsebrühe, dem Thymian und den Lorbeerblättern mit in den Topf geben und für 20 Minuten köcheln lassen.
        \step In der Zwischenzeit die Kartoffeln schälen und in einem weiteren Topf in reichlich Salzwasser für etwa 15 Minuten kochen. Den Ofen auf \SI{200}{\celsius} Ober-/Unterhitze vorheizen.
        \step Wenn die Kartoffeln gar sind, das Wasser abgießen und dann mit Margarine und Muskat würzen.
        Dann die Kartoffeln zu Brei machen.
        Die Linsen mit Salz und Pfeffer abschmecken und in einer Auflaufform verteilen.
        Die Linsen dann mit dem Kartoffelbrei bedecken, sodass eine ebene Fläche entsteht.
        Alles zusammen für \SI[]{20}{\minute} backen.

    }

    % \suggestion[TITEL EINES VORSCHLAGS]{
	% 	VORSCHLAG (DURCH HORIZONTALE LINIE VOM REZEPT GETRENNT)
    % }

    % \hint{
    %     HINWEIS (IN EINEM KASTEN UNTEN AUF DER SEITE)
    % }

\end{recipeDP}