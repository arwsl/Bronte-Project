\begin{recipeDP}
    [
        preparationtime = {\SI{45}{\minute}},
        bakingtime = {\SI{20}{\minute}},
        bakingtemperature = {\protect\bakingtemperature{fanoven=\SI{200}{\celsius}}},
        portion = {4 Portionen},
        source = {zuckerjagdwurst.com}
    ]
    {Rotkohl-Kroketten-Auflauf}

    \graph
        {
            big=Hauptgerichte/Ofengerichte/Rotkohl-Kroketten-Auflauf/big.png,
            small=Hauptgerichte/Ofengerichte/Rotkohl-Kroketten-Auflauf/small.png
        }

    % \introduction{
    %     EINLEITUNG
    % }

    \ingredients{
        \multicolumn{2}{l}{\textbf{Für die Soße}} \\
        \SI{200}{\g} & Möhren \\
        \SI{400}{\g} & Champignons \\
        2 & Zwiebeln \\
        \SI{2}{\EL} & Olivenöl \\
        \SI{1}{\EL} & Agavendicksaft \\
        \SI{1}{\EL} & Tomatenmark \\
        \SI{1}{\EL} & Mehl \\
        \SI{60}{\ml} & Rotwein \\
        \SI{200}{\ml} & Gemüsebrühe \\
        \SI{200}{\ml} & Kochcreme \\
        \SI{1}{\TL} & Zimt \\
        \SI{1}{\TL} & Muskat \\
         & Salz \\
         & Pfeffer \\
        \\
        \multicolumn{2}{l}{\textbf{Außerdem}} \\
        \SI{400}{\g} & Rotkohl (aus dem Glas) \\
        \SI{200}{\g} & Reibekäse \\
        \SI{1}{\EL} & Öl \\
        \SI{1}{\EL} & Wasser \\
        \SI{1}{\kg} & Kroketten
    }

    \preparation{
        \step Karotten und Zwiebel schälen, Pilze putzen und alles klein schneiden.
        In einer Pfanne Olivenöl erhitzen und das Gemüse anrösten, bis es gebräunt ist.
        Agavendicksaft und Tomatenmark hinzugeben und 1 bis 2 Minuten mit anrösten.
        \step Das Gemüse mit Mehl bestreuen, verrühren und mit Rotwein ablöschen.
        Mit Gemüsebrühe und Sahne aufgießen und aufköcheln lassen.
        Die Soße mit Zimt und Muskat würzen und mit Salz und Pfeffer abschmecken.
        Die Soße circa 5 Minuten köcheln lassen, bis sie etwas angedickt ist.
        \step Den Backofen auf \SI{200}{\celsius} (Umluft) vorheizen.
        Die Pilzsoße in eine Auflaufform geben.
        Den Rotkohl in einem Sieb abtropfen lassen und gleichmäßig auf der Pilzsoße verteilen.
        Den Reibekäse mit Öl und Wasser vermischen und gleichzeitig auf dem Rotkohl verteilen.
        Zum Schluss die Kroketten auf den Käse legen.
        Den Auflauf dann für 20 bis 25 Minuten backen, bis die Kroketten braun werden und knusprig sind.
    }

    % \suggestion[TITEL EINES VORSCHLAGS]{
	% 	VORSCHLAG (DURCH HORIZONTALE LINIE VOM REZEPT GETRENNT)
    % }

    % \hint{
    %     HINWEIS (IN EINEM KASTEN UNTEN AUF DER SEITE)
    % }

\end{recipeDP}