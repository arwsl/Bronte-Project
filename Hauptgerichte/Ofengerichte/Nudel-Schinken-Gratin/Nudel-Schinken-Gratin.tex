\begin{recipeDP}
    [
        preparationtime = {\SI{40}{\minute}},
        bakingtime = {\SI{15}{\minute}},
        bakingtemperature = {\protect\bakingtemperature{topbottomheat=\SI{200}{\celsius}}},
        portion = {4 Portionen},
        source = {zuckerjagdwurst.com}
    ]
    {Nudel-Schinken-\addtoidx{Gratin}}

    \graph
    {
            big=Hauptgerichte/Ofengerichte/Nudel-Schinken-Gratin/big.jpg,
            small=Hauptgerichte/Ofengerichte/Nudel-Schinken-Gratin/small.jpg
    }

    % \introduction{
    %     EINLEITUNG
    % }

    \ingredients{
        \SI{500}{\g} & Nudeln (Penne) \\
        \SI{400}{\g} & Räuchertofu \index{Tofu!Räucher-} \\
        1 & \addtoidx{Brokkoli} \\
        2 & Zwiebeln \\
        \SI{700}{\ml} & Sojamilch \\
        \SI{200}{\ml} & Soja-Cuisine \\
        \SI{150}{\ml} & Wasser \\
        \SI{200}{\g} & Reibekäse \\
        \SI{4}{\EL} & Margarine \\
        & Pflanzenöl \\
        & Sojasoße \\
        & Salz \\
        & Pfeffer \\
        & Muskat
    }

    \preparation{
        \step Räuchertofu in kleine Würfel schneiden. Kleine Röschen vom Brokkoli abschneiden und die Zwiebel schälen und fein würfeln. Pflanzenöl in einer Pfanne erhitzen und die Räuchertofu-Würfel bei mittlerer Hitze \ca 3 bis 4 Minuten anbraten. Einen Schuss Sojasoße dazugeben und noch weitere 2 Minuten anbraten, bis der Räuchertofu leicht knusprig ist. Danach beiseitestellen.
        \step Parallel dazu gesalzenes Wasser aufkochen lassen und die Brokkoliröschen im siedenden Wasser 2 Minuten blanchieren. Danach aus dem Wasser nehmen und mit kaltem Wasser abschrecken. Den blanchierten Brokkoli ebenfalls beiseitestellen.
        \step Margarine in einem Topf bei mittlerer Hitze schmelzen. Die Zwiebelwürfel darin anschwitzen, bis sie glasig sind. Danach pflanzliche Milch, pflanzliche Sahne und Wasser dazugeben, salzen und aufkochen lassen. Sobald die Flüssigkeit kocht, Nudeln dazugeben und nach Packungsangabe kochen. In der Zwischenzeit den Backofen schon mal auf \SI{200}{\celsius} Ober- und Unterhitze vorheizen.
        \step Die Hälfte des veganen Reibekäses dazugeben und einrühren, bis er geschmolzen ist. Die Nudeln in Soße mit Salz, Pfeffer und etwas Muskat abschmecken. Danach den Brokkoli und Räuchertofu untermengen.
        \step Die Mischung in eine leicht eingefettete Auflaufform geben, mit dem restlichen Käse bedecken und bei \SI{200}{\celsius} (Ober- und Unterhitze) 15 Minuten überbacken.
    }

    % \suggestion[TITEL EINES VORSCHLAGS]{
	% 	VORSCHLAG (DURCH HORIZONTALE LINIE VOM REZEPT GETRENNT)
    % }
    %
    % \hint{
    %     HINWEIS (IN EINEM KASTEN UNTEN AUF DER SEITE)
    % }

\end{recipeDP}
