\begin{recipeDP}
    [
        preparationtime = {\SI{45}{\minute}},
        % bakingtime = {\SI{ZEIT}{\minute} bis \SI{ZEIT}{\minute}},
        % bakingtemperature = {\protect\bakingtemperature{fanoven=\SI{TEMPERATUR}{\celsius}}},
        portion = {4 Portionen},
        source = {blog.eimele.com}
    ]
    {\addtoidx{Brokkoli} und \addtoidx{Rosenkohl} Stir-Fry}

    \graph
        {
            big=Hauptgerichte/Pfannengerichte/Rosenkohl_Stir-Fry/big.jpg,
            small=Hauptgerichte/Pfannengerichte/Rosenkohl_Stir-Fry/small.jpg
        }

    % \introduction{
    %     EINLEITUNG
    % }

    \ingredients{
        \SI{250}{\ml} & Gemüsebrühe \\
        1 & Schalotte \\
        \SI{1}{\EL} & geriebener Ingwer \\
        2 Zehen & Knoblauch \\
        \SI{300}{\g} & Rosenkohl \\
        1 Kopf & Brokkoli \\
        1 Dose & braune Dosenlinsen (\SI{400}{\g}) \\
        \SI{40}{\g} & Mandeln \\
        \SI{140}{\g} & Quinoa \\
        \\
        \multicolumn{2}{l}{\textbf{Soße}} \\
        \SI{3}{\EL} & Sojasoße \\
        \SI[parse-numbers = false]{\nicefrac{1}{2}}{\EL} & Ahornsirup \\
        \SI{1,5}{\EL} & Sesamöl
    }

    \preparation{
        \step Zuerst die Schalotte fein hacken, den Ingwer und den Knoblauch zerreiben und den Rosenkohl säubern und halbieren. Den Brokkoli in kleine Röschen schneiden und die Dosenlinsen abtropfen und gut abspülen.
        Den Quinoa nach Packungsangaben zubereiten.
        \step Danach in einer großen Pfanne oder in einem Wok auf mittlerer Hitze die Gemüsebrühe mit der Schalotte, dem Ingwer, dem Knoblauch und dem Rosenkohl für etwa \SI{8}{\minute} kochen. Währenddessen immer wieder umrühren.
        \step Wenn der Rosenkohl etwas weich geworden ist, den Brokkoli und die Linsen dazu geben und weitere \SI{10}{\minute} garen. Während der Brokkoli gar wird, die Zutaten für die Soße vermischen.
        \step Zuletzt die Soße und die Mandeln in das Gemüse geben, gut vermengen und nochmal \SI{4}{\minute} kochen. Zum Servieren mit Chiliflocken servieren.
    }

    % \suggestion[TITEL EINES VORSCHLAGS]{
	% 	VORSCHLAG (DURCH HORIZONTALE LINIE VOM REZEPT GETRENNT)
    % }

    % \hint{
    %     HINWEIS (IN EINEM KASTEN UNTEN AUF DER SEITE)
    % }

\end{recipeDP}
