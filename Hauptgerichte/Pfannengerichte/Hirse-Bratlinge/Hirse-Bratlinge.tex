\begin{recipeDP}
    [
        preparationtime = {\SI{30}{\minute}},
        % bakingtime = {\SI{ZEIT}{\minute} bis \SI{ZEIT}{\minute}},
        % bakingtemperature = {\protect\bakingtemperature{fanoven=\SI{TEMPERATUR}{\celsius}}},
        portion = {12 Stück},
        source = {veganblatt.com}
    ]
    {\addtoidx{Hirse}-\addtoidx{Bratlinge}}

    \graph
        {
            % big=TEIL/KAPITEL/REZEPT/big.jpg,
            small=Hauptgerichte/Pfannengerichte/Hirse-Bratlinge/small.jpg
        }

    \introduction{
        Hirse ist nicht nur glutenfrei, sondern bringt auch ansonsten eine Menge Mineralstoffe mit sich. Noch dazu schmeckt sie sehr lecker.
    }

    \ingredients{
        \SI{150}{\g} & Hirse \\
        \SI{300}{\ml} & Wasser \\
        \SI{1}{\EL} & Chiasamen \\
        \SI{3}{\EL} & Wasser \\
        \SI{3}{\EL} & Dinkelvollkornmehl \\
         & Petersilie \\
         & Salz \\
         & Pfeffer \\
         & Öl
    }

    \preparation{
        \step Die Hirse in die kochende Gemüsebrühe einrühren und auf kleiner Flamme mit Deckel für \SI{7}{\minute} köcheln lassen. Danach für weitere \SI{10}{\minute} zugedeckt quellen lassen.
        \step Aus Chiasamen und Waser ein Chia-EI anrühren: also beides vermengen und einige Minuten quellen lassen. Das Mehl mit dem Chia-Ei und der fertigen Hirse vermischen. Es soll eine formbare Masse entstehen. Mit Petersilie und nach Belieben mit Salz und Pfeffer abschmecken.
        \step Eine Pfanne mit dem Öl aufstellen und die Laibchen von beiden Seiten knusprig goldbraun braten.
    }

    \suggestion[Linsen-Bratlinge]{
		Mit grünen oder Berg-\addtoidx{Linsen} lässt sich die Hirse auch ersetzen. Die Linsen garen und dann mit einem Kartoffelstampfer alles vermengen. Dabei ist in der Regel etwas mehr Mehl notwendig.
    }

    \hint{
        Die Bratlinge lassen sich auch gut in einem Burger genießen!
    }

\end{recipeDP}
