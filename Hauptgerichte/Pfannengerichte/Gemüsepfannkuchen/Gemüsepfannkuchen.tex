\begin{recipeDP}
    [
        preparationtime = {\SI{20}{\minute}},
        bakingtime = {\SI{30}{\minute}},
        % bakingtemperature = {\protect\bakingtemperature{fanoven=\SI{TEMPERATUR}{\celsius}}},
        portion = {5 Stück (2 Personen)},
        source = {@eatmoreplants.no}
    ]
    {Koreanische Gemüsepfannkuchen\index{Pfannkuchen!Gemüse-}}

    \graph
        {
            big=Hauptgerichte/Pfannengerichte/Gemüsepfannkuchen/big.jpg,
            % small=TEIL/KAPITEL/REZEPT/small.jpg
        }

    % \introduction{
    %     EINLEITUNG
    % }

    \ingredients{
        \multicolumn{2}{l}{\textbf{Teig}} \\
        \SI{375}{\g} & Mehl \\
        \SI{4}{\TL} & Backpulver \\
        \SI{8}{\EL} & Speisestärke \\
        \SI{2}{\TL} & Salz \\
        2 Prisen & Kurkuma \\
        \SI{700}{\ml} & Wasser \\
        \\
        \multicolumn{2}{l}{\textbf{Gemüse}} \\
        4 & Möhren \\
        2 & kleine Zucchini \\
        1 & Zwiebel \\
        \SI{300}{\g} & \addtoidx{Spitzkohl} \\
        \\
        \multicolumn{2}{l}{\textbf{Dip}} \\
		\SI{3}{\EL} & Sojasoße \\
		\SI{3}{\EL} & Wasser \\
		\SI{3}{\TL} & Reisessig \\
		\SI{1}{\EL} & Rübensirup \\
		\SI{1}{\TL} & Chiliflocken \\
    }

    \preparation{
        \step In einer sehr großen Schüssel die trockenen Zutaten für den Teig vermischen. Dann nach un nach das Wasser einrühren - möglichst klumpenfrei. Den Teig beiseite stellen.
        \step Das Gemüse in dünne Streifen schneiden: Die Möhren am besten grob raspeln, die Zucchini und die Zwiebeln fein zerschneiden und auch den Spitzkohl in möglichst nicht zu lange Streifen zerteilen. Das geschnittene Gemüse dann unter den Teig heben und alles gut mit teig bedecken.
        \step Eine große Pfanne dann auf etwa Stufe 6 von 9 erwärmen und die Pfannkuchen von beiden Seiten goldbraun anbraten. Für den Dip nebenbei alle Zutaten gründlich miteinander verrühren. Die fertigen Pfannkuchen in kleine Quadrate zerschneiden und mit Dip genießen.
    }

    % \suggestion[TITEL EINES VORSCHLAGS]{
	% 	VORSCHLAG (DURCH HORIZONTALE LINIE VOM REZEPT GETRENNT)
    % }
    %
    % \hint{
    %     HINWEIS (IN EINEM KASTEN UNTEN AUF DER SEITE)
    % }

\end{recipeDP}
