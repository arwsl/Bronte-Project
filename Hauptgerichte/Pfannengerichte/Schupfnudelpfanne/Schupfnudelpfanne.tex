\begin{recipeDP}
    [
        preparationtime = {\SI{30}{\minute}},
        % bakingtime = {\SI{ZEIT}{\minute} bis \SI{ZEIT}{\minute}},
        % bakingtemperature = {\protect\bakingtemperature{fanoven=\SI{TEMPERATUR}{\celsius}}},
        portion = {2 Portionen},
        source = {@kochen.mit.liza}
    ]
    {Schupfnudelpfanne}

    \graph
        {
            big=Hauptgerichte/Pfannengerichte/Schupfnudelpfanne/big.jpg,
            small=Hauptgerichte/Pfannengerichte/Schupfnudelpfanne/small.jpg
        }

    % \introduction{
    %     EINLEITUNG
    % }

    \ingredients{
        \SI{500}{\g} & \addtoidx{Schupfnudeln} \\
        \SI{250}{\g} & Hafersahne \\
        1 Block & Hirtenkäse \\
        \SI{300}{\g} & Cherrytomaten \\
        \SI{2}{\EL} & Tomatenmark \\
         & \addtoidx{Pinienkerne} \\
        \SI[parse-numbers = false]{\nicefrac{1}{2}}{\TL} & ital. Kräuter \\
        \SI[parse-numbers = false]{\nicefrac{1}{2}}{\TL} & Paprikapulver \\
        \SI[parse-numbers = false]{\nicefrac{1}{2}}{\TL} & Salz \\
        \SI[parse-numbers = false]{\nicefrac{1}{2}}{\TL} & Oregano \\
        5 Blätter & Basilikum
    }

    \preparation{
        \step Pinienkerne in der Pfanne anrösten und danach aus der Pfanne nehmen.
        \step In der Pfanne die Schupfnudeln in etwas Öl anbraten.
        Tomaten und Tomatenmark dazu geben.
        Die Gewürze mit der Hafersahne vermengen und mit in die Pfanne geben.
        Alles kurz aufkochen lassen, vom Herd nehmen und Pinienkerne, Feta und Basilikum dazu geben.
    }

    % \suggestion[TITEL EINES VORSCHLAGS]{
	% 	VORSCHLAG (DURCH HORIZONTALE LINIE VOM REZEPT GETRENNT)
    % }

    % \hint{
    %     HINWEIS (IN EINEM KASTEN UNTEN AUF DER SEITE)
    % }

\end{recipeDP}