\begin{recipeDP}
    [
        preparationtime = {\SI{40}{\minute}},
        % bakingtime = {\SI{ZEIT}{\minute} bis \SI{ZEIT}{\minute}},
        % bakingtemperature = {\protect\bakingtemperature{fanoven=\SI{TEMPERATUR}{\celsius}}},
        portion = {4 Portionen},
        source = {senseofveggies.de}
    ]
    {Kartoffel-Brokkoli-Pfanne mit Zitronencreme}

    \graph
        {
            big=Hauptgerichte/Pfannengerichte/Kartoffel-Brokkoli-Pfanne/big.jpg,
            small=Hauptgerichte/Pfannengerichte/Kartoffel-Brokkoli-Pfanne/small.jpg
        }

    % \introduction{
    %     EINLEITUNG
    % }

    \ingredients{
        \SI{2}{\kg} & \addtoidx{Kartoffeln}, festkochend \\
        \SI{1}{\kg} & \addtoidx{Brokkoli} \\
        2 & Zwiebel \\
        4 & Knoblauchzehen \\
        \SI{2}{\TL} & Thymian \\
        \SI{6}{\EL} & Olivenöl \\
        \SI{400}{\ml} & Kochcreme \\
        \SI{75}{\g} & Cashews \\
        \SI{2}{\EL} & Petersilie \\
        1 & Bio-\addtoidx{Zitrone} \\
         & Salz \\
         & Pfeffer
    }

    \preparation{
        \step Zuerst die Cashews in heißem Wasser einweichen lassen, die Kartoffeln schälen, in \SI{3}{\cm} große Würfel schneiden und den Brokkoli in kleine Röschen zerteilen (TK Brokkoli auftauen und ebenfalls zerkleinern).
        Die Zwiebel in den Knoblauch klein schneiden.
        \step Die Kartofeln in reichlich Salzwasser kochen (im Schnellkochtopf für 5 Minuten kochen, auf erster Druckstufe, dann vorsichtig belüften).
        Den Brokkoli ebenfals kurz, für 4 Minuten in Salzwasser garen und dann mit kaltem Wasser abschrecken (TK Brokkoli kann auch direkt verwendet werden.)
        \step Für die Zitronencreme die Cashews abgeißen und mit der Kochcreme, der Petersilie, dem Thymian, dem Zitronenabrieb und dem Saft der Zitrone, etwas Salz und Pfeffer pürieren bis eine glatte Soße entsteht.
        Unter Umständen noch Wasser hinzu geben.
        \step In einer großen Pfanne das Öl erhiten, die Zwiebeln und den Knoblauch anschwitzen und dann die Kartoffeln hinzugeben und einige Minuten mitbraten.
        Dann den Brokkoli hinzugeben und kurz mitbraten.
        Mit Salz, Pfeffer und Thymain abschmecken und mit der Zitronencreme servieren.
    }

    % \suggestion[TITEL EINES VORSCHLAGS]{
	% 	VORSCHLAG (DURCH HORIZONTALE LINIE VOM REZEPT GETRENNT)
    % }

    % \hint{
    %     HINWEIS (IN EINEM KASTEN UNTEN AUF DER SEITE)
    % }

\end{recipeDP}