\begin{recipe}
    [
        preparationtime = {\SI{45}{\minute}},
        % bakingtime = {\SI{12}{\minute} bis \SI{15}{\minute}},
        % bakingtemperature = {\protect\bakingtemperature{fanoven=\SI{180}{\degree}}},
        portion = {10 Stück},
        source = {@vegangotgame.co}
    ]
    {Kartoffel-\addtoidx{Zucchini}-\addtoidx{Bratlinge}}

    % \graph
    %     {
    %         small=Recipes/MainCourses/BBQChicken/Small.jpg,
    %         big=example-image
    %     }

    % \introduction{einleitung}

    \ingredients{
    1 & Zucchini \\
    \SI{150}{\g} & \addtoidx{Kartoffeln} \\
    1 & kleine Zwiebel \\
    \SI{40}{\g} & rote Paprika, gewürfelt \\
    \SI{30}{\g} & Grünkohl\index{Gruenkohl@Grünkohl}, zerkleinert \\
    \SI{65}{\g} & Mehl \\
    \SI[parse-numbers = false]{\nicefrac{1}{2}}{\TL} & Salz \\
    \SI[parse-numbers = false]{\nicefrac{1}{2}}{\TL} & Knoblauchpulver \\
    \SI[parse-numbers = false]{\nicefrac{1}{2}}{\TL} & versch. Kräuter \\
     & Pfeffer
    }

    \preparation{
        \step Die Zucchini und die geschälten Kartoffeln raspeln und mit einem sauberen Tuch möglichst viel Flüsigkeit ausdrücken. Die fein gewürfelte Paprika und Zwiebel sowie den klein geschnittenen Grünkohl mit den Raspeln zusammen vermengen.

        \step Salz, Pfeffer, Knoblauch, Kräuter und Mehl einrühren, bis eine homogene Masse entsteht. Wenn der Teig zu feucht ist, noch etwa mehl hinzu geben, sodass der Teig gut zusammen hält. Den Teig dann in \SI{5}{\cm} bis \SI{8}{\cm} Bratlinge formen.

        \step Die Bratlinge entweder frittieren oder in einer heißen Pfanne mit etwas Öl ausbraten, jede Seite etwa \SI{4}{\minute}.
    }

    % \suggestion[Title of Suggestion]{
	% 	Suggestion
    % }
    %
    % \hint{Hint}

\end{recipe}
