\begin{recipeDP}
    [
        preparationtime = {\SI{40}{\minute}},
        % bakingtime = {\SI{ZEIT}{\minute} bis \SI{ZEIT}{\minute}},
        % bakingtemperature = {\protect\bakingtemperature{fanoven=\SI{TEMPERATUR}{\celsius}}},
        portion = {4 Portionen},
        source = {kptncook.com}
    ]
    {Thai-Reisbällchen mit Erdnüssen}

    \graph
        {
            big=Hauptgerichte/Pfannengerichte/Thai-Reisbaellchen/big.jpg,
            small=Hauptgerichte/Pfannengerichte/Thai-Reisbaellchen/small.jpg
        }

    % \introduction{
    %     EINLEITUNG
    % }

    \ingredients{
        \SI{300}{\g} & Thai Jasmin Reis \index{Reis!Jasmin} \\
         & Salz \\
        \SI{80}{\g} & \addtoidx{Ingwer} \\
        \SI{30}{\g} & \addtoidx{Erdnussbutter} \\
        4 & Frühlingszwiebeln \\
        \SI{40}{\g} & Erdnüsse, geröstet, ungesalzen \\
        4 & Chilischoten \\
        4 & Knoblauchzehen \\
        \SI{4}{\EL} & Sonnenblumenöl \\
        \SI{4}{\EL} & Thai-Chilisoße (\addtoidx{Sriracha}) \\
        \SI{4}{\EL} & Agavendicksaft \\
        \SI{8}{\EL} & Sojasoße \\
    }

    \preparation{
        \step Zuerst den Reis nach Packungsangabe in gesalzenem Wasser kochen, sodass am Ende kein Wasser mehr bleibt.
        Während der Reis kocht, den Ingwer schälen und fein reiben, die Chilis klein schneiden, die Frühlingszwiebeln in Ringe schneiden und den Knoblach fein hacken.
        \step Den fertigen Reis mit dem Ingwer vermengen und dann mit angefeuchteten Fingern mundgerechte Reisbällchen formen.
        \step Für die Soße das Öl mit der Sriracha-Soße, der Erdnussbutter der Sojasoße, dem Agavendicksaft, dem Knoblauch und den Chilis in einer großen Pfanne erhitzen.
        Wenn sich alles gut vermischt hat, die Reisbällchen hinein geben und gut anbraten, dass sie leicht braun werden und die Soße annehmen.
        \step Mit den Erdnüssen und Frühlingszwiebeln bestreut servieren!
    }

    % \suggestion[TITEL EINES VORSCHLAGS]{
	% 	VORSCHLAG (DURCH HORIZONTALE LINIE VOM REZEPT GETRENNT)
    % }

    \hint{
        Für den Reis bietet sich die Quellreis-Methode an: dazu die \SI{300}{\g} Jasmin Reis in \SI{450}{\ml} Wasser kochen, bis das ganze Wasser vom Reis aufgenommen wurde.
    }

\end{recipeDP}