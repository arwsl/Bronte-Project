\begin{recipeDP}
    [
        preparationtime = {\SI{45}{\minute}},
        % bakingtime = {\SI{ZEIT}{\minute} bis \SI{ZEIT}{\minute}},
        % bakingtemperature = {\protect\bakingtemperature{fanoven=\SI{TEMPERATUR}{\celsius}}},
        portion = {4 Portionen},
        source = {rainbowplantlife.com}
    ]
    {Fenchel-Kichererbsen Eintopf}

    \graph
        {
            big=Hauptgerichte/Schnellkochtopf/Fenchel_Kichererbsen_Eintopf/big.jpg,
            small=Hauptgerichte/Schnellkochtopf/Fenchel_Kichererbsen_Eintopf/small.jpg
        }

    \introduction{
        Die Kichererbsen sollten über Nacht im Kühlschrank in reichlich Wasser einweichen!
    }

    \ingredients{
        \SI{200}{\g} & getrocknete Kichererbsen \index{Kichererbsen!getrocknet} \\
        1 & Zwiebel \\
        1 & großer \addtoidx{Fenchel} \\
        4 & Knoblauchzehen \\
        \SI{1}{\EL} & Thymian \\
        \SI{950}{\ml} & Gemüsebrühe \\
        \SI{225}{\g} & brauner Reis \index{Reis!braun} \\
        \SI[parse-numbers = false]{\nicefrac{1}{2}}{\TL} & Chiliflocken \\
        2 & Lorbeerblätter \\
        \SI{1}{\TL} & Salz \\
        \SI{1}{\TL} & Pfeffer \\
        \SI{400}{\g} & Tomatenstücke \\
        \SI{200}{\TL} & passierte Tomaten \\
        3 Handvoll & Spinat oder Grünkohl \\
        \SI{1}{\EL} & Zitronensaft \\
         & Petersilie
    }

    \preparation{
        \step Am Vortag die Kichererbsen in reichlich Wasser einweichen lassen (am besten über Nacht im Kühlschrank).
        \step Zuerst die Gemüsebrühe in den Schnellkochtopf geben.
        Dann die eingeweichen und gut abgespülten Kichererbsen, die grob gewürfelte Zwiebel, den geschnittenen Fenchel (ohne Stiele), den fein gehackten Knoblauch, den Thymian, den Reis, Chiliflocken, Lorbeerblätter, Salz und Pfeffer hinzugeben und verrühren.
        \step Dann die Tomatenstücke und die passierten Tomaten darauf geben ohne weiter umzurühren (die Tomaten würden sonst anbrennen).
        Dann den Topf verschließen und bei hohem Druck (auf der zweiten Stufe) für 12 Minuten kochen (die Zeit zählt erst, wenn die Druckstufe erreicht wurde).
        \step Danach mit natürlichem Abkühlen für 10 Minuten den Druck abbauen lassen.
        Dann den restlichen Druck per Ventil heraus lassen.
        \step Nun noch Zitronensaft und das Grünzeug hinein geben.
        Kurz andünsten lassen, bis alles gar ist. 
    }

    % \suggestion[TITEL EINES VORSCHLAGS]{
	% 	VORSCHLAG (DURCH HORIZONTALE LINIE VOM REZEPT GETRENNT)
    % }

    % \hint{
    %     HINWEIS (IN EINEM KASTEN UNTEN AUF DER SEITE)
    % }

\end{recipeDP}