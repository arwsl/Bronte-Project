\begin{recipeDP}
    [
        preparationtime = {\SI{45}{\minute}},
        % bakingtime = {\SI{ZEIT}{\minute} bis \SI{ZEIT}{\minute}},
        % bakingtemperature = {\protect\bakingtemperature{fanoven=\SI{TEMPERATUR}{\celsius}}},
        portion = {6 Portionen},
        source = {@vegan\_high\_protein}
    ]
    {Pad Thai}

    \graph
        {
            big=Hauptgerichte/Nudelgerichte/Pad_Thai/big.jpg,
            small=Hauptgerichte/Nudelgerichte/Pad_Thai/small.jpg
        }

    % \introduction{
    %     EINLEITUNG
    % }

    \ingredients{
        \SI{400}{\g} & \addtoidx{Reisnudeln} \\
        \SI{400}{\g} & \addtoidx{Tofu} oder \addtoidx{Tempeh} \\
        \SI{500}{\g} & Brokkoli \\
        3 & Paprika \\
        8 & Möhren \\
        1 Stück & Ingwer \\
        4 & Knochlauchzehen \\
        1 & Zwiebel \\
        \SI{2}{\EL} & Curry-Paste \\
        \SI{2}{\EL} & Miso-Paste \\
        \SI{2}{\EL} & Erdnussmus \\
        1 & Limette \\
        \SI{400}{\ml} & Kokosmilch \\
        \SI{400}{\ml} & Wasser
    }

    \preparation{
        \step Zuerst die Zwiebel klein schneidnn und mit derm Knoblauch und dem Tofu oder Tempeh  in einem großen Topf anbraten.
        Währenddessen die Möhren und die Paprika in feine Stifte schneiden.
        \step Dann in einer Schale, mit etwas von dem Wasser die Gewürzpasten mit der Limette, dem Erdnusmus und dem Ingwer vermischen.
        gemiensam mit dem restlchen Wasser und der Kokosmilch dann in den Topf geben.
        Die rohen Reisnudeln darauf legen und mit dem Gemüse bedecken.
        Ales zusammen köcheln lassen bis das Gemüse und die Reisnudeln gar sind.
        Zwischendurch umrühren.
    }

    % \suggestion[TITEL EINES VORSCHLAGS]{
	% 	VORSCHLAG (DURCH HORIZONTALE LINIE VOM REZEPT GETRENNT)
    % }

    % \hint{
    %     HINWEIS (IN EINEM KASTEN UNTEN AUF DER SEITE)
    % }

\end{recipeDP}