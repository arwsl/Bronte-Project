\begin{recipe}
    [
        preparationtime = {\SI{10}{\minute}},
        % bakingtime = {\SI{12}{\minute} bis \SI{15}{\minute}},
        % bakingtemperature = {\protect\bakingtemperature{fanoven=\SI{180}{\degree}}},
        portion = {1 Portion},
        source = {@fitgreenmind}
    ]
    {Lazy Noodles}

    % \graph
    %     {
    %         small=Recipes/MainCourses/BBQChicken/Small.jpg,
    %         big=example-image
    %     }

    % \introduction{einleitung}

    \ingredients{
        \textbf{Soße}\\
        \SI{125}{\ml} & Wasser \\
        \SI{4}{\EL} & Soja Soße \\
        \SI{2}{\EL} & Süße (z.B. Ahornsirup) \\
        \SI{1}{\EL} & Weißweinessig oder Zitronensaft \\
        \SI[parse-numbers = false]{\nicefrac{1}{2}}{\EL} & Tahin \\
        \SI[parse-numbers = false]{\nicefrac{1}{2}}{\TL} & Chiliflocken \\
        \SI{2}{\EL} & Maisstärke \\
        \\
        \textbf{sonst} \\
        \SI{250}{\g} & gekochte Nudeln \\
         & Petersilie
    }

    \preparation{
        \step Alle Zutaten für die Soße in einem Mixer miteinander verquirlen, bis eine homogene Soße entsteht.

        \step Die fertige in einer Pfanne erhitzen, wenn sie anfängt anzudicken, die Nudeln hinzugeben. Verrühren bis alles mit dickflüssiger Soße bedeckt ist. Dann mit Petersilie anrichten.
    }

    % \suggestion[Title of Suggestion]{
	% 	Suggestion
    % }

    % \hint{Hint}

\end{recipe}
