\begin{recipeDP}
    [
        preparationtime = {\SI{30}{\minute}},
        % bakingtime = {\SI{ZEIT}{\minute} bis \SI{ZEIT}{\minute}},
        % bakingtemperature = {\protect\bakingtemperature{fanoven=\SI{TEMPERATUR}{\celsius}}},
        portion = {4 Portionen},
        source = {sallys-blog.de}
    ]
    {One Pot Champignon-Brokkoli-Nudeln}

    \graph
        {
            big=Hauptgerichte/Nudelgerichte/One_Pot_Champignon-Nudeln/big.jpg,
            small=Hauptgerichte/Nudelgerichte/One_Pot_Champignon-Nudeln/small.jpg
        }

    % \introduction{
    %     EINLEITUNG
    % }

    \ingredients{
        \SI{250}{\g} & braune Champignons \index{Champignons!braune} \\
        1 & Zwiebel \\
        1 & Knoblauchzehe \\
        1 & \addtoidx{Brokkoli} \\
        \SI{500}{\g} & Nudeln \\
        \SI{1}{\l} & Gemüsebrühe \\
        \SI{200}{\ml} & \addtoidx{Soja-Cuisine} \\
        \SI[parse-numbers = false]{\nicefrac{1}{2}}{\TL} & Pfeffer \\
        \SI{200}{\g} & Blattspinat \\
         & Salz \\
         & Chiliflocken
    }

    \preparation{
        \step Zuerst die Champignons in Scheiben schneiden und den Brokkoli in kleine Röschen schneiden. Den Stiel des Brokkolis und die Zwiebel würfeln. Den Knoblauch fein zerkleinern. Das alles mit den Nudeln zusammen in einen großen Topf geben.
        \step Das Gemüse und die Nudeln mit Pfeffer und Salz würzen und mit Wasser und Sahne übergießen. Alles aufkochen lassen und anschließend bei mittlerer Hitze \SI{10}{\minute} köcheln lassen.
        \step Dann den Blattspinat hinzugeben und kurz weiter köcheln lassen, bis der Spinat weich ist. Am Ende mit Salz und Chiliflocken abschmecken.
    }

    % \suggestion[TITEL EINES VORSCHLAGS]{
	% 	VORSCHLAG (DURCH HORIZONTALE LINIE VOM REZEPT GETRENNT)
    % }
    %
    % \hint{
    %     HINWEIS (IN EINEM KASTEN UNTEN AUF DER SEITE)
    % }

\end{recipeDP}
