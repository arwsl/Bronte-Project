% \begingroup
% \makeatletter
% \renewenvironment{recipe}[2][]
% {% initialisation
%     \setkeys{recipe}{preparationtime, bakingtime, bakingtemperature, portion, calory, source}
%     \setkeys{picture}{small, big, smallpicturewidth=\xcb@smallpicturewidth, bigpicturewidth=\xcb@bigpicturewidth} % load the default values
%     \def\xcb@hook@pregraph{}
%     \def\xcb@hook@pretitle{}
%     \def\xcb@introduction{}
%     \def\xcb@hook@prepreparation{}
%     \preparation{}
%     \def\xcb@hook@postpreparation{}
%     \def\xcb@hook@preingredients{}
%     \ingredients{}
%     \def\xcb@hook@postingredients{}
%     \def\xcb@suggestion{}
%     \def\xcb@hint{}
%
%     \def\xcb@recipename{#2}
%     \setkeys{recipe}{#1}  % reading the optional parameters
%
%     \setcounter{xcb@newpagefoot}{1}
%     \setcounter{xcb@newpagehead}{\value{page}}
% }
% {% this part is executed at \end{recipe}
% %% FIRST BLOCK
%     \xcb@hook@pregraph
%     \if@twoside
%         \ifodd\arabic{page}
%             \begin{minipage}[T]{\xcb@picture@bigwidth}
%                 \xcb@picture@big
%             \end{minipage}
%             \hfill
%             \begin{minipage}[T]{\xcb@picture@smallwidth}
%                 \xcb@picture@small
%             \end{minipage}
%         \else
%             \begin{minipage}[T]{\xcb@picture@smallwidth}
%                 \xcb@picture@small
%             \end{minipage}
%             \hfill
%             \begin{minipage}[T]{\xcb@picture@bigwidth}
%                 \xcb@picture@big
%             \end{minipage}
%         \fi
%     \else
%         \begin{minipage}[T]{\xcb@picture@bigwidth}
%             \xcb@picture@big
%         \end{minipage}
%         \hfill
%         \begin{minipage}[T]{\xcb@picture@smallwidth}
%             \xcb@picture@small
%         \end{minipage}
%     \fi
%
% %% SECOND BLOCK
%     \xcb@hook@pretitle
%     \recipesection[\normalsize\xcb@recipename]%
%     {\hspace{-1em}\textcolor{\xcb@color@recipename}{\xcb@font@recipename\xcb@recipename}}
%     \xcb@cmd@recipeoverview
%
%     \xcb@introduction
%
% %% THIRD BLOCK
%     \columnratio{0.66}
%     \begin{paracol}{2}
%         \xcb@hook@prepreparation
%
%         \xcb@preparation
%
%         \xcb@hook@postpreparation
%
%         \xcb@suggestion
%
%         \vfill
%
%         \xcb@cmd@wrapfill
%         \xcb@hint
%         \setcounter{xcb@newpagefoot}{0}
%       \switchcolumn
%             \xcb@hook@preingredients
%
%             \xcb@ingredients
%
%             \xcb@hook@postingredients
%     \end{paracol}
% }
%
% \renewcommand*{\ingredients}[2][\empty]
% {% The optional argument contains the number of lines
%     \def\xcb@ingredientslines{#1}
%     \def\xcb@ingredients
%     {%
%         \xcb@name@inghead
%         \\[1em]
%         {\xcb@fontsize@ing\color{\xcb@color@ing}
%         \begin{supertabular}{r>{\raggedright\arraybackslash}p{3cm}}
%             #2
%         \end{supertabular}}
%     }
% }
% \makeatother

\begin{recipeDP}
    [
        preparationtime = {\SI{4}{\hour}},
        % bakingtime = {\SI{ZEIT}{\minute} bis \SI{ZEIT}{\minute}},
        % bakingtemperature = {\protect\bakingtemperature{fanoven=\SI{TEMPERATUR}{\celsius}}},
        portion = {61 Stück},
        source = {Chefkoch}
    ]
    {\addtoidx{Ravioli} mit Pilz-Tomaten-Füllung}

    \graph
        {
            big=Hauptgerichte/Nudelgerichte/Ravioli_Pilz-Tomaten/big.jpg,
            small=Hauptgerichte/Nudelgerichte/Ravioli_Pilz-Tomaten/small.jpg
        }

    % \introduction{
    %     EINLEITUNG
    % }

    \ingredients{
        \multicolumn{2}{l}{\textbf{Nudelteig}} \\
        \SI{250}{\g} & Weizenmehl (Typ 1050) \\
        \SI{150}{\g} & Hartweizengrieß \\
        \SI{1}{\TL} & Salz \\
        \SI{4}{\EL} & Olivenöl \\
        \SI{160}{\g} & Wasser\\
        \\
        \multicolumn{2}{l}{\textbf{Füllung}} \\
        3 & Schalotten \\
        1 Prise & Zucker \\
        2 Zehen & Knoblauch \\
        \SI{3}{\EL} & Olivenöl \\
        \SI{250}{\g} & \addtoidx{Champignons} \\
        \SI{75}{\g} & sonnengetrocknete Tomaten\index{Tomaten!sonnengetrocknet} \\
        \SI{3}{\EL} & Tomatenmark \\
         & Weißwein \\
        \SI{175}{\g} & weiße Bohnen\index{Bohnen!weiße} \\
         & Petersilie \\
         & Pfeffer
    }

    \preparation{
        \step Für den Nudelteig zuerst das Weizenmehl, den Hartweizengrieß und das Salz in einer großen Schüssel vermischen. In die Mitte eine Mulde drücken. Anschließend in die Mulde das Olivenöl und das Wasser hineingeben und mit einer Gabel vom Rand her nach innen nach und nach verrühren. Dann den Teig mit den Händen ca. 5 - 10 Minuten lang kräftig verkneten. Es soll ein relativ fester Teig dabei entstehen. Eine Kugel formen, in Klarsichtfolie einschlagen und eine halbe Stunde ruhen lassen.
        \step Für die Füllung Schalotten, Knoblauch, Champignons und Tomaten fein hacken. Schalotten und Knoblauch in Olivenöl anschwitzen, eine Prise Zucker dazugeben. Wenn die Schalotten leicht gebräunt sind, erst Champignons, dann Tomaten zugeben und braten, bis die Pilze gar sind. Tomatenmark zugeben und unter Rühren mit anrösten, dann mit Weißwein ablöschen. Die Bohnen im Mixer pürieren und untermengen, es soll eine Paste entstehen. Mit gehackter Petersilie, Salz und Pfeffer abschmecken.
        \step Danach den Teig mit dem Nudelholz oder mit der Nudelmaschine dünn ausrollen und Kreise mit etwa \SI{8}{\cm} Durchmesser ausstechen. Die geschnittene Pasta noch einmal 15 Minuten ruhen lassen. Danach etwa einen Esslöffel Füllung in die Mitte der Teigkreise geben und den Rand etwas anfeuchten. Die Ravioli zuklappen und mit einer Gabel verschließen.
        \step Abschließend die Ravioli für etwa \SI{5}{\minute} in sprudelnd kochendem Salzwasser gar kochen.
    }

    % \suggestion[TITEL EINES VORSCHLAGS]{
	% 	VORSCHLAG (DURCH HORIZONTALE LINIE VOM REZEPT GETRENNT)
    % }
    %
    % \hint{
    %     HINWEIS (IN EINEM KASTEN UNTEN AUF DER SEITE)
    % }

\end{recipeDP}
