\begin{recipeDP}
    [
        preparationtime = {\SI{35}{\minute}},
        % bakingtime = {\SI{ZEIT}{\minute} bis \SI{ZEIT}{\minute}},
        % bakingtemperature = {\protect\bakingtemperature{fanoven=\SI{TEMPERATUR}{\celsius}}},
        portion = {4 Portionen},
        source = {kptncook.com}
    ]
    {Mafaldine mit Brokkoli-Soße}

    \graph
        {
            big=Hauptgerichte/Nudelgerichte/Mafaldine_mit_Brokkoli/big.jpg,
            small=Hauptgerichte/Nudelgerichte/Mafaldine_mit_Brokkoli/small.jpg
        }

    % \introduction{
    %     EINLEITUNG
    % }

    \ingredients{
        2 & \addtoidx{Brokkoli} \\
        2 & Zwiebel \\
        \SI{30}{\g} & \addtoidx{Basilikum} \\
        \SI{15}{\g} & Hefeflocken \\
        \SI{800}{\ml} & Sojamilch \\
        \SI{500}{\g} & Mafaldine \\
        4 & Knoblauchzehen \\
         & Salz \\
         & Pfeffer \\
        \SI{400}{\ml} & Gemüsebrühe \\
        \SI{4}{\EL} & Olivenöl \\
    }

    \preparation{
        \step Zuerst die Zwiebel schälen und fein würfeln, den Knoblauch hacken und den gewaschenen Brokkoli in Röschen schneiden (auch den Strung in Stücke teilen).
        Den Basilikum waschen und abtropfen lassen.
        \step In einem großen Topf mit etwas Salzwasser die Nudeln nach Packungsangabe kochen.
        Wenn die Nudeln gar sind, etwas von dem Nudelwasser auffangen und den Rest abießen.
        \step In einem großen Topf den Brokkoli mit den Zwiebeln und dem Knoblauch für 3 Minuten anbraten.
        Dann den Basilikum mit der Milch und der Gemüsebrühe dazu geben und alles für 7 Minuten köcheln lassen.
        \step Wenn der Brokkoli gar ist, den Herd ausstellen und den Brokkoli im Topf pürieren.
        Mit Salz und Pfeffer abschmecken und dann die Hefeflocken dazu geben.
        Die Nudeln mit in den Topf geben und nach Geschmack etwas \SI[]{100}{\ml} Nudelwasser dazu geben.
        Alles gut vermengen und servieren.
    }

    % \suggestion[TITEL EINES VORSCHLAGS]{
	% 	VORSCHLAG (DURCH HORIZONTALE LINIE VOM REZEPT GETRENNT)
    % }

    % \hint{
    %     HINWEIS (IN EINEM KASTEN UNTEN AUF DER SEITE)
    % }

\end{recipeDP}