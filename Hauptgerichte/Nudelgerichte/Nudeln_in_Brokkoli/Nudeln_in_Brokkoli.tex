\begin{recipeDP}
    [
        preparationtime = {\SI{40}{\minute}},
        % bakingtime = {\SI{ZEIT}{\minute} bis \SI{ZEIT}{\minute}},
        % bakingtemperature = {\protect\bakingtemperature{fanoven=\SI{TEMPERATUR}{\celsius}}},
        portion = {4 Portionen},
        source = {KptnCook}
    ]
    {Nudeln in \addtoidx{Brokkoli}-\addtoidx{Basilikum}-Soße}

    \graph
        {
            big=Hauptgerichte/Nudelgerichte/Nudeln_in_Brokkoli/big.jpg,
            small=Hauptgerichte/Nudelgerichte/Nudeln_in_Brokkoli/small.jpg
        }

    % \introduction{
    %     EINLEITUNG
    % }

    \ingredients{
        2 & Brokkoli \\
        1 & Zwiebel \\
        \SI{20}{\g} & frischer Basilikum \\
        \SI{3}{\TL} & Basilikum, gerebelt \\
        \SI{3}{\EL} & Hefeflocken \\
        \SI{1}{\l} & Pflanzenmilch \\
        \SI{500}{\g} & \addtoidx{Mafaldine} oder \addtoidx{Bandnudeln} \\
        4 & Knoblauchzehen \\
         & Salz \\
         & Pfeffer \\
        \SI{500}{\ml} & Gemüsebrühe \\
        \SI{4}{\EL} & Olivenöl
    }

    \preparation{
        \step Zunächst die Zwiebel und den Knoblauch schälen und fein hacken. Dann den Brokkoli waschen und in grobe Teile zerlegen. Den frischen Basilikum waschen.
        \step In einem großen Topf das Öl erwärmen und die Zwiebel und den Knoblauch darin anbraten. Den Brokkoli hinzugeben und für \SI{3}{\minute} garen. Danach den trockenen und den frischen Basilikum dazugeben und mit der Milch und der Gemüsebrühe aufkochen. Dann für \SI{7}{\minute} auf mittlerer Hitze köcheln lassen. Nebenbei in einem zweiten Topf Nudelwasser aufsetzen. Darin die Nudeln nach Packungsangabe kochen.
        \step Den Brokkoli dann von der Hitze nehmen und cremig pürieren. Die Hefeflocken unterrühren und mit Salz und Pfeffer abschmecken. Zuletzt die fertigen Nudeln hineingeben und nach bedarf noch etwas Nudelwasser beifügen bis die Soße eine geschmeidige Konsistenz erhält. Mit etwas Olivenöl und Basilikum servieren.
    }

    % \suggestion[TITEL EINES VORSCHLAGS]{
	% 	VORSCHLAG (DURCH HORIZONTALE LINIE VOM REZEPT GETRENNT)
    % }

    \hint{
        Es eignen sich auch übrig gebliebene Lasagneplatten. Diese nur etwas zerbrechen vor dem Kochen!
    }

\end{recipeDP}
