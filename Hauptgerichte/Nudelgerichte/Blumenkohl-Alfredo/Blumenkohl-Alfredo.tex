\begin{recipeDP}
    [
        preparationtime = {\SI{50}{\minute}},
        % bakingtime = {\SI{ZEIT}{\minute} bis \SI{ZEIT}{\minute}},
        % bakingtemperature = {\protect\bakingtemperature{fanoven=\SI{TEMPERATUR}{\celsius}}},
        portion = {4 Portionen},
        source = {@youeatlikearabbit}
    ]
    {\addtoidx{Blumenkohl}-Alfredo}

    \graph
        {
            big=Hauptgerichte/Nudelgerichte/Blumenkohl-Alfredo/big.jpg,
            % small=TEIL/KAPITEL/REZEPT/small.jpg
        }

    % \introduction{
    %     EINLEITUNG
    % }

    \ingredients{
        1 & Blumenkohl \\
        5 Zehen & Knoblauch \\
        1 & Zitrone \\
        \SI[parse-numbers = false]{\nicefrac{3}{4}}{\TL} & Salz \\
        \SI{3}{\EL} & Hefeflocken \\
        \SI{1}{\EL} & Zwiebelpulver \\
        \SI{2}{\EL} & Speisestärke \\
        \SI{476}{\ml} & Nudelwasser \\
        \SI{1}{EL} & Gemüsebrühenpulver \\
        \SI{70}{\g} & Cashews \\
        \SI{500}{\g} & Spaghetti \\
        \SI{200}{\g} & Blattspinat \\
        \SI{100}{\g} & Cherrytomaten \\
         & Salz \\
         & Pfeffer
    }

    \preparation{
        \step Die Cashews in heißem Wasser für \SI{15}{\minute} einweichen. In einem großen Topf Wasser erhitzen und sobald es kocht die klein geschnittenen Blumenkohlröschen hineingeben und für etwa 5 Minuten kochen - bis sie gar sind. danach mit einem Sieblöffel heraus nehmen und die Nudeln in das kochende Wasser hineingeben. Nach Packungsangaben kochen. Nebenbei die Knoblauchzehen in einem anderen Top mit etwas Öl anbraten.
        \step In den Topf mit den Knoblauchzehen die gegarten Blumenkohlröschen, den Saft der Zitrone, Salz, Hefeflocken, Zwiebelpulver, Speisestärke, Cashews, Gemüsebrühenpulver und das Nudelwasser geben. Alles gut pürieren, bis eine glatte Flüssigkeit entsteht.
        \step Die Nudeln abgießen und mit der Alfredo-Soße übergießen. Den Blattspinat und die Tomaten leicht anschwitzen und mit in die Nudeln geben. Mit Salz und Pfeffer abschmecken.
    }

    % \suggestion[TITEL EINES VORSCHLAGS]{
	% 	VORSCHLAG (DURCH HORIZONTALE LINIE VOM REZEPT GETRENNT)
    % }
    %
    % \hint{
    %     HINWEIS (IN EINEM KASTEN UNTEN AUF DER SEITE)
    % }

\end{recipeDP}
