\begin{recipeDP}
    [
        preparationtime = {\SI{20}{\minute}},
        bakingtime = {\SI{30}{\minute} bis \SI{40}{\minute}},
        bakingtemperature = {\protect\bakingtemperature{fanoven=\SI{180}{\celsius}}},
        portion = {4 Portionen},
        source = {@healthygirlkitchen}
    ]
    {\addtoidx{Nudeln} mit gebackenem \addtoidx{Hummus}}

    \graph
        {
            % big=TEIL/KAPITEL/REZEPT/big.jpg,
            small=Hauptgerichte/Nudelgerichte/Nudeln_mit_gebackenem_Hummus/small.jpg
        }

    % \introduction{
    %     EINLEITUNG
    % }

    \ingredients{
        \SI{300}{\g} & Hummus \\
        \SI{500}{\g} & Nudeln \\
        \SI{150}{\g} & Cherry-Tomaten\index{Tomaten!Cherry-} \\
        1 & Zucchini \\
        \SI{250}{\g} & Champignons \\
        \SI{2}{\TL} & Knoblauch \\
        \SI{50}{\ml} & Weißwein \\
        \SI{50}{\ml} & Gemüsebrühe \\
        \SI{1}{\TL} & Salz \\
        \SI{3}{\EL} & Olivenöl \\
        \SI{1}{\TL} & Oregano \\
        \SI{1}{\TL} & Thymian
    }

    \preparation{
        \step Den Hummus in einer großen Auflaufform mittig platzieren. Drum herum das klein geschnittene Gemüse verteilen. Mit dem Knoblauch, dem Weißwein, der Gemüsebrühe, dem Öl und den Gewürzen vermischen.
        \step Die Auflaufform für \SI{30}{\minute} bis \SI{40}{\minute} bei \SI{180}{\celsius} im Backofen garen. Wenn die letzten \SI{10}{\minute} anbrechen, die Nudeln nach Packungsanleitung zubereiten. Abschließend die Nudeln in die fertige Auflaufform geben und servieren.
    }

    % \suggestion[TITEL EINES VORSCHLAGS]{
	% 	VORSCHLAG (DURCH HORIZONTALE LINIE VOM REZEPT GETRENNT)
    % }
    %
    % \hint{
    %     HINWEIS (IN EINEM KASTEN UNTEN AUF DER SEITE)
    % }

\end{recipeDP}
