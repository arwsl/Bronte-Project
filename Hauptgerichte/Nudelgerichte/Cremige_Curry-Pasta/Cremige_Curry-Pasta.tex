\begin{recipeDP}
    [
        preparationtime = {\SI{45}{\minute}},
        % bakingtime = {\SI{ZEIT}{\minute} bis \SI{ZEIT}{\minute}},
        % bakingtemperature = {\protect\bakingtemperature{fanoven=\SI{TEMPERATUR}{\celsius}}},
        portion = {5 Portionen},
        source = {@rosakochtgruen}
    ]
    {Cremige \addtoidx{Curry}-Pasta}

    \graph
        {
            big=Hauptgerichte/Nudelgerichte/Cremige_Curry-Pasta/big.jpg,
            small=Hauptgerichte/Nudelgerichte/Cremige_Curry-Pasta/small.jpg
        }

    % \introduction{
    %     EINLEITUNG
    % }

    \ingredients{
        \SI{500}{\g} & Nudeln (Tagliatelle) \\
        2 & Knoblauchzehen \\
        2 & Schalotten \\
         & Öl \\
        \SI{2}{\TL} & Paprikapulver \\
        \SI{360}{\g} & veganes "`\addtoidx{Hühnchen}"' \\
        \SI{2}{\TL} & Curry \\
        3 & Paprika \\
        \SI{400}{\g} & Ananasstücke\index{Ananas!-stücke} \\
        \SI{200}{\g} & Cherry-tomaten\index{Tomaten!Cherry-} \\
        \SI{400}{\g} & Sahne \\
         & Salz \\
         & Pfeffer
    }

    \preparation{
        \step Zuerst die Zwiebeln und den Knoblauch würfeln. Die Paprika in grobe Stücke oder kurze Streifen schneiden und die Tomaten halbieren. Dann in einer großen Pfanne mit Öl die Zwiebeln und den Knoblauch anbraten.
        \step Wenn die Zwiebeln glasig sind, das Hühnchen und die Ananas hinzu geben. Beides mit Curry- und Paprikapulver würzen, gut umrühren und für etwa \SI{10}{\minute} auf mittlerer Stufe anbraten. Währenddessen Nudelwasser in einem sehr großen Topf an aufkochen und die Nudeln nach Packungsangabe zubereiten.
        \step Danach mit der Sahne ablöschen, gut unterrühren und zum Abschluss die Tomaten und die Paprika dazugeben. Alles zusammen noch kurz für etwa \SI{5}{\minute} erwärmen damit die Paprika noch etwas knackig bleibt. Mit Salz und Pfeffer abschmecken.
        \step Abschließend die Nudeln abgießen und das Curry darüber in den großen Topf geben. Alles gut vermengen und servieren.
    }

    % \suggestion[TITEL EINES VORSCHLAGS]{
	% 	VORSCHLAG (DURCH HORIZONTALE LINIE VOM REZEPT GETRENNT)
    % }

    \hint{
        Bei der Paprika am besten verschiedene Farben verwenden! Das vegane "`Hühnchen"' kann auch gut durch Tofu ersetzt werden - ganz nach Geschmack. Bei der Sahne am besten auf einen hohen Fettanteil achten (zum Beispiel von Schlagfix).
    }

\end{recipeDP}
