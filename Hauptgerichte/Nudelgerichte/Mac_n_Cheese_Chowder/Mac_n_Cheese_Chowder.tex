\begin{recipeDP}
    [
        preparationtime = {\SI{60}{\minute}},
        % bakingtime = {\SI{ZEIT}{\minute} bis \SI{ZEIT}{\minute}},
        % bakingtemperature = {\protect\bakingtemperature{fanoven=\SI{TEMPERATUR}{\celsius}}},
        portion = {4 Portionen},
        source = {@thevegansara}
    ]
    {Mac\textquotesingle n\textquotesingle Cheese \addtoidx{Chowder}}

    \graph
        {
            big=Hauptgerichte/Nudelgerichte/Mac_n_Cheese_Chowder/big.jpg,
            small=Hauptgerichte/Nudelgerichte/Mac_n_Cheese_Chowder/small.jpg
        }

    % \introduction{
    %     EINLEITUNG
    % }

    \ingredients{
        \SI{500}{\g} & \addtoidx{Nudeln} \\
        \\
        \multicolumn{2}{l}{\textbf{Cashew-Creme:}} \\
        \SI{120}{\g} & Cashewkerne \\
        \SI{475}{\ml} & Gemüsebrühe \\
        \SI{150}{\g} & eingelegte, geröstete Paprika\index{Paprika!geröstet} \\
        \SI{4}{\EL} & \addtoidx{Hefeflocken} \\
        \\
        \multicolumn{2}{l}{\textbf{Suppe:}} \\
        \SI{5}{\EL} & Butter \\
        2 & Zwiebeln \\
        \SI{5}{\EL} & Mehl \\
        \SI{4}{\EL} & Senf \\
        \\
        \SI{1}{\l} & Gemüsebrühe \\
        4 Stangen & \addtoidx{Sellerie} \\
        2 & Möhren \\
        \SI{500}{\g} & \addtoidx{Brokkoli} \\
        \\
        \SI{700}{\ml} & Milch \\
        \SI{2}{\EL} & Essig \\
         & Salz \\
         & Pfeffer
    }

    \preparation{
        \step Für die Cashew-Creme zuerst die Cashews, mit der Brühe, den Paprikastücken und Hefeflocken in einem Mixer pürieren bis die Soße cremig ist. Dann beiseite stellen. Das Gemüse säubern und klein schneiden, den Brokkoli in kleine Röschen zerteilen.
        \step In einem sehr großen Topf die Butter schmelzen und die kleingeschnittenen Zwiebeln etwa \SI{5}{\minute} darin andünsten. Wenn die Zwiebeln glasig sind, den Senf hinzu geben und vorsichtig das Mehl mit einem Schneebesen einrühren. Die Mehlschwitze auf mittlerer Stufe etwa \SI{2}{\minute} erwärmen. Nebenher die Nudeln nach Packungsangabe zubereiten und dann das Wasser abgießen.
        \step Die Mehlschwitze mit etwas von der Brühe cremig rühren. Wenn keine Klümpchen mehr übrig sind, den Rest der Gemüsebrühe dazugeben und das klein geschnittene Gemüse darin aufkochen und etwa \SI{20}{\minute} köcheln.
        \step Abschließend die Milch, die Cashew-Creme und den Essig hinzugeben und mit Salz, Pfeffer und Hefeflocken abschmecken. Die Nudeln im dem Chowder servieren.
    }

    % \suggestion[TITEL EINES VORSCHLAGS]{
	% 	VORSCHLAG (DURCH HORIZONTALE LINIE VOM REZEPT GETRENNT)
    % }
    %
    % \hint{
    %     HINWEIS (IN EINEM KASTEN UNTEN AUF DER SEITE)
    % }

\end{recipeDP}
