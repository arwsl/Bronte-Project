\begin{recipeDP}
    [
        preparationtime = {\SI{30}{\minute}},
        % bakingtime = {\SI{ZEIT}{\minute} bis \SI{ZEIT}{\minute}},
        % bakingtemperature = {\protect\bakingtemperature{fanoven=\SI{TEMPERATUR}{\celsius}}},
        portion = {4 Portionen},
        source = {@cookingforpeanuts}
    ]
    {Tofu-Tomaten-\addtoidx{Nudeln}}

    \graph
        {
            % big=TEIL/KAPITEL/REZEPT/big.jpg,
            small=Hauptgerichte/Nudelgerichte/Tofu_Tomaten_Nudeln/small.jpg
        }

    % \introduction{
    %     EINLEITUNG
    % }

    \ingredients{
        \SI{500}{\g} & Nudeln \\
        \SI{400}{\g} & Naturtofu\index{Tofu!Natur-} \\
        \SI{70}{\ml} & Olivenöl \\
        2 & Knoblauchzehen \\
        \SI{20}{\g} & Hefeflocken \\
        \SI{2}{\EL} & weiße \addtoidx{Miso}-Paste \\
        \SI{180}{\g} & sonnengetrocknete Tomaten\index{Tomaten!sonnengetrocknet} \\
        & Zitronensaft \\
        & Chiliflocken \\
    }

    \preparation{
        \step Die Nudeln nach Packungsanleitung kochen, bis sie gar sind. Von dem Kochwasser \SI{400}{\ml} aufheben, den rest abgießen.
        \step Während die Nudeln kochen, mit einem Mixer den Tofu mit dem Olivenöl, dem Knoblauch, den Hefeflocken und der Miso-Paste vermischen. Das Nudelwasser kann dann eingerührt werden, zunächst nur \SI{200}{\ml}. Die Soße soll cremig werden, nach Bedarf bs zu \SI{400}{\ml} dazugeben.
        \step Mit einem Teigschaber die Soße in einen Topf geben und auf mittlerer Stufe erhitzen. Währenddessen die sonnengetrockneten Tomaten etwas vom Öl befreien und grob zerschneiden. Wenn die Soße warm ist, die Nudeln und die Tomatenstücke hinzugeben und vermengen.
        \step Zum servieren mit Zitronensaft beträufeln und mit Chiliflocken würzen.
    }

    % \suggestion[TITEL EINES VORSCHLAGS]{
	% 	VORSCHLAG (DURCH HORIZONTALE LINIE VOM REZEPT GETRENNT)
    % }
    %
    % \hint{
    %     HINWEIS (IN EINEM KASTEN UNTEN AUF DER SEITE)
    % }

\end{recipeDP}
