\begin{recipeDP}
    [
        preparationtime = {\SI{40}{\minute}},
        % bakingtime = {\SI{ZEIT}{\minute} bis \SI{ZEIT}{\minute}},
        % bakingtemperature = {\protect\bakingtemperature{fanoven=\SI{TEMPERATUR}{\celsius}}},
        portion = {4 Portionen},
        source = {zuckerjagdwurst.com}
    ]
    {\addtoidx{Tortellini}-Suppe}

    \graph
        {
            big=Hauptgerichte/Nudelgerichte/Tortellini-Suppe/big.jpg,
            small=Hauptgerichte/Nudelgerichte/Tortellini-Suppe/small.jpg
        }

    % \introduction{
    %     EINLEITUNG
    % }

    \ingredients{
        \SI{500}{\g} & Tortellini \\
        1 & Zwiebel \\
        1 & Knoblauchzehe \\
        4 & Möhren \\
        1 & \addtoidx{Pastinake} \\
        1 Stange & \addtoidx{Lauch} \\
        \SI{100}{\g} & Spinat \\
        \SI{2}{\EL} & Mehl \\
        \SI{200}{\ml} & Gemüsebrühe \\
        \SI{400}{\ml} & Soja-Cuisine \\
        \SI{600}{\ml} & Wasser \\
        \SI{1}{\TL} & Chiliflocken \\
        \SI{2}{\EL} & \addtoidx{Hefe!-flocken} \\
        1 Prise & Zucker \\
         & Salz \\
         & Pfeffer \\
         & Muskat \\
         & Zitronensaft \\
         & Öl
    }

    \preparation{
        \step Zwiebel und Knoblauch schälen und fein würfeln. Möhren und Pastinake schälen, ebenfalls in kleine Würfel schneiden. Lauch waschen und in dünne Ringe schneiden. Spinat ebenfalls waschen und grob hacken oder auftauen.
        \step Etwas Pflanzenöl in einem großen Topf bei mittlerer Hitze erwärmen. Zuerst die Zwiebelwürfel für drei Minuten anschwitzen, bis sie glasig sind. Danach Pastinake und Möhren dazugeben, mit einer Prise Zucker bestreuen und fünf Minuten anbraten, bis die Möhre leicht karamellisiert. Nun Knoblauch und Chiliflocken dazugeben und ein paar Minuten anbraten, bis alles leicht gebräunt ist.
        \step Das Gemüse mit Mehl bestäuben und 2 Minuten erhitzen. Danach Gemüsebrühe und Wasser unter Rühren aufgießen, die pflanzliche Sahne dazugeben und alles gut verrühren. Die Suppe einmal aufkochen lassen.
        \step Sobald die Flüssigkeit kocht, vegane Tortellini und Lauch dazugeben und alles mit Salz, Pfeffer und optional mit Hefeflocken würzen. Alles so lange köcheln lassen, bis die Tortellini gar sind (Packungsangabe beachten), dabei immer mal wieder umrühren. Falls die Suppe zu dick wird, etwas Wasser dazugeben. In den letzten 2 Minuten den frischen Spinat dazugeben. Die Tortellini-Suppe mit etwas Zitronensaft beträufeln und zum Servieren noch mal abschmecken und mit frisch geriebenem Muskat bestreuen.
    }

    % \suggestion[TITEL EINES VORSCHLAGS]{
	% 	VORSCHLAG (DURCH HORIZONTALE LINIE VOM REZEPT GETRENNT)
    % }
    %
    % \hint{
    %     HINWEIS (IN EINEM KASTEN UNTEN AUF DER SEITE)
    % }

\end{recipeDP}
