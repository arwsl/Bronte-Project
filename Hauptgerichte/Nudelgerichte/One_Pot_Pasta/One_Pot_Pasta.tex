\begin{recipeDP}
    [
        preparationtime = {\SI{35}{\minute}},
        % bakingtime = {\SI{ZEIT}{\minute} bis \SI{ZEIT}{\minute}},
        % bakingtemperature = {\protect\bakingtemperature{fanoven=\SI{TEMPERATUR}{\celsius}}},
        portion = {4 Portionen},
        source = {@plantbasedcoupable}
    ]
    {One Pot Pasta}

    \graph
        {
            big=Hauptgerichte/Nudelgerichte/One_Pot_Pasta/big.jpg,
            small=Hauptgerichte/Nudelgerichte/One_Pot_Pasta/small.jpg
        }

    % \introduction{
    %     EINLEITUNG
    % }

    \ingredients{
        1 & Zwiebel \\
        2 & Paprika \\
        2 & Zucchini \\
        \SI{3}{\EL} & Olivenöl \\
        \SI{800}{\g} & gehackte Tomaten \\
        \SI{400}{\g} & Kokosmilch \\
        \SI{2}{\TL} & Curry-Paste \\
        \SI{500}{\g} & Nudeln \\
        \SI{500}{\ml} & Wasser \\
         & Salz \\
         & Pfeffer \\
         \SI{250}{\g} & Erbsen (Tiefkühl-)
    }

    \preparation{
        \step Zuerst die Zwiebel klein würfeln, die Paprikas und die Zucchinis würfeln.
        Alles zusammen mit dem Öl in einem großen Topf anbraten
        \step Nach 5 Minuten die Tomaten, die Kokosmilch und die Curry Paste hinzugeben und gut verrühren.
        Einmal aufkochen und dann die Nudeln hinein geben.
        Alles mit Wasser aufgießem, sodas die Nudeln mit Wasser bedeckt sind.
        Mit Salz und Pfeffer würzen und garen bis die Nudeln die gewüschte Konsistenz haben.
        Kurz vor Ende noch die Erbesen heingeben, sodass sie aufgetaut sind.
    }

    % \suggestion[TITEL EINES VORSCHLAGS]{
	% 	VORSCHLAG (DURCH HORIZONTALE LINIE VOM REZEPT GETRENNT)
    % }

    % \hint{
    %     HINWEIS (IN EINEM KASTEN UNTEN AUF DER SEITE)
    % }

\end{recipeDP}