\begin{recipeDP}
    [
        preparationtime = {\SI{20}{\minute}},
        % bakingtime = {\SI{12}{\minute} bis \SI{15}{\minute}},
        % bakingtemperature = {\protect\bakingtemperature{fanoven=\SI{180}{\celsius}}},
        portion = {4 Portionen},
        source = {@healthygirlkitchen}
    ]
    {Italienischer Nudeltopf}

    \graph
        {
            % big=example-image,
            small=Hauptgerichte/Nudelgerichte/Italienischer_Nudeltopf/small.jpg
        }

    % \introduction{einleitung}

    \ingredients{
        \SI{500}{\g} & Nudeln \\
        \SI{1,2}{\l} & Wasser \\
        \SI{600}{\ml} & passierte Tomaten \\
        \SI{90}{\g} & Oliven \\
        \SI{30}{\g} & sonnengetrocknete Tomaten \\
        \SI{2}{\EL} & Kapern \\
        \SI{2}{\TL} & Basilikum \\
        \SI{1}{\TL} & Oregano \\
        \SI{1}{\TL} & Knoblauch \\
        \SI{100}{\g} & Spinat \\
         & Pfeffer \\
         & Salz \\
         & Chiliflocken
    }

    \preparation{
        \step Alle Zutaten in einem Topf oder einer großen Pfanne zusammen aufkochen. Mit Salz, Pfeffer und Chiliflocken würzen. Nach dem Aufkochen auf mittlerer Hitze etwa \SI{10}{\minute} weiter köcheln lassen. Währenddessen regelmäßig umrühren damit nichts anbrennt.
    }

    % \suggestion[Title of Suggestion]{
	% 	Suggestion
    % }

    % \hint{Hint}

\end{recipeDP}
