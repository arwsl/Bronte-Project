\begin{recipeDP}
    [
        preparationtime = {\SI{45}{\minute}},
        % bakingtime = {\SI{12}{\minute} bis \SI{15}{\minute}},
        % bakingtemperature = {\protect\bakingtemperature{fanoven=\SI{180}{\degree}}},
        portion = {5 Portionen},
        source = {@cookingforpeanuts}
    ]
    {\addtoidx{Brokkoli}, \addtoidx{Kichererbsen} in heller Soße}

    \graph
        {
            big=Hauptgerichte/Nudelgerichte/Brokkoli_und_Kicherebsen/big.jpg
            small=Hauptgerichte/Nudelgerichte/Brokkoli_und_Kicherebsen/small.jpg
        }

    \introduction{Eine helle Soße, die gut zu Nudeln passt}

    \ingredients{
        \SI{150}{\g} & \addtoidx{Cashews} \\
         & Wasser \\
        \\
        \SI{300}{\g} & Brokkoli \\
        \SI{45}{\g} & \addtoidx{Hefeflocken} \\
        \SI{1}{\TL} & Knoblauchpulver \\
        \SI{1}{\TL} & Zwiebelpulver \\
        \SI{1}{\TL} & Pfeffer \\
        \SI{350}{\milli\l} & Wasser\\
        \SI{340}{\g} & gekochte Kichererbsen \\
        \\
         & Nudeln
    }

    \preparation{
        \step Zuerst die Cashews in kochendem Wasser einlegen, sodass die \SI{2}{\cm} bedeckt sind. Dann etwa \SI{20}{\minute} ziehen lassen.

        \step in der Zwischenzeit den Brokkoli vorbereiten: säubern und in kleine Stücke zerlegen. Dann in kochendem Wasser etwa \SI{3}{\minute} blanchieren. Dann unter kaltem Wasser abschrecken. In dem heißen Wasser des Brokkolis anschlie0end die Nudeln nach Bedarf kochen.

        \step Für die helle Soße die Cashew abgießen und mit den Hefeflocken, den Gewürzen und etwa \SI{350}{\ml} Wasser zusammen pürieren, bis eine glatte Soße entsteht. Die fertige Soße in einer großen Pfanne erhitzen, bis sie langsam eindickt. Dann Die Kichererbsen und den Brokkoli unterrühren. Mit Nudeln servieren.
    }

    % \suggestion[Title of Suggestion]{
	% 	Suggestion
    % }
    %
    \hint{Damit der Brokkoli beim Kochen seine Farbe behält, etwa einen halben Teelöffel Natron ins Kochwasser geben.}

\end{recipeDP}
