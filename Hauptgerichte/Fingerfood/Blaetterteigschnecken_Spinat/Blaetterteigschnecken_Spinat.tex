\begin{recipeDP}
    [
        preparationtime = {\SI{15}{\minute}},
        bakingtime = {\SI{20}{\minute}},
        bakingtemperature = {\protect\bakingtemperature{fanoven=\SI{160}{\celsius}}},
        portion = {15 Schnecken},
        source = {kitchensplace.de}
    ]
    {Blätterteigschnecken mit Spinat}

    \graph
        {
            big=Hauptgerichte/Fingerfood/Blaetterteigschnecken_Spinat/big.jpg,
            small=Hauptgerichte/Fingerfood/Blaetterteigschnecken_Spinat/small.jpg
        }

    % \introduction{
    %     EINLEITUNG
    % }

    \ingredients{
        1 & Blätterteigrolle \index{Blätterteig}\\
        \SI{400}{\g} & frischer \addtoidx{Spinat} \\
        \SI{60}{\g} & \addtoidx{Cashews} \\
        1 & Knoblauchzehe \\
        \SI{40}{\g} & Olivenöl \\
        \SI{2}{\EL} & Frischkäse \\
         & Salz \\
         & Pfeffer \\
    }

    \preparation{
        \step Den Blattspinat zerrupfen und in einer tiefen Pfanne mit etwas Olivenöl anschwitzen, bis er zerfällt. Mit etwas Meersalz und Pfeffer würzen.
        \step Die Knoblauchzehe leicht hacken und zusammen mit den Cashewkernen, Olivenöl, etwas Frischkäse und mindestens der Hälfte des Spinats (vorher ausdrücken!) im Mixer oder mit dem Pürierstab zu einer gleichmäßigen Creme pürieren. Mit etwas Salz abschmecken.
        \step Den Blätterteig ausrollen und mit der Creme beschmieren. Anschließend mit dem restlichen Spinat belegen, nachdem die Flüssigkeit des Spinats auch hier vorher etwas ausgedrückt wurde.
        \step Die Rolle langsam zusammen rollen und anschließend etwa \SI{1,5}{\cm} dicke Scheiben abschneiden. Diese auf ein Backblech legen und bei \SI{160}{\celsius} Umluft für \ca 15 bis 20 Minuten backen.
    }

    % \suggestion[TITEL EINES VORSCHLAGS]{
	% 	VORSCHLAG (DURCH HORIZONTALE LINIE VOM REZEPT GETRENNT)
    % }
    %
    % \hint{
    %     HINWEIS (IN EINEM KASTEN UNTEN AUF DER SEITE)
    % }

\end{recipeDP}
