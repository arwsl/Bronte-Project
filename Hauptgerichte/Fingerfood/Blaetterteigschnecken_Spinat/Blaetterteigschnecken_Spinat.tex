\begin{recipeDP}
    [
        preparationtime = {\SI{20}{\minute}},
        bakingtime = {\SI{20}{\minute}},
        bakingtemperature = {\protect\bakingtemperature{fanoven=\SI{160}{\celsius}}},
        portion = {15},
        source = {kitchensplace.de}
    ]
    {Blätterteigschnecken mit Spinat}

    \graph
        {
            big=Hauptgerichte/Fingerfood/Blaetterteigschnecken_Spinat/big.jpg,
            small=Hauptgerichte/Fingerfood/Blaetterteigschnecken_Spinat/small.jpg
        }

    % \introduction{
    %     EINLEITUNG
    % }

    \ingredients{
        \SI[]{275}{\g} & \addtoidx{Blätterteig} \\
        \SI{400}{\g} & Spinat (frisch) \\
        \SI{60}{\g} & Cashewkerne\index{Cashews} \\
        1 & Knoblauchzehe \\
        \SI{40}{\g} & Olivenöl \\
        \SI{2}{\TL} & Frischkäse \\
         & Salz \\
         & Pfeffer \\
    }

    \preparation{
        \step Den Blattspinat in einer tiefen Pfanne mit etwas Olivenöl anschwitzen, bis er zerfällt.
        Mit etwas Meeraalz und Pfeffer würzen.
        \step Die Knoblauchzehe leicht hacken und zusammen mit den Cashewkernen, Olivenöl und etwas Frischkäse im Mixer oder mit dem Pürierstab zu einer gleichmäßigen Creme pürieren.
        Mit etwas Salz abschmecken.
        \step Den Blätterteig ausrollen und mit der Cashewcreme beschmieren.
        Anschließend mit Spinat belegen, nachdem die Flüssigkeit des Spinats vorher etwas ausgedrückt wurde.
        \step Die Rolle langsam zusammen rollen und anschließend 1 bis 2 Zentimeter dicke Scheiben abschneiden. Diese auf ein Backblech legen und bei \SI{160}{\celsius} Umluft für etwa 15 bis 20 Minuten backen. 
    }

    % \suggestion[TITEL EINES VORSCHLAGS]{
	% 	VORSCHLAG (DURCH HORIZONTALE LINIE VOM REZEPT GETRENNT)
    % }

    % \hint{
    %     HINWEIS (IN EINEM KASTEN UNTEN AUF DER SEITE)
    % }

\end{recipeDP}