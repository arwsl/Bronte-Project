\begin{recipeDP}
    [
        preparationtime = {\SI{45}{\minute}},
        % bakingtime = {\SI{ZEIT}{\minute} bis \SI{ZEIT}{\minute}},
        % bakingtemperature = {\protect\bakingtemperature{fanoven=\SI{TEMPERATUR}{\celsius}}},
        % portion = {PORTIONEN/MENGE},
        source = {veganblatt.com}
    ]
    {Hirse Laibchen}

    \graph
        {
            big=Hauptgerichte/Fingerfood/Hirse_Laibchen/big.jpg,
            small=Hauptgerichte/Fingerfood/Hirse_Laibchen/small.jpg
        }

    % \introduction{
    %     EINLEITUNG
    % }

    \ingredients{
        \SI{180}{\g} & \addtoidx{Hirse} \\
        \SI{360}{\ml} & Gemüsebrühe \\
        \SI{3}{\TL} & \addtoidx{Flohsamenschalen} \\
        \SI{1}{\EL} & Speisestärke \\
        \SI{2}{\EL} & Mehl \\
        1 Prise & Salz \\
         & Petersilie \\
         & Öl zum Braten
    }

    \preparation{
        \step Die Hirse in die kochende Gemüsebrühe einrühren und auf kleiner Flamme mit Deckel für 7 Minuten köcheln lassen.
        Danach für weitere 10 min zugedeckt quellen lassen.

        \step Den Ei-Ersatz (Flohsamenschalen und Speisestärke) sowie das Mehl einrühren.
        Mit Petersilie und etwas Salz würzen, nach Belieben weitere Gewürze verwenden.
        Es soll eine formbare Masse entstehen.
        Mit feuchten Händen Laibchen daraus formen.

        \step Eine Pfanne mit dem Öl aufstellen und die Laibchen von beiden Seiten knusprig goldbraun braten.
    }

    \suggestion[Servieren]{
		Mit Salat, Tomatensauce und Petersilie bestreut servieren oder in einen Burger packen und genießen!
    }

    % \hint{
    %     HINWEIS (IN EINEM KASTEN UNTEN AUF DER SEITE)
    % }

\end{recipeDP}
