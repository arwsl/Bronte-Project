\begin{recipeDPToTest}
    [
        preparationtime = {\SI{15}{\minute}},
        bakingtime = {\SI{12}{\minute} bis \SI{15}{\minute}},
        bakingtemperature = {\protect\bakingtemperature{topbottomheat=\SI{200}{\celsius}}},
        portion = {6 Portionen},
        source = {deutschlandistvegan.de}
    ]
    {\addtoidx{Blätterteig}-Schnecken mit \addtoidx{Spinat}}

    \graph
        {
            % big=example-image
            small=Hauptgerichte/Fingerfood/Blätterteig-Schnecken_mit_Spinat/small.jpg
        }

    % \introduction{einleitung}

    \ingredients{
        1 Paket & Blätterteig \\
        1 & Zwiebel \\
        \SI{2}{\EL} & Olivenöl \\
        \SI{200}{\g} & Spinat \\
        \SI{2}{\EL} & Pflanzensahne \\
        \SI{2}{\EL} & \addtoidx{Hefeflocken} \\
        \SI{1}{\EL} & Zitronensaft \\
         & Salz \\
         & Pfeffer \\
         & Chiliflocken
    }

    \preparation{
        \step Zunächst Zwiebel fein würfeln und mit Olivenöl in einer Pfanne glasig anbraten. Anschließend Spinat klein schneiden, mit den restlichen Zutaten in die Pfanne geben und für 3-5 Minuten dünsten lassen. Dann gut abkühlen lassen.
        \step Blätterteig gleichmäßig mit der Füllung bestreichen, dabei einen 2 cm breiten Rand frei lassen. Anschließend mit der langen Seite einrollen und mit einem scharfen Messer fingerbreite Scheiben abschneiden.
        \step Danach die Blätterteig-Schnecken mit Spinat auf ein Backblech geben und im vorgeheizten Backofen bei \SI{200}{\celsius} für 12-15 Minuten goldbraun backen.
    }

    % \suggestion[Title of Suggestion]{
	% 	Suggestion
    % }
    %
    % \hint{Hint}

\end{recipeDPToTest}
