\begin{recipeDP}
    [
        preparationtime = {\SI{110}{\minute}},
        bakingtime = {\SI{15}{\minute} bis \SI{20}{\minute}},
        bakingtemperature = {\protect\bakingtemperature{topbottomheat=\SI{180}{\celsius}}},
        % portion = {PORTIONEN/MENGE},
        source = {@annibacktvegan}
    ]
    {Laugenkonfekt}

    \graph
        {
            big=Hauptgerichte/Fingerfood/Laugenkonfekt/big.jpg,
            small=Hauptgerichte/Fingerfood/Laugenkonfekt/small.jpg
        }

    % \introduction{
    %     EINLEITUNG
    % }

    \ingredients{
        \SI{25}{\g} & vegane Butter \\
        \SI{225}{\g} & Mehl \\
        \SI{1}{\TL} & Zucker \\
        \SI{1}{\TL} & Salz \\
		\SI[parse-numbers = false]{\nicefrac{1}{2}}{\pck} & Trockenhefe \index{Hefe!Trocken-}\\
        \SI{60}{\ml} & Wasser \\
        \SI{60}{\ml} & Sojamilch \\
        \\
        \SI{1,5}{\l} & Wasser \\
        \SI{3}{\EL} & \addtoidx{Natron} \\
        \\
         & grobes Salz \\
         & Sonnenblumenkerne \\
         & Sesam
    }

    \preparation{
        \step Zuerst Mehl in eine Schüssel geben und eine Mulde in die Mitte drücken.
        Da hinein den Zucker, die Hefe und das Wasser (lauwarm) füllen. 
        Die Zutaten in der Mulde vorsichtig vermischen und abgedeckt stehen lassen.

        \step In der Zwischenzeit die Pflanzenmilch mit der Butter zusamen in der Mirkowelle erwärmen, bis die Butter geschmolzen ist.
        Danach die warme (nicht heiße) Flüssigkeit mit in die Mulde zum Vorteig geben.
        Alles gut miteinander verkneten - der Teig sollte nicht mehr kleben.

        \step Den Hefeteig für \SI[]{60}{\minute} an einem warmen Ort gehen lassen.
        Kurz vor Ende der Zeit in einem Topf das Wasser mit dem Natron zusammen zum Kochen bringen.

        \step Den Teig gut durchkneten und in 3 gleiche Teile teilen. 
        Diese Teile zu Rollen formen und mit einem Messer jeweils \SI[]{3}{\centi\meter} dicke Stücke abschneiden.
        Ein Backblech mit Backpapier bereit stelen.

        \step Mit einer Schaumkelle die Teigstücke für 20 bis 30 Sekunden in das kochende Natronwasser geben.
        Anschließend abtropfen lassen und auf das Backpapier legen.
        Sofort die Toppings drauf verteilen und leicht andrücken.
        Die Laufenstücke im Backofen bei \SI{180}{\celsius} Ober-/Unterhitze für \SI{15}{\minute} bis \SI{20}{\minute} backen.
    }

    % \suggestion[TITEL EINES VORSCHLAGS]{
	% 	VORSCHLAG (DURCH HORIZONTALE LINIE VOM REZEPT GETRENNT)
    % }

    % \hint{
    %     HINWEIS (IN EINEM KASTEN UNTEN AUF DER SEITE)
    % }

\end{recipeDP}