\begin{recipeDPToTest}
    [
        preparationtime = {\SI{10}{\minute}},
        bakingtime = {\SI{10}{\minute} bis \SI{12}{\minute}},
        bakingtemperature = {\protect\bakingtemperature{topbottomheat=\SI{200}{\celsius}}},
        portion = {6 Portionen},
        source = {deutschlandistvegan.de}
    ]
    {\addtoidx{Käsestangen}\index{Käse}}

    \graph
        {
            % big=example-image
            small=Hauptgerichte/Fingerfood/Kaesestangen/small.jpg
        }

    % \introduction{einleitung}

    \ingredients{
        1 Paket & \addtoidx{Blätterteig} \\
        \SI{100}{\ml} & Pflanzensahne \\
        \SI{3}{\EL} & \addtoidx{Hefeflocken} \\
        \SI{1}{\EL} & Zitronensaft \\
        \SI[parse-numbers = false]{\nicefrac{1}{2}}{\TL} & Senf \\
        \SI[parse-numbers = false]{\nicefrac{1}{2}}{\TL} & Paprikapulver \\
        \SI[parse-numbers = false]{\nicefrac{1}{2}}{\TL} & Zwiebelpulver \\
         & Salz \\
         & Pfeffer
    }

    \preparation{
        \step Alle Zutaten (außer Blätterteig) zu einer cremigen Masse verrühren. Danach Blätterteig halbieren, eine Hälfte mit der Masse bestreichen und die zweite Blätterteig-Hälfte darauf geben. Mit den Händen etwas andrücken.
        \step Dann je 1 cm breite Streifen schneiden, die Enden gegeneinander eindrehen und auf ein Backblech geben. Anschließend die Käsestangen im vorgeheizten Backofen bei \SI{200}{\celsius} für 10-12 Minuten goldbraun backen.
    }

    % \suggestion[Title of Suggestion]{
	% 	Suggestion
    % }
    %
    % \hint{Hint}

\end{recipeDPToTest}
