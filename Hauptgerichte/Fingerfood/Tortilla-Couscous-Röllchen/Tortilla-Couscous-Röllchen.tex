\begin{recipeDP}
    [
        preparationtime = {\SI{25}{\minute}},
        % bakingtime = {\SI{12}{\minute} bis \SI{15}{\minute}},
        % bakingtemperature = {\protect\bakingtemperature{fanoven=\SI{180}{\celsius}}},
        % portion = {40 Stück},
        source = {deutschlandistvegan.de}
    ]
    {\addtoidx{Tortilla}-\addtoidx{Couscous}-Röllchen}

    \graph
        {
            % big=example-image
            small=Hauptgerichte/Fingerfood/Tortilla-Couscous-Röllchen/small.jpg
        }

    \introduction{Leckere Röllchen gefüllt mit Couscous und Spinat dürfen auf eurem nächsten Party-Buffet für veganes Fingerfood nicht fehlen!}

    \ingredients{
        \SI{120}{\g} & Couscous \\
        \SI{300}{\ml} & kochendes Wasser \\
        1 Handvoll & \addtoidx{Spinat} \\
        1 Zweig & Petersilie \\
        \SI{2}{\EL} & Zitronensaft \\
        \SI{1}{\TL} & Salz \\
        \SI[parse-numbers = false]{\nicefrac{1}{2}}{\TL} & Chili-Flocken \\
         & Pfeffer \\
        \\
        4 bis 5 & Tortilla-Wraps
    }

    \preparation{
        \step Den Couscous mit dem kochenden Wasser übergießen und ziehen lassen, bis das Wasser komplett aufgesaugt ist.
        \step Die restlichen Zutaten mit einem Mixer fein pürieren und unter den Couscous rühren.
        \step Die Couscous-Füllung gleichmäßig auf die Tortilla-Wraps streichen. Dann die Wraps zusammenrollen und fingerdicke Scheiben runterschneiden.
    }

    % \suggestion[Title of Suggestion]{
	% 	Suggestion
    % }

    \hint{Am besten mit kleinen Spießen und einer Soße (etwa Barbecuesoße) servieren.}

\end{recipeDP}
