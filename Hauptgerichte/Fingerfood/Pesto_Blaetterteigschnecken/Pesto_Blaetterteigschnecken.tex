\begin{recipeDP}
    [
        preparationtime = {\SI{30}{\minute}},
        bakingtime = {\SI{15}{\minute}},
        bakingtemperature = {\protect\bakingtemperature{topbottomheat=\SI{225}{\celsius}}},
        portion = {26 Stück},
        source = {freudeamkochen.at}
    ]
    {Blätterteigschnecken mit rotem \addtoidx{Pesto}}

    \graph
        {
            big=Hauptgerichte/Fingerfood/Pesto_Blaetterteigschnecken/big.jpg,
            % small=TEIL/KAPITEL/REZEPT/small.jpg
        }

    % \introduction{
    %     EINLEITUNG
    % }

    \ingredients{
        \SI{50}{\g} & Cashewkerne\index{Cashews} \\
        \SI{200}{\g} & getrocknete Tomaten \index{Tomaten!getrocknet} \\
        1 & Knoblauchzehe \\
        \SI[parse-numbers = false]{\nicefrac{1}{2}}{Zweig} & Rosmarin \\
        \SI[]{1}{\EL} & Balsamicoessig \\
        \SI[]{2}{\EL} & Öl \\
        \\
        \SI[]{275}{\g} & \addtoidx{Blätterteig} \\
         & Sojadrink
    }

    \preparation{
        \step Zuerst den Backofen auf \SI{225}{\celsius} Ober-/Unterhitze vorheizen. Den Blätterteig aus dem Kühlschrank nehmen.
        \step Für das Pesto die Cashewkerne in einer beschichteten Pfanne ohne Fett kurz rösten und anschließend im Universalzerkleinerer fein hacken. Die abgetropften Tomaten grob schneiden, den geschälten Knoblauch grob hacken und die Nadeln des gewaschenen Rosmarins fein hacken. Tomaten, Knoblauch, Rosmarin, Balsamicoessig und das Öl zu den Cashewkernen in den Universalzerkleinerer geben und alles mixen, bis sich eine homogene Masse gebildet hat.
        \step Den Blätterteig aus der Packung nehmen, ausrollen, aber noch am Papier lassen und das Tomatenpesto auf den Teig verteilen, wobei auf der oberen Längsseite \SI[]{1}{\cm} frei bleibt. Den Blätterteig von der unteren Längsseite aus mit Hilfe des Papiers fest aufrollen, mit einem Pinsel die freie Teigstelle mit dem Sojadrink einstreichen, die Rolle schließen und noch einmal fest rollen. Die Rolle rundherum mit dem Sojadrink bestreichen und mit einem sehr scharfen Messer vorsichtig in \SI[]{1,5}{\cm} dicke Scheiben schneiden.
        \step Die Schnecken auf das Backblech legen und 15 Minuten backen. Da das Tomatenpesto leicht dunkel wird, zwischendurch einen Kontrollblick in den Backofen werfen.
    }

    % \suggestion[TITEL EINES VORSCHLAGS]{
	% 	VORSCHLAG (DURCH HORIZONTALE LINIE VOM REZEPT GETRENNT)
    % }

    % \hint{
    %     HINWEIS (IN EINEM KASTEN UNTEN AUF DER SEITE)
    % }

\end{recipeDP}