\begin{recipeDP}
    [
        preparationtime = {\SI{45}{\minute}},
        % bakingtime = {\SI{25}{\minute}},
        % bakingtemperature = {\protect\bakingtemperature{fanoven=\SI{180}{\degree}}},
        portion = {3 Portionen},
        source = {Chefkoch: roddenberry}
    ]
    {Rote Linsen-Curry mit Süßkartoffeln}

    \graph
        {
            big=Hauptgerichte/Eintoepfe/Rote_Linsen_Curry/big.png,
            small=Hauptgerichte/Eintoepfe/Rote_Linsen_Curry/small.png
        }

    \introduction{vegan, gesund, schnell und günstig}

    \ingredients{
        \SI{450}{\g} & Süßkartoffeln \\
        1 & rote Paprika \\
        \SI{200}{\g} & rote Linsen \\
        2 & Zwiebeln \\
        2 & Knoblauchzehen \\
        \SI{1}{\EL} & Olivenöl \\
        \SI{500}{\ml} & Kokosmilch \\
        \SI{250}{\ml} & Gemüsebrühe \\
        \SI{2}{\EL} & Tomatenmark \\
        \SI{2}{\TL} & Currypulver \\
        \SI{2}{\TL} & Kurkuma \\
        \SI{1}{\TL} & Garam Masala \\
         & Salz \\
         & Pfeffer
    }

    \preparation{
        \step Die Süßkartoffeln schälen und würfeln. Die Paprikaschote würfeln, den Knoblauch schälen und die Zwiebeln klein schneiden.
        \step Das Olivenöl in einem großen Topf erhitzen und die Zwiebeln glasig anschwitzen. Den Knoblauch durch die Presse in den Topf drücken, kurz mitbraten und dann Süßkartoffeln und Paprika in den Topf geben.
        \step Das Tomatenmark und die Gewürze dazugeben, kurz mitbraten, anschließend die Linsen hinzufügen. Nun alles mit Kokosmilch und Gemüsebrühe aufgießen und ca. 25 min. köcheln lassen.
    }

    % \suggestion[Title of Suggestion]{
	% 	Suggestion
    % }

    \hint{Das Curry schmeckt am nächsten Tag meist noch besser und lässt sich auch gut einfrieren.}

\end{recipeDP}
