\begin{recipeDP}
    [
        preparationtime = {\SI{45}{\minute}},
        bakingtime = {\SI{25}{\minute}},
        bakingtemperature = {\protect\bakingtemperature{fanoven=\SI{180}{\celsius}}},
        portion = {4 Portionen},
        source = {@petadeutschland}
    ]
    {Kürbisrisotto\index{Risotto}}

    \graph
        {
            big=Hauptgerichte/Eintoepfe/Kuerbisrisotto/big.jpg,
            small=Hauptgerichte/Eintoepfe/Kuerbisrisotto/small.jpg
        }

    % \introduction{einleitung}

    \ingredients{
        1 & rote Zwiebel \\
        1 & Knoblauchzehe \\
        \SI{2}{\EL} & Olivenöl \\
        \SI{2}{\Ta} & Risottoreis\index{Reis!Risotto-} \\
        \SI{1}{\Ta} & Weißwein \\
        \SI{4}{\Ta} & Gemüsebrühe \\
        \SI{1}{Dose} & Kürbispüree \\
        \SI{1}{\TL} & Ingwer, gehackt \\
        \SI{1}{Prise} & Zimt \\
        \SI{4}{\EL} & Rosmarin \\
        \SI{4}{\EL} & Basilikum, gehackt \\
        1 & Hokkaidokürbis\index{Kuerbis@Kürbis!Hokkaido-} \\
        \SI{1}{\EL} & Olivenöl (zum Backen) \\
        \SI{4}{\EL} & Walnüsse, gehackt
    }

    \preparation{
        \step Den Kürbis waschen, schneiden und in zwei Hälften teilen. Die Kürbiskerne entfernen. Eine Hälfte in kleine Würfel schneiden, die andere in lange Streifen. Die Schale ist essbar und muss nicht entfernt werden. Den Backofen auf \SI{180}{\celsius} vorheizen.

        \step Olivenöl in einem großen Topf erhitzen und Zwiebel und Knoblauch hinzugeben. Alles goldbraun anbraten und dann den Risottoreis hinzugeben. Das Ganze unter Rühren für \SI{5}{\minute} anrösten, danach die Kürbiswürfel hinzugeben. Dann nach und nach, unter ständigem Rühren, den Weißwein zum Reis dazu geben. Bei mittlerer Hitze kochen lassen. Danach langsam und unter ständigem Rühren die Gemüsebrühe hinzugeben. Reis aufkochen. Kürbispüree hinzufügen und noch fünf weitere Minuten kochen.

        \step Risotto vom Ofen nehmen und Ingwer, Zimt, Basilikum, die Hälfte des Rosmarins und die Hälfte der Walnüsse unterrühren. Das Risotto in eine Auflaufform geben, und über darüber die Kürbisstreifen. Diese mit dem restlichen Olivenöl bestreichen. Für \SI{25}{\minute}, oder bis die Kürbisstreifen leicht goldbraun gebrannt sind und der Reis weich ist, backen lassen. Das Risotto aus dem Ofen nehmen und den Rest der Walnüsse und des Rosmarins zurückgeben.
    }

    % \suggestion[Title of Suggestion]{
	% 	Suggestion
    % }
    %
    % \hint{Hint}

\end{recipeDP}
