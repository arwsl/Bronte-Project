\begin{recipeDP}
    [
        preparationtime = {\SI{25}{\minute}},
        % bakingtime = {\SI{12}{\minute} bis \SI{15}{\minute}},
        % bakingtemperature = {\protect\bakingtemperature{fanoven=\SI{180}{\degree}}},
        portion = {4 Portionen},
        source = {chefkoch.de}
    ]
    {Bunter Kichererbseneintopf}

    \graph
        {
            % small=Recipes/MainCourses/BBQChicken/Small.jpg,
            big=Hauptgerichte/Eintoepfe/Bunter_Kichererbseneintopf/big.jpg
        }

    % \introduction{einleitung}

    \ingredients{
        \SI{1}{Stange} & Lauch, groß \\
        2 & Paprikaschoten, gelb, rot \\
        2 & Knoblauchzehen \\
        \SI{800}{\g} & Dosentomaten \\
        \SI{2}{\EL} & Tomatenmark \\
        \SI{400}{\g} & \addtoidx{Kichererbsen}, aus der Dose \\
        \SI{1}{\l} & Gemüsebrühe \\
        \SI{225}{\g} & Blattspinat \\
        \SI{1}{\TL} & Paprikapulver, scharf \\
        \SI{1}{\TL} & Paprikapulver, mild \\
         & Salz \\
         & Pfeffer
    }

    \preparation{
        \step Das Gemüse waschen, den Lauch in dünne Ringe schneiden, die Paprikaschoten würfeln. Die Kichererbsen gut abspülen, abtropfen lassen.

        \step Das Olivenöl in einem großen Topf erhitzen, Lauchscheiben und Paprikawürfel darin unter Rühren \ca \SI{5}{\minute} scharf anbraten. Die Knoblauchzehen dazu pressen, das Tomatenmark dazu geben und beides kurz mit braten. Mit den etwas zerkleinerten Tomaten und der Gemüsebrühe auffüllen und die Kichererbsen sowie das Paprikapulver zufügen.

        \step Alles aufkochen und dann bei kleinerer Hitze ca. 5 Minuten kochen lassen. Den Blattspinat etwas zerkleinern, in die Suppe geben und nochmal \ca \SI{3}{\minute} kochen lassen. Alles mit Salz und Pfeffer abschmecken.
    }

    % \suggestion[Title of Suggestion]{
	% 	Suggestion
    % }
    %
    % \hint{Hint}

\end{recipeDP}
