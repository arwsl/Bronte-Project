\begin{recipeDP}
    [
        preparationtime = {\SI{40}{\minute}},
        % bakingtime = {\SI{ZEIT}{\minute} bis \SI{ZEIT}{\minute}},
        % bakingtemperature = {\protect\bakingtemperature{fanoven=\SI{TEMPERATUR}{\celsius}}},
        portion = {5 Portionen},
        source = {@veganewunder}
    ]
    {Gnocchi-Suppe mit Bohnen}

    \graph
        {
            big=Hauptgerichte/Suppen_und_Eintoepfe/Gnocchi-Suppe_Bohnen/big.jpg,
            small=Hauptgerichte/Suppen_und_Eintoepfe/Gnocchi-Suppe_Bohnen/small.jpg
        }

    % \introduction{
    %     EINLEITUNG
    % }

    \ingredients{
        \SI{4}{\EL} & Olivenöl \\
        2 & Zwiebeln \\
        4 & Koblauchzehen \\
        \SI{1}{\EL} & Kräuter der Provence \\
        \SI{2}{\EL} & Tomatenmark \\
        \SI{280}{\g} & getrocknete Tomaten in Öl \index{Tomaten!sonnengetrocknet} \\
        \SI{800}{\g} & Dosentomaten in Stücken \\
        \SI{2}{\l} & Gemüsebrühe \\
        \SI{1}{\kg} & \addtoidx{Gnocchi} \\
        \SI{400}{\g} & Weiße Bohnen \\
        \SI{300}{\g} & Spinat \\
         & Salz \\
         & Pfeffer
    }

    \preparation{
        \step Olivenöl in einem Topf erhitzen.
        Zwiebeln würfeln, dazugeben und 3 Minuten glasig braten.
        Knoblauch hacken, getrocknete Tomaten klein schneiden, mit Tomatenmark sowie Kräuter der Provence \ca 2 Minuten im Topf mit anrösten.
        \step Mit Tomaten in Stücken sowie Gemüsebrühe ablöschen und 10 Minuten bei geschlossenem Deckel köcheln lassen.
        Dann die Gnocchi dazugeben, 2 weitere Minuten köcheln lassen.
        \step Den gewaschenen Blattspinat, die vorgekochten weißen Bohnen, Salz, Pfeffer und optional Basilikum dazugeben.
        Weitere 5 Minuten köcheln lassen und genießen!
    }

    % \suggestion[TITEL EINES VORSCHLAGS]{
	% 	VORSCHLAG (DURCH HORIZONTALE LINIE VOM REZEPT GETRENNT)
    % }

    % \hint{
    %     HINWEIS (IN EINEM KASTEN UNTEN AUF DER SEITE)
    % }

\end{recipeDP}