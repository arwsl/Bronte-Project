\begin{recipeDP}
    [
        preparationtime = {\SI{40}{\minute}},
        % bakingtime = {\SI{ZEIT}{\minute} bis \SI{ZEIT}{\minute}},
        % bakingtemperature = {\protect\bakingtemperature{fanoven=\SI{TEMPERATUR}{\celsius}}},
        portion = {4 Portionen},
        source = {@elavegan}
    ]
    {Indisches Kürbis\index{Kürbis!Hokkaido-}-\addtoidx{Curry}}

    \graph
        {
            big=Hauptgerichte/Suppen_und_Eintoepfe/Kuerbis-Curry/big.jpg,
            small=Hauptgerichte/Suppen_und_Eintoepfe/Kuerbis-Curry/small.jpg
        }

    % \introduction{
    %     EINLEITUNG
    % }

    \ingredients{
        \SI[parse-numbers = false]{\nicefrac{1}{2}}{\EL} & Öl \\
        \SI{110}{\g} & Zwiebel \\
        \SI{150}{\g} & Möhren \\
        1 & Paprika \\
        3 & Knoblauchzehen \\
        \SI{5}{\cm} & Ingwer \\
        \\
        \SI[parse-numbers = false]{\nicefrac{1}{2}}{\EL} & Currypulver \\
        \SI[parse-numbers = false]{\nicefrac{3}{4}}{\TL} & Kurkuma \\
        \SI{1}{\TL} & Kreuzkümmel \\
        \SI{1}{\TL} & Salz \\
        \SI[parse-numbers = false]{\nicefrac{1}{2}}{\EL} & Räucherpaprika \\
        \SI[parse-numbers = false]{\nicefrac{1}{2}}{\EL} & Pfeffer \\
        \SI{400}{\g} & Passata \\
        \SI[parse-numbers = false]{\nicefrac{1}{2}}{} & Hokkaido-Kürbis \\
        \SI{1}{\TL} & Gemüsebrühe-Pulver \\
        \SI{400}{\ml} & Kokosmilch \\
        \SI{150}{\g} & Spinat \\
         & Petersilie \\
         & Limettensaft \\
        \\
        \SI{300}{\g} & Naturreis
    }

    \preparation{
        \step Öl in einer Pfanne bei mittlerer Hitze erhitzen und die gewürfelte Zwiebel, klein geschnittene Karotte, die gewürfelte Paprika und den klein geschnittenen Kürbis etwa 7 Minuten lang anbraten.
        \step Den Knoblauch pressen und mit dem Ingwer hineingeben. Eine weitere Minute anbraten. Alle Gewürze, Passata, Gemüsebrühe-Pulver und die Kokosmilch hinzufügen. Alles gut verrühren. Nebenbei den Reis in einem extra Topf in knapp \SI{600}{\ml} Wasser zum kochen bringen, dann nur noch köcheln lassen bis der Reis gar ist.
        \step Die Mischung zum Köcheln bringen. Auf kleiner Flamme etwa 8 Minuten köcheln lassen, bis das Gemüse weich ist. (Optional: Jetzt lassen sich gekochte Kichererbsen für zusätzliches Eiweiß hinzufügen.)
        \step Zum Schluss den Spinat dazugeben und weitere 2 Minuten köcheln lassen, dann den Herd ausschalten. Schließlich abschmecken und die Gewürze anpassen. Je nach Geschmack mehr Salz, Pfeffer oder rote Chiliflocken hinzufügen. Mit dem Reis servieren.
    }



    % \suggestion[TITEL EINES VORSCHLAGS]{
	% 	VORSCHLAG (DURCH HORIZONTALE LINIE VOM REZEPT GETRENNT)
    % }

    % \hint{
    %     HINWEIS (IN EINEM KASTEN UNTEN AUF DER SEITE)
    % }

\end{recipeDP}
