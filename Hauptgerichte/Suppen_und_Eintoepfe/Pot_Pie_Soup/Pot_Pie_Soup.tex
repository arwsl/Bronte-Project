\begin{recipeDP}
    [
        preparationtime = {\SI{15}{\minute}},
        bakingtime = {\SI{ZEIT}{\25}},
        % bakingtemperature = {\protect\bakingtemperature{fanoven=\SI{TEMPERATUR}{\celsius}}},
        portion = {6 Portionen},
        source = {serenetrail.com}
    ]
    {\addtoidx{Pot Pie Soup}}

    \graph
        {
            big=Hauptgerichte/Suppen_und_Eintoepfe/Pot_Pie_Soup/big.jpg,
            small=Hauptgerichte/Suppen_und_Eintoepfe/Pot_Pie_Soup/small.jpg
        }

    % \introduction{
    %     EINLEITUNG
    % }

    \ingredients{
        \SI{1}{\EL} & Margarine \\
        \SI[parse-numbers = false]{\nicefrac{1}{2}}{} & Zwiebel \\
        3 & Möhren \\
        3 & Knoblauchzehen \\
        \SI[]{45}{\g} & Mehl \\
        \SI[]{500}{\ml} & Sojamilch \\
        \SI[]{1}{\EL} & Speisestärke \\
        3 & mittlere Kartoffeln \\
        \SI[]{700}{\ml} & Gemüsebrühe \\
        \SI[]{1}{\TL} & Hähnchengewürz \\
        \SI[parse-numbers = false]{\nicefrac{1}{2}}{\TL} & Salz \\
        \SI[parse-numbers = false]{\nicefrac{1}{2}}{\TL} & Pfeffer \\
        \SI[]{1}{\TL} & Rosmarin \\
        \SI[]{150}{\g} & \addtoidx{Erbsen}, gefroren \\
        \SI[]{280}{\g} & veganes Hühnchen \index{Hühnchen (vegan)}
    }

    \preparation{
        \step Zuerst die Zwiebel, die Möhren, den Knoblauch, die Kartoffeln, und den Rosamrin nach Belieben klein schneiden.
        \step In einem großen Topf auf mittelhoher Hitze dann die Margarine schmelzen und dann die Möhen mit den Zwiebeln zusammen hinein geben.
        Den Knoblauch hinzu geben und für etwa \SI[]{4}{\minute} dünsten.
        \step Dann die Hitze reduzieren, das Mehl hineingeben und gut verteilen.
        Die Speisestärke mit einem kleinen Teil der pflanzlichen Milch anrühren und mit der gesamten Milch, der Brühe, den Gewürzen und den Katoffeln zum Rest in den Topf geben.
        \step Für etwa \SI[]{15}{\minute} köcheln lassen, bis die Kartofeln gar sind, dabei immer wieder umrühren, damit nichts anbrent.
        Dann das vegane Hühnchen in mundgerechten Stücken zusammen mit den Erbsen in die Suppe geben.
        Weitere \SI[]{5}{\minute} kochen lassen und dann anrichten.
    }

    % \suggestion[TITEL EINES VORSCHLAGS]{
	% 	VORSCHLAG (DURCH HORIZONTALE LINIE VOM REZEPT GETRENNT)
    % }

    % \hint{
    %     HINWEIS (IN EINEM KASTEN UNTEN AUF DER SEITE)
    % }

\end{recipeDP}
