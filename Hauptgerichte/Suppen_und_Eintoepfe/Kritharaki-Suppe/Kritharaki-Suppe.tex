\begin{recipeDP}
    [
        preparationtime = {\SI{15}{\minute}},
        bakingtime = {\SI{35}{\minute}},
        % bakingtemperature = {\protect\bakingtemperature{fanoven=\SI{180}{\celsius}}},
        portion = {6 Portionen},
        source = {rainbowplantlife.com}
    ]
    {\addtoidx{Kritharaki}-Suppe mit \addtoidx{Kichererbsen}}

    \graph
        {
            big=Hauptgerichte/Suppen_und_Eintoepfe/Kritharaki-Suppe/big.jpg,
            small=Hauptgerichte/Suppen_und_Eintoepfe/Kritharaki-Suppe/small.jpg
        }

    \introduction{Diese Kritharaki-Suppe ist eine herzhafte und einfache Suppe mit Kichererbsen, Kritharaki, Gewürzen und frischen Kräutern, die sich gut für einen Schnellkochtopf eignet.}

    \ingredients{
        \SI{2}{\EL} & natives Olivenöl extra \\
        1 & gelbe Zwiebel \\
        4 & Knoblauchzehen \\
        \SI{2}{\TL} & Paprikapulver \\
        \SI{2}{\TL} & Kreuzkümmel \\
        \SI{1}{\TL} & Chiliflocken \\
        \SI{4}{\EL} & Tomatenmark \\
        \SI{1}{\l} & Gemüsebrühe \\
        \SI{500}{\ml} & Wasser \\
        \SI{250}{\g} & Kritharaki\index{Orzo|see{Kritharaki}} \\
        \SI{450}{\g} & Kichererbsen (Dose) \\
        2 & Lorbeerblätter \\
        \SI{1}{\TL} & Rosmarin \\
        \SI{1,5}{\TL} & Oregano \\
        \SI{1,5}{\TL} & Salz \\
         & Pfeffer \\
         \SI{800}{\g} & Tomatenstücke \\
         10 & sonnengetrocknete Tomaten \\
         \SI{3}{\TL} & Kapern \\
         & Petersilie \\
         & Zitronenabrieb
    }

    \preparation{
        \step Olivenöl in einem großen Topf erhitzen, dann die gewürfelte Zwiebel hinzugeben und für \SI{5}{\minute} braten, bis die Zwiebel leicht braun ist. Den gepressten Knoblauch dazugeben und nochmal für \SI{2}{\minute} braten.
        \step Paprikapulver, Kreuzkümmel, Chiliflocken und Tomatenmark hinzugeben. Ständig umrühren, damit nichts anbrennt und für \SI{2}{minute} erwärmen bis das Tomatenmark dunkler geworden ist. Danach mit einen Schuss der Gemüsebrühe ablöschen und den Topf frei kratzen. Wenn alles in der Brühe ausgelöst ist, die restliche Gemüsebrühe dazugeben. Das Wasser und die Kritharaki, die Kichererbsen, Lorbeerblätter, Rosmarin, Oregano, Salz, Pfeffer und die Tomaten aus der Dose hinzugeben. Gut miteinander verrühren und aufkochen.
        \step Wenn die Suppe einmal aufgekocht ist, auf mittler bis geringe Hitze stellen und die sonnengetrockneten Tomaten mit den Kapern und der Petersilie dazugeben. Mit Salz und Pfeffer abschmecken und mit frischer Petersilie und etwas Zitronenabrieb servieren.
    }

    % \suggestion[Title of Suggestion]{
	% 	Suggestion
    % }

    % \hint{Hint}

\end{recipeDP}
