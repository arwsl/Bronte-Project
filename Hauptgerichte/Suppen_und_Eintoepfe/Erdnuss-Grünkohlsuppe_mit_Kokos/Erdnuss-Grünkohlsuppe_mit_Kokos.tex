\begin{recipeDP}
    [
        preparationtime = {\SI{60}{\minute}},
        % bakingtime = {\SI{ZEIT}{\minute} bis \SI{ZEIT}{\minute}},
        % bakingtemperature = {\protect\bakingtemperature{fanoven=\SI{TEMPERATUR}{\celsius}}},
        portion = {4 Portionen},
        source = {seriouseats.com}
    ]
    {Erdnuss-Grünkohlsuppe mit Kokos}

    \graph
        {
            big=Hauptgerichte/Suppen_und_Eintoepfe/Erdnuss-Grünkohlsuppe_mit_Kokos/big.jpg,
            small=Hauptgerichte/Suppen_und_Eintoepfe/Erdnuss-Grünkohlsuppe_mit_Kokos/small.jpg
        }

    \introduction{
        Diese von Westafrika inspirierte Erdnusssuppe ist eine kreative Fusion, die traditionelle Maafe-Aromen mit Thai-Einflüssen verbindet.
    }

    \ingredients{
        \multicolumn{2}{l}{\textbf{Aromaten}} \\
        6 & Knoblauchzehen \\
        \SI{40}{\g} & Ingwer \\
        \SI{45}{\g} & Koriander \\
        1 & Jalapeño-Chili \\
        & Meersalz \\
        \\
        \multicolumn{2}{l}{\textbf{Suppenbasis}} \\
        \SI{800}{\ml} & \addtoidx{Kokosmilch} \\
        \SI{30}{\ml} & Öl \\
        6 & Frühlingszwiebeln \\
        \SI{1}{\TL} & Kurkuma \\
        \SI{1000}{\ml} & Gemüsebrühe \\
        \\
        \SI{225}{\g} & Erdnüsse \\
        \SI{10}{\g} & Zucker \\
        \\
        \SI{225}{\g} & Süßkartoffeln\index{Kartoffeln!Süß-} \\
        \SI{140}{\g} & Lacinato-\addtoidx{Grünkohl} \\
        \SI{15}{\ml} & Limettensaft \\
        & Pfeffer \\
        & Scharfe Sauce \\
        & Reis
    }

    \preparation{
        \step Knoblauch, Ingwer, Korianderstiele und die Hälfte der Chili in einem Mörser vermischen.
        Mit etwa Salz bestreuen und zu einer groben Paste zerstoßen.
        Die verbleibende Chili-Hälfte dünn schneiden und beiseite stellen.
        \step Eine Dose Kokosmilch öffnen und etwa 3 Esslöffel festes Fett von der Oberfläche in einen großen Topf löffeln.
        Die Hälfte des Öls hinzufügen.
        Bei mittlerer bis hoher Hitze erhitzen, häufig umrühren, bis das Kokosfett sich trennt und die Feststoffe zu zischen beginnen (etwa 4 Minuten).
        Weiter kochen, ständig umrühren, bis die Feststoffe hell goldbraun werden, etwa 1 Minute länger.
        \step Die zerstoßene Knoblauch-Ingwer-Mischung hinzufügen und unter Rühren etwa eine Minute kochen.
        Die Hälfte der geschnittenen Frühlingszwiebeln und der Kurkuma hinzufügen und unter Rühren kochen, bis alles leicht erweicht.
        \step Den Rest beider Kokosmilchdosen zusammen mit der Brühe hinzufügen und gewürfelte Süßkartoffeln hinzufügen.
        Zum Kochen bringen und auf niedriger Hitze köcheln lassen, bis die Süßkartoffeln gar sind, etwa 15 Minuten.
        Währenddessen die Erdnüsse und die restlichen 15 ml Öl in einer Pfanne bei mittlerer Hitze rösten, bis die Erdnüsse dunkelgoldbraun sind.
        \step Eine halbe Tasse der Erdnüsse auf ein Schneidebrett geben und grob hacken.
        Zum Garnieren beiseite stellen.
        Die Hälfte der verbleibenden Erdnüsse in den Mörser und Stößel geben.
        Mit einer Prise Salz und einer Prise Zucker bestreuen.
        Zu einer groben Paste zerstoßen, dann in den Topf geben.
        Mit der anderen Hälfte der verbleibenden Erdnüsse wiederholen.
        \step Wenn die Kartoffeln gar sind, die Suppe mit einem Stabmixer direkt im Topf pürieren, bis sie größtenteils glatt ist.
        Zum Köcheln bringen und Grünkohl einrühren. Kochen, bis der Grünkohl zart ist, etwa 5 Minuten.
        Limettensaft einrühren. Suppe nach Geschmack mit Salz, Pfeffer und scharfer Soße würzen. Koriander­blätter einrühren, etwas zum Garnieren beiseite stellen. Suppe mit Reis servieren, garniert mit Koriander, gehackten Erdnüssen, Frühlingszwiebeln und geschnittenen Chilis.
    }

    % \suggestion[TITEL EINES VORSCHLAGS]{
	% 	VORSCHLAG (DURCH HORIZONTALE LINIE VOM REZEPT GETRENNT)
    % }

    \hint{
        Die Dosen mit Kokosmilch nicht schütteln, bevor man sie öffnet, da das Fett getrennt bleiben soll. Ein stabiler Granitmörser und -stößel verleihen besseren Geschmack als eine Küchenmaschine, aber eine Küchenmaschine kann verwendet werden, wenn man keinen hat. Die Suppe schmeckt aufgewärmt am nächsten Tag noch besser. Für beste Ergebnisse mit rot, schwarz oder braunem Reis servieren und mit frischen Elementen für zusätzliche Textur und Geschmack garnieren.
    }

\end{recipeDP}