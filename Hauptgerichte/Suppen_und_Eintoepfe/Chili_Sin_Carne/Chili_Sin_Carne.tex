\begin{recipeDP}
    [
        preparationtime = {\SI{50}{\minute}},
        % bakingtime = {\SI{ZEIT}{\minute} bis \SI{ZEIT}{\minute}},
        % bakingtemperature = {\protect\bakingtemperature{fanoven=\SI{TEMPERATUR}{\celsius}}},
        portion = {4 Portionen},
        source = {elavegan.com}
    ]
    {\addtoidx{Chili} Sin Carne}

    \graph
        {
            big=Hauptgerichte/Suppen_und_Eintoepfe/Chili_Sin_Carne/big.jpg,
            small=Hauptgerichte/Suppen_und_Eintoepfe/Chili_Sin_Carne/small.jpg
        }

    % \introduction{
    %     EINLEITUNG
    % }

    \ingredients{
        1 & Zwiebel \\
        4 Zehen & Knoblauch \\
        1 & grüne Paprika \index{Paprika!grüne}\\
        3 & Möhren \\
        \SI{560}{\g} & gehackte Tomaten \index{Tomaten!gehackte} \\
        \SI{6}{\EL} & Tomatenmark \\
        \SI{500}{\ml} & Gemüsebrühe \\
        \SI{650}{\g} & gekochte Bohnen \\
        \SI{2}{\TL} & Rübensirup \\
        \SI{1}{\TL} & Kreuzkümmel \\
        \SI{1}{\TL} & Zwiebelpulver \\
        \SI{1}{\TL} & Knoblauchpulver \\
        \SI{1}{\TL} & Salz \\
        \SI[parse-numbers = false]{\nicefrac{1}{2}}{\TL} & Pfeffer \\
        \SI[parse-numbers = false]{\nicefrac{1}{2}}{\TL} & Paprikapulver \\
        \SI[parse-numbers = false]{\nicefrac{1}{4}}{\TL} & Chiliflocken \\
         & Öl \\
        \\
        \SI{250}{\g} & Reis
    }

    \preparation{
        \step Eine große und tiefe Pfanne (oder einen Topf) mit etwas Öl erhitzen. Die gewürfelte Zwiebel und die grob gewürfelte grüne Paprika für etwa 5 Minuten bei mittlerer Hitze anbraten, den Knoblauch hinzufügen und alles bei mittlerer Hitze für 2 weitere Minuten anbraten lassen, dabei hin und wieder umrühren, damit nichts anbrennt.
        \step Das Tomatenmark mit dem Rübensirup hinzugeben und etwas karamellisieren lassen. Danach die Tomaten hinzugeben und alles für 5 Minuten leicht köcheln lassen. Den Reis mist der doppelten Menge Wasser aufsetzen: erst aufkochen und dann auf niedriger Stufe ziehen lassen.
        \step Alle restlichen Zutaten hinzugeben und das Chili auf kleiner Flamme mindestens 30 Minuten köcheln lassen, damit alles gut durchziehen kann. Empfohlener Schritt: \ca 1 bis \SI{1,5}{} Tassen des Chilis in einen anderen Topf gießen und diesen Teil mit einem Pürierstab pürieren. Das pürierte Chili zurück in den großen Topf geben und gut umrühren.
        \step Das Chili am Ende mit dem Reis als Beilage servieren. Alternativ passen auch Kartoffeln, Nudeln oder Brot.
    }

    % \suggestion[TITEL EINES VORSCHLAGS]{
	% 	VORSCHLAG (DURCH HORIZONTALE LINIE VOM REZEPT GETRENNT)
    % }
    %
    % \hint{
    %     HINWEIS (IN EINEM KASTEN UNTEN AUF DER SEITE)
    % }

\end{recipeDP}
