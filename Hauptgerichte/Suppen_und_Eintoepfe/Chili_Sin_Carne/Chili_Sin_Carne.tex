\begin{recipeDP}
    [
        preparationtime = {\SI{45}{\minute}},
        % bakingtime = {\SI{ZEIT}{\minute} bis \SI{ZEIT}{\minute}},
        % bakingtemperature = {\protect\bakingtemperature{fanoven=\SI{TEMPERATUR}{\celsius}}},
        portion = {4 Portionen},
        source = {elavegan.com}
    ]
    {\addtoidx{Chili} Sin Carne}

    \graph
        {
            big=Hauptgerichte/Suppen_und_Eintoepfe/Chili_Sin_Carne/big.jpg,
            small=Hauptgerichte/Suppen_und_Eintoepfe/Chili_Sin_Carne/small.jpg
        }

    % \introduction{
    %     EINLEITUNG
    % }

    \ingredients{
        1 & Zwiebel \\
        4 & Knoblauchzehen \\
        1 & grüne Paprika \\
        2 & Möhren \\
        \SI{560}{\g} & Dosentomaten \\
        \SI{5}{\EL} & Tomatenmark \\
        \SI{500}{\ml} & Gemüsebrühe \\
        \SI{650}{\g} & \addtoidx{Bohnen} aus der Dose \\
        \SI{2}{\TL} & Zucker \\
        \SI{1}{\TL} & Kreuzkümmel \\
        \SI{1}{\TL} & Zwiebelpulver \\
        \SI{1}{\TL} & Knoblauchpulver \\
        \SI{1}{\TL} & Salz \\
        \SI[parse-numbers = false]{\nicefrac{1}{2}}{\TL} & Pfeffer \\
        \SI[parse-numbers = false]{\nicefrac{1}{4}}{\TL} & Paprikapulver \\
        \SI[parse-numbers = false]{\nicefrac{1}{4}}{\TL} & Chilipulver \\
        1 & Chilischote \\
        \SI{2}{\TL} & Öl \\
    }

    \preparation{
        \step Eine große Pfanne (oder einen Topf) mit dem Öl erhitzen.
        Die klein geschnittene Zwiebel und die gewürfelte grüne Paprika für etwa 5 Minuten bei mittlerer Hitze anbraten, den Knoblauch hinein pressen und alles bei mittlerer Hitze für 2 weitere Minuten anbraten lassen, dabei hin und wieder umrühren, damit nichts anbrennt.

        \step Die Tomaten hinzugeben und alles für 5 Minuten leicht köcheln lassen.
        Währenddessen die Möhren grob raspeln.
        Dann alle restlichen Zutaten hinzugeben und das Chili auf kleiner Flamme mindestens 30 Minuten köcheln lassen, damit alles gut durchziehen kann.

        \step Mit Kartoffeln, Reis oder Nudeln servieren. Zum Garnieren eignen sich Koriander oder Petersilie.
    }

    % \suggestion[Servieren]{
		
    % }

    % \hint{
    %     HINWEIS (IN EINEM KASTEN UNTEN AUF DER SEITE)
    % }

\end{recipeDP}