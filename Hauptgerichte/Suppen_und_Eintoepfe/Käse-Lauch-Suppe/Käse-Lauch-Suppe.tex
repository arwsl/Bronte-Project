\begin{recipeDP}
    [
        preparationtime = {\SI{45}{\minute}},
        % bakingtime = {\SI{ZEIT}{\minute} bis \SI{ZEIT}{\minute}},
        % bakingtemperature = {\protect\bakingtemperature{fanoven=\SI{TEMPERATUR}{\celsius}}},
        portion = {4 Portionen},
        source = {cheapandcheerfulcooking.com}
    ]
    {Käse-\addtoidx{Lauch}-Suppe mit Hack}

    \graph
        {
            big=Hauptgerichte/Suppen_und_Eintoepfe/Käse-Lauch-Suppe/big.jpg,
            small=Hauptgerichte/Suppen_und_Eintoepfe/Käse-Lauch-Suppe/small.jpg
        }

    % \introduction{
    %     EINLEITUNG
    % }

    \ingredients{
        \multicolumn{2}{l}{\textbf{Suppe}} \\
        2 Stangen & Lauch \\
        1 Zehe & Knoblauch \\
        1 & Zwiebel \\
        \SI{2}{\EL} & Mehl \\
        \SI{800}{\ml} & Gemüsebrühe \\
        \SI{200}{\ml} & \addtoidx{Soja-Cuisine} \\
        \SI{8}{\EL} & Hefeflocken \\
        \SI{1}{\TL} & Paprikapulver \\
        1 Prise & Muskat \\
         & Pflanzenöl \\
         & Salz \\
         & Pfeffer \\
        \\
        \multicolumn{2}{l}{\textbf{Hack-Einlage}} \\
		\SI{100}{\g} & \addtoidx{Sojagranulat} \\
        \SI{3}{\EL} & Sojasoße \\
         & Pflanzenöl
    }

    \preparation{
        \step Das Sojagranulat in einem Topf mit kochendem Wasser übergießen und 5-10 Minuten ziehen lassen. In ein Sieb abgießen, mit kaltem Wasser durchspülen und überschüssige Flüssigkeit ausdrücken. Lauch putzen und in Ringe schneiden. Zwiebel und Knoblauch schälen und fein hacken.
        \step Einen guten Schuss Pflanzenöl einem großen Topf bei mittlerer bis hoher Temperatur erhitzen. Lauch und Zwiebel für etwa 3 Minuten anbraten, dann Knoblauch dazu geben und ebenfalls kurz mit anbraten. Das Mehl hineingeben, kurz anschwitzen und alles mit Brühe ablöschen. Ein Mal kurz aufkochen lassen und dann bei niedriger Temperatur und geschlossenem Deckel \ca 10 Minuten köcheln lassen.
        \step Das Sojahack in einer Pfanne mit einem guten Schuss Pflanzenöl bei hoher Temperatur anbraten. Mit Sojasauce ablöschen. Die Suppe mit veganer Sahne, Hefeflocken, Paprikapulver, Muskat, Salz und Pfeffer abschmecken. Zusammen mit dem veganen Hackfleisch servieren.
    }

    % \suggestion[TITEL EINES VORSCHLAGS]{
	% 	VORSCHLAG (DURCH HORIZONTALE LINIE VOM REZEPT GETRENNT)
    % }
    %
    % \hint{
    %     HINWEIS (IN EINEM KASTEN UNTEN AUF DER SEITE)
    % }

\end{recipeDP}
