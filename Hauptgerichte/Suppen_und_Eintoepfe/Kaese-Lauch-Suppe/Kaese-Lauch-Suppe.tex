\begin{recipeDP}
    [
        preparationtime = {\SI{45}{\minute}},
        % bakingtime = {\SI{ZEIT}{\minute} bis \SI{ZEIT}{\minute}},
        % bakingtemperature = {\protect\bakingtemperature{fanoven=\SI{TEMPERATUR}{\celsius}}},
        portion = {4 Portionen},
        source = {cheapandcheerfulcooking.com}
    ]
    {Käse-Lauch-Suppe mit Hack}

    \graph
        {
            big=Hauptgerichte/Suppen_und_Eintoepfe/Kaese-Lauch-Suppe/big.jpg,
            small=Hauptgerichte/Suppen_und_Eintoepfe/Kaese-Lauch-Suppe/small.jpg
        }

    % \introduction{
    %     EINLEITUNG
    % }

    \ingredients{
        \multicolumn{2}{l}{\textbf{Suppe}} \\
        3 Stangen & \addtoidx{Lauch} \\
        1 Zehe & Konoblauch \\
        1 & Zwiebel \\
        \SI{2}{\EL} & Mehl \\
        \SI{800}{\ml} & Gemüsebrühe \\
        \SI{200}{\ml} & \addtoidx{Kochcreme} \\
        \SI{8}{\EL} & \addtoidx{Hefeflocken} \\
        \SI{1}{\TL} & Paprikapulver \\
        1 Prise & Muskat \\
         & Salz \\
         & Pfeffer \\
        \\
        \multicolumn{2}{l}{\textbf{Hack-Einlage}} \\
		\SI{100}{\g} & \addtoidx{Sojagranulat} \\
        \SI{3}{\EL} & Sojasoße \\
        \SI{1}{\TL} & \addtoidx{Liquid Smoke} (optional) \\
         & Bratöl
    }

    \preparation{
        \step Das Sojagranulat in einem Topf mit kochendem Wasser übergießen und 5 bis 10 Minuten ziehen lassen.
        Dann abgießen, mit kaltem Wasser durchspülen und überschüssige Flüssigkeit ausdrücken.

        \step Lauch putzen und in Ringe schneiden.
        Zwiebel und Knoblauch schälen und fein hacken.

        \step Einen guten Schuss Pflanzenöl in einem großen Topf bei mittlerer bis hoher Temperatur erhitzen.
        Lauch und Zwiebel für etwa 3 Minuten anbraten, dann Knoblauch dazu geben und ebenfalls kurz mit anbraten.
        Das Mehl hineingeben, kurz anschwitzen und alles mit Brühe ablöschen.
        Ein Mal kurz aufkochen lassen und dann bei niedriger Temperatur und geschlossenem Deckel ca. 10 Minuten köcheln lassen.

        \step Währenddessen das Sojahack in einer Pfanne mit einem guten Schuss Pflanzenöl bei relativ hoher Temperatur anbraten.
        Mit Sojasauce und Liquid Smoke ablöschen.

        \step Die Suppe mit veganer Sahne, Hefeflocken, Paprikapulver, Muskat, Salz und Pfeffer abschmecken.
        Zusammen mit dem veganen Hack servieren.
    }

    % \suggestion[TITEL EINES VORSCHLAGS]{
	% 	VORSCHLAG (DURCH HORIZONTALE LINIE VOM REZEPT GETRENNT)
    % }

    % \hint{
    %     HINWEIS (IN EINEM KASTEN UNTEN AUF DER SEITE)
    % }

\end{recipeDP}