\begin{recipeDPToTest}
    [
        preparationtime = {\SI{60}{\minute}},
        % bakingtime = {\SI{12}{\minute} bis \SI{15}{\minute}},
        % bakingtemperature = {\protect\bakingtemperature{fanoven=\SI{180}{\celsius}}},
        portion = {4 Portionen},
        source = {@petadeutschland}
    ]
    {Kürbisgulasch\index{Gulasch!Kürbis-}}

    \graph
        {
            big=Hauptgerichte/Suppen_und_Eintoepfe/Kuerbisgulasch/big.png,
            % small=Recipes/MainCourses/BBQChicken/Small.jpg
        }

    \introduction{Dieses Kürbisgulasch strotzt nur so vor Aromen und ist ein wohlig-wärmendes Herbst- und Wintergericht.}

    \ingredients{
        1 & Hokkaidokürbis\index{Kuerbis@Kürbis!Hokkaido-} (\ca \SI{1}{\kg}) \\
        1 & gelbe Paprika \\
        \SI{220}{\g} & \addtoidx{Kichererbsen} (aus der Dose) \\
        \SI{250}{\g} & \addtoidx{Linsen} (rote und/oder braune) \\
        \SI{450}{\ml} & \addtoidx{Schwarzbier} \\
        6 & Soft-Datteln, entsteint \\
        2 & Zwiebeln \\
        3 & Knoblauchzehen \\
        \SI{750}{\ml} & Gemüsebrühe \\
        \SI{6}{\EL} & Tomatenmark \\
        \SI{4}{\EL} & Olivenöl \\
        \SI{2}{\TL} & Paprikapulver, edelsüß \\
        \SI{2}{\TL} & Paprikapulver, rosenscharf \\
         & Salz \\
         & Pfeffer \\
         & Petersilie
    }

    \preparation{
        \step Kichererbsen abgießen und mit den Linsen kalt abbrausen. Hokkaido waschen, halbieren und Kerne entfernen. Den Kürbis in etwa 1 cm große Würfel schneiden (mit Schale). Paprika waschen und ebenfalls würfeln.
        \step Zwiebeln und Knoblauch schälen und fein würfeln. In einem großen Topf mit 4 EL Olivenöl andünsten. Tomatenmark und Paprikapulver einrühren. Kichererbsen und Linsen zufügen.
        \step Mit Gemüsebrühe und Schwarzbier aufgießen und etwa 20 Minuten mit Deckel köcheln lassen. Kürbis- und Paprikawürfel zugeben und weitere 30 Minuten köcheln lassen.
        \step Die Datteln kleinschneiden und etwa 10 Minuten vor Ende der Garzeit in den Topf geben. Mit Salz und Pfeffer abschmecken. Vor dem Servieren mit gehackter Petersilie bestreuen.
    }

    % \suggestion[Title of Suggestion]{
	% 	Suggestion
    % }
    %
    % \hint{Hint}

\end{recipeDPToTest}
