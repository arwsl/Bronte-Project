\begin{recipeDP}
    [
        preparationtime = {\SI{45}{\minute}},
        % bakingtime = {\SI{12}{\minute} bis \SI{15}{\minute}},
        % bakingtemperature = {\protect\bakingtemperature{fanoven=\SI{180}{\celsius}}},
        portion = {4 Portionen},
        source = {@byanjushka}
    ]
    {Pilzsuppe}

    \graph
        {
            % big=example-image
            small=Hauptgerichte/Suppen_und_Eintoepfe/Pilzsuppe/small.jpg
        }

    % \introduction{einleitung}

    \ingredients{
        \multicolumn{2}{l}{\textbf{Suppe}} \\
        \SI{50}{\g} & getrocknete Pilze \\
        \SI{500}{\ml} & Wasser \\
        \\
        \SI{2}{\EL} & Öl \\
        2 & Knoblauchzehen \\
        1 & Zwiebel \\
        \SI{500}{\g} & \addtoidx{Champignons} \\
        \SI{500}{\ml} & Gemüsebrühe \\
        \SI{250}{\ml} & Sojasahne \\
         & Salz \\
         & Pfeffer \\
        \\
        \multicolumn{2}{l}{\textbf{dazu servieren}} \\
        \SI{500}{\g} & Nudeln \\
        \SI{250}{\g} & Champignons \\
         & frische Petersilie
    }

    \preparation{
        \step Zuerst die getrockneten Pilze kurz abspülen und dann für etwa \SI{30}{\minute} in \SI{500}{\ml} Wasser einweichen.
        \step Die eingeweichten Pilze abgießen und das Wasser auffangen. Dann in einem großen Topf das Öl erhitzen, den Knoblauch und die Zwiebeln dazugeben und für etwa \SI{3}{\minute} braten, bis die Zwiebeln glasig werden. Nun die grob geschnittenen und die eingeweichten Pilze in den Topf geben und etwa \SI{8}{\minute} braten.
        \step Die Gemüsebrühe und das Wasser der getrockneten Pilze hinzugeben, alles aufkochen und dann \SI{10}{\minute} bis \SI{15}{\minute} köcheln lassen. Danach alles mit einem Stabmixer pürieren, bis eine glatte Suppe entsteht. Die Sojasahne einrühren und mit Salz und Pfeffer würzen.
        \step Die Nudeln nach Packungsangabe kochen und die restlichen Pilze in einer Pfanne anbraten. Auch hier mit Salz und Pfeffer würzen.
        \step Zum servieren die Nudeln in eine Schale geben, die Suppe darüber gießen und mit den angebratenen Pilzen und Petersilie garnieren.
    }

    % \suggestion[Title of Suggestion]{
	% 	Suggestion
    % }

    \hint{Die getrockneten Pilz und das dazu benutzte Wasser können durch \SI{250}{\g} Pilze und etwa \SI{250}{\g} Gemüsebrühe ersetzt werden.}

\end{recipeDP}
