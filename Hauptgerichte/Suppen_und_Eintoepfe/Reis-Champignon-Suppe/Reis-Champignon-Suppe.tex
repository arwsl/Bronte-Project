\begin{recipeDP}
    [
        preparationtime = {\SI{40}{\minute}},
        % bakingtime = {\SI{ZEIT}{\minute} bis \SI{ZEIT}{\minute}},
        % bakingtemperature = {\protect\bakingtemperature{fanoven=\SI{TEMPERATUR}{\celsius}}},
        portion = {4 Portionen},
        source = {@vegamelon}
    ]
    {Reis-Champignon-\addtoidx{Suppe}}

    \graph
        {
            % big=TEIL/KAPITEL/REZEPT/big.jpg,
            small=Hauptgerichte\Suppen_und_Eintoepfe\Reis-Champignon-Suppe\small.jpg
        }

    % \introduction{
    %     EINLEITUNG
    % }

    \ingredients{
        \SI{3}{\EL} & Olivenöl \\
        \SI{1}{\TL} & Thymian \\
        \SI[parse-numbers = false]{\nicefrac{1}{2}}{\TL} & Oregano \\
        2 & Schalotten \\
        2 Zehen & Knoblauch \\
        \SI{300}{\g} & \addtoidx{Reis} \\
        \SI{700}{\g} & \addtoidx{Champignons} \\
        \SI{700}{\ml} & Gemüsebrühe \\
        \SI{200}{\g} & Soja-Sahne
         & Tahin \\
         & Salz \\
         & Pfeffer \\
         & Zitronensaft
    }

    \preparation{
        \step Zuerst die Pilze säubern und in grobe Scheiben schneiden, die Schalotten fein würfeln. In einem großen Topf dann das Olivenöl erwärmen und die Schalotten dazugeben, den Knoblauch hinein pressen und für etwa \SI{5}{\minute} glasig dünsten. Danach die Champignons, den Reis und die Gewürze dazu geben, unterrühren und gemeinsam anbraten.
        \step Wenn die Pilze etwa Flüssigkeit abgegeben haben, die Gemüsebrühe hinein geben. Alles aufkochen und dann auf niedriger Stufe köcheln lassen (gelegentlich umrühren). Wenn der Reis gar ist, die Soja-Sahne dazu geben, nach Bedarf auch etwas Tahin. Die Suppe kann nach Belieben mit mehr Gemüsebrühe noch etwas verflüssigt werden. Abschließend mit Salz, Pfeffer und Zitronensaft abschmecken.
    }

    % \suggestion[TITEL EINES VORSCHLAGS]{
	% 	VORSCHLAG (DURCH HORIZONTALE LINIE VOM REZEPT GETRENNT)
    % }
    %
    % \hint{
    %     HINWEIS (IN EINEM KASTEN UNTEN AUF DER SEITE)
    % }

\end{recipeDP}
