\begin{recipeDP}
    [
        preparationtime = {\SI{30}{\minute}},
        % bakingtime = {\SI{ZEIT}{\minute} bis \SI{ZEIT}{\minute}},
        % bakingtemperature = {\protect\bakingtemperature{fanoven=\SI{TEMPERATUR}{\celsius}}},
        portion = {4 Portionen},
        source = {biancazapatka.com}
    ]
    {Gelbe Linsen\index{Linsen!gelb} \addtoidx{Curry} Suppe}

    \graph
        {
            big=Hauptgerichte/Suppen_und_Eintoepfe/Gelbe_Linsen_Suppe/big.jpg,
            small=Hauptgerichte/Suppen_und_Eintoepfe/Gelbe_Linsen_Suppe/small.jpg
        }

%     % \introduction{
%     %     EINLEITUNG
%     % }

    \ingredients{
        \SI[parse-numbers = false]{\nicefrac{1}{2}}{\EL} & Öl \\
        1 & Zwiebel \\
        3 & Konlauchzehen \\
        1 Stück & Ingwer \\
        2 & Möhren \\
        \SI[]{1}{\EL} & Currypulver \\
        \SI[parse-numbers = false]{\nicefrac{1}{2}}{\TL} & Kurkuma \\
        \SI[]{1}{\TL} & Chilipulver \\
        \SI[]{180}{\g} & gelbe Linsen \\
        \SI[]{700}{\ml} & Gemüsebrühe \\
        \SI[]{200}{\ml} & Kokosmilch \\
        \SI[]{30}{\ml} & Sojasoße \\
         & Salz \\
         & Pfeffer \\
         & Petersilie oder Koriander \\
         & Sesam \\
         & Limetten oder Zitronen \\
         & Fladenbrot oder Reis
    }

    \preparation{
        \step Das Öl in einer großen Pfanne oder Topf erhitzen und die Zwiebel 2-3 Minuten anbraten, bis sie glasig und leicht gebräunt sind.
        Dann Knoblauch und Ingwer hinzugeben und eine weitere Minute anbraten.
        Als nächstes die Möhren hinzufügen und weitere 2 bis 3 Minuten dünsten.
        Nun Currypulver, Kurkuma und Chili oder Paprikapulver darüber streuen und eine Minute anrösten.

        \step Die Linsen in ein feinmaschiges Sieb geben und unter fließendem Wasser abspülen und abtropfen lassen.
        In die Pfanne geben und 2 Minuten anschwitzen.
        Nun das Ganze mit Gemüsebrühe, Kokosmilch und Sojasoße aufgießen, verrühren und zum Kochen bringen.
        Die Hitze auf niedrige Stufe reduzieren und die Suppe abgedeckt \ca 15 Minuten köcheln lassen, bis die Linsen gar sind.
        Sollte die Suppe zu dick werden,noch mehr Gemüsebrühe oder Kokosmilch hinzufügen.

        \step Anschließend die Suppe abschmecken.
        Die Linsensuppe in Schüsseln anrichten und nach Belieben mit frischer Petersilie oder Koriander, Sesam und einem Spritzer frischen Limetten- oder Zitronensaft garnieren. Dazu schmeckt Fladenbrot, Naan oder Reis.
    }

%     % \suggestion[TITEL EINES VORSCHLAGS]{
% 	% 	VORSCHLAG (DURCH HORIZONTALE LINIE VOM REZEPT GETRENNT)
%     % }

    \hint{
        Statt Karotten kann man auch Kürbis oder Süßkartoffeln nehmen.
        Restliche Suppe kann bis zu 5 Tage im Kühlschrank aufbewahrt oder bis zu 3 Monaten eingefroren werden. Beim Aufwärmen ein wenig Flüssigkeit hinzufügen.
    }

\end{recipeDP}