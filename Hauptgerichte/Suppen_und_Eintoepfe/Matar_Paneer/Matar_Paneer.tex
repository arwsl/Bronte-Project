\begin{recipeDP}
    [
        preparationtime = {\SI{40}{\minute}},
        % bakingtime = {\SI{ZEIT}{\minute} bis \SI{ZEIT}{\minute}},
        % bakingtemperature = {\protect\bakingtemperature{fanoven=\SI{TEMPERATUR}{\celsius}}},
        portion = {4 Portionen},
        source = {veganmiche.blog}
    ]
    {Matar Paneer}

    \graph
        {
            big=Hauptgerichte/Suppen_und_Eintoepfe/Matar_Paneer/big.jpg,
            small=Hauptgerichte/Suppen_und_Eintoepfe/Matar_Paneer/small.jpg
        }

    \introduction{
        Einfaches cremiges \addtoidx{Curry} mit \addtoidx{Tofu} und \addtoidx{Erbsen}
    }

    \ingredients{
        \SI{450}{\g} & Naturtofu \\
        \SI{200}{\g} & Tiefkühlerbsen \\
        \\
        2 & weiße Zwiebeln \\
        \SI{2}{\cm} & Ingwer \\
        5 & Knoblauchzehen \\
        2 & grüne Chilis \\
        \SI{100}{\ml} & Wasser \\
        \SI{1}{\TL} & Kreuzkümmel \\
        \SI{500}{\ml} & passierte Tomaten \\
        \SI{1}{\TL} & Koriander \\
        \SI{1}{\TL} & Kurkuma \\
        \SI{1}{\TL} & Currypulver \\
        \SI{1}{\TL} & Salz \\
        \SI{1}{\TL} & Pfeffer \\
        \\
        \SI{50}{\g} & Cashews \\
        \SI{60}{\ml} & Wasser \\
        \SI{300}{\g} & Reis oder Quinoa
    }

    \preparation{
        \step Zuerst die Cashews mit heißem Wasser übergießen und mindestens \SI{10}{\minute} einweichen lassen. Die Zwiebeln mit dem Ingwer, dem Knoblauch, den Chilis und \SI{100}{\ml} Wasser pürieren. Den Reis oder Quinoa in heißem Waser zubereiten.
        \step In einem großen Topf den Kreuzkümmel in etwas Öl erwärmen und nach einer Minute die Zwiebelsoße dazu geben. Zusammen \SI{5}{\minute} köcheln lassen und dann mit den passierten Tomaten ablöschen. Die Gewürze hinzugeben und weitere \SI{5}{\minute} köcheln lassen.
        \step Den Tofu in Würfel schneiden und mit den Erbsen in das Curry geben. Gemeinsam \SI{15}{\minute} köcheln lassen. Währenddessen die eingeweichten Cashews mit \SI{60}{\ml} Wasser zu einer cremigen Soße pürieren.
        \step Abschließend die Cashew-Soße einrühren und das Curry mit Reis oder Quinoa servieren.
    }

    % \suggestion[TITEL EINES VORSCHLAGS]{
	% 	VORSCHLAG (DURCH HORIZONTALE LINIE VOM REZEPT GETRENNT)
    % }

    % \hint{
    %     HINWEIS (IN EINEM KASTEN UNTEN AUF DER SEITE)
    % }

\end{recipeDP}
