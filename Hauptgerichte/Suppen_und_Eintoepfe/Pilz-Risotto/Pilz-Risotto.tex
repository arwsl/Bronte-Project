\begin{recipeDP}
    [
        preparationtime = {\SI{45}{\minute}},
        % bakingtime = {\SI{ZEIT}{\minute} bis \SI{ZEIT}{\minute}},
        % bakingtemperature = {\protect\bakingtemperature{fanoven=\SI{TEMPERATUR}{\celsius}}},
        portion = {4 Portionen},
        source = {zuckerjagdwurst.com}
    ]
    {Pilz-\addtoidx{Risotto}}

    \graph
        {
            big=Hauptgerichte/Suppen_und_Eintoepfe/Pilz-Risotto/big.jpg,
            small=Hauptgerichte/Suppen_und_Eintoepfe/Pilz-Risotto/small.jpg
        }

    % \introduction{
    %     EINLEITUNG
    % }

    \ingredients{
        \SI{30}{\g} & getrocknete Steinpilze \index{Pilze!Stein-} \\
        \SI{800}{\ml} & Wasser \\
        4 & Schalotten \\
        2 & Knoblauchzehen \\
        \SI[]{1}{\TL} & Thymian \\
        \SI[]{500}{\g} & braune \addtoidx{Champignons} \\
        \SI[]{5}{\EL} & Margarine \\
        \SI[]{250}{\g} & Risottoreis \indent{Reis!Risotto-} \\
        \SI[]{2}{\EL} & weißer Balsamico \\
         & Salz \\
         & Pfeffer \\
         & Hefeflocken
    }

    \preparation{
        \step Die getrockneten Steinpilze in heißem Wasser einweichen und mindestens 30 Minuten ziehen lassen.
        Danach durch ein Sieb abgießen, die Brühe aber unbedingt aufheben.
        In der Zwischenzeit Schalotten und Knoblauch schälen und fein würfeln.
        Champignons putzen und vierteln oder mit den Händen in kleine Stücke reißen.
        \step Pflanzenöl in einer Pfanne erhitzen und die Pilze \ca 10 Minuten anbraten, bis sie gebräunt sind.
        Danach die eingeweichten Steinpilze dazugeben und weitere 3 Minuten braten.
        Mit Salz und Pfeffer abschmecken.
        \step Parallel dazu vegane Butter in einem großen Topf bei kleiner Hitze zerlassen und die Schalotten darin 5 Minuten anschwitzen.
        Reis und Knoblauch dazugeben und 6 Minuten anschwitzen, bis der Reis leicht glasig wird; dabei immer wieder rühren, damit nichts anbrennt.
        \step Den Balsamicoessig hinzugeben und kurz einköcheln lassen.
        Ab diesem Zeitpunkt nach und nach die Pilzbrühe dazugeben, aber immer nur so viel, dass der Reis bedeckt ist.
        Danach unter regelmäßigem Rühren bei kleiner Hitze köcheln lassen, bis die Flüssigkeit vom Reis aufgesogen wurde.
        Erst dann wieder Brühe hinzugeben und den Vorgang wiederholen.
        Der Risottoreis ist nach \ca 15 bis 20 Minuten gar.
        (Kurz vor Ende lassen sich auch gut noch \SI[]{100}{\g} Tiefkühlerbsen untermengen.)
        \step Das Risotto mit Salz und Pfeffer abschmecken.
        Die gebratenen Pilze zusammen mit Thymian untermengen und Hefeflocken unterrühren.
    }

    % \suggestion[TITEL EINES VORSCHLAGS]{
	% 	VORSCHLAG (DURCH HORIZONTALE LINIE VOM REZEPT GETRENNT)
    % }

    % \hint{
        
    % }

\end{recipeDP}