\begin{recipeDP}
    [
        preparationtime = {\SI{40}{\minute}},
        % bakingtime = {\SI{ZEIT}{\minute} bis \SI{ZEIT}{\minute}},
        % bakingtemperature = {\protect\bakingtemperature{fanoven=\SI{TEMPERATUR}{\celsius}}},
        portion = {4 Portionen},
        source = {zuckerjagdwurst.com}
    ]
    {Lasagnesuppe}

    \graph
        {
            big=Hauptgerichte/Suppen_und_Eintoepfe/Lasagnesuppe/big.jpg,
            small=Hauptgerichte/Suppen_und_Eintoepfe/Lasagnesuppe/small.jpg
        }

    % \introduction{
    %     EINLEITUNG
    % }

    \ingredients{
        10 & \addtoidx{Lasagneplatten} \\
        \SI{250}{\g} & \addtoidx{Sojagranulat} \\
        1 & Zwiebel \\
        1 & Knoblauchzehe \\
        3 & Möhren \\
        4 & Selleriestangen \index{Sellerie} \\
        \SI{1}{\EL} & Sojasoße \\
        \SI{4}{\EL} & Tomatenmark \\
        \SI{1}{\EL} & Ahornsirup \\
        \SI{2}{\TL} & Basilikum \\
        \SI{2}{\TL} & Oregano \\
        \SI{2}{\TL} & Cayennepfeffer \\
        \SI{1}{\TL} & Paprikapulver \\
        \SI{100}{\ml} & Rotwein \\
        \SI{800}{\g} & gestückelte Tomaten \\
        \SI{1}{\l} & Gemüsebrühe \\
        \SI{15}{\g} & frischer Basilikum \\
         & Salz \\
         & Pfeffer
    }

    \preparation{
        \step Zwiebel und Knoblauch schälen und hacken. Möhren und Selleriestange fein würfeln. Das getrocknete Granulat einweichen, wie auf der Packung angegeben.
        \step Zuerst werden das ausgedrückte Sojagranulat und die Zwiebeln parallel angebraten: Pflanzenöl in einer Pfanne über mittlerer Hitze erwärmen und den pflanzlichen Hackersatz \ca 5~Minuten scharf anbraten, bis er gut gebräunt ist. Sojasauce dazugeben und noch weitere 2 bis 3~Minuten anbraten. Danach die Hitze ausschalten. Parallel dazu etwas Pflanzenöl in einem großen Topf erhitzen und die Zwiebelwürfel auf kleiner Hitze 3 bis 4~Minuten anschwitzen, bis sie glasig sind. Danach den Knoblauch dazugeben und eine weitere Minute anbraten. Tomatenmark und Ahornsirup dazugeben und ca. 3 Minuten leicht anbraten, sodass das Tomatenmark dunkler wird.
        \step Möhren, Sellerie, getrockneten Oregano, Basilikum, Cayennepfeffer und Paprikapulver zu den Zwiebeln geben und alles zusammen 5~Minuten anbraten. Mit veganem Rotwein ablöschen und anschließend Dosentomaten, Gemüsebrühe und den gebratenen Hackersatz dazugeben. Alles einmal aufkochen lassen und etwa 20 Minuten köcheln lassen.
        \step Die trockenen Lasagneplatten grob auseinanderbrechen und zusammen mit frischem Basilikum in die Suppe geben. Die Suppe auf mittlerer Stufe weiterköcheln lassen, bis die Lasagneplatten gar sind, das dauert nur ca. 5 Minuten. Die Suppe noch einmal abschmecken. Servieren lässt sich die Suppe mit etwas Joghurt und frischem Basilikum.
    }

    % \suggestion[TITEL EINES VORSCHLAGS]{
	% 	VORSCHLAG (DURCH HORIZONTALE LINIE VOM REZEPT GETRENNT)
    % }

    % \hint{
    %
    % }

\end{recipeDP}
